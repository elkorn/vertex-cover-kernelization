\section{Algorytm Chen, Kanj, Xia}\label{s_ckx}
\subsection{Wstęp}
\par{
  Zaprezentowane w poprzednich rozdziałach pracy algorytmy związane są przede wszystkim z modyfikacją struktury grafu wejściowego w celu przyspieszenia procesu rozwiązywania problemu pokrycia wierzchołkowego.
  Każdy z przestawionych dotychczas algorytmów może być wykorzystywany niezależnie --- ich działanie jest ortogonalne względem siebie.
  Fakt ten nie uszedł uwadze autorów pracy~\cite{ImprovedBounds10}, którzy dostrzegli potencjał optymalizacji drzemiący w zastosowaniu starannie zaprojektowanego i przeanalizowanego modelu redukcji dziedziny do jądra problemu.
  Model ten oparty jest na stosunkowo prostym algorytmie rekurencyjnym działającym zgodnie z zasadą drzewa poszukiwań z ograniczeniami, popartym wyczerpującą analizą mogących zaistnieć po drodze przypadków.
  Wykorzystane w ramach działania algorytmu funkcje redukujące dziedzinę są od siebie mocno uzależnione oraz uruchamiane w ściśle określonej kolejności w celu zachowania poprawności logiki wysokopoziomowej i maksymalizacji pozytywnych efektów ich wykonywania.
}
\par{
  Wedle autorów, główną motywacją do stworzenia niniejszego algorytmu było udowodnienie wagi gruntownej analizy dziedziny algorytmu (w przeciwieństwie do znacznie powszechniejszej analizy opartej o charakterystyczne przypadki) w celu uzyskania jak najdokładniejszych wartości pesymistycznej złożoności obliczeniowej.
  Zastosowane podejście czerpie inspirację z niezywkle efektywnej metodyki analitycznej ``mierz i zwyciężaj'', zaproponowanej w artykule~\cite{conf/icalp/FominGK05}.

  Drugim głównym celem pracy~\cite{ImprovedBounds10} jest studium najlepszej możliwej złożoności obliczeniowej osiągalnej dla problemu pokrycia wierzchołkowego.
  Wobec odkryć dokonanych w pracach~\cite{Impagliazzo2001512} oraz~\cite{Cai2003789} wyjawiających, iż dopóki nie zostanie udowodnione, że wszystkie problemy $\mathcal{SNP}$ są rozwiązywalne w czasie krótszym niż wykładniczy, istnieje pewna stała $c_0 > 1$, dla której problem pokrycia wierzchołkowego nie może być rozwiązany w czasie $c_0^kn^{O(1)}$.
  Uwaga poświęcona jest zatem próbom jak najbardziej znacznego zmniejszenia stałej $c > 1$, dla której problem pokrycia wierzchołkowego rozwiązywalny jest w czasie $c^kn^{O(1)}$.
}
\par {
  Otrzymany algorytm jest nieskomplikowany i jednolity, zachowuje również formę poddającą się w łatwy sposób dokładnej analizie.
  Kręgosłup algorytmu stanowi utrzymywana na bieżąco kolejka priorytetowa ``pożytecznych'' struktur służacych do redukowania dziedziny problemu.
  Na każdym z etapów algorytm wybiera strukturę o najwyższym priorytecie, zapewniającą najwięcej korzyści.
  Wybrana struktura przetwarzana jest w sposób \emph{jednolity} oraz \emph{nieświadomy} jej kształtu.
  W wyniku analiz udało się doprowadzić do stanu logiki, w której istnieją jedynie dwa możliwe sposoby (odzwierciedlone dwoma gałęziami algorytmu) przetwarzania dowolnej struktury.
  Pozostałe operacje nie posiadają żadnych rozgałęzień --- stanowią działania pomocnicze ``oczyszczające'' graf (podobnie jak algorytmy przetwarzania wstępnego omówione w podrozdziale~\ref{Section_preprocessing}) w celu zapewnienia największej możliwej wydajności następujących operacji z rozgałęzieniami.

  Złożoność czasowa algorytmu wynosi $O(1.2738^k + kn)$, co stawia go w światowej czołówce najszybszych algorytmów rozwiązujących parametryzowany wariant problemu pokrycia wierzchołkowego.
}
\subsection{Podstawowe pojęcia i struktury}\label{ss_ckx_preliminaries}
\par{
  W dalszym toku rozumowanie przyjęte zostaną następujące oznaczenia dla grafu $G=(V, E)$:
  \begin{itemize}
    \item $|G|$ stanowi liczbę wierzchołków $|V|$.
    \item $\tau(G)$ stanowi rozmiar najmniejszego pokrycia wierzchołkowego grafu $G$.
  \end{itemize}
  Dla pewnego wierzchołka $v \in V$:
  \begin{itemize}
    \item $N(v)$ stanowi zbiór wierzchołków sąsiednich względem wierzchołka $v$, to znaczy połączonych krawędzią z tym wierzchołkiem.
    \item $N[v]$ stanowi zbiór $N(v) \bigcup \{v\}$.
    \item $d(v)$ stanowi stopień wierzchołka $v$.
  \end{itemize}
  Dla pewnego zbioru wierzchołków $S \in V$:
  \begin{itemize}
    \item $N(S)$ stanowi sumę zbiorów wierzchołków sąsiednich dla każdego wierzchołka $v_S \in S$ z wyłączeniem $v_S$, to znaczy $N(S)=\bigcup_{v_S\in S}N(v_S) \setminus S$.
    \item $N[S]$ oznacza zbiór wierzchołków $N(S) \bigcup S$.
  \end{itemize}

  Przyjęte zostaje następujące założenie, oparte na drugim twierdzeniu Nemhausera-Trottera (twierdzenie~\ref{nt_lp}) oraz pojęciu NT-dekompozycji, opisanej w rozdziale~\ref{ss_nt_decomposition}.
  \begin{proposition}
    Istnieje algorytm o złożoności czasowej $O(kn + k^3)$ który z instancji problemu pokrycia wierzchołkowego $(G, k)$, gdzie $|G|=n$ konstruuje inną instancję problemu  pokrycia wierzchołkowego $(G_1, k_1)$, dla której $k_1 \leq k$ oraz $|G_1| \leq 2k_1$ oraz $\tau(G) \leq k$ wtedy i tylko wtedy gdy $\tau(G_1) \leq k_1$.
  \end{proposition}

  \begin{definition}
    Instancja $(G_1, k_1)$, otrzymana przez NT-dekompozycję instancji $(G, k)$, stanowi \emph{jądro} instancji $(G, k)$.
  \end{definition}
}
\par{
  Logika głównego algorytmu zrealizowana jest jako drzewo poszukiwań, zgodne z techniką opisywaną w podrozdziale~\ref{ss_branch_and_bound}.
  Każda iteracja podejmuje próby redukcji wartości parametru $k$ instancji wejściowej problemu pokrycia wierzchołkowego $(G, k)$ przez identyfikację pewnego zbioru wierzchołków $S$, który w całości przynależy do optymalnego pokrycia wierzchołkowego grafu $G$ lub też w całości jest z niego wykluczony.
  W przypadku odnalezienia takiego zbioru, algorytm usuwa należące do niego wierzchołki z dziedziny i rekurencyjnie wywołuje sam siebie na zredukowanych instancjach problemu.

  Sposób redukcji instancji problemu oparty jest o następujące prawidłowości zaobserwowane w zachowaniu pokrycia wierzchołkowego w grafie.
  \begin{theorem}
    Dla pewnego wierzchołka $v$ w grafie $G$ istnieje optymalne pokrycie wierzchołkowe grafu $G$ zawierające zbiór $N(v)$ lub co najwyżej $|N(v)| - 2$ wierzchołków należących do zbioru $N(v)$.
  \end{theorem}
  \begin{bproof}
    Jeżeli pewne optymalne pokrycie wierzchołkowe $C$ grafu $G$ zawiera $|N(v)|-1$ wierzchołków należących do zbioru $N(v)$, to musi ono zawierać również wierzchołek $v$, ponieważ krawędź $(w \in N(v) \setminus C, v)$ również musi zostać pokryta.
    Stworzyć można zatem inne optymalne pokrycie wierzchołkowe $C^\prime$ grafu $G$ poprzez zastąpienie wierzchołka $v$ w zbiorze $C$ przez wierzchołek $w$, czego wynikiem jest optymalne pokrycie wierzchołkowe grafu $G$ zawierające $N(v)$.
  \end{bproof}
  \begin{theorem}
    Dla pewnych wierzchołków $u$ oraz $v$ połączonych krawędzią $(u, v) \in E$ w grafie $G=(V, E)$ istnieje optymalne pokrycie wierzchołkowe grafu $G$ zawierające wierzchołek $v$ lub wykluczające wierzchołek $v$ oraz co najmniej jeden wierzchołek sąsiadujący z wierzchołkiem $u$.
  \end{theorem}
  \begin{bproof}
    Jeżeli każde optymalne pokrycie wierzchołkowe $C$ grafu $G$ wykluczałoby wierzchołek $v$, zawierając zarazem wszystkie wierzchołki sąsiadujące z wierzchołkiem $u$ w grafie $G$, to pokrycie $C$ musi również zawierać wierzchołek $u$ --- krawędź $(u, v)$ musi zostać pokryta.
    W tej sytuacji pokrycie $C^\prime=(C \setminus \{u\}) \bigcup \{v\}$ stanowiłoby zgodne z definicją~\ref{def_vc} optymalne pokrycie wierzchołkowe zawierające $v$, przeczące założeniu że każde optymalne pokrycie wierzchołkowe musi wykluczać wierzchołek $v$.
  \end{bproof}
  \begin{definition}
    W przypadku odnalezienia w grafie $G$ pewnej instancji $(G, k)$ problemu pokrycia wierzchołkowego zbioru $S$, który w całości przynależy do optymalnego pokrycia wierzchołkowego grafu $G$ lub też w całości jest z niego wykluczony mówi się, że algorytm \emph{rozgałęzia się na zbiorze $S$}.
    Oznacza to, iż algorytm konstruuje dwie zredukowane instancje problemu.
    \begin{enumerate}
      \item Instancja zawierająca zbiór $S$ w częściowym pokryciu wierzchołkowym.
      \item Instancja wykluczająca zbiór $S$ z częściowego pokrycia wierzchołkowego i zawierająca wszystkie wierzchołki sąsiadujące ze zbiorem $S$ w częściowym pokryciu wierzchołkowym.
    \end{enumerate}
    Jeżeli zbiór $S$ składa się z jednego wierzchołka $v$ mówi się, że algorytm \emph{rozgałęzia się na wierzchołku $v$}.
  \end{definition}
}
\subsubsection{\textbf{Operacja strukturyzacji}}\label{sss_ckx_struction}
\par{
  Wprowadzona przez autorów pracy~\cite{ImprovedBounds10} operacja strukturyzacji stanowi uściślenie operacji strukturyzacji występującej w~pracy~\cite{Ebengger:1984}.
  Operacja strukturyzacji jest pierwszą z podstawowych, nierozgałęziających się funkcji pomagających w utrzymywaniu ``czystej'' struktury grafu dla maksymalizacji efektywności algorytmu głównego.
  Operację tę szczegółowo opisuje pseudokod~\ref{alg_ckx_struction} w podrozdziale~\ref{sss_internals_ckx}.
}
\par{
  \begin{definition}
    Dla pewnych dwóch wierzchołków $u$ oraz $v$ grafu $G=(V, E)$ zbiór $\{u, v\}$ określa się mianem \emph{antykrawędzi} w grafie $G$ jeżeli $(u, v) \notin E$.
  \end{definition}

  Przyjąć należy pewien wierzchołek $v_0$ należący do grafu $G$ oraz posiadający zestaw wierzchołków sąsiednich $N(v_0)=\{v_1, v_2, \ldots, v_p\}$.
  Graf strukturyzowany $G^\prime$ konstruowany jest według następujących kroków.
  \begin{enumerate}
    \item Usuń wierzchołki $v_N \in N[v_0]$ z grafu G i dla każdej antykrawędzi $\{v_i, v_j\}$ w grafie $G$, gdzie $0 < i < j \leq p$ wprowadź nowy wierzchołek $v_{ij}$ do grafu strukturyzowanego $G^\prime$.
    \item Jeżeli $i=j, r\neq s$ oraz $(v_r, v_s) \in E$, dodaj do grafu strukturyzowanego $G^\prime$ krawędź $(v_{ir}, v_{js})$.
    \item Jeżeli $i \neq j$, dodaj do grafu strukturyzowanego $G^\prime$ krawędź $(v_{ir}, v_{js})$.
    \item Dla każdego wierzchołka $u \notin N[v_0]$ dodaj krawędź $(v_{ij}, u)$ jeżeli $(v_i, u) \in E$ lub $(v_j, u) \in E$.
  \end{enumerate}
  \begin{theorem}
    Dla pewnego wierzchołka $v_0$ grafu $G=(V, E)$ o sąsiedztwie $N(v_0); |N(v_0)|=p$ założyć należy istnienie co najwyżej $p-1$ antykrawędzi w zbiorze $N(v_0)$.
    Jeżeli za $G^\prime$ przyjmuje się graf otrzymany w yniku strukturyzacji wierzchołka $v_0$ w grafie $G$, to spełniona jest nierówność $\tau(G) \leq \tau(G^\prime) - 1$.
  \end{theorem}
  \begin{bproof}
    Przyjąć należy notację $\alpha(G)$ oraz $\alpha(G^\prime)$ jako oznaczenie liczebności największego zbioru niezależnego (w rozumieniu definicji~\ref{def_independent_set}) w grafach $G$ oraz $G^\prime$.
    W~pracy~\cite{Ebengger:1984} udowodniono, iż $\alpha(G^\prime)=\alpha(G) - 1$.
    Niech przez $n$ oraz $n^\prime$ oznaczone będą liczby wierzchołków odpowiednio grafu $G$ oraz $G^\prime$.
    W związku z założeniem, iż istnieje co najwyzej $p - 1$ antykrawędzi w zbiorze $N(v_0)$, liczba nowo utworzonych w grafie $G^\prime$ wierzchołków może wynosić co najwyżej $p - 1$.
    Ponieważ z grafu $G$ usunieto $p+1$ wierzchołków, to znaczy zbiór $N[v_0]$, łatwo zauważyć, iż $n^\prime \leq n-2$.
    Z charakterystyki zbioru niezależnego oraz pokrycia wierzchołkowego wywnioskować można, iż dla dowolnego grafu $H$ spełniona jest równość $\alpha(H)+\tau(H)=|H|$.
    Na podstawie powyższych obserwacji stwierdzić można, iż zachowana jest następująca własność.
    \[\tau(G^\prime)=n^\prime-\alpha(G^\prime)\leq (n-2)-(\alpha(G)-1)=\tau(G)-1\]
  \end{bproof}
}