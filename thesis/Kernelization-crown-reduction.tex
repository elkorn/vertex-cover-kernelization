\subsection{Redukcja koron}
\label{ss_kernelization_crown_reduction}
\par{
  Pojęcie \emph{korony grafu} spopularyzowane zostało przede wszystkim dzięki dorobkowi naukowemu M. Fellowsa oraz F. Abu-Khzam.
  Niniejsza praca czerpie z literatury tychże autorów, głównie z pracy~\cite{KernelizationAlgorithms04}.
  Znaczenie koron w grafach jest kluczowe szczególnie dla parametryzowanego wariantu problemu pokrycia wierzchołkowego --- specyficzna budowa tej struktury otwiera drogę do efektywnej redukcji dziedziny poszukiwań do jądra problemu pokrycia wierzchołkowego w grafie.
  Wiele algorytmów wiodących prym pod względem złożoności obliczeniowej opiera się w dużym stopniu właśnie na identyfikacji oraz przetwarzaniu koron lub struktur podobnych koronom.
  Prócz algorytmu analizowanego w niniejszym podrozdziale, podjęto się również analizy i implementacji algorytmu zaproponowanego w pracy~\cite{ImprovedBounds10}, któremu poświęcono podrozdział~\ref{s_ckx}.
}
\subsubsection{\textbf{Kontekst struktur koron w grafach}}
\label{sss_kernelization_crown_context}
\par{
  Bardzo ważna dla opisywanej koncepcji jest praca~\cite{chlebik:crown}, gdzie usystematyzowano miejsce koron wśród szerszej klasy tzw. \emph{struktur zmniejszających zaangażowanie} w grafach.
  \begin{definition}
    Dla grafu $G=(V, E)$ i pewnego podzbioru $U \subseteq V$ ze zbiorem wierzchołków sąsiednich $N(U)$ \emph{strukturą zmniejszającą zaangażowanie} nazywa się uporządkowaną parę $(I, N(I))$ podzbiorów zbioru wierzchołków $V$ spełniającą następujące własności.
    \begin{enumerate}
      \item $I \neq \emptyset$ stanowi niezależny zbiór wierzchołków w grafie $G$.
      \item $N(I)$ stanowi optymalne pokrycie wierzchołkowe grafu zaindukowanego z grafu $G$ zbiorem wierzchołków $I \cup N(I)$.
    \end{enumerate}
  \end{definition}
  Możliwość identyfikacji struktury $(I, N(I))$ zmniejszającej zaangażowanie w grafie $G=(V, E)$ jest ważna ze względu na fakt, iż każdy podzbiór $C=N(I)\cupC^\prime$, gdzie $C^\prime$ stanowi optymalne pokrycie wierzchołkowe dla grafu zaindukowanego z grafu $G$ zbiorem wierzchołków $V \setminus (I\cup N(I))$ stanowi optymalne pokrycie wierzchołkowe grafu $G$.
  Zmniejszenie zaangażowania za pomocą struktury $(I, N(I))$ polega na tym, iż algorytm korzystający z tejże struktury angażuje się w odnajdywanie wyłącznie rozwiązań $C$ spełniających własność $C \cap (I \cup N(I)) = N(I)$, usuwa zbiór $I \cup N(I)$ z dziedziny poszukiwań w grafie $G$, redukując tym samym instancję problemu pokrycia wierzchołkowego do pomniejszonego grafu zaindukowanego z grafu $G$ zbiorem wierzchołków $V \setminus (I \cup N(I))$.

  Charakterystycznym rodzajem struktur zmiejszających zaangażowanie w grafie są tak zwane \emph{NT-dekompozycje}, stanowiące wynik \emph{NT-redukcji} --- operacji opartej na sformułowaniu problemu pokrycia wierzchołkowego jako egzemplarza problemu programowania liniowego i rozwiązania go zgodnie z twierdzeniem~\ref{nt_lp}.\footnote{Sformułowanie problemu jako relaksacji liniowej jest jednym ze sposobów podejścia do identyfikacji struktur zmniejszających zaangażowanie, istnieją również inne sposoby (przykładem może być algorytm J.F. Bussa) --- jednakże wspólnym mianownikiem wszystkich tych metod są korzenie sięgające twierdzenia~\ref{nt_lp}.}
  \begin{definition}\thlabel{def_nt_decomposition}
    Mianem \emph{NT-dekompozycji} (dekompozycji Nemhausera-Trottera) określa się specjalny przypadek struktury $(I, N(I))$ zmniejszającej zaangażowanie w grafie $G=(V, E)$ zidentyfikowanej przez zastosowanie procesu \emph{NT-redukcji} (redukcji Nemhausera-Trottera).
    Proces ten posługuje się funkcją $x: V \rightarrow \{0, \frac{1}{2}, 1\}$ różną od $x \equiv \frac{1}{2}$ określającą przynależność wierzchołków do optymalnego pokrycia wierzchołkowego.
  \end{definition}
  Przyjmując $V_i^x=\{u \in V| x(u)=i\}$ dla każdej wartości $i\in \{0, \frac{1}{2}, 1\}$ zaobserwować można, iż zbiór $V_0^x$ jest niezależny i niepusty w grafie $G$, oraz że zachowana zostaje własność $V_1^x = N(V_0^x)$.
  Szeroko dostępny dowód twierdzenia~\ref{nt_lp} pokazuje, iż zbiór $V_1^x$ stanowi optymalne pokrycie wierzchołkowe grafu zaindukowanego z grafu $G$ zbiorem wierzchołków $V_0^x \cup N(V_0^x)$ --- co prowadzi do wniosku, że para $(V_0^x, N(V_0^x))$ stanowi prawidłową strukturę zmniejszającą zaangażowanieg w grafie $G$.
  Zaangażowanie zostaje zmniejszone do rozwiązań mających część wspólną ze zbiorem $V_0^x \cup V_1^x$ w zbiorze $V_1^x$, a przestrzeń poszukiwań zredukowana zostaje do grafu zaindukowanego z grafu $G$ zbiorem wierzchołków $V_\frac{1}{2}^x$.
}
\par{
  Autorzy pracy~\cite{chlebik:crown} obalają stwierdzenie jakoby koncepcja koron miała być ortogonalna względem NT-redukcji --- postuluje się, że korony stanowią wyspecjalizowaną podklasę NT-dekompozycji o właściwościach opisanych w następującym podrozdziale.
}
\subsubsection{\textbf{Właściwości koron}}
\label{sss_kernelization_crown_main}
\begin{definition}\thlabel{def_crown}
  Mianem \emph{korony} grafu $G=(V, E)$ nazywa się uporządkowaną parę
  podzbiorów wierzchołków $(I, H), I \subseteq V, H \subset V$, zachowujących
  następujące własności.
  \begin{enumerate}
    \item $I \neq \emptyset$ stanowi zbiór niezależny w $G$.
    \item $H=N(I)$.
    \item Istnieje skojarzenie $M=\{e_0, e_1, \ldots, e_p\}, \forall_{e_M=(u,v) \in
      M}: (u\in I \land v\in H) \lor (u \in H \land v \in I)$ takie, że
      $\forall_{v_h \in H}\exists_{e_M=(u,v)\in M}: u = v_h \oplus v = v_h$.
    \item Spełniona jest nierówność $|H| \leq |I|$. (własność przechodnia z własności 3)
  \end{enumerate}
\end{definition}
\begin{definition}
  Zbiór $H$ określa się mianem \emph{głowy korony}.
\end{definition}
\begin{definition}\thlabel{def_crown_head}
  Mianem \emph{szerokości korony} określa się liczebność zbioru stanowiącego jej głowę $|H|$.
\end{definition}
\begin{definition}\thlabel{def_strict_crown}
  Mianem korony \emph{ścisłej} określa się koronę $(I, H)$, dla której spełniona jest nierówność $|H| < |I|$.
\end{definition}
\begin{definition}\thlabel{def_equal_crown}
  Mianem korony \emph{równej} określa się koronę $(I, H)$, dla której spełniona jest równość $|H| = |I|$.
\end{definition}
\begin{theorem}\thlabel{th_crown_vc}
  Jeżeli graf $G=(V,E)$ zawiera koronę $(I,H)$, istnieje optymalna pokrywa 
  wierzchołkowa $C_{\textnormal{OPT}} \in V, H \in C_{\textnormal{OPT}}, I \notin C_{\textnormal{OPT}}$.
\end{theorem}
\begin{bproof}
  Z własności 3.\ definicji~\ref{def_crown}.\ wynika, że każda pokrywa 
  wierzchołkowa $C$ musi zawierać przynajmniej jeden wierzchołek $v_H \in H$.
  Na tej podstawie stwierdzić można, że $|C|\geq|H|$.
  Taki rozmiar pokrywy osiągnąć można przez umieszczenie $C=C\cup H$.
  Należy w~tym miejscu zaznaczyć, że wierzchołki $v_H$ są użyteczne w~kontekście
  możliwości pokrywania krawędzi $e \notin M$, podczas gdy wierzchołki $v_I \in
  I$ nie mają tej cechy.
  Mając to na uwadze łatwo zauważyć, że $|C \cup H| \leq |C \bigcup
  I|$.
  Wniosek płynący z~tej obserwacji jest jednoznaczny: istnieje optymalna pokrywa
  wierzchołkowa $C_{\textnormal{OPT}}; H \in C_{\textnormal{OPT}}, I \notin C_{\textnormal{OPT}}$.
\end{bproof}
W celu odnalezienia korony w~grafie, zastosować można następujący algorytm.
\begin{algorithm}
  \caption{Algorytm odnajdujący koronę w~grafie $G$}\label{alg_findCrown}
  \begin{algorithmic}[1]
    \Function{findCrown}{G, k}
    \State{$M_1\leftarrow$ największe skojarzenie $G$}
    \State{$O\leftarrow v \in V, \neg\exists_{e_{M_1}=(u, w) \in M_1}: u=v \lor w=v$}
    \If{$|M_1| \geq k$}
    \State\textbf{return} nil\Comment{$\neg\exists{C_{\textnormal{OPT}} \in V}: |C_{\textnormal{OPT}}| \leq k$}
  \EndIf
  \State{$M_2 \leftarrow$ maksymalne skojarzenie na krawędziach $O\leftrightarrow N(O)$}
  \If{$|M_2| > k$}
  \State{\textbf{return} nil\Comment{$\neg\exists{C_{\textnormal{OPT}} \in V}: |C_{\textnormal{OPT}}|\leq k$}}
\EndIf
\State{$I_0 \leftarrow v_O\in O, \neg\exists_{e_{M_2}=(u,v)\in M_2}: u=v_O\lor v=v_O$}
\State($n \leftarrow 0$)
\While{$I_{n-1} \neq I_n$}\label{findCrown_while}
\State{$H_n \leftarrow N(I_n)$}\label{findCrown_makeH}
\State{$I_{n+1} \leftarrow I_n\cup N_{M_2}(H_n)$}\label{findCrown_makeI}
\State{$n \leftarrow n+1$}
\EndWhile
\State{\textbf{return} $(I_n,H_n)$}\Comment{$n=N$}
  \EndFunction
\end{algorithmic}
\end{algorithm}\\
Rezultatem działania algorytmu jest korona $(I,H); I=I_N, H=H_N$.

\begin{theorem}
  Algorytm~\ref{alg_findCrown}.\ jest w stanie odnaleźć koronę pod warunkiem, że
  $I_0\neq\emptyset$.
\end{theorem}
\begin{bproof} (Spełnienie własności 1.\ definicji~\ref{def_crown}.)
  \par{
    Bazując na fakcie, iż $M_1$ stanowi największe skojarzenie $G$, stwierdzić
    można, że zarówno $O$ jak i $I \subset O$ stanowią zbiory niezależne.
  }
\end{bproof}
\begin{bproof} (Spełnienie własności 2.\ definicji~\ref{def_crown}.)
  \par{
    Z definciji wynika $H=N(I_{N-1})$.
    Z warunku zakończenia pętli~\algref{alg_findCrown}{findCrown_while} wynika 
    $I=I_N=I_{N-1}$.
    Na tej podsawie widocznym jest, że $H=N(I)$.
  }
\end{bproof}
\begin{bproof} (Spełnienie własności 3.\ definicji~\ref{def_crown}., dowód przez
  sprzeczność)\par{
    Założyć należy istnienie elementu $h \in H, \neg\exists_{e_{M_2}=(u,v) \in
  M_2}: u=h \lor v = h$.
  Rezultatem budowy $H$ byłaby zatem ścieżka rozszerzająca $P$ o nieparzystej
  długości. 
  Warunkiem przynależności $h \in H$ jest istnienie nieskojarzonego wierzchołka
  $v_O \in O$, stanowiącego początek tejże ścieżki.
  W takim wypadku, wynikiem linii~\ref{findCrown_makeH}.\ algorytmu byłaby
  zawsze krawędź nieskojarzona, podczas gdy wynikiem
  linii~\ref{findCrown_makeI}.\ byłaby  krawędź stanowiąca część~skojarzenia.
  Proces ten powtarzałby się do momentu osiągnięcia wierzchołka $h$.
  Utworzona ścieżka rozpościera się zatem pomiędzy dwoma nieskojarzonymi
  wierzchołkami, będąc zarazem $M_2$-przemienną.
  Istnienie takiej ścieżki oznaczałoby możliwość zwiększenia skojarzenia $M_2$
  przez wykonanie operacji $M_2=M_2\oplus P$, co stoi w opozycji do
  założenia, iż $M_2$ stanowi skojarzenie maksymalne.
  Obserwacja ta prowadzi do stwierdzenia, iż każdy wierzchołek $h \in H$ musi
  być skojarzony w $M_2$.
  Właściwe skojarzenie użyte w strukturze korony to skojarzenie $M_2$, z
  dziedziną ograniczoną do krawędzi $H \leftrightarrow I$.
}
\end{bproof}

\par{
  Rezultatem jednej iteracji algorytmu redukcji korony jest graf
  $G^\prime=(V^\prime, E^\prime);\\V^\prime=V\setminus H \setminus I, E^\prime = E
  \setminus \{H\leftrightarrow I\}$.

  Rozmiar dziedziny problemu ulega zmniejszeniu do wartości
  $n^\prime=n-|I|-|H|$, natomiast wartość parametru spada do $k^\prime=k-|H|$,
  z~uwagi na fakt, że każdy z wierzchołków $h \in H$ musi należeć do optymalnej
  pokrycia wierzchołkowego, co udowodniono dla twierdzenia~\ref{th_crown_vc}.
  Należy zaobserwować, iż jeżeli w~grafie istnieje maksymalne skojarzenie
  $M_{MAX}, |M_{MAX}| > k$, wyklucza to istnienie optymalnej pokrywy
  wierzchołkowej $C_{\textnormal{OPT}}, |C_{\textnormal{OPT}}|\leq k$.
  Zatem, jeżeli rozmiar dowolnego z~odnalezionych skojarzeń $M_1, M_2$ jest
  większy niż $k$, algorytm może zakończyć działanie, udzielając ekwiwalentu 
  odpowiedzi negatywnej.
  Zależność ta pozwala również zdefiniować górną granicę rozmiaru grafu 
  wynikowego $|G^\prime|$.
}
\begin{theorem}\thlabel{th_crown_domain_reduction}
  $|M_1| \leq k, |M_2| \leq k \implies |V^\prime \setminus I \setminus H|
  \leq 3k$.
\end{theorem}
\begin{bproof}
  Ponieważ skojarzenie $M_1, |M_1| \leq k$ stanowi zbiór krawędzi, wnioskować
  można, iż $V_{M_1}=\{v, u|v, u \in V, (u,v)\in M_1 \lor (v,u) \in M_1\}, |V_{M_1}| \leq
  2k$.
  Z tego wynika, iż $|O| \geq n-2k$.
  W związku z~faktem, że $|M_2| \leq k$, istnieje co najwyżej $k$ wierzchołków
  $v_O \in O$ skojarzonych przez $M_2$.
  Łatwo zauważyć, iż w~takim wypadku istnieje co najmniej $n-3k$ wierzchołków
  $v_O \in O$ nieskojarzonych przez $M_2$ --- są one zawarte w $I_0$, a~zatem
  także i~w~$I$.
  Ten tok rozumowania prowadzi do wniosku, iż $|V \setminus I \setminus H|
  \leq 3k$.
\end{bproof}
\par{
  Należy zauważyć, że kształt odnalezionej przez algorytm korony podyktowany
  jest trukturą wybranego największego skojarzenia $M_1$.
  Prowadzi to do wniosku, iż pożądane jest wykonywanie algorytmu redukcji korony
  wielokrotnie, wykorzystując różne największe skojarzenia tak, by zlokalizować 
  i~zredukować jak największą ilość koron, co pozwoli na maksymalną redukcję
  dziedziny problemu.
  Rozsądnym krokiem jest również wykonanie jednej iteracji algorytmów 
  przetwarzania wstępnego przez każdą kolejną iteracją redukcji 
  koron --- usunięcie korony z~dużym prawdopodobieństwem prowadzić będzie do 
  powstania wierzchołków niskiego stopnia, redukowalnych przez przetwarzanie
  wstępne.
  Najbardziej złożoną obliczeniowo częścią algorytmu jest odnalezienie
  skojarzenia maksymalnego $M_2$, zrealizowane w niniejszej pracy za pomocą 
  algorytmu kwiatów Edmondsa.\footnote{
    Oryginalna implementacja, opisywana
    w~\cite{KernelizationAlgorithms04} oparta jest o~przeformułowanie problemu do
    egzamplarza zadania przepływu w sieci, rozwiązanej za pomocą algorytmu Dinica,
    o~wynikowej złożoności czasowej $O(n^{\frac{5}{2}})$.
  }
}

\begin{theorem}
  Implementacja algorytmu redukcji korony grafu $G=(V,E); |V|=n,|E|=m$ 
  w~oparciu o~algorytm kwiatów Edmondsa znajduje koronę w~czasie $O(n^{4})$.
\end{theorem}
\begin{bproof}
  Z~teoretycznego punktu widzenia, dwie najbardziej obciążające operacje to
  odnalezienie skojarzenia maksymalnego $M_2$ oraz odnalezienie skojarzenia
  największego $M_1$.
  W celu odnalezienia skojarzenia największego $M_1$ grafu należy
  sprawdzić wszystkie krawędzie $e\in E$ w celu poszukiwania wspólnych
  wierzchołków.
  Wykorzystując koncepcję zaznaczania wierzchołków krawędzi dołączanych do 
  skojarzenia jako odwiedzonych, złożoność operacji sprowadza się do $O(m)$.
  W grafie może znajdować się maksymalnie $O(n^{2})$ krawędzi.
  W celu odnalezienia skojarzenia maksymalnego zastosowano algorytm Edmondsa,
  którego złożoność wynosi $O(n^{4})$, co opisano 
  w~podrozdziale~\ref{ss_edmonds}.
  Powyższe obserwacje prowadzą do wniosku, iż algorytm redukcji korony jest 
  w~stanie znaleźć koronę w~grafie w~czasie $O(n^{4} + n^{2})=O(n^{4})$.
\end{bproof}
