\subsection{Redukcja koron}\label{ss_kernelization_crown_reduction}
\begin{definition}\thlabel{def_crown}
  Mianem \emph{korony} grafu $G=(V, E)$ nazywa się uporządkowaną parę
  podzbiorów wierzchołków $(I, H), I \subseteq V, H \subset V$, zachowujących
  następujące własności.
  \begin{enumerate}
    \item $I \neq \emptyset$ stanowi zbiór niezależny w $G$.
    \item $H=N(I)$.
    \item Istnieje skojarzenie $M=\{e_0, e_1, \ldots, e_p\}, \forall_{e_M=(u,v) \in
      M}: (u\in I \land v\in H) \lor (u \in H \land v \in I)$ takie, że
      $\forall_{v_h \in H}\exists_{e_M=(u,v)\in M}: u = v_h \oplus v = v_h$.
  \end{enumerate}
\end{definition}
\begin{definition}
  Zbiór $H$ określa się mianem \emph{głowy korony}.
\end{definition}
\begin{definition}
  Mianem \emph{szerokości korony} określa się $\|H\|$.
\end{definition}
\begin{theorem}\thlabel{th_crown_vc}
  Jeżeli graf $G=(V,E)$ zawiera koronę $(I,H)$, istnieje optymalna pokrywa 
  wierzchołkowa $VC_{OPT} \in V, H \in VC_{OPT}, I \notin VC_{OPT}$.
\end{theorem}
\begin{bproof}
  Z własności 3.\ definicji~\ref{def_crown}.\ wynika, że każda pokrywa 
  wierzchołkowa $VC$ musi zawierać przynajmniej jeden wierzchołek $v_H \in H$.
  Na tej podstawie stwierdzić można, że $\|VC\|\geq\|H\|$.
  Taki rozmiar pokrywy osiągnąć można przez umieszczenie $VC=VC\bigcup H$.
  Należy w~tym miejscu zazanczyć, że wierzchołki $v_H$ są użyteczne w~kontekście
  możliwości pokrywania krawędzi $e \notin M$, podczas gdy wierzchołki $v_I \in
  I$ nie posiadają tej cechy.
  Mając to na uwadze łatwo zauważyć, że $\|VC \bigcup H\| \leq \|VC \bigcup
  I\|$.
  Wniosek płynący z tej obserwacji jest jednoznaczny: istnieje optymalna pokrywa
  wierzchołkowa $VC_{OPT}; H \in VC_{OPT}, I \notin VC_{OPT}$.
\end{bproof}
W celu odnalezienia korony w grafie, zastosować można następujący algorytm.
\begin{algorithm}
  \caption{Algorytm odnajdujący koronę w grafie $G$}\label{alg_findCrown}
  \begin{algorithmic}[1]
    \Function{findCrown}{G, k}
    \State{$M_1\leftarrow$ największe dopasowanie $G$}
    \State{$O\leftarrow v \in V, \neg\exists_{e_{M_1}=(u, w) \in M_1}: u=v \lor w=v$}
    \If{$\|M_1\| \geq k$}
    \State\textbf{return} nil\Comment{$\neg\exists{VC_{OPT} \in V}: \|VC_{OPT}\| \leq k$}
  \EndIf
  \State{$M_2 \leftarrow$ maksymalne dopasowanie na krawędziach $O\leftrightarrow N(O)$}
  \If{$\|M_2\| > k$}
  \State{\textbf{return} nil\Comment{$\neg\exists{VC_{OPT} \in V}: \|VC_{OPT}\|\leq k$}}
\EndIf
\State{$I_0 \leftarrow v_O\in O, \neg\exists_{e_{M_2}=(u,v)\in M_2}: u=v_O\lor v=v_O$}
\State($n \leftarrow 0$)
\While{$I_{n-1} \neq I_n$}\label{findCrown_while}
\State{$H_n \leftarrow N(I_n)$}\label{findCrown_makeH}
\State{$I_{n+1} \leftarrow I_n\bigcup N_{M_2}(H_n)$}\label{findCrown_makeI}
\State{$n \leftarrow n+1$}
\EndWhile
\State{\textbf{return} $(I_n,H_n)$}\Comment{$n=N$}
  \EndFunction
\end{algorithmic}
\end{algorithm}
Rezultatem działania algorytmu jest korona $(I,H); I=I_N, H=H_N$.

\begin{theorem}
  Algorytm~\ref{alg_findCrown}.\ jest w stanie odnaleźć koronę pod warunkiem, że
  $I_0\neq\emptyset$.
\end{theorem}
\begin{bproof} (Spełnienie własności 1.\ definicji~\ref{def_crown}.)
  \par{
    Bazując na fakcie, iż $M_1$ stanowi największe dopasowanie $G$, stwierdzić
    można, że zarówno $O$ jak i $I \subset O$ stanowią zbiory niezależne.
  }
\end{bproof}
\begin{bproof} (Spełnienie własności 2.\ definicji~\ref{def_crown}.)
  \par{
    Z definciji wynika $H=N(I_{N-1})$.
    Z warunku zakończenia pętli\algref{findCrown}{findCrown_while} wynika 
    $I=I_N=I_{N-1}$.
    Na tej podsawie widocznym jest, że $H=N(I)$.
  }
\end{bproof}
\begin{bproof} (Spełnienie własności 3.\ definicji~\ref{def_crown}., dowód przez
  sprzeczność)\par{
    Założyć należy istnienie elementu $h \in H, \neg\exists_{e_{M_2}=(u,v) \in
  M_2}: u=h \lor v = h$.
  Rezultatem budowy $H$ byłaby zatem ścieżka rozszerzająca $P$ o nieparzystej
  długości. 
  Warunkiem przynależności $h \in H$ jest istnienie nieskojarzonego wierzchołka
  $v_O \in O$, stanowiącego początek tejże ścieżki.
  W takim wypadku, wynikiem linii~\ref{findCrown_makeH}.\ algorytmu byłaby
  zawsze krawędź nieskojarzona, podczas gdy wynikiem
  linii~\ref{findCrown_makeI}.\ byłaby  krawędź stanowiąca część skojarzenia.
  Proces ten powtarzałby się do momentu osiągnięcia wierzchołka $h$.
  Utworzona ścieżka rozpościera się zatem pomiędzy dwoma nieskojarzonymi
  wierzchołkami, będąc zarazem $M_2$-przemienną.
  Istnienie takiej ścieżki oznaczałoby możliwość zwiększenia dopasowania $M_2$
  poprzez wykonanie operacji $M_2=M_2\oplus P$, co stoi w opozycji do
  założenia, iż $M_2$ stanowi skojarzenie maksymalne.
  Obserwacja ta prowadzi do stwierdzenia, iż każdy wierzchołek $h \in H$ musi
  być skojarzony w $M_2$.
  Właściwe dopasowanie użyte w strukturze korony to dopasowanie $M_2$, z
  dziedziną ograniczoną do krawędzi $H \leftrightarrow I$.
}
\end{bproof}

Rezultatem jednej iteracji algorytmu redukcji korony jest graf
$G\prime=(V\prime, E\prime);\\V\prime=V\setminus H \setminus I, E\prime = E
\setminus \{H\leftrightarrow I\}$.

Rozmiar dziedziny problemu ulega zmniejszeniu do wartości
$n\prime=n-\|I\|-\|H\|$, natomiast wartość parametru spada do $k\prime=k-\|H\|$,
z uwagi na fakt, że każdy z wierzchołków $h \in H$ musi należeć do optymalnej
pokrywy wierzchołkowej, co udowodniono dla twierdzenia~\ref{th_crown_vc}.
Należy zaobserwować, iż jeżeli w grafie istnieje maksymalne skojarzenie
$M_{MAX}, \|M_{MAX}\| > k$, wyklucza to istnienie optymalnej pokrywy
wierzchołkowej $VC_{OPT}, \|VC_{OPT}\|\leq k$. Zatem, jeżeli rozmiar dowolnego 
z odnalezionych skojarzeń $M_1, M_2$  jest większy niż $k$, algorytm może
zakończyć działanie, udzielając ekwiwalentu odpowiedzi negatywnej.
Zależność ta pozwala również zdefiniować górną granicę rozmiaru grafu wynikowego
$G\prime$.

\begin{theorem}
  $\|M_1\| \leq k, \|M_2\| \leq k \implies \|V\prime \setminus I \setminus H\|
  \leq 3k$.
\end{theorem}
\begin{bproof}
  Ponieważ skojarzenie $M_1, \|M_1\| \leq k$ stanowi zbiór krawędzi, wnioskować
  można, iż $V_{M_1}=\{v, u|v, u \in V, (u,v)\in M_1 \lor (v,u) \in M_1\}, \|V_{M_1}\| \leq
  2k$.
  Z tego wynika, iż $\|O\| \geq n-2k$.
  W związku z faktem, iż $\|M_2\| \leq k$ istnieje co najwyżej $k$ wierzchołków
  $v_O \in O$ skojarzonych przez $M_2$.
  Łatwo zauważyć, iż w takim wypadku istnieje co najmniej $n-3k$ wierzchołków
  $v_O \in O$ nieskojarzonych przez $M_2$---są one zawarte w $I_0$, a zatem
  także w $I$.
  Ten tok rozumowania prowadzi do wniosku, iż $\|V \setminus I \setminus H\|
  \leq 3k$.
\end{bproof}


