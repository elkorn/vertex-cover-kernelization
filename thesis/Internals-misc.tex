\subsection{Narzędzia pomocnicze}
\label{ss_technologies_misc}
\subsubsection{\textbf{Pakiet Graphviz}}
\label{sss_technologies_misc_graphviz}
\par{
  Pakiet Graphviz (skrót od \emph{Graph Visualization Software}) jest zbiorem narzędzi,których rozwój zapoczątkowała firma AT\&T Labs Research, służących do rysowania grafów oraz udostępniających funkcje pakietu w ramach bibliotek dla aplikacji.
  Pakiet opublikowany jest w ramach licencji EPL.
  Wizualizacja grafów odbywa się na podstawie modeli w języku skryptowym DOT.
  W zakres możliwości pakietu wchodzą również przydatne funkcje wyróżniania elementów prezentowanego grafu poprzez kolorowanie węzłów lub krawędzi, możliwość tabelarycznego rozmieszczania węzłów oraz definiowanie własnych stylów krawędzi czy też wstawianie hiperłącz oraz dowolnych kształtów.
  Narzędzia wchodzące w skład pakietu:
  \begin{itemize}
    \item \texttt{dot} --- służy do tworzenia hierarchicznych obrazów grafów skierowanych,
    \item \texttt{neato} --- służy do tworzenia obrazów grafów nieskierowanych o niewielkiej liczbie węzłów (do ok. 100) za pomocą algorytmu minimalizującego globalną funkcję energii grafu,
    \item \texttt{fdp} --- podobnie jak \texttt{neato}, jednak funkcja rozmieszczająca działa w oparciu o redukcję sił połączeń w grafie,
    \item \texttt{sfdp} --- zmodyfikowana wersja \texttt{fdp}, przystosowana do obrazowania grafów o wysokiej liczbie węzłów,
    \item \texttt{twopi} oraz \texttt{circo} --- służą do obrazowania grafów w układzie kołowym,
    \item \texttt{dotty} --- stanowi graficzny interfejs użytkownika do wizualizacji oraz edycji grafów,
    \item \texttt{lefty} --- programowalny graficzny element kontrolujący wyswietlający grafy odczytane z pliku DOT i umożliwiający użytkownikowi wykonywanie na nich operacji za pomocą myszy.
  \end{itemize}
}
