\subsection{Formulacja problemu jako programu liniowego }\label{section_kernelization_lp_formulation}

Proces rozwiązywania problemu pokrycia wierzchołkowego może być zoptymalizowany
przy zastosowaniu heurystyki zaproponowanej w~\cite{hochbaum82}, opartej 
o~programowanie całkowitoliczbowe. \\
Opisywany algorytm~\cite[rozdz.~4.2.2]{abukhzam03} korzysta z~\cite{hochbaum82} 
w~następujący sposób.

\subsubsection{\textbf{Rozwiązanie problemu oryginalnego}}\label{ss_lp_original}

Każdemu wierzchołkowi $u \in V$ grafu $G=(V,E)$ przypisać należy wartość $X_u
\in \{0, 1\}$, z zachowaniem następujących własności:
\begin{enumerate}
  \item $\sum_{u}X_u = \min$,
  \item $\{u,v\} \in E \implies X_U + X_v \geq 1$.
\end{enumerate}

Funkcja celu programu liniowego zwraca dolną granicę rozmiaru pokrywy 
wierzchołkowej $|VC|$.
Zbiór rozwiązań prawdopodobnych składa się z~funkcji $V \to \{0, 1\}$,
spełniających warunek 2.
W związku z faktem, iż programowanie całkowitoliczbowe samo w sobie stanowi
problem NP-zupełny, dokonać należy relaksacji do postaci programu liniowego, co
zapewni szerszy zakres prawdopodobnych rozwiązań.

W~\cite{khuller02} zaproponowana została relaksacja poprzez zamianę wartości 
$X_u \in \{0,1\}$ na $X_u \geq 0$.
Należy zauważyć, iż wartość $OPT_{LP}$ zwracana przez rozwiązanie postaci 
liniowej jest zawsze ograniczona z dołu przez wartość $OPT_{IP}$ zwracaną przez 
rozwiązanie postaci całkowitoliczbowej.
Co więcej, w~\cite{khuller02} udowodniono, że $OPT_{IP} \leq 2*OPT_{LP}$.
Zależność ta oparta jest na twierdzeniu Nemhausera-Trottera korzystającym
z~własności, że w~dowolnym ekstremum rozwiązania relaksacji programu
całkowitoliczbowego do postaci liniowej zmienne przyjmują wartość 
$X_u \in \{0, \frac{1}{2}, 1\}$.

Definiując $V_0 = \{u : X_u=0\}, V_{\frac{1}{2}}=\{u: X_u=\frac{1}{2}\},
V_1=\{u: X_u=1\}$, twierdzenie zapisać można jak następuje.

\begin{theorem}[Pierwsze twierdzenie Nemhausera-Trottera]\thlabel{nt_lp}
  Istnieje optymalne rozwiązanie $OPT$ o następujących właściwościach:
  \begin{itemize}
    \item[(a)] $OPT \subset V_1 \bigcup V_{\frac{1}{2}}$.
    \item[(b)] $V_1 \subset OPT$.
  \end{itemize}
\end{theorem}

W celu wyspecjalizowania powyższej relaksacji do przypadku rozwiązania
sparametryzowanego problemu pokrycia wierzchołkowego, zdefiniować należy zbiór 
$\{ X_u : u \in V \}$ zawierający wartości przypisywane wierzchołkom grafu 
$G=(V,E)$ przez funkcję celu oraz zbiory:\\


$P=\{u \in V : X_u>\frac{1}{2}\}$,\par
$Q=\{u \in V : X_u=\frac{1}{2}\}$,\par
$R=\{u \in V : X_u<\frac{1}{2}\}$.\\


Istotą redukcji dziedziny problemu do jądra jest dołączenie wszystkich
wierzchołków $u_P \in P$ do częściowej pokrywy wierzchołkowej $VC$ oraz 
usunięcie z~niej wszystkich wierzchołków $u_R \in R$.
Graf wynikowy $G\prime=(V\prime, E\prime)$ zaindukowany jest elementami $Q$: 
$V\prime=Q; E\prime=\{e=(v, w)| e \in E, \{v, w\} \in Q\}$.

\begin{theorem}
  Istnieje optymalna pokrywa wierzchołkowa
  $VC, VC \in G;\\P \subset VC, VC \bigcap R = \emptyset$.
\end{theorem}
\begin{bproof}
  Należy założyć pewne rozwiązanie całkowitoliczbowej formulacji problemu 
  pokrycia wierzchołkowego $OPT_{IP}$ oraz zbiory 
  ${A = P \setminus OPT_{IP}, B = R \bigcap OPT_{IP}}$.
  % Dowód przeprowadzony będzie zatem przy założeniu $|A|\neq |B|$.
  Zauważyć należy, że $N(B) \bigcap Q = \emptyset$, co zapewnia właściwość 2.
  formulacji, której rozwiązaniem jest $OP_{IP}$.


  Jeżeli $|A|<|B|$, zastąpienie $B$ przez $A$ w $OPT_{IP}$ spowodowałoby
  odkrycie przynajmniej jednej krawędzi grafu---wykluczając tym samym
  tak otrzymaną pokrywę jako rozwiązanie.
  W prypadku gdy $|A|>|B|$, swiadczyłoby to, że istnieje możliwość
  otrzymania rozwiązania formulacji liniowej lepszego niż $OPT_{IP}$ poprzez
  ustanowienie $\epsilon &= \min\{X_v-\frac{1}{2}: v \in A\}$, a~następnie
  zastąpienie $\forall{u \in B}:X_u \leftarrow X_u + \epsilon$; 
  $\forall{v \in A}: X_v \leftarrow X_v -\epsilon$.
  Jest to niemożliwe ze względu na fakt, iż wynik $OPT_{IP}$ stanowi optymalne 
  rozwiązanie formulacji liniowej w~oparciu o~twierdzenie
  Nemhausera-Trottera~\ref{nt_lp}.

  Nasuwa się konkluzja, iż jedyny przypadek z~jakim można mieć w tym miejscu do 
  czynienia to $|A|=|B|$.
  Przypadek ten jest trywialny---w~celu orzymania optymalnej pokrywy
  wierzchołkowej wystarczy zastąpić zbiór $A$ zbiorem $B$.
\end{bproof}

Prezentowany algorytm redukuje dziedzinę do jądra problemu o~rozmiarze
$n\prime=|V|-|P|-|R|$.

Wartość wynikowa parametru określającego maksymalny rozmiar optymalnej pokrywy
wierzchołkowej zmniejszona zostaje do $k\prime=k-|P|$.

\begin{theorem}
  Nie istnieje optymalna pokrywa wierzchołkowa $VC\prime_{OPT}\in G\prime,|VC\prime_{OPT}|>\Sigma_{u\in Q}X_u=\frac{|Q|}{2}$.
\end{theorem}
\begin{bproof}
  Należy mieć na uwadze fakt, iż rozmiar funkcji celu formulacji liniowej 
  ogranicza od dołu rozmiar funkcji celu formulacji całkowitoliczbowej.
  W przeciwnym wypadku, procedura rozwiązująca początkową formulację liniową
  problemu, której wynik stanowi zbiór $Q$, nie byłaby w stanie zapewnić
  optymalnego rozwiązania, co byłoby sprzeczne z założeniami formulacji.
\end{bproof}

W świetle powyższego dowodu stwierdzić można, że w sytuacji gdy
$|Q|>2k\prime$, można zakończyć działanie całego procesu poszukiwania pokrywy
wierzchołkowej $VC_{OPT}, |VC_{OPT}|\leq k$, udzielając odpowiedzi negatywnej.

Warto nadmienić, że powyższe sformułowanie algorytmu jest niepraktyczne dla
grafów o~dużym zagęszczeniu ze względu na liczbę warunków ograniczających 
formulacji równą $|E|$.

Sensowną optymalizacją podejścia dla takich przypadków jest przekształcenie
problemu z~minimalizacyjnego do dualnego problemu maksymalizacyjnego, 
w~którym liczba warunków ograniczających równa będzie $|V|$.

\subsubsection{\textbf{Rozwiązanie problemu dualnego}}

W oparciu o~obserwację, iż koszt dowolnego prawdopodobnego rozwiązania problemu
dualnego do oryginalnej formulacji liniowej problemu pokrycia
wierzchołkowego~\ref{ss_lp_original} stanowi dolną granicę dla optimum
\ref{ss_lp_original} poprzez słabą dualność. 

Konstrukcja formulacji liniowej dualnego problemu maksymalizacyjnego wygląda
następująco.

Każdej krawędzi $e=(u,v) \in E$ grafu $G=(V,E)$ przypisać należy wartość
$Y_(u,v) \geq 0$, z zachowaniem następujących własności:
\begin{enumerate}
  \item $\sum_{(u,v)}Y_{(u,v)} = \max$,
  \item $\forall_{v \in V}:\sum_{u:(u,v)}Y_{(u,v)}\eq 1$,
  \item $\forall_{e=(u,v) \in E}: Y_{(u,v)} \geq 0$.
\end{enumerate}

Łatwo zauważyć, iż ma się tu do czynienia z problemem odnalezienia maksymalnego
skojarzenia grafu.
