\subsection{Sformułowanie problemu pokrycia wierzchołkowego jako problemu programowania liniowego}\label{section_kernelization_lp_formulation}

Proces rozwiązywania problemu pokrycia wierzchołkowego może być rozwiązany
przy zastosowaniu heurystyki zaproponowanej w pracy~\cite{hochbaum82}, opartej 
na~programowaniu całkowitoliczbowym.
Opisywany algorytm~\cite[rozdz.~4.2.2]{abukhzam03} korzysta z~ heurystyki proponowanej w~pracy~\cite{hochbaum82} 
w~następujący sposób.

\subsubsection{\textbf{Rozwiązanie problemu oryginalnego}}\label{ss_lp_original}
Każdemu wierzchołkowi $u \in V$ grafu $G=(V,E)$ należy przypisać wartość $X_u
\in \{0, 1\}$, z zachowaniem następujących własności:
\begin{enumerate}
  \item $\sum_{u \in V}X_u = \min$,
  \item $\{u,v\} \in E \implies X_u + X_v \geq 1$.
\end{enumerate}

Funkcja celu programu liniowego zwraca dolną granicę rozmiaru pokrycia wierzchołkowego $|C|$.
Zbiór rozwiązań prawdopodobnych składa się z~funkcji $V \to \{0, 1\}$,
spełniających warunek 2.
W związku z faktem, iż programowanie całkowitoliczbowe samo w sobie stanowi
problem $\mathcal{NP}$-zupełny, dokonać należy relaksacji do postaci programu liniowego, co
zapewni szerszy zakres prawdopodobnych rozwiązań.

W~pracy~\cite{khuller02} zaproponowana została relaksacja przez zamianę wartości 
$X_u \in \{0,1\}$ na $X_u \geq 0$.
Należy zauważyć, iż wartość $O_{LP}$ (ang. \emph{Linear Programming, LP}) zwracana przez rozwiązanie postaci 
liniowej jest zawsze ograniczona z dołu przez wartość $O_{\textnormal{IP}}$ zwracaną przez 
rozwiązanie postaci całkowitoliczbowej.
Co więcej, w~pracy~\cite{khuller02} udowodniono, że $O_{\textnormal{IP}} \leq 2*O_{LP}$.
Zależność ta wynika z Twierdzenia Nemhausera--Trottera korzystającego
z~własności, że w~dowolnym ekstremum rozwiązania relaksacji programu
całkowitoliczbowego do postaci liniowej zmienne przyjmują wartość 
$X_u \in \{0, \frac{1}{2}, 1\}$.

Definiując $V_0 = \{u : X_u=0\}, V_{\frac{1}{2}}=\{u: X_u=\frac{1}{2}\},
V_1=\{u: X_u=1\}$, Twierdzenie zapisać można jak następuje.

\begin{theorem}[Pierwsze Twierdzenie Nemhausera--Trottera]\thlabel{nt_lp}
  Istnieje optymalne rozwiązanie $O$ sformułowania problemu pokrycia wierzchołkowego jako problemu programowania liniowego o następujących właściwościach.
  \begin{enumerate}[(a)]
    \item $O \subset V_1 \cup V_{\frac{1}{2}}$.
    \item $V_1 \subset O$.
  \end{enumerate}
\end{theorem}
W celu sprowadzenia powyższej relaksacji do przypadku rozwiązania
sparametryzowanego problemu pokrycia wierzchołkowego, zdefiniować należy zbiór 
$\{ X_u : u \in V \}$ zawierający wartości przypisywane wierzchołkom grafu 
$G=(V,E)$ przez funkcję celu oraz zbiory:\\
\begin{align*}
P&=\left\{u \in V | X_u>\frac{1}{2}\right\},\\
Q&=\left\{u \in V | X_u=\frac{1}{2}\right\},\\
R&=\left\{u \in V | X_u<\frac{1}{2}\right\}.
\end{align*}
Istotą redukcji dziedziny problemu do jądra jest dołączenie wszystkich
wierzchołków $u_P \in P$ do częściowego pokrycia wierzchołkowego $C$ oraz 
usunięcie z~niego wszystkich wierzchołków $u_R \in R$.
Graf wynikowy $G^\prime=(V^\prime, E^\prime)$ zaindukowany jest elementami $Q$: 
zbiorem wierzchołków $V^\prime=Q$ oraz zbiorem krawędzi\\$E^\prime=\{e=(v, w)| e \in E, \{v, w\} \in Q\}$.
\begin{theorem}
  Istnieje optymalne pokrycie wierzchołkowe $C \in G$, dla którego spełnione są własności $P \subset C$ oraz $C \cap R = \emptyset$.
\end{theorem}
\begin{bproof}
  Należy założyć pewne rozwiązanie całkowitoliczbowego sformułowania problemu 
  pokrycia wierzchołkowego $O_{\textnormal{IP}}$ oraz zbiory 
  ${A = P \setminus O_{\textnormal{IP}}, B = R \cap O_{\textnormal{IP}}}$.
  Zachodzi $N(B) \cap Q = \emptyset$, co zapewnia właściwość 2 sformułowania, którego rozwiązaniem jest $O_{\textnormal{IP}}$.

  Jeżeli $|A|<|B|$, to zastąpienie zbioru $B$ przez $A$ w rozwiązaniu $O_{\textnormal{IP}}$ (ang. \emph{Integer Programming, IP}) spowodowałoby odkrycie przynajmniej jednej krawędzi grafu --- wykluczając tym samym tak otrzymane pokrycie wierzchołkowe jako rozwiązanie.
  W prypadku gdy $|A|>|B|$, musiałaby istnieć możliwość otrzymania rozwiązania sformułowania liniowego lepszego niż $O_{\textnormal{IP}}$ przez ustanowienie $\epsilon &= \min\{X_v-\frac{1}{2}: v \in A\}$, a~następnie
  zastąpienie $\forall{u \in B}:X_u \leftarrow X_u + \epsilon$ oraz $\forall{v \in A}: X_v \leftarrow X_v -\epsilon$.
  Jest to niemożliwe, ponieważ wynik $O_{\textnormal{IP}}$ stanowi optymalne rozwiązanie sformułowania liniowego w~oparciu o~pierwsze twierdzenie Nemhausera--Trottera~(\ref{nt_lp}).
  Nasuwa się konkluzja, że jedyny przypadek z~jakim można mieć w tym miejscu do czynienia to $|A|=|B|$.
  Przypadek ten jest trywialny --- w~celu orzymania optymalnego pokrycia wierzchołkowego wystarczy zastąpić zbiór $A$ zbiorem $B$.
\end{bproof}
Prezentowany algorytm redukuje dziedzinę do jądra problemu o~rozmiarze $n^\prime=|V|-|P|-|R|$.
Wartość wynikowa parametru określającego maksymalny rozmiar optymalnego pokrycia wierzchołkowego zmniejszona zostaje do $k^\prime=k-|P|$.
\begin{theorem}
  Nie istnieje optymalne pokrycie wierzchołkowe $C^\prime_{\textnormal{OPT}}\in G^\prime$ o rozmiarze $|C^\prime_{\textnormal{OPT}}|>\Sigma_{u\in Q}X_u=\frac{|Q|}{2}$.
\end{theorem}
\begin{bproof}
  Zauważmy, że rozmiar funkcji celu sformułowania liniowego ogranicza od dołu rozmiar funkcji celu sformułowania całkowitoliczbowego.
  W przeciwnym wypadku, procedura rozwiązująca początkowe sformułowanie liniowe problemu, którego wynik stanowi zbiór $Q$, nie byłaby w stanie zapewnić optymalnego rozwiązania, co byłoby sprzeczne z założeniami sformułowania.
\end{bproof}
\par{
  W świetle powyższego dowodu widać, że można zakończyć działanie procesu poszukiwania pokrycia wierzchołkowego $C_{\textnormal{OPT}}$ o liczebności $|C_{\textnormal{OPT}}|\leq k$, udzielając odpowiedzi negatywnej gdy $|Q|>2k^\prime$.
  Warto dodać, że powyższe sformułowanie algorytmu jest niepraktyczne dla grafów o~dużym zagęszczeniu ze względu na liczbę warunków ograniczających sformułowania równą $|E|$.
  Właściwym podejściem do takich przypadków jest przekształcenie problemu z~minimalizacyjnego do dualnego problemu maksymalizacyjnego, w~którym liczba warunków ograniczających równa będzie $|V|$.
}
\subsubsection{\textbf{Rozwiązanie problemu dualnego}}
\par{
  Ponieważ koszt dowolnego prawdopodobnego rozwiązania problemu
  dualnego do oryginalnego sformułowania liniowego problemu pokrycia
  wierzchołkowego~(\ref{ss_lp_original}) stanowi dolną granicę dla optimum
  (\ref{ss_lp_original}) poprzez słabą dualność. 
  Konstrukcja sformułowania liniowego dualnego problemu maksymalizacyjnego wygląda
  następująco.\\
  Każdej krawędzi $e=(u,v) \in E$ grafu $G=(V,E)$ należy przypisać wartość
  $Y_{(u,v)} \geq 0$, z zachowaniem następujących własności:
  \begin{enumerate}
    \item $\sum_{(u,v) \in E}Y_{(u,v)} = \max$,
    \item $\forall_{v \in V}:\sum_{u:(u,v)\in E}Y_{(u,v)} \leq 1$,
    \item $\forall_{(u,v) \in E}: Y_{(u,v)} \geq 0$.
  \end{enumerate}
}