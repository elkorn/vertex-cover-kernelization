\chapter{Specyfikacja wewnętrzna implementacji opisanych algorytmów}\label{s_internals}
\section{Wykorzystane techniki i technologie}\label{ss_internals-technologies}
W przeciągu ostatniej dekady proces tworzenia oprogramowania uległ znaczącym
transformacjom.
Ogromna popularyzacja sektora technologii informacyjnych, którą zawdzięczamy głównie rozwojowi internetu, przyczyniła się do skupienia dużo większej uwagi na kwestiach zarówno sposobów wytwarzania oraz charakteru oprogramowania jak i na narzędziach temu służących.

W środowisku informatyki biznesowej odchodzi się od klasycznych, liniowych procesów tworzenia oprogramowania na rzecz szerokiego grona rozwiązań zaliczanych do grupy tzw. metodyk zwinnych. 
Głównym celem każdej metodyk zwinnych jest zwiększenie ilości dostarczanych produktów (w tym wypadku rozwiązań informatycznych) w danym czasie, zachowując jednocześnie jak najwyższy poziom ich jakości.
Prawidłowo wdrożone metodyki zwinne pozwalają na zacieśnienie pętli komunikacyjnej pomiędzy interesariuszami biznesowymi oraz technicznymi projektów, owocując częstszą wymianą bardziej szczegółowych informacji.
Bezpośrednio przekłada się to na zwiększoną przezroczystość procesu realizacji projektu.
Konsekwencja ta otwiera nowe pole możliwości związanych z planowaniem, podziałem prac, raportowaniem i prognozowaniem postępów prac oraz weryfikacją prognoz poprzez monitorowanie stanu realizacji za pośrednictwem metryk.

\par{
Zwiększona częstotliwość dostarczania wartości biznesowej wiąże się z potrzebą wynajdowania bardziej nowoczesnych narzędzi oraz praktyk wspierających taki tryb operowania.
Wobec zmieniających się wymagań dziedziny technologii informacyjnych, z biegiem lat na znaczeniu zyskały dawne koncepcje takie jak:

\begin{itemize}
  \item enkapsulacja,
  \item wyższe poziomy abstrakcji kodu,
  \item paradygmat programowania obiektowego,
  \item paradygmat programowania funkcyjnego,
  \item programowanie systemowe,
  \item programowanie współbieżne i asynchroniczne.
\end{itemize}
}

\par{
Wraz z adaptacją tychże pomysłów, opisywanych już w pracach sprzed zgoła czterdziestu laty, równolegle narodziły się i na ich podstawie rozwijane są nowe idee, do których zaliczyć można:
\begin{itemize}
  \item Paradygmat programowania hybrydowego, łączący w sobie elementy funkcyjne zarówno jak i obiektowe. Dzięki zaletom elementów paradygmatu obiektowego, wysoce i zrozumiale dla człowieka zorganizowane dane mogą być przetwarzane przy pomocy potężnych szkicy logicznych, złożonych z funkcji za pomocą implementacji pojęć należących do paradygmatu funkcyjnego takich jak monady i kombinatory.
  \item Biblioteki wysokopoziomowe, enkapsulujące funkcjonalności narzędziowe i działania związane mocno z architekturą komputera. Biblioteki wysokopoziomowe zwiększają efektywność inżyniera oprogramowania poprzez wprowadzenie użytecznych i zrozumiałych idiomów programistycznych, często wywodzących się bezpośrednio z matematyki lub realizujących zestawy niskopoziomowych operacji sięgających języka pośredniego lub kodu assemblerowego jako całość logiczną oraz nazwaną w sposób zrozumiały dla człowieka. Ważną funkcją bibliotek wysokopoziomowych jest również dążenie do maskowania rozbieżności związanych z docelową platformą uruchomieniową kodu, co stanowi znaczące wsparcie dla programisty, pozwalając na pisanie w większości przypadków homogenicznego kodu, który nie jest uzależniony od platformy, na której ma zostać użyty.
  \item Automatyczne zarządzanie pamięcią przy pomocy tzw. mechanizmów odśmiecania pamięci (\emph{Garbage Collection}, popularnie skracane do \emph{GC}). Praktyka pokazuje, iż w dużym odsetku systemów przeznaczenia ogólnego, przez które rozumie się większość zastosowań biznesowych i przemysłowych, nie wymaga tak ywsokiej precyzji, jaką oferuje ręczne zarządzanie pamięcią. Biorąc pod uwagę duży nakład pracy programisty potrzebny do prawidłowego ręcznego zarządzania pamięcia, bardziej opłacalne okazuje się poświęcenie ułamka wydajności systemu poprzez wprowadzenie cyklicznego odśmiecania pamięci w oparciu o pewne reguły rozpoznawania bloków do usunięcia, zwalniając jednocześnie z tego obowiązku inżyniera. Dla systemów zawierających komponenty bardziej wyspecjalizowane, lub wymagające przetwarzania w czasie rzeczywistym, nowoczesne języki oferują podejście mieszane, pozwalając na ręczne zarządzanie pamięcią w krytycznych miejscach aplikacji.
  \item Wprowadzanie mechanizmów synchronizacji wątków oraz prymitywów do zarządzania asynchronicznym lub współbieżnym wykonywaniem kodu jako obywateli pierwszej klasy nowoczesnych języków programowania. Popularnymi obecnie koncepcjami w tej dziedzinie są:
  \begin{itemize}
    \item ̆\emph{funkcje zwrotne} (Callback functions), przekazywane jako punkty powrotu w momencie zakończenia przetwarzania asynchronicznego,
    \item \emph{kanały} (channels), stanowiące abstrakcje w postaci obiektów umożliwiających dwukierunkową transmisję danych dowolnego typu pomiędzy komponentami systemu lub systemami,
    \item \emph{zdarzenia} (events), będące abstrakcją w postaci obiektów niosących informacje o zajściu określonych warunkóœ w asynchronicznym przepływie sterowania aplikacji.
  \end{itemize}
\end{itemize}
}

\par{
  Pojęcie ``narzędzia'' w XXI. wieku znacznie zyskało na pojemności i~nie ogranicza się już wyłącznie aspektów samego języka lub nieodzownych elementów ściśle z~nim związanych jak kompilatory czy linkery.
  Narastający nacisk kładzie się również na rozwijanie tak zwanych \emph{ekosystemów} wokół języków programowania.
  Pojęcie ekosystemu stanowi dość liberalne określenie grupy funkcjonalności oraz aplikacji służących wsparciu programisty przez m.in.:
  \begin{itemize}
    \item automatyzację podstawowych zadań,
    \item analizę a nawet zmiany lub przepisywanie kodu na podstawie badania drzewa składni abstrakcyjnej,
    \item uruchamianie testów jednostkowych oraz sprawnościowych,
    \item tworzenie profili wydajnościowych aplikacji ze względu na zużycie czasu procesora lub pamięci RAM w oparciu o śledzenie przepływu kontroli w aplikacji,
    \item podpowiedzi oraz dopełnianie na bieżąco pisanego kodu.
  \end{itemize}
}

\par{
  Większe zaplecze narzędziowe umożliwia wykorzystywanie grup aplikacji należących do ekosystemu w celu spełnienia założeń leżących u podstaw metodyk zwinnych, skupiających się na jak najczęstym dostarczaniu wartościowych produktów lub komponentów produktu wysokiej jakości, w ramach przejrzystego procesu twórczego.
}

\subsection{Język programowania Go} % (fold)
\label{sss_go}
\par{
Go jest statycznie typowanym, imperatywnym, strukturalnym językiem programowania, którego historia rozpoczęła się w 2007. roku w firmie Google. Autorzy oryginalnej specyfikacji wraz z implementacją to: Rob Pike, Robert Griesemer oraz Ken Thompson.
Najnowsza stabilna wersja języka na dzień pisania niniejszej pracy to 1.3.3.
}
\par{
Składnia Go silnie nawiązuje do języka C --- dokonano jednak wielu modyfikacji skupiających się przede wsyzstkim na jej uproszczeniu, eliminacji możliwości popełniania błędów oraz zwiększeniu zwięzłości.
Dużo uwagi poświęca się również pielęgnacji ekosystemu wokół Go w celu uczynienia go narzędziem jak najłatwiejszym i najbardziej praktycznym w użyciu.
Dla osiągnięcia tych założeń zastosowano wzorce znane zarówno z języków statycznie jak i dynamicznie typowanych.
}
\par{Deklaracja i inicjalizacja zmiennych odbywa się przy pomocy mechanizmu domniemania typów, w większości przypadków zwalniającego programistę z obowiązku jawnego oznaczania typu zmiennych oraz metod. Zamiast zapisu \texttt{int~x~=~0;}, znanego z języka C, stosuje się tu krótszy zapis \texttt{x~:=~0}. Warto również zwrócić uwagę na brak wymagania stawiania średników jako zakończeń wyrażeń.
}
 \par{Mimo możliwości korzystania ze wskaźników, bezpośredni dostęp do nich jest niemożliwy, co zapobiega błędom związanym z niezgodnością typów. W połączeniu ze statycznym typowaniem oznacza to, iż programista nie jest w stanie wprowadzić rozbieżności na poziomie typów zmiennych prowadzących do awarii aplikacji niewykrytych przez kompilator. Konsekwencją zablokowania bezpośredniego dostępu do wskaźników jest również brak możliwości wykonywania na nich działań arytmetycznych.
 }
 \par{
Dzięki mechanizmowi Garbage Collection, język zapobiega wyciekom pamięci wynikającym z nieprawidłowego zarządzania wskaźnikami. W aktualnej wersji wykorzystywana jest współbieżna wersja algorytmu \textit{mark and sweep}.
}
\par{
 W związku z faktem, iż Go jest językiem strukturalnym, brak w nim pojęcia obiektu. Uproszczonym odpowiednikiem jest struktura, definiowana słowem kluczowym \texttt{struct}.
}
  \par{
  Jedną z najbardziej radykalnych decyzji podczas tworzenia specyfikacji języka stanowi rezygnacja ze standardowego mechanizmu dziedziczenia. W miejsce dziedziczenia wirtualnego zastosowano system interfejsów, gdzie dana struktura implementuje określony interfejs wtedy i tylko wtedy gdy wystawia pełen zestaw publicznych metod zgodnych z jego deklaracją. Odpowiednikiem dziedziczenia klasycznego w Go jest osadzanie typów.
  Przykład osadzenia typów \texttt{Reader} i \texttt{Writer} w nowo utworzonym interfejsie \texttt{ReaderWriter}.
  \begin{lstlisting}
    type ReaderWriter interface {
      Reader
      Writer
    }
  \end{lstlisting}
  Struktury implementujące interfejs \texttt{ReaderWriter} implementują również interfejsy osadzone.
  Osadzanie typów w strukturach wygląda podobnie, należy jednak oznaczyć je symbolem wskaźnika.
  \begin{lstlisting}
    type ReaderWriter struct {
      *Reader
      *Writer
    }
  \end{lstlisting}
  Kluczowa różnica pomiędzy dziedziczeniem a osadzaniem typów polega na właściwości, iż metody typu osadzanego zostają włączone do typu zewnętrznego~---~jednak podczas wywołania danej metody, jej odbiorcą jest instancja typu osadzonego. \cite{godoc:embedding}
}
\par {
W celu uproszczenia składni głównie względem C++, argumentowanym przez jednego z autorów w notatce \cite{Pike:LessIsMore}, wyłączono ze specyfikacji wiele funkcji oferowanych przez podobne języki, poza wymienionymi wcześniej.
  \begin{itemize}
    \item Przeciążanie metod i operatorów,
    \item cykliczne zależności pomiędzy pakietami,
    \item asercje,
    \item programowanie generyczne.
  \end{itemize}
}
\par{
  Dla wygody programisty wprowadzono do Go zestaw podstawowych typów, wyrażających elementy brakujące zdaniem autorów w czystym C.
  \begin{itemize}
    \item \emph{Plastry} (slices), zapisywane jako \texttt{[]typ}, wskazują na tablicę obiektów przechowywanych w pamięci, przechowując wskaźnik do początku danego plastra, jego długość oraz \emph{pojemność}, określającą liczebność elementów plastra, która wymagać alokacji dodatkowej pamięci w celu rozszerzenia odpowiadającej tablicy.
    \item Niezmienne ciągi znaków (typ \texttt{string}), zawierające przeważnie tekst w kodowaniu UTF-8. Mogą jednak przechowywać dowolne bajty.
    \item Tablica haszująca, zapisywana jako \texttt{map[typ\_klucza]typ\_wartości}.
  \end{itemize}
}
\par{
  Go jest językiem kompilowanym do kodu bajtowego. Istnieją dwa oficjalnie wspierane zestawy kompilatorów:
  \begin{itemize}
    \item \texttt{gc} wraz z narzędziami dla architektur \texttt{amd64} oraz \texttt{i386}, posiadający wydajny optymalizator, oraz dla architektury \texttt{ARM},
    \item \texttt{gccgo}, będący nakładką na \texttt{gcc}.
  \end{itemize}
  Każdy z zestawów wspiera platformy DragonFly BSD, FreeBSD, Linux, NetBSD, OpenBSD, OS X (Darwin), Plan 9, Solaris oraz Windows. Wyjątkiem jest kompilator \texttt{gc} na architekturę \texttt{ARM}, wspierający wyłącznie platformy Linux, FreeBSD oraz NetBSD. \cite{godoc:compilers}
  Łańcuch narzędzi związany z procesem kompilacji tworzy statycznie linkowane, natywne binarne pliki wykonywalne bez zewnętrznych zależności.
}

\par{
  Ostatnim elementem, o którym należy wspomnieć, są wbudowane mechanizmy służące do programowania współbieżnego.
}
% subsection go (end)

\section{Szczegóły techniczne implementacji}\label{s_internals_implementation}
\subsection{Architektura aplikacji}\label{ss_internals_architecture}
\par{
  Zgodnie z kanonem podyktowanym przez specyfikację języka Go, kod aplikacji podzielony został na \emph{pakiety}.
  Należy zaznaczyć, iż pakiety w rozumienia języka Go są czymś innym niż pakiety opisywane w podrozdziale~\ref{ss_internals_misc}.
  W kontekście obecnego rozdziału przez pakiety rozumie się logiczne jednostki organizacji kodu w języku Go, pakiety omawiane w podrozdziale~\ref{ss_internals_misc} stanowią zewnętrzne biblioteki, z których korzystano w ramach prac implementacyjnych.
  Projektując strukturę kodu, nacisk położony został na logiczne rozdzielenie części programu związanych z poszczególnymi aspektami domeny problemu.
  W związku z tym zdefiniowano następujące pakiety:
  \begin{itemize}
    \item \texttt{graph} --- pakiet zawierający strukturę reprezentującą graf wraz ze zbiorem metod i funkcji do odczytywania związanych z nim danych, tworzenia oraz modyfikacji jego konstrukcji;
    \item \texttt{preprocessing} --- pakiet zawierający zestaw funkcji realizujących zadanie przetwarzania wstępnego grafu, realizując koncepcje opisane w podrozdziale~\ref{Section_preprocessing};
    \item \texttt{kernelization} --- pakiet zawierający implementacje koncepcji związanych z redukcją dziedziny do jądra problemu pokrycia wierzchołkowego, opisanych w podrozdziale~\ref{s_kernelization};
    \item \texttt{vc} --- pakiet zawierający implementacje algorytmów rozwiązujących problem pokrycia wierzchołkowego: podstawowego algorytmu rekurencyjnego opisanego pseudokodem~\ref{alg_VC1} w podrozdziale~\ref{s_vertex_cover_domain} oraz implementacji własnej metody opartej na drzewie poszukiwań z ograniczeniami, opisanej w podrozdziale~\ref{ss_internals_bnb};
    \item \texttt{matching} --- pakiet zawierający funkcje związane z poszukiwaniem skojarzeń w grafie, opisywane w podrozdziale~\ref{s_supplementary_algorithms}
    \item \texttt{containers} --- pakiet zawierający implementacje wykorzystywanych w pracy struktur danych nie udostępnionych przez biblioteki standardowe języka Go takich jak kolejka priorytetowa i stos;
    \item \texttt{graphviz} --- pakiet stanowiącą ograniczoną do minimum w kontekście oferowanych funkcji bibliotekę opakowującą pakiet Graphviz, opisywany w podrozdziale~\ref{sss_internals_misc_graphviz};
    \item \texttt{utility} --- pakiet zawierający funkcje pomocnicze, niezwiązane konkretnie z żadnym aspektem domeny pracy, przydatne jednak podczas implementacji opisywanych koncepcji, jak na przykład wypisywanie dowolnych wartości do standardowego wyjścia w celu ułatwienia analizy wykonywania programu;
    \item \texttt{main} --- pakiet stanowiący punkt wejściowy programu. Tutaj znajdują się testy logiczne oraz wydajnościowe stanowiące źródło danych do badań eksperymentalnych przeprowadzonych w rozdziale~\ref{results}.
  \end{itemize}
}
\par{
  Organizacja kodu zorientowana na pakiety wykazuje kilka kluczowych zalet, z których główną jest luźne powiązanie komponentów odpowiadających za poszczególne aspekty domeny problemu oraz funkcje potrzebne do ich obsługi.
  Dzięki temu o wiele łatwiej jest pisać testy jednostkowe, izolując poszczególne części logiki w celu weryfikacji poprawności ich działania. 
  Bezpośrednio przekłada się to na rozszerzalność kodu dla osób wykorzystujących lub rozwijających go w przyszłości celem podjęcia nowych wyzwań z zakresu opisywanej w niniejszej pracy problematyki.
  Kod otoczony siecią testów jednostkowych jest wytrzymały --- można poddawać go dowolnym modyfikacjom bez obaw o utracenie poprawności działania poszczególnych jego elementów.
  Dodatkową zaletą płynącą z tego faktu jest zwiększona przejrzystość wynikająca z podziału całości logiki na jasno określone części o dużej czytelności oraz wartości.
  Testy jednostkowe pełnią również rolę dokumentacji i przykładów zastosowania funkcji, a także oczekiwanych względem testowych danych wejściowych wyników.
}
\subsection{Opis wybranych pakietów}\label{ss_internals_important_packages}
\subsubsection{\textbf{Pakiet \texttt{graph}}}
\par{
  W obliczu charakterystyki pracy to znaczy podjęcia się analizy i implementacji szerokiej gamy technik związanych z parametryzacją złożoności obliczeniowej pokrycia wierzchołkowego, kwestią wymagającą uwagi jest sposób reprezentacji grafu.
  W związku ze zróżnicowaniem podejść do problemu, reprezentacja struktur grafowych za pomocą wyłącznie macierzy incydencji lub też listy incydencji okazuje się nie tyle niewystarczająca, co nieoptymalna.
  Podczas gdy niektóre z technik operują na grafie przede wszystkim za pomocą wierzchołków (głównie metody oparte na sformułowaniu problemu jako egzemplarza przepływu w sieci), inne zdecydowanie traktują graf jako zbiór krawędzi pomiędzy pewnymi punktami.
  Jako iż jednym z pomniejszych celów pracy jest stworzenie ogólnej platformy użytecznej do testowania rozwiązań różnych problemów związanych z grafami, zdecydowano się zaimplementować strukturę grafu jako reprezentację zawierającą zarówno macierz incydencji jak i listę krawędzi.
}
\par{
  \begin{lstlisting}[language=go, caption=Typy reprezentujące wierzchołki grafu.]
  type Vertex int
  type Vertices []Vertex
  \end{lstlisting}
  W celu uniknięcia potrzeby uważania tablicy haszujących, gdzie klucze stanowią wierzchołki, zdecydowano się na wprowadzenie metody \textit{\lstinline{Vertex.ToInt() int}}, konwertującej dowolny wierzchołek do odpowiadającej mu wartości całkowitoliczbowej.
  Zabieg ten pozwala na zastosowanie zwykłych tablic w miejsce tablic haszujących, co pozwala na obniżenie złożoności obliczeniowej związanych z takimi strukturami operacji.

  Wartości wierzchołków często występują obok liczb całkowitych, pełniących zazwyczaj w tym kontekście rolę indeksów tablic.
  Aby łatwiej było odróżnić egzemplarze wierzchołków od zwykłych liczb całkowitych, przyjęto konwencję numeracji wartości wierzchołków grafu od 1.
  \begin{lstlisting}[language=go, caption=Typy reprezentujące krawędzie grafu.]
  type Edge struct {
    From      Vertex
    To        Vertex
    isDeleted bool
  }

  type Edges []*Edge
  \end{lstlisting}

  Krawędzie grafu nie są reprezentowane idiomatycznie, to znaczy jako pary wierzchołków stanowiących ich zakończenia.
  Wynika to z praktycznej potrzeby związanej ze zwijaniem krawędzi w ramach operacji przetwarzania wstępnego (podrozdział~\ref{Section_preprocessing}), działania algorytmu Edmondsa (podrozdział~\ref{ss_edmonds_blossom}) oraz algorytmu Chen, Kanj, Xia (podrozdział~\ref{s_ckx}).
  Praktyka pokazuje, iż znacznie łatwiejsze i bardziej wydajne jest oznaczanie krawędzi grafu jako usuniętych zamiast właściwego ich usuwania --- głównie ze względu na fakt, iż w większości przypadków zostają one w późniejszym momencie przywrócone do grafu na przykład w celu określenia pełnej postaci odnalezionej ścieżki powiększającej.

  \begin{lstlisting}[language=go, caption=Struktura reprezentująca graf.]
  type Graph struct {
    Vertices                Vertices
    Edges                   Edges
    CurrentVertexIndex      int
    IsVertexDeleted         []bool
    degrees                 []int
    neighbors               [][]*Edge
    numberOfVertices        int
    numberOfEdges           int
    isRegular               bool
    needToComputeRegularity bool
  }
  \end{lstlisting}

  Podczas implementacji rozwiązań opisywanych w niniejszej pracy problemów napotkano na trzy podstawowe sposoby poruszania się po grafie.
  Każdy z tych sposobów znajduje odzwierciedlenie w implementacji struktury --- istnieje metoda realizująca dany sposób przy zachowaniu najmniejszej możliwej złożoności dla grafu $G=(V, E)$: $O(1)$, $O(|V|)$, lub $O(|E|)$.
  \\\\\underline{Iteracja po wszystkich wierzchołkach}\\
  \par{
      Metoda \textit{\lstinline{ForAllVertices(action func(v Vertex, chan<- done bool))}} iteruje po kolekcji \textit{\lstinline{Vertices}}, zapewniając  podstawową złożoność operacji na poziomie $O(|V|)$, z wyłączeniem złożoności operacji realizowanych przez funkcję \textit{\lstinline{action}}. Kanał \textit{\lstinline{done}} służy do informowania o potrzebie przerwania iteracji.\\
      Metody \textit{\lstinline{ForAllVerticesOfDegree(degree int, action func(v Vertex, chan<- done bool))}} oraz \textit{\lstinline{ForAllVerticesOfDegreeGeq(degree int, action func(v Vertex, chan<- done bool))}} korzystają z metody \textit{\lstinline{ForAllVertices}}, nakładając dodatkowe ograniczenia związane ze stopniem wierzchołków, które mają zostać uwzględnione w iteracjach.
  }
  \\\\\underline{Iteracja po wszystkich krawędziach}\\\
  \par{
      Metoda \textit{\lstinline{ForAllEdges(action func(edge *Edge, done chan<- bool))}} iteruje po kolekcji \textit{\lstinline{Edges}}, zachowując złożoność operacji na poziomie $O(|E|)$ z wyłączeniem złożoności funkcji \textit{\lstinline{action}}.
  }
  \\\\\underline{Uzyskiwanie dostępu do krawędzi pomiędzy danymi wierzchołkami}\\
  \par{
      Metoda \textit{\lstinline{GetEdgeByCoordinates(from, to int) *Edge}} korzysta z macierzy incydencji \textit{\lstinline{neighbors}}, pobierając z niej krawędź łączącą wierzchołki, których wartości skonwertowane do liczb całkowitych odpowiadają podanym współrzędnym.

      Przy okazji omawiania tejże metody należy zwrócić uwagę na następujące dwie kwestie.
      \begin{enumerate}
        \item Macierz incydencji \textit{\lstinline{neighbors}} zamiast wartości logicznych typu \textit{\lstinline{bool}} zawiera wskaźniki na krawędzie istniejące w kolekcji \textit{\lstinline{Edges}}.
        Rozwiązanie to podyktowane jest znów praktyką --- w zdecydowanej większości przypadków fakt istnienia danej krawędzi wiązał się z potrzebą wykonania działań z nią związanych, gdzie przechowywanie wskaźników na krawędzie pozwoliło na uniknięcie dodatkowych jawnych odwołań do kolekcji \textit{\lstinline{Edges}}.
        \item W związku z faktem, iż w ramach niniejszej pracy podjęto się analizy rozwiązań wersji problemów dotyczących grafu o nieskierowanych krawędziach, macierz \textit{\lstinline{neighbors}} zawiera wskaźnik do tej samej krawędzi zarówno dla współrzędnych wprost $(x, y)$, jak i zestawu transponowanego $(y, x)$.
        Rozwiązanie takie zostało przyjęte, mimo pewnego narzutu zajętości pamięci, za pożądane głównie ze względu na spójność właściwej struktury grafu z jego logiczną reprezentacją.
        Dodatkowo, zachowanie takiego rozłożenia wskaźników pomaga w bardziej zrozumiały sposób posługiwać się strukturą sieci przepływowej.
      \end{enumerate}
  }
  \\\\\underline{Operacje wykonywane na zbiorze wierzchołków sąsiednich}\\
  \par{
    Bardzo szeroki zakres działań implementowanych algorytmów w grafie $G=(V,E)$ opiera się na operacjach wykonywanych na sąsiedztwie wierzchołka $N(v); v \in V$.
    Zdecydowanie najczęściej realizowana jest iteracja po wierzchołkach sąsiednich, na drugim miejscu znajdują się operacje wymagające uzyskania sąsiedztwa jako kolekcji lub zbioru w rozumieniu struktury danych.
    W tym celu bardzo pomocna okazuje się macierz incydencji.
    Dla pierwszego przypadku została zaimplementowana metoda \textit{\lstinline{ForAllNeigbors(v Vertex, action func(edge *Edge, done chan<- bool))}}, oparta na rozumowaniu analogicznym do tego, na którym oparto metodę \textit{\lstinline{ForAllEdges}}, jednak iterująca po danym rzędzie macierzy \textit{\lstinline{neighbors}}, indeksowanym przekształconą do liczby całkowitej wartością wierzchołka $v$ z ograniczeniem uwzględniania w iteracjach wyłącznie istniejących wskaźników (to znaczy \textit{\lstinline{edge != nil}}).

    Drugi przypadek obsługiwany jest przez metody \textit{\lstinline{GetNeighbors(v Vertex) Neighbors}} oraz \textit{\lstinline{GetNeighborsWithSet(v Vertex) (Neighbors, mapset.Set)}}.

    Metoda \textit{\lstinline{GetNeighbors}} zwraca odpowiedni rząd macierzy incydencji opakowany w typ pomocniczy, natomiast metoda \textit{\lstinline{GetNeighborsWithSet}} dodatkowo w ramach wewnętrznej iteracji konstruuje zbiór wierzchołków (w rozumieniu struktury danych) stanowiących sąsiedztwo.
    Wszystkie opisane metody związane z operacjami na sąsiedztwie wierzchołka $v$ zachowują złożoność $O(|N(v)|)$ z wyłączeniem złożoności funkcji \textit{\lstinline{action}}.
  }
}
\subsubsection{\textbf{Pakiet \texttt{kernelization} --- algorytm Chen, Kanj, Xia}}
\label{sss_internals_ckx}
\par{
  Algorytm Chen, Kanj, Xia, rozwiązujący problem pokrycia wierzchołkowego, opisany w~podrozdziale~\ref{s_ckx} przed każdym rozgałęzieniem wykorzystuje proste, nierozgałęziające się funkcje redukujące przestrzeń poszukiwań przez usunięcie z niej wierzchołków, o których przynależności do optymalnego pokrycia wierzchołkowego zadecydować można bez posługiwania się krotkami.
  Pierwszą z tych funkcji stanowi strukturyzacja, postępująca zgodnie z pseudokodem~\ref{alg_ckx_struction}.
}

\subsubsection{\textbf{Pakiet \texttt{vc} --- algorytm \texttt{BranchAndBound}}}
\label{ss_internals_bnb}
\par{
  W celu weryfikacji poprawności działania implementacji koncepcji związanych z redukcją dziedziny do jądra problemu pokrycia wierzchołkowego opisanych w podrozdziale~\ref{s_kernelization} wymagane jest zaimplementowanie algorytmu rozwiązującego sam problem pokrycia wierzchołkowego.

  W podrozdziale~\ref{s_vertex_cover_domain} przytoczono prosty, rekurencyjny algorytm siłowy służący do tego celu.
  Jednak w związku z prominencją koncepcji drzewa poszukiwań z ograniczeniami w dziedzinie parametrycznej złożoności obliczeniowej oraz zainteresowaniem technikaliami tej koncepcji, podjęto decyzję o stworzeniu własnego algorytmu na niej opartego, o potencjalnie mniejszej złożoności obliczeniowej od wersji naiwnej przez podążanie wyłącznie za obiecującymi gałęziami poszukiwań.

  \begin{definition}
    \emph{Rozwiązanie kandydackie} w grafie $G=(V, E)$ określa zbiór wierzchołków $C=\{v_1, v_2, \ldots\, v_p\} \neq \emptyset$, gdzie $C \subseteq V$, spełniający warunki częściowego pokrycia wierzchołkowego.
  \end{definition}
  \begin{definition}\thlabel{def_bnb_promising_solution}
    W grafie $G=(V,E)$ dla drzewa poszukiwań $T$ z pewnym najlepszym rozwiązaniem kandydackim $r_b$ rozwiązanie \emph{obiecujące} to rozwiązanie kandydackie $r$ spełniające jedną z następujących własności w podanej kolejności.
    \begin{enumerate}
      \item Liczba krawędzi w zbiorze $E$ pokrytych przez rozwiązanie $r$ jest większa niż liczba krawędzi w zbiorze $E$ pokrytych przez obecne najlepsze rozwiązanie $r_b$.
      \item Liczba krawędzi w zbiorze $E$ pokrytych przez rozwiązanie $r$ jest równa liczbie krawędzi w zbiorze $E$ pokrytych przez obecne najlepsze rozwiązanie $r_b$ i liczebność rozwiązania $|r|$ jest mniejsza od liczebności obecnego najlepszego rozwiązania $|r_b|$.
    \end{enumerate}
  \end{definition}
  \begin{definition}
     W grafie $G=(V, E)$ dla drzewa poszukiwań $T$ z pewnym obiecującym rozwiązaniem $r_b$ \emph{obiecująca gałąź} $t \subseteq T$ zawiera obiecujące rozwiązanie $r_b$.
  \end{definition}
}
\par{
  Algorytm działa zgodnie z pseudokodem~\ref{alg_VC2}.
  Zmienna $Q$ stanowi kolejkę priorytetową działającą według własności określanych Definicją~\ref{def_bnb_promising_solution}.
  \begin{algorithm}
    \caption{Algorytm odnajdujący pokrycie wierzchołkowe --- drzewo poszukiwań z ograniczeniami}\label{alg_VC2}
    \begin{algorithmic}[1]
      \Function{BranchAndBoundVC}{G, k}

        \algorithmicrequire{graf $G=(V, E)$, maks. rozmiar pokr. wierzch. $k$}

        \algorithmicensure{odnalezione pokrycie wierzchołkowe}

        \State $Q \gets \emptyset$\Comment{Q --- kolejka priorytetowa}
        \State $VS \gets \Call{Wierzchołki}{E}$\Comment{Zbiór wierzchołków krawędzi grafu G}
        \State{Wstaw $\emptyset$ do $Q$}
        \State $b \gets \infty$
        \While{$Q \neq \emptyset$}
          \State{$r \gets$ kolejne rozwiązanie z $Q$}
          \If{$r$ jest obiecujące względem $b$}\label{bnb_promising_check1}
            \If{$r$ pokrywa wszystkie krawędzie w $G$}
            \State {$b \gets \min\{|r_{VC}\supseteq r|\}$ takiego, że $r_{VC}$ zawiera pokrycie wierzch.}
            \If{$b \leq k$}
              \State{\textbf{return} $r$}
            \EndIf
            \Else
              \For{$v \in VS$}\label{bnb_forLoop}
                \If{$v \in r$}
                  \State{Idź do kroku~\ref{bnb_forLoop}}
                \EndIf
                \State$r_{tmp} \gets r \cup \{v\}$
                \If{$r_{tmp}$ jest obiecujące względem $b$}\label{bnb_promising_check2}
                  \State{Wstaw $r_{tmp}$ do $Q$}
                \EndIf
              \EndFor
            \EndIf
          \EndIf
        \EndWhile
        \State{\textbf{return nil}}\Comment{Nie istnieje pokrycie o liczebności $\leq k$}
      \EndFunction
    \end{algorithmic}
  \end{algorithm}

  Zastosowany algorytm cechuje się dość kiepską złożonością teoretyczną --- w pewnym najbardziej pesymistycznym przypadku będzie on rozpatrywał wszystkie możliwe podzbiory wierzchołków grafu $G=(V, E)$ o rozmiarze co najwyżej $k$ elementów, co oznacza złożoność $O{|V| \choose k}$.
  W praktyce złożoność obliczeniowa algorytmu jest bardziej skomplikowana matematycznie ze względu na udział dystrybucji stopni wierzchołków grafu i nieco bardziej korzystna.
  Dzięki zastosowaniu warunków~\algref{alg_VC2}{bnb_promising_check1} oraz~\algref{alg_VC2}{bnb_promising_check2} w momencie odnalezienia rozwiązania -- kandydata oferującego lepsze parametry niż najlepsze odnalezione dotychczas rozwiązanie, ustanowiona (lub zmodyfikowana) zostaje granica wartości funkcji celu.
  Rozwiązania oferujące lokalne maksimum wartości funkcji celu gorsze od ustalonej granicy zostają automatycznie odrzucone jeżeli zostały umieszczone w kolejce priorytetowej przed ustaleniem nowej granicy wartości funkcji celu (wiersz~\algref{alg_VC2}{bnb_promising_check1}).
  Dodatkowo w celu zmniejszenia liczby rozwiązań -- kandydatów przechowywanych w kolejce podobna weryfikacja realizowana jest w wierszu~\algref{alg_VC2}{bnb_promising_check2}, przy czym rozwiązania oferujące gorsze maksimum lokalne funkcji celu nie zostają uwzględnione w dalszych poszukiwaniach.
  Drogą testów wydajnościowych na różnych rodzajach grafów --- a w szczególności na grafach pełnych $G=(V, E)$, gdzie $|E|=|V|^2$ --- udało się wykazać, że algorytm w zdecydowanej większości przypadków musi rozpatrzyć ok. 15\% rozwiązań kandydackich w celu udzielenia dokładnej odpowiedzi.
  Mimo, że zapewnia to pewną poprawę względem wariantu siłowego, przedstawiony algorytm nie może stanowić substytutu dla opisywanych w rozdziale~\ref{Chapter_Domain} technik ze względu na wciąż wykładniczą pesymistyczną złożoność obliczeniową.
}
\section{Środowisko wykonawcze}
\par{
  Docelową platformą wykonawczą dla stworzonej w~ramach niniejszej pracy aplikacji jest system Linux.
  Poprawność funkcjonowania programu została zweryfikowana na systemach Arch Linux oraz Ubuntu 14.04.
  W celu wykonania załączonej aplikacji wymagane jest zainstalowanie następujących zależności:
  \begin{itemize}
    \item Środowisko wykonawcze języka Go w~wersji co najmniej 1.3.3.
    \item Niestandardowe biblioteki:
    \begin{itemize}
      \item \texttt{github.com/deckarep/golang-set}
      \item \texttt{github.com/lukpank/go-glpk/glpk}
    \end{itemize}
    \item Pakiet GLPK.
  \end{itemize} 
  Aby móc generować grafy losowe należy zainstalować środowisko wykonawcze języka Python w~wersji co najmniej 3.4.2 oraz pakiet \texttt{graph-tool}.
  W celu tworzenia wizualizacji grafów za pomocą skryptu generacji grafów losowych lub zaimplementowanej struktury \textit{\lstinline{GraphVisualizer}} należy zainstalować bibliotekę \texttt{graphviz}.
}