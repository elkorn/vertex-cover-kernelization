\subsection{Usuwanie wierzchołków wysokiego stopnia}\label{section_kernelization_high-degree}

\begin{theorem}
  Każdy wierzchołek $v$ stopnia $d(v) > k$ musi należeć do optymalnego pokrycia wierzchołkowego 
  $C$ rozmiaru $|C| \leq k$.
\end{theorem}
\begin{bproof}
  W~celu uzyskania pokrycia wierzchołkowego podgrafu $G^\prime=(V^\prime,E^\prime)$
  grafu $G=(V,E)$, gdzie $V^\prime=\{v_0, v_1, \ldots, v_k, v_{k+1}\} \subseteq V$, $d(v_0)=k$ oraz \\
  $E^\prime=\{(v_0,v_1), (v_0,v_2), \ldots, (v_0,v_{k+1})\}$ należy pokryć każdą krawędź $e \in E^\prime$.
  Zakładając istnienie zbiorów $V_1=\{v_0,v_1\}$, $V_2=\{v_0,v_2\}$, $\ldots $,
  $V_k=\{v_0,v_k\}$, $V_{k+1}=\{v_0,v_{k+1}\}$, zbiór $C_1=V_1 \cup V_2 \cup \ldots \cup V_k \cup V_{k+1}$ stanowi pokrycie wierzchołkowe, jednak jego rozmiar wynosi $|C_1| = k + 1$.
  Jeżeli $C_2=V_1 \cap V_2 \cap \ldots \cap V_k \cap V_{k+1} \neq \emptyset$, to zbiór $C_2$ nadal stanowi pokrycie wierzchołkowe grafu $G^\prime$ o~liczebności
  $|C_2|=1$.
  Gdyby usunąć wierzchołek należący do zbioru $C_2$, zbiór $C_3=V_1 \oplus V_2 \oplus \ldots \oplus V_k \oplus V_{k+1}$ nadal
  również stanowi pokrycie wierzchołkowe, jednak jego rozmiar wynosi $|C_3|=k$.
  Dowolny zbiór $C_4=C_3 \setminus \{v\}$ nie spełnia jednak warunków pokrycia wierzchołkowego $G^\prime$.
  Na tej podstawie stwierdzić można, iż dowolne pokrycie wierzchołkowe $C$ grafu $G$ o~rozmiarze $|C| \leq k$ wykluczające wierzchołek $v_0$ musi zawierać całe jego
  sąsiedztwo.
  Pokrycie to nie może być jednak optymalne dla wartości parametru $k > 1$.
  Obserwacja ta prowadzi do wniosku, że optymalne pokrycie wierzchołkowe $C_{\textnormal{opt}}$ grafu $G$ musi zawierać każdy wierzchołek $v \in V$ stopnia $d(v) \geq k$.
\end{bproof}

\begin{theorem}
  Procedura usuwania wierzchołków wysokiego stopnia realizowana jest w~czasie $O(n^2)$.
\end{theorem}
\begin{bproof}
  Aby określić stopień dowolnego wierzchołka $v$ w~grafie $G=(V,E)$ należy dokonać $O(|V|)$ porównań w~celu wyznaczenia jego sąsiedztwa.
  Operacja musi zostać zrealizowana $|V|$ razy w~celu określenia stopnia wszystkich wierzchołków $v \in V$, co w~rezultacie daje złożoność $O (|V|^2)$.
\end{bproof}

Zastosowanie algorytmu usuwania wierzchołków wysokiego stopnia w~połączeniu z~technikami przetwarzania wstępnego ogranicza rozmiar dziedziny problemu przez
to, że każdy wierzchołek $v \in V^\prime$ jest stopnia $3 \leq d(v) \leq k^\prime$.

\begin{theorem}
  Jeżeli $G^\prime$ stanowi graf o~pokryciu wierzchołkowym rozmiaru $k^\prime$, który nie zawiera wierzchołka $v$ stopnia $d(v) > k^\prime \lor d(v) > 3$, to
  zawiera on co najwyżej $\frac{k^{\prime2}}{3} + k^\prime$ wierzchołków.
\end{theorem}
\begin{bproof}
  Przyjąć należy $C$ jako pokrycie wierzchołkowe grafu $G^\prime$ rozmiaru $|C|=k^\prime$.
  Dopełnienie $\overline{C}$ zbioru $C$ stanowi niezależny zbiór $|V^\prime|-k^\prime$ wierzchołków.
  Przyjąć należy zbiór $F=\{f_0,f_1, \ldots, f_p\} \subseteq E^\prime$ krawędzi pokrytych przez $\overline{C}$.
  Ponieważ zachodzi $\forall_{v \in \overline{C}}{d(v) \geq 3}$, każdy wierzchołek $v \in \overline{C}$ musi mieć przynajmniej 3 wierzchołki sąsiednie
  $w \in C$.
  Prowadzi to do wniosku, iż $|F| \geq 3(|V^\prime| - k^\prime)$.
  Liczba krawędzi pokrytych przez zbiór $C$ nie może być mniejsza niż $|F|$ --- nie może być również większa niż $k^\prime|C|$, ponieważ po wykonaniu przetwarzania wstępnego oraz procedury usuwania wierzchołków wysokiego stopnia spełniona musi być własność $\forall_{v \in V}:{|N(v)|\leq k^\prime}$.
  Ponieważ rozmiar pokrycia wierzchołkowego wynosi $|C|=k^\prime$, zbiór $F$ jest rozmiaru $3(|V^\prime|-k^\prime)\leq|F|\leq~k^{\prime2}$, co daje $|V^\prime|\leq\frac{k^{\prime2}}{3}+k^\prime$.
\end{bproof}