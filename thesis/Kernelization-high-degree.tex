\subsection{Usuwanie węzłów wysokiego stopnia}\label{section_kernelization_high-degree}

\begin{theorem}
  Każdy wierzchołek $v; d(v) > k$ musi należeć do optymalnej pokrywy wierzchołkowej 
  $VC; \|VC\| \leq k$.
\end{theorem}
\begin{bproof}
  W~celu uzyskania pokrywy wierzchołkowej podgrafu $G\prime=(V\prime,E\prime)$
  grafu $G=(V,E)$, gdzie $V\prime=\{v_0, v_1, \ldots, v_k, v_{k+1}\}, V\prime
  \subseteq V, d(v_0)=k$ oraz \\
  $E\prime=\{(v_0,v_1), (v_0,v_2), \ldots, (v_0, v_k), (v_0,v_{k+1})\}$,
  należy pokryć każdą krawędź $e \in E\prime$.
  Łatwo zauważyć, iż \\ jeżeli $ve_1=\{v_0,v_1\}, ve_2=\{v_0,v_2\}, \ldots,
  ve_k=\{v_0,v_k\},ve_{k+1}=\{v_0,v_{k+1}\}$,
  to ${VC_1=ve_1 \bigcup ve_2 \bigcup \ldots \bigcup ve_k \bigcup ve_{k+1}}$ spełnia warunki 
  wymagane do uzyskania statusu pokrywy wierzchołkowej, jednak $\|VC_1\| = k +1$.\\
  Jeżeli $VC_2=ve_1 \bigcap ve_2 \bigcap \ldots \bigcap ve_k \bigcap ve_{k+1}; VC_2 \neq \emptyset$,
  stwierdzić można, iż $VC_2$ nadal stanowi pokrywę wierzchołkową $G\prime$ oraz
  $\|VC_2\|=1$.
  Gdyby usunąć wierzchołek należący do $VC_2$, $VC_3=ve_1 \oplus ve_2 \oplus \ldots \oplus ve_k \oplus ve_{k+1}$ nadal
  również stanowi pokrywę wierzchołkową, jednak $\|VC_3\|=k$.
  Dowolny zbiór $VC_4=VC_3 \setminus \{v\}$ nie spełnia jednak warunków pokrywy
  wierzchołkowej $G\prime$.
  Na tej podstawie stwierdzić można, iż dowolna pokrywa wierzchołkowa 
  $VC, \|VC\| \leq k$ grafu $G$ niezawierająca $v_0$, musi zawierać całe jego
  sąsiedztwo. Pokrywa ta nie może być jednak optymalna dla $k > 1$.
  Obserwacja ta prowadzi do wniosku, iż optymalna pokrywa wierzchołkowa
  $\|VC_{opt}\|$ grafu $G$ musi zawierać każdy wierzchołek $v \in V, d(v) \geq k$.
\end{bproof}

\begin{theorem}
  Procedura usuwania wierzchołków wysokiego stopnia realizowana jest w czasie
  $O(n^2)$.
\end{theorem}
\begin{bproof}
  Aby określić stopień dowolnego wierzchołka $v$ w grafie $G=(V,E)$, należy 
  dokonać $O(n), n=\|V\|$ porównań w celu wyznaczenia jego sąsiedztwa.
  Operacja musi zostać zrealizowana $n$ razy w celu określenia stopnia
  wszystkich wierzchołków $G$, co w rezultacie daje złożoność $O (n^2)$.
\end{bproof}

Zastosowanie algorytmu usuwania wierzchołków wysokiego stopnia w połączeniu z
technikami przetwarzania wstępnego ogranicza rozmiar dziedziny problemu przez
fakt, iż każdy wierzchołek $v, v \in V\prime$ jest stopnia $d(v)$ takiego, iż
$3 \leq d(v) \leq k\prime$.

\begin{theorem}
  Jeżeli mianem $G\prime$ określa się graf o pokrywie wierzchołkowej rozmiaru
  $k\prime$, który nie zawiera wierzchołka $v, d(v) > k\prime \lor d(v) > 3$, to
  wtedy $n\prime \leq \frac{k\prime^2}{3} + k\prime$.
\end{theorem}
\begin{bproof}
  Przyjąć należy ${VC,\|VC\|=k\prime}$ jako pokrywę wierzchołkową grafu
  $G\prime$.
  Dopełnienie $\overline{VC}$ zbioru $VC$ stanowi niezależny zbiór
  $n\prime-k\prime$ wierzchołków.
  Przyjąć należy zbiór $F=\{f_0,f_1, \ldots, f_p\}, f \in F \Rightarrow f \in E\prime$
  krawędzi pokrytych przez $\overline{VC}$.
  W związku z faktem, iż $\forall_{v \in \overline{VC}}{d(v) \geq 3}$, każdy
  wierzchołek $v \in \overline{VC}$ musi posiadać przynajmniej 3 wierzchołki sąsiednie
  $w \in C$.
  Prowadzi to do wniosku, iż $\|F\| \geq 3(n\prime - k\prime)$.
  Liczba krawędzi pokrytych przez $C$ nie może być mniejsza niż $\|F\|$ --- nie
  może być również większa niż $k\prime\|C\|$, ponieważ po wykonaniu
  przetwarzania wstępnego oraz procedury usuwania węzłów wysokiego stopnia,
  $\forall_{v \in V}{\|N(v)\|\leq~k\prime}$.
  Pamiętając, że $\|C\|=k\prime$, można stwierdzić, iż
  ${3(n\prime-k\prime)\leq\|F\|\leq~k\prime^2}$, co sprowadza się do
  ${n\prime\leq\frac{k\prime^2}{3}+k\prime}$.
\end{bproof}


