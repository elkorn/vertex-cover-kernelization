\subsection{Usuwanie wierzchołków wysokiego stopnia}\label{section_kernelization_high-degree}

\begin{theorem}
  Każdy wierzchołek $v$, gdzie $d(v) > k$ musi należeć do optymalnego pokrycia wierzchołkowego 
  $C; |C| \leq k$.
\end{theorem}
\begin{bproof}
  W~celu uzyskania pokrycia wierzchołkowego podgrafu $G^\prime=(V^\prime,E^\prime)$
  grafu $G=(V,E)$, gdzie $V^\prime=\{v_0, v_1, \ldots, v_k, v_{k+1}\}, V^\prime
  \subseteq V, d(v_0)=k$ oraz \\
  $E^\prime=\{(v_0,v_1), (v_0,v_2), \ldots, (v_0, v_k), (v_0,v_{k+1})\}$,
  należy pokryć każdą krawędź $e \in E^\prime$.
  Łatwo zauważyć, iż zakładając istnienie zbiorów \[V_1=\{v_0,v_1\}, V_2=\{v_0,v_2\}, \ldots,
  V_k=\{v_0,v_k\},V_{k+1}=\{v_0,v_{k+1}\}\]
  zbiór $C_1=V_1 \cup V_2 \cup \ldots \cup V_k \cup V_{k+1}}$ spełnia warunki 
  wymagane do uzyskania statusu pokrycia wierzchołkowego, jednak $|C_1| = k +1$.\\
  Jeżeli $C_2=V_1 \cap V_2 \cap \ldots \cap V_k \cap V_{k+1}; C_2 \neq \emptyset$,
  stwierdzić można, iż zbiór $C_2$ nadal stanowi pokrycie wierzchołkowe $G^\prime$ oraz
  $|C_2|=1$.
  Gdyby usunąć wierzchołek należący do $C_2$, $C_3=V_1 \oplus V_2 \oplus \ldots \oplus V_k \oplus V_{k+1}$ nadal
  również stanowi pokrycie wierzchołkowe, jednak $|C_3|=k$.
  Dowolny zbiór $C_4=C_3 \setminus \{v\}$ nie spełnia jednak warunków pokrywy
  wierzchołkowej $G^\prime$.
  Na tej podstawie stwierdzić można, iż dowolna pokrycie wierzchołkowe 
  $C, |C| \leq k$ grafu $G$ niezawierająca $v_0$, musi zawierać całe jego
  sąsiedztwo. Pokrywa ta nie może być jednak optymalna dla $k > 1$.
  Obserwacja ta prowadzi do wniosku, iż Optymalne pokrycie wierzchołkowe
  $|C_{opt}|$ grafu $G$ musi zawierać każdy wierzchołek $v \in V, d(v) \geq k$.
\end{bproof}

\begin{theorem}
  Procedura usuwania wierzchołków wysokiego stopnia realizowana jest w czasie
  $O(n^2)$.
\end{theorem}
\begin{bproof}
  Aby określić stopień dowolnego wierzchołka $v$ w grafie $G=(V,E)$, należy 
  dokonać $O(n), n=|V|$ porównań w celu wyznaczenia jego sąsiedztwa.
  Operacja musi zostać zrealizowana $n$ razy w celu określenia stopnia
  wszystkich wierzchołków $G$, co w rezultacie daje złożoność $O (n^2)$.
\end{bproof}

Zastosowanie algorytmu usuwania wierzchołków wysokiego stopnia w połączeniu 
z~technikami przetwarzania wstępnego ogranicza rozmiar dziedziny problemu przez
fakt, iż każdy wierzchołek $v, v \in V^\prime$ jest stopnia $d(v)$ takiego, iż
$3 \leq d(v) \leq k^\prime$.

\begin{theorem}
  Jeżeli przez $G^\prime$ określa się graf o pokryciu wierzchołkowym rozmiaru
  $k^\prime$, który nie zawiera wierzchołka $v, d(v) > k^\prime \lor d(v) > 3$, to
  wtedy $n^\prime \leq \frac{k^\prime^2}{3} + k^\prime$.
\end{theorem}
\begin{bproof}
  Przyjąć należy ${C,|C|=k^\prime}$ jako pokrycie wierzchołkowe grafu
  $G^\prime$.
  Dopełnienie $\overline{C}$ zbioru $C$ stanowi niezależny zbiór
  $n^\prime-k^\prime$ wierzchołków.
  Przyjąć należy zbiór $F=\{f_0,f_1, \ldots, f_p\}, f \in F \Rightarrow f \in E^\prime$
  krawędzi pokrytych przez $\overline{C}$.
  W związku z tym, iż $\forall_{v \in \overline{C}}{d(v) \geq 3}$, każdy
  wierzchołek $v \in \overline{C}$ musi mieć przynajmniej 3 wierzchołki sąsiednie
  $w \in C$.
  Prowadzi to do wniosku, iż $|F| \geq 3(n^\prime - k^\prime)$.
  Liczba krawędzi pokrytych przez zbiór $C$ nie może być mniejsza niż $|F|$ --- nie
  może być również większa niż $k^\prime|C|$, ponieważ po wykonaniu
  przetwarzania wstępnego oraz procedury usuwania wierzchołków wysokiego stopnia,
  $\forall_{v \in V}{|N(v)|\leq~k^\prime}$.
  Biorąc pod uwagę, że $|C|=k^\prime$, można stwierdzić, iż
  ${3(n^\prime-k^\prime)\leq|F|\leq~k^\prime^2}$, co daje ${n^\prime\leq\frac{k^\prime^2}{3}+k^\prime}$.
\end{bproof}


