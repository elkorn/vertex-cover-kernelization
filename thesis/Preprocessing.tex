\section{Techniki przetwarzania wstępnego}\label{Section_preprocessing}

Przed przystąpieniem do właściwego procesu zawężania dziedziny grafu $G=(V,E)$
rozpatrywanej w poszukiwaniu optymalnej pokrywy wierzchołkowej, wykonać można 
(i~należy) zestaw prostych procedur o~złożoności czasowej $O(n^2)$.
Procedury te pozwalają na pozbycie się z~grafu wierzchołków trywialnych do 
rozpatrzenia w kontekście przynależności do optymalnej pokrywy wierzchołkowej,
o~których przynależności do pokrywy decydują proste cechy związane
z~najbliższym sąsiedztwem danwego wierzchołka.
Wynikiem każdej z procedur jest graf $G\prime=(V\prime, E\prime); V\prime
\subseteq V, E\prime \subseteq E$.


Procedury składowe przetwarzania wstępnego:

\begin{enumerate}
  \item Usunięcie wszystkich wierzchołków izolowanych.


    W oparciu o~definicję~\ref{def_isolated_vertex} zauważyć można, iż zawarcie
    izolowanego wierzchołka $u$ w zbiorze pretendującym do miana pokrywy
    wierzchołkowej nie wpływa na pokrycie krawędzi grafu przez ten zbiór.
    Nie ma zatem uzasadnienia dla rozpatrywania wierzchołków izolowanych w
    kontekście dziedziny problemu.
    Utworzyć zatem należy graf $G\prime$, gdzie $V\prime = V
    \setminus u$.
    rozmiar dziedziny problemu zostaje zredukowany do $n\prime=n-1$.
    Operację należy powtarzać tak dopóki $V$ zawiera wierzchołki izolowane.

  \item Usunięcie wszystkich wierzchołków stopnia 1.


    Jeżeli w grafie $G$ istnieje wierzchołek $v$ stopnia 1., można 
    stwierdzić, iż istnieje optymalna taka pokrywa wierzchołkowa $VC_{opt}$, że
    $v \notin VC{opt}, u \in VC_{opt}; (v,u) \in E$.
    Graf $G\prime$ utworzyć można zatem przez usunięcie z dziedziny $v$ oraz
    $u$, zarówno jak i krawędzi $(v,u)$.
    Po wykonaniu tej operacji usunąć należy również wierzchołki sąsiednie dla
    $\{v,u\}$, których stopień wyniósł 0.
    Jedna iteracja redukuje rozmiar dziedziny problemu do $n\prime=n-x$, gdzie
    $x$ stanowi liczebność~zbioru usuniętych wierzchołków.
    Ze względu na usuniętą krawędź, rozmiar parametru $k$ ulega zmniejszeniu do
    $k\prime=k-1$.
    Operację należy powtarzać do momentu usunięcia z grafu wszystkich
    wierzchołków stopnia 1.

  \item Usunięcie wierzchołków stopnia 2., o połączonym sąsiedztwie, wraz 
    z~sąsiedztwem.   
    \begin{theorem}
      Procedura usuwa z~grafu wierzchołek zbędny w~optymalnej pokrywie
      wierzchołkowej.
    \end{theorem}
    \begin{proof}
      W celu uzyskania pokrywy wierzchołkowej podgrafu $G\prime=(V\prime,E\prime)$
      grafu ${G=(V,E); V\prime=\{u, v, w\}, V\prime \subseteq V; E\prime=\{(u,v),
      (u,w), (w,v)\}, E\prime \subseteq E}$, należy pokryć każdą krawędź $e \in
      E\prime$.\\ 
      Łatwo zauważyć, iż jezeli ${ve_1=\{u,v\}, ve_2=\{u,w\}, ve_3=\{w,v\}}$,\\ 
      to ${VC_1=ve_1 \cup ve_2 \cup ve_3}$ spełnia warunki wymagane do uzyskania 
      statusu pokrywy wierzchołkowej.
      Usunięcie dowolnego wierzchołka $x \in V\prime$ nie zmienia faktu, iż 
      ${VC_1=V\prime \setminus \{x\}}$ nadal stanowi pokrywę wierzchołkową
      $G\prime$.
      Usunięcie dowolnego wierzchołka $y \in VC$ powoduje jednak, iż zbiór
      $VC_2=VC_1 \setminus \{y\}$ nie pokrywa jednej z krawędzi $e \in E\prime$.
      Na tej podstawie stwierdzić można, iż dowolna pokrywa wierzchołkowa
      $VC$ zawierać musi przynajmniej 2 wierzchołki należące do $V\prime$, oraz
      że istnieje optymalna pokrywa wierzchołkowa $VC_{opt}$ grafu $G\prime$
      zawierająca dokładnie 2 wierzchołki należące do $V\prime$.
      Co więcej, jeżeli $d(u)=2$, zawarcie wierzchołka $u$ w pokrywie
      wierzchołkowej $VC_{opt}$ gwarantowałoby pokrycie jedynie krawędzi
      $\{(u,v), (u,w), (v,w)\}$, podczas gdy usunięcie $u$ z pokrywy gwarantuje 
      pokrycie nie tylko tych krawędzi ale także i wszystkich krawędzi $e \in E$
      przystających do $\{w, v\}$.
      Na tej podstawie stwierdzić można, iż istnieje optymalna pokrywa
      wierzchołkowa $VC_{opt}$ grafu $G$ taka, że $\{v,w\} \in VC_{opt}$.
    \end{proof}
    W związku z powyższym, graf $G\prime$ otrzymuje się przez usunięcie z grafu
    $G$ wierzchołków $\{u,v,w\}$ oraz krawędzi pomiędzy nimi.
    Następnie, usunąć należy wierzchołki, których stopień wyniósł 0.
    Jedna iteracja redukuje rozmiar dziedziny problemu do wartości
    $n\prime=n-x$, gdzie $x$ stanowi liczebność zbioru usuniętych wierzchołków.
    Ze względu na usunięte krawędzie, rozmiar parametru $k$ ulega zmniejszeniu 
    do wartości $k\prime=k-2$.
    Operację należy powtarzać dopóki w~grafie istnieją wierzchołki stopnia 2.\
    o~połączonym sąsiedztwie.

  \item Zwinięcie rozłącznego sąsiedztwa wierzchołków stopnia 2.
    \begin{theorem}
      Procedura usuwa z~grafu wierzchołki nie mogące lub muszące przynależeć do 
      optymalnej pokrywy wierzchołkowej.
    \end{theorem}
    \begin{proof}
      W celu uzyskania pokrywy wierzchołkowej podgrafu $G\prime=(V\prime,E\prime)$
      grafu ${G=(V,E); V\prime=\{u, v, w\}, V\prime \subseteq V; E\prime=\{(u,v),
      (u,w)\}, E\prime \in E}$,
      należy pokryć każdą krawędź $e \in E\prime$. 
      Łatwo zauważyć, iż jezeli $ve_1=\{u,v\}, ve_2=\{u,w\}$, 
      to $VC_1=ve_1 \cup ve_2$ spełnia warunki wymagane do uzyskania statusu
      pokrywy wierzchołkowej.
      W celu minimalizacji pokrywy, jeżeli $VC_2=ve_1 \cap ve_2; VC_2 \neq \emptyset$,
      stwierdzić można, iż $VC_2$ nadal stanowi pokrywę wierzchołkową $G\prime$.
      Gdyby usunąć wierzchołki należące do $VC_2$, $VC_3=ve_1 \oplus ve_2$ nadal
      również stanowi pokrywę wierzchołkową, jednak $|VC_2| < |VC_3|$.
      Na tej podstawie stwierdzić można, iż istnieje optymalna pokrywa
      wierzchołkowa $VC_{opt}; \{v,w\} \notin VC_{opt}, u \in VC_{opt}$.
    \end{proof}

    W związku z~powyższym, graf $G\prime$ otrzymuje się przez wykonywanie
    następujących operacji do momentu eliminacji wszystkich wierzchołków stopnia
    2.\ o~rozłącznym sąsiedztwie.

    Dla wierzchołka $u; d(u)=2$ o~rozłącznym sąsiedztwie $(w,v)$:
    \begin{itemize}
      \item Każdą krawędź $(v,v_i); v_i \in V \land v_i \neq u$ zastąp 
        krawędzią $(u, v_i)$.
      \item Każdą krawędź $(w,w_i); w_i \in V \land w_i \neq u$ zastąp
        krawędzią $(u, w_i)$.
      \item Usuń wierzchołki $v$ oraz $w$.
    \end{itemize}
    Jedna iteracja redukuje rozmiar dziedziny problemu do wartości
    $n\prime=n-2$.
    Ze względu zastąpienie krawędzi stanowiących o sąsiedztwie wierzchołka $u$,
    wartość parametru $k$ ulega zmniejszeniu do $k\prime=k-1$.

\end{enumerate}

