\section{Techniki przetwarzania wstępnego}\label{Section_preprocessing}

Przed przystąpieniem do procesu zawężania dziedziny grafu $G=(V,E)$
rozpatrywanej w~poszukiwaniu optymalnego pokrycia wierzchołkowego w~opraciu o~algorytmy kernelizacji wykonać można 
(i~należy) zestaw prostych procedur o~złożoności czasowej $O(n^2)$.
Procedury te pozwalają na pozbycie się z~grafu wierzchołków trywialnych do 
rozpatrzenia w~kontekście przynależności do optymalnego pokrycia wierzchołkowego,
o~których przynależności do pokrycia decydują proste cechy związane
z~najbliższym sąsiedztwem danego wierzchołka.
Wynikiem każdej z~procedur jest graf $G^\prime=(V^\prime, E^\prime)$ składający się ze zbiorów $V^\prime\subseteq V$ oraz $E^\prime \subseteq E$.

Procedury składowe przetwarzania wstępnego:
\begin{enumerate}
  \item Usunięcie wszystkich wierzchołków izolowanych.

    W oparciu o~Definicję~\ref{def_isolated_vertex} zauważyć można, że zawarcie
    izolowanego wierzchołka $u$ w zbiorze pretendującym do miana pokrycia
    wierzchołkowego nie wpływa na pokrycie krawędzi grafu przez ten zbiór.
    Nie ma zatem uzasadnienia dla rozpatrywania wierzchołków izolowanych w
    kontekście dziedziny problemu.
    Utworzyć zatem należy graf $G^\prime$, gdzie $V^\prime = V \setminus u$.
    Rozmiar dziedziny problemu zostaje zredukowany do $n^\prime=n-1$.
    Operację należy powtarzać dopóty, dopóki $V$ zawiera wierzchołki izolowane.

  \item Usunięcie wszystkich wierzchołków stopnia 1.

    Jeżeli w~grafie $G$ istnieje wierzchołek $v$ stopnia 1, to istnieje optymalne pokrycie wierzchołkowe $C_{\textnormal{opt}}$, dla którego istnieje krawędź $(v,u) \in E$, łącząca wierzchołki $v \notin VC{\textnormal{opt}}$ oraz $u \in C_{\textnormal{opt}}$.
    Graf $G^\prime$ utworzyć można zatem przez usunięcie z~dziedziny wierzchołków $v$ oraz
    $u$, a także krawędzi $(v,u)$.
    Po wykonaniu tej operacji usunąć należy również wierzchołki sąsiadujące z~wierzchołkami
    $\{v,u\}$, których stopień~wyniósł 0.
    Jedna iteracja redukuje rozmiar dziedziny problemu do $n^\prime=n-x$, gdzie
    $x$ stanowi liczebność~zbioru usuniętych wierzchołków.
    Ze względu na usuniętą krawędź rozmiar parametru $k$ ulega zmniejszeniu do
    $k^\prime=k-1$.
    Operację należy powtarzać do momentu usunięcia z~grafu wszystkich
    wierzchołków stopnia 1.

  \item Usunięcie wierzchołków stopnia 2 o~połączonym sąsiedztwie wraz z~wierzchołkami sąsiednimi.   
    \begin{theorem}
      Procedura usuwa z~grafu wierzchołek zbędny w~optymalnym pokryciu wierzchołkowym.
    \end{theorem}
    \begin{proof}
      W celu uzyskania pokrycia wierzchołkowego podgrafu $G^\prime=(V^\prime,E^\prime)$
      grafu $G=(V,E); V^\prime=\{u, v, w\}$ złożonego ze zbiorów $ V^\prime \subseteq V$ oraz $E^\prime=\{(u,v), (u,w), (w,v)\} \subseteq E$ należy pokryć każdą krawędź $e \in E^\prime$.\\ 
      Jeżeli przyjmiemy zbiory ${V_1=\{u,v\}, V_2=\{u,w\}, V_3=\{w,v\}}$,\\ 
      to zbiór ${C_1=V_1 \cup V_2 \cup V_3}$ spełnia warunki wymagane do uzyskania 
      statusu pokrycia wierzchołkowego.
      Usunięcie dowolnego wierzchołka $x \in V^\prime$ z dziedziny nie wpływa na to, że 
      ${C_1=V^\prime \setminus \{x\}}$ nadal stanowi pokrycie wierzchołkowe grafu $G^\prime$.
      Usunięcie dowolnego wierzchołka $y \in VC$ powoduje jednak, iż zbiór
      $C_2=C_1 \setminus \{y\}$ nie pokrywa jednej z~krawędzi $e \in E^\prime$.
      Na tej podstawie stwierdzić można, że dowolne pokrycie wierzchołkowe
      $VC$ zawierać musi co najmniej 2 wierzchołki należące do zbioru $V^\prime$, a zatem
      istnieje optymalne pokrycie wierzchołkowe $C_{\textnormal{opt}}$ grafu $G^\prime$
      zawierające dokładnie 2 wierzchołki należące do $V^\prime$.
      Jeżeli $d(u)=2$, to zawarcie wierzchołka $u$ w~pokryciu wierzchołkowym $C_{\textnormal{opt}}$ gwarantowałoby pokrycie jedynie krawędzi
      $\{(u,v), (u,w), (v,w)\}$, podczas gdy usunięcie wierzchołka $u$ ze zbioru $C_{\textnormal{opt}}$ gwarantuje 
      pokrycie nie tylko tych krawędzi ale także i~wszystkich krawędzi $e \in E$ przystających do $\{w, v\}$.
      Na tej podstawie stwierdzić można, iż istnieje takie optymalne pokrycie wierzchołkowe $C_{\textnormal{opt}}$ grafu $G$, że $\{v,w\} \in C_{\textnormal{opt}}$.
    \end{proof}
    W związku z~powyższym graf $G^\prime$ otrzymuje się przez usunięcie z~grafu
    $G$ wierzchołków $\{u,v,w\}$ oraz krawędzi pomiędzy nimi.
    Następnie usunąć należy wierzchołki, których stopień wyniósł 0.
    Jedna iteracja redukuje rozmiar dziedziny problemu do wartości
    $n^\prime=n-x$, gdzie $x$ stanowi liczebność zbioru usuniętych wierzchołków.
    Ze względu na usunięte krawędzie rozmiar parametru $k$ ulega zmniejszeniu 
    do wartości $k^\prime=k-2$.
    Operację należy powtarzać dopóty, dopóki w~grafie istnieją wierzchołki stopnia 2\
    o~połączonym sąsiedztwie.

  \item Zwinięcie rozłącznego sąsiedztwa wierzchołków stopnia 2.
    \begin{theorem}
      Procedura usuwa z~grafu wierzchołki, które nie mogą lub muszą przynależeć do 
      optymalnego pokrycia wierzchołkowego.
    \end{theorem}
    \begin{proof}
      Dla pewnego grafu $G=(V,E)$ w celu uzyskania pokrycia wierzchołkowego podgrafu $G^\prime=(V^\prime,E^\prime)$ o zbiorze wierzchołków $V^\prime=\{u, v, w\} \subseteq V$ i zbiorze krawędzi $E^\prime=\{(u,v), (u,w)\} \subseteq E$ należy pokryć każdą krawędź $e \in E^\prime$. 
      Zauważmy, że dla zbiorów $V_1=\{u,v\}$ i $V_2=\{u,w\}$ zbiór $C_1=V_1 \cup V_2$ stanowi pokrycie wierzchołkowe.
      Jeżeli $C_2=V_1 \cap V_2 \neq \emptyset$, to zbiór $C_2$ nadal stanowi pokrycie wierzchołkowe grafu $G^\prime$.
      Gdyby usunąć wierzchołki należące do $C_2$, zbiór $C_3=V_1 \oplus V_2$ również stanowi pokrycie wierzchołkowe, jednak $|C_2| < |C_3|$.
      Na tej podstawie stwierdzić można, że istnieje takie optymalne pokrycie wierzchołkowe $C_{\textnormal{opt}}$ spełniające własności $\{v,w\} \notin C_{\textnormal{opt}}$ oraz $u \in C_{\textnormal{opt}}$.
    \end{proof}

    W związku z~powyższym graf $G^\prime$ otrzymuje się przez wykonywanie
    następujących operacji do momentu eliminacji wszystkich wierzchołków stopnia
    2\ o~rozłącznym sąsiedztwie.

    Dla wierzchołka $u$ stopnia $d(u)=2$ o~rozłącznym sąsiedztwie $(w,v)$:
    \begin{itemize}
      \item Każdą krawędź $(v,v_i); v_i \in V \land v_i \neq u$ zastąp 
        krawędzią $(u, v_i)$.
      \item Każdą krawędź $(w,w_i); w_i \in V \land w_i \neq u$ zastąp
        krawędzią $(u, w_i)$.
      \item Usuń z grafu wierzchołki $v$ oraz $w$.
    \end{itemize}
    Jedna iteracja redukuje rozmiar dziedziny problemu do wartości
    $n^\prime=n-2$.
    Ze względu na zastąpienie krawędzi stanowiących o~sąsiedztwie wierzchołka $u$,
    wartość parametru $k$ ulega zmniejszeniu do $k^\prime=k-1$.

\end{enumerate}

