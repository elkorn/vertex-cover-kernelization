\chapter{Wstęp}\label{Chapter_Wstep}

  \section{Cel}\label{Section_Aim}

  \section{Układ pracy}\label{Section_Layout}

\chapter{Zagadnienie }\label{Chapter_Domain}
  \section{Opis dziedziny problemu}\label{Section_Problematyka}

  \section{Podstawowe definicje}\label{Section_Definicje}

  \section{Techniki przetwarzania wstępnego}\label{Section_preprocessing}

  Przed przystąpieniem do właściwego procesu zawężania dziedziny grafu
  rozpatrywanej w poszukiwaniu pokrywy wierzchołkowej, wykonać można (i~należy)
  zestaw prostych procedur o~niewielkiej złożoności czasowej.
  Procedury te pozwalają na pozbycie się z~grafu wierzchołków trywialnych,
  o~których przynależności do pokrywy decydują proste cechy związane
  z~najbliższym sąsiedztwem danwego wierzchołka.


  Procedury składowe przetwarzania wstępnego:
  \begin{enumerate}
    \item Usunięcie wszystkich wierzchołków izolowanych
      % Are cross-references allowed? E.g. refer here to the definition of an
      % isolated vertex.
    \item Usunięcie wsyzstkich wierzchołków stopnia 1., a~następnie usunięcie
      wszystkich wierzchołków izolowanych.
    \item Usunięcie wszystkich wierzchołków stopnia 2., których sąsiedztwo jest
      połączone, wraz z~sąsiedztwem. Następnie, usunięcie wszystkich
      wierzchołków izolowanych.
    \item Dopóki istnieją wierzchołki stopnia 2., dla każdego wierzchołka $u$ 
      stopnia 2. o~niepołączonym sąsiedztwie $(w,v)$
      \begin{itemize}
        \item[-] Każdą krawędź $(v,v_i)$ zastąp krawędzią $(u, v_i)$.
        \item[-] Każdą krawędź $(w,w_i)$ zastąp krawędzią $(u, w_i)$.
        \item[-] Usuń wierzchołki $v$ oraz $w$.
      \end{itemize}
  \end{enumerate}

  \begin{lemma}
    Krawędź $e=(u,v)$ uznaje się za \emph{przystającą} do wierzchołka $v$,
    jeżeli $u=v \lor w=v$.
  \end{lemma}

  \begin{lemma}
    Krawędź $e=(a,b)$ uznaje się za \emph{pokrytą} przez zbiór wierzchołków \\
    $V=\{a_0, a_1, \ldots, a_p\}$, jeżeli przystaje ona do co najmniej jednego
    wierzchołka $v \in V$.
  \end{lemma}

  \begin{lemma}
    Pokrywą wierzchołkową grafu $G=(V,E)$ nazywa się taki zbiór wierzchołków
    $VC \subseteq V$, że każda krawędź $e \in E$ jest pokryta przez $VC$.
  \end{lemma}

  \begin{theorem}
    Procedura składowa 4. przetwarzania wstępnego usuwa z~grafu wierzchołki nie 
    mogące przynależeć do optymalnej pokrywy wierzchołkowej.
  \end{theorem}
  \begin{proof}
    W celu uzyskania pokrywy wierzchołkowej podgrafu $G\prime=(V\prime,E\prime)$
    grafu $G=(V,E)$, gdzie $V\prime=\{u, v, w\}, V\prime \subseteq V$ oraz 
    $E\prime=\{(u,v), (u,w)\}$, należy pokryć każdą krawędź $e \in E\prime$. 
    Łatwo zauważyć, iż jezeli $ve_1=\{u,v\}, ve_2=\{u,w\}$, \\
    to $VC_1=ve_1 \cup ve_2$ spełnia warunki wymagane do uzyskania statusu
    pokrywy wierzchołkowej.
    W celu minimalizacji pokrywy, jeżeli $VC_2=ve_1 \cap ve_2; VC_2 \neq \emptyset$,
    stwierdzić można, iż $VC_2$ nadal stanowi pokrywę wierzchołkową $G\prime$.
    Gdyby usunąć wierzchołki należące do $VC_2$, $VC_3=ve_1 \oplus ve_2$ nadal
    również stanowi pokrywę wierzchołkową, jednak $\|VC_2\| < \|VC_3\|$.
  \end{proof}

\chapter{Wybrane algorytmy }\label{Chapter_Algorytmy}

