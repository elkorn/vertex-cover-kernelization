\chapter{Wstęp}\label{Chapter_Wstep}

  \section{Cel}\label{Section_Aim}

  \section{Układ pracy}\label{Section_Layout}

\chapter{Zagadnienie }\label{Chapter_Domain}
  \section{Opis dziedziny problemu}\label{Section_Problematyka}

  \section{Podstawowe definicje}\label{Section_Definicje}

  \section{Techniki przetwarzania wstępnego}\label{Section_preprocessing}

  Przed przystąpieniem do właściwego procesu zawężania dziedziny grafu
  rozpatrywanej w poszukiwaniu pokrywy wierzchołkowej, wykonać można (i~należy)
  zestaw prostych procedur o~niewielkiej złożoności czasowej.
  Procedury te pozwalają na pozbycie się z~grafu wierzchołków trywialnych,
  o~których przynależności do pokrywy decydują proste cechy związane
  z~najbliższym sąsiedztwem danwego wierzchołka.


  Elementy składowe przetwarzania wstępnego:
  \begin{enumerate}
    \item Usunięcie wszystkich wierzchołków izolowanych
      % Are cross-references allowed? E.g. refer here to the definition of an
      % isolated vertex.
    \item Usunięcie wsyzstkich wierzchołków stopnia 1., a~następnie usunięcie
      wszystkich wierzchołków izolowanych.
    \item Usunięcie wszystkich wierzchołków stopnia 2., których sąsiedztwo jest
      połączone, wraz z sąsiedztwem. Następnie, usunięcie wszystkich
      wierzchołków izolowanych.
    \item Dopóki istnieją wierzchołki stopnia 2., dla każdego wierzchołka $u$ 
      stopnia 2. o niepołączonym sąsiedztwie $(w,v)$:
      \begin{itemize}
        \item[-] Każdą krawędź $(v,v_i)$ zastąp krawędzią $(u, v_i)$.
        \item[-] Każdą krawędź $(w,w_i)$ zastąp krawędzią $(u, w_i)$.
        \item[-] Usuń wierzchołki $v$ oraz $w$.
      \end{itemize}
  \end{enumerate}


\chapter{Wybrane algorytmy }\label{Chapter_Algorytmy}

