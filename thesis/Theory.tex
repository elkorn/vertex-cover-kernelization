\chapter{Wstęp}\label{Chapter_Wstep}

\section{Cel}\label{Section_Aim}

\section{Układ pracy}\label{Section_Layout}

\chapter{Zagadnienie }\label{Chapter_Domain}
\section{Opis dziedziny problemu}\label{Section_Problematyka}

\section{Podstawowe definicje}\label{Section_Definicje}

\begin{lemma}
  Krawędź $e=(u,v)$ uznaje się za \emph{przystającą} do wierzchołka $v$,
  jeżeli $u=v \lor w=v$.
\end{lemma}

\begin{lemma}
  Krawędź $e=(a,b)$ uznaje się za \emph{pokrytą} przez zbiór wierzchołków \\
  $V=\{a_0, a_1, \ldots, a_p\}$, jeżeli przystaje ona do co najmniej jednego
  wierzchołka $v \in V$.
\end{lemma}

\begin{lemma}
  Pokrywą wierzchołkową grafu $G=(V,E)$ nazywa się taki zbiór wierzchołków
  $VC \subseteq V$, że każda krawędź $e \in E$ jest pokryta przez $VC$.
\end{lemma}

\section{Techniki przetwarzania wstępnego}\label{Section_preprocessing}

Przed przystąpieniem do właściwego procesu zawężania dziedziny grafu
rozpatrywanej w poszukiwaniu pokrywy wierzchołkowej, wykonać można (i~należy)
zestaw prostych procedur o~niewielkiej złożoności czasowej.
Procedury te pozwalają na pozbycie się z~grafu wierzchołków trywialnych,
o~których przynależności do pokrywy decydują proste cechy związane
z~najbliższym sąsiedztwem danwego wierzchołka.


Procedury składowe przetwarzania wstępnego:
\begin{enumerate}
  \item Usunięcie wszystkich wierzchołków izolowanych
    % Are cross-references allowed? E.g. refer here to the definition of an
    % isolated vertex.
  \item Usunięcie wsyzstkich wierzchołków stopnia 1., a~następnie usunięcie
    wszystkich wierzchołków izolowanych.
  \item Usunięcie wszystkich wierzchołków stopnia 2., których sąsiedztwo jest
    połączone, wraz z~sąsiedztwem. Następnie, usunięcie wszystkich
    wierzchołków izolowanych.
  \item Dopóki istnieją wierzchołki stopnia 2., dla każdego wierzchołka $u$ 
    stopnia 2. o~niepołączonym sąsiedztwie $(w,v)$
    \begin{itemize}
      \item[-] Każdą krawędź $(v,v_i)$ zastąp krawędzią $(u, v_i)$.
      \item[-] Każdą krawędź $(w,w_i)$ zastąp krawędzią $(u, w_i)$.
      \item[-] Usuń wierzchołki $v$ oraz $w$.
    \end{itemize}
\end{enumerate}

\begin{theorem}
  Procedura składowa 4. przetwarzania wstępnego usuwa z~grafu wierzchołki nie 
  mogące przynależeć do optymalnej pokrywy wierzchołkowej.
\end{theorem}
\begin{proof}
  W celu uzyskania pokrywy wierzchołkowej podgrafu $G\prime=(V\prime,E\prime)$
  grafu $G=(V,E)$, gdzie $V\prime=\{u, v, w\}, V\prime \subseteq V$ oraz 
  $E\prime=\{(u,v), (u,w)\}$, należy pokryć każdą krawędź $e \in E\prime$. 
  Łatwo zauważyć, iż jezeli $ve_1=\{u,v\}, ve_2=\{u,w\}$, \\
  to $VC_1=ve_1 \cup ve_2$ spełnia warunki wymagane do uzyskania statusu
  pokrywy wierzchołkowej.
  W celu minimalizacji pokrywy, jeżeli $VC_2=ve_1 \cap ve_2; VC_2 \neq \emptyset$,
  stwierdzić można, iż $VC_2$ nadal stanowi pokrywę wierzchołkową $G\prime$.
  Gdyby usunąć wierzchołki należące do $VC_2$, $VC_3=ve_1 \oplus ve_2$ nadal
  również stanowi pokrywę wierzchołkową, jednak $\|VC_2\| < \|VC_3\|$.
\end{proof}

\section {Techniki zawężania dziedziny}\label{Section_kernelization}

Proces zawężania dziedziny realizowany jest przed właściwym poszukiwaniem
pokrywy wierzchołkowej. Polega on na usunięciu z dziedziny problemu
wierzchołków, które z~pewnością nie należą lub z~pewnością należą do optymalnej
pokrywy wierzchołkowej grafu wejściowego o~rozmiarze $k\prime \leq k$.
Dzięki parametryzacji maksymalną wartością oczekiwanego rozmiaru wyniku,
istnieje możliwość zakończenia przetwarzania z~odpowiedzią negatywną bez
angażowania właściwej logiki wyszukiwania pokrywy wierzchołkowej, co pozytywnie
wpływa na średnią złożoność obliczeniową procesu.

Opisane algorytmy zawężające dziedzinę problemu są od siebie niezależne.
Wynikiem działania każdego z~nich jest zbiór wierzchołków nietrywialnych do
rozpatrzenia z~perspektywy podjęcia decyzji o~przynależności do optymalnej
pokrywy wierzchołkowej.

\subsection {Usuwanie węzłów wysokiego stopnia}\label{section_kernelization_high-degree}

\begin{theorem}
  Każdy wierzchołek $v; d(v) > k $ musi należeć do optymalnej pokrywy wierzchołkowej 
  $VC; \|VC\| \leq k$.
\end{theorem}
\begin{proof}
  W~celu uzyskania pokrywy wierzchołkowej podgrafu $G\prime=(V\prime,E\prime)$
  grafu $G=(V,E)$, gdzie $V\prime=\{v_0, v_1, \ldots, v_k, v_{k+1}\}, V\prime
  \subseteq V$ oraz \\
  $E\prime=\{(v_0,v_1), (v_0,v_2), \ldots, (v_0, v_k), (v_0,v_{k+1})\}$,
  należy pokryć każdą krawędź $e \in E\prime$.
  Łatwo zauważyć, iż \\ jeżeli $ve_1=\{v_0,v_1\}, ve_2=\{v_0,v_2\}, \ldots,
  ve_k=\{v_0,v_k\},ve_{k+1}=\{v_0,v_{k+1}\}$,\\
  to $VC_1=ve_1 \cup ve_2 \cup \ldots \cup ve_k \cup ve_{k+1}$ spełnia warunki 
  wymagane do uzyskania statusu pokrywy wierzchołkowej, jednak $\|VC_1\| = k +1$.\\
  Jeżeli $VC_2=ve_1 \cap ve_2 \cap \ldots \cap ve_k \cap ve_{k+1}; VC_2 \neq \emptyset$,
  stwierdzić można, iż $VC_2$ nadal stanowi pokrywę wierzchołkową $G\prime$ oraz
  $\|VC_2\|=1$.
  Gdyby usunąć wierzchołek należący do $VC_2$, $VC_3=ve_1 \oplus ve_2 \oplus \ldots \oplus ve_k \oplus ve_{k+1}$ nadal
  również stanowi pokrywę wierzchołkową, jednak $\|VC_3\|=k$, a~zatem $VC_3$ nie
  może należeć do pokrywy wierzchołkowej o~rozmiarze $k\prime \leq k$.
\end{proof}

\chapter{Wybrane algorytmy }\label{Chapter_Algorytmy}

