\RequirePackage{ifpdf}
\newif\ifelektroniczna
\newif\ifjednostronna

%\elektronicznatrue
\elektronicznafalse

% \jednostronnafalse
\jednostronnatrue

\ifjednostronna
\def\strony{oneside,openany}
\else
\def\strony{twoside,openright}
\fi

\ifpdf
\documentclass[pdftex,12pt,a4paper,\strony,colorlinks,nocenter,noupper,crosshair]{thesis}
\usepackage[pdftex]{graphicx}
\usepackage[pdftex]{hyperref}
\hypersetup{colorlinks,%
  citecolor=black,%
  filecolor=black,%
  linkcolor=black,%
  urlcolor=black,%
pdftex}
\pdfcompresslevel=1
\else
\documentclass[12pt,a4paper,\strony,nocenter,noupper,crosshair]{thesis}
\usepackage{graphicx}
\fi

\usepackage{wrapfig}
\usepackage{lscape}
\usepackage{rotating}
\usepackage{url}
\usepackage{TitlePage}
\usepackage{theoremref}
\usepackage{mathtools}
\usepackage{tikz}
\usepackage{algorithm}% http://ctan.org/pkg/algorithms
\usepackage{algpseudocode}% http://ctan.org/pkg/algorithmicx
\usepackage{enumerate}
\makeatletter
\renewcommand{\ALG@name}{Pseudokod}
\renewcommand{\listalgorithmname}{Zbiór pseudokodów}
\renewcommand{\algorithmicrequire}{\textbf{Argumenty: }}
\renewcommand{\algorithmicensure}{\textbf{Wynik: }}
\makeatother
\usetikzlibrary{calc, shapes, backgrounds}
\usepackage[utf8]{inputenc}
\def\rodzaj{Praca magisterska}

\def\wydzial{Automatyki, Elektroniki i~Informatyki}

\def\tytul{Parametryzowana złożoność obliczeniowa}
\def\tytulpdf{Parametryzowana złożoność obliczeniowa}

\def\autor{Autor: Korneliusz Adam Caputa}
\def\promotor{Kierujący pracą: prof.~dr~hab.~inż. Zbigniew Czech}
\def\konsultant{}
\def\data{Gliwice, listopad 2014}
\def\slowakluczowe{Fixed point tractability,graph theory,vertex cover,kernelization}

\graphicspath{{./pictures/}}

\DeclareUnicodeCharacter{00A0}{~}

\ifpdf
\ifelektroniczna
\usepackage[
  pdfusetitle=true,
  pdfsubject={\tytulpdf},
  pdfkeywords={\slowakluczowe},
  pdfcreator={\autor},
  pdfstartview=FitV,
  linkcolor=blue,
  citecolor=red,
]{hyperref}
\fi
\fi

\usepackage{layout}

\usepackage{t1enc,amsmath}
\usepackage[OT4,plmath]{polski}
\usepackage{helvet}

%\usepackage{anysize}
%\marginsize{3cm}{2.5cm}{2.5cm}{2.5cm}%LPGD
%\setlength{\textheight}{24cm}
%\usepackage{multirow}
%\ifpdf\usepackage{pdflscape}\else\usepackage{lscape}\fi
%\usepackage{longtable}
\usepackage[hmarginratio=3:2]{geometry}
%GATHER{thesis.bib}
%\usepackage[twoside]{geometry}
%\geometry{ lmargin=3.5cm, rmargin=2.5cm, tmargin=3cm, bmargin=3cm,
%headheight=1cm, headsep=0.5cm, footskip=0pt }
%\def\fixme#1{}f

\textwidth 150mm
\textheight 225mm
\usepackage{amsfonts}
\usepackage{amsthm}
\usepackage{subfig}
\usepackage{caption}
\captionsetup[subfigure]{justification=centerfirst}
\usepackage{cite}
\usepackage{listings}
\lstset{
  language={Go},
  captionpos=b,
  inputencoding=utf8
}
\usepackage{cleardpempty}
\usepackage{float}
\usepackage{textcomp}
\def\vec#1{\ensuremath{\mathbf{#1}}}
\def\ang#1{ang.~\emph{#1}}
\def\lat#1{lac.~\emph{#1}}
\def\e{\ensuremath{\textrm{\normalfont{}e}}}
\def\degree{\ensuremath{^{\circ}}\protect}
\def\fixme#1{\marginpar{\tiny{}#1}}
\def\labelitemi{--}
\def\labelitemii{--}
\def\labelitemiii{--}

\newtheorem{theorem}{\parindent=0pt{\textbf{Twierdzenie}}}[section]
\newtheoremstyle{named}{}{}{\itshape}{}{\bfseries}{.}{.5em}{\thmname{#1}\thmnumber{ #2}\textnormal{\thmnote{#3}}}
\theoremstyle{named}
\newtheorem*{namedtheorem}{Twierdzenie}
\newtheorem{property}{\parindent=0pt{\textbf{Własność}}}[section]
\newtheorem{lemma}{\parindent=0pt{\textbf{Lemat}}}[section]
\newtheorem{corollary}{\parindent=0pt{\textbf{Wniosek}}}[section]
\newtheorem{definition}{\parindent=0pt{\textbf{Definicja}}}[section]
\newtheorem{proposition}{\parindent=0pt{\textbf{Założenie}}}[section]
\newtheorem{conjecture}{\parindent=0pt{\textbf{Przpyuszczenie}}}[section]
\newtheorem{note}{\parindent=0pt{\textbf{Uwaga}}}[chapter]
\newenvironment{bproof}{\parindent=0pt{\textbf{Dowód.} }}{\begin{flushright}$\square$\end{flushright}}

  \def\captionlabeldelim{.}
  \linespread{1}
  \chapterfont{\Huge\bfseries}
  \sectionfont{\bfseries\Large}
  \subsectionfont{\bfseries\large}
  \institutionfont{\bfseries}%\mdseries}
  \def\captionlabelfont{\bfseries}

  \renewcommand{\figureshortname}{Rys.}
  \renewcommand{\tableshortname}{Tab.}

  \renewcommand\floatpagefraction{.9}
  \renewcommand\topfraction{.9}
  \renewcommand\bottomfraction{.9}
  \renewcommand\textfraction{.1}
  \setcounter{totalnumber}{50}
  \setcounter{topnumber}{50}
  \setcounter{bottomnumber}{50}

  \newcommand{\topcaption}{%
    \setlength{\abovecaptionskip}{0pt}%
    \setlength{\belowcaptionskip}{10pt}%
  \caption}

  \newcommand\scalemath[2]{\scalebox{#1}{\mbox{\ensuremath{\displaystyle #2}}}}

  \hyphenation{wew-nętrz-nej}

  % \makeatletter
  % \renewcommand\fs@ruled{\def\@fs@cfont{\bfseries}\let\@fs@capt\floatc@ruled
  %   \def\@fs@pre{\hrule height1.0pt depth0pt \kern2pt}
  %   \def\@fs@post{\vskip-1.5\baselineskip\kern2pt\hrule\relax}%
  %   \def\@fs@mid{\kern2pt\hrule\kern2pt}
  % \let\@fs@iftopcapt\iftrue}
  % \makeatother

  \floatstyle{ruled}
  % \newfloat{sample}{thp}{lop}
  % \floatname{sample}{Przykład}

  \begin{document}

  \bibliographystyle{plain}
  \frontmatter
  \stronatytulowa
  %\cleardoublepage
  %\maketitle
  %\tocbibname

  \tableofcontents % \listoffigures \listoftables \listof{sample}{Spis przykładów}
  %\listofacros
  %\input{abbrev_body}
  %\newpage
  %\input{spis_oznaczen}

  \mainmatter
  \chapter{Wstęp}\label{Chapter_Introduction}
\section{Cel}\label{Section_Aim}
\par{
  Jedną z~cech charakterystycznych problemów należących do klasy $\mathcal{NP}$ jest trudność rozwiązania ich (w przeciwieństwie do weryfikacji poprawności danej odpowiedzi) w~czasie umożliwiającym zastosowanie w~skali spotykanej w~praktyce w~informatyce oraz innych dziedzinach takich jak przemysł, biznes czy biologia.
  Richard Karp określa takiej trudności problem jako \emph{satysfakcjonująco} rozwiązany w~sytuacji gdy pewien algorytm jest w~stanie znaleźć jego rozwiązanie wykonując skończoną ilość kroków, ograniczoną pewnym wielomianem, którego zmienną stanowi rozmiar danych wejściowych --- mówi się wtedy o rozwiązaniu otrzymanym w~\emph{czasie wielomianowym}.
  Natura problemów tego stopnia trudności bardzo często związana jest z~domeną elementów policzalnych.
  Popularne i przydatne zarówno w~szerszym kontekście badań algorytmicznych jak i w~praktyce okazują się być zadania polegające na określaniu charakterystycznych właściwości grafów, macierzy całkowitoliczbowych, rodzin skończonych zbiorów, wzorów logicznych i podobnych im struktur.
}
\par{
  Niniejsza praca skupia się na analizie, implementacji oraz opisie wybranych metod rozwiązywania problemu pokrycia wierzchołkowego grafu z~wykorzystaniem technik zaliczanych do grupy metod parametryzacji.
  Problem pokrycia wierzchołkowego należy do klasy problemów $\mathcal{NP}$-zupełnych, stanowiących podzbiór problemów klasy $\mathcal{NP}$.
  Problem pokrycia wierzchołkowego został uwzględniony w~zestawie 21 problemów $\mathcal{NP}$-zupełnych w~pracy~\cite{DBLP:Karp10} Richarda Karpa z~roku 1972.
  Studium klasy problemów $\mathcal{NP}$-zupełnych od ponad 50 lat stanowi bardzo aktywną i obszerną dziedzinę algorytmiki oraz teorii obliczeń.
  Mimo niezwykle bogatego dorobku naukowego związanego z~analizą problemów $\mathcal{NP}$-zupełnych pytanie czy klasy problemów $\mathcal{NP}$ i $\mathcal{P}$ są jednoznaczne nadal pozostaje otwarte --- nie udzielono na nie popartej konstruktywnymi dowodami odpowiedzi.
  Fakt ten warunkuje dalsze postępy w~tej dziedzinie i jednocześnie zachęca do zgłębiania opisywanej tematyki, oferując szerokie pole dla nowego wkładu w~jej rozwój.
  Głównym celem pracy jest przedstawienie fundamentalnej teoretycznej wiedzy dotyczącej podejścia do rozwiązywania problemów $\mathcal{NP}$-zupełnych opartego o techniki parametryzacji na przykładzie problemu pokrycia wierzchołkowego grafu oraz materializacji koncepcji teoretycznych w~postaci implementacji opisywanych algorytmów.
  Za cele dodatkowe pracy uznaje się przedstawienie skomplikowanej matematycznie problematyki w~sposób przystępny dla czytelnika o tle inżynierskim, stanowiące zrozumiałą podporę w~dalszych badaniach lub adaptacji przedstawionych rozwiązań do konkretnych problemów praktycznych oraz utworzenie możliwie jednolitej platformy ułatwiającej implementację i badania eksperymentalne nad algorytmami związanymi z~problemami poruszającymi tematykę grafów.
}
\section{Układ pracy}\label{Section_Layout}
\par{
  Praca podzielona została na cztery główne rozdziały, związane ściśle z~poszczególnymi etapami prowadzonych prac.
}
\subsection{Rozdział teoretyczny --- opis zagadnienia}
\par{
  Rozdział~\ref{Chapter_Domain} zawiera opis teoretyczny oraz analizę koncepcji związanych z~poruszaną tematyką.

  Podrozdział~\ref{Section_Domain} stanowi szczegółowe wprowadzenie do domeny problemu pokrycia wierzchołkowego oraz oględnie przedstawia uznane grupy technik ograniczania złożoności obliczeniowej algorytmów rozwiązujących problemy $\mathcal{NP}$-zupełne.

  Podrozdziały~\ref{s_methods} oraz~\ref{s_kernelization} skupiają się na opisie i analizie konkretnych technik należących do grup opisanych w~podrozdziale~\ref{Section_Domain} --- w~szczególności do grupy technik parametryzacji --- wykorzystanych do realizacji założonych w~niniejszej pracy celów.

  Podrozdział~\ref{s_definitions} stanowi zbiór definicji podstawowych pojęć wykorzystywanych w~dalszych częściach pracy, których znajomość jest wymagana do zrozumienia prezentowanego toku rozumowania.
  Pojęcia zdefiniowane w~ramach podrozdziału~\ref{s_definitions} są wykorzystywane w~analizie wszystkich następujących koncepcji --- dlatego też zostały zagregowane w~poprzedzającym ją miejscu.
  Pojęcia związane bezpośrednio z~konkretnym algorytmem zawarte są w~odpowiadającym mu podrozdziale.

  Podrozdział~\ref{Section_preprocessing} przybliża proste techniki modyfikacji struktury grafu stanowiące uzupełnienie mające na celu zwiększenie efektywności algorytmów opisywanych w~podrozdziale~\ref{s_kernelization}.

  Podrozdział~\ref{s_kernelization} przedstawia analizę poszczególnych technik redukcji dziedziny do jądra problemu pokrycia wierzchołkowego zaproponowanych w~literaturze źródłowej.

  Podrozdział~\ref{s_ckx} poświęcony jest w~całości algorytmowi zaproponowanemu w~pracy~\cite{ImprovedBounds10} --- algorytm ten stanowi wyczerpującą całość, wykorzystującą i łączącą opisane w~poprzedzających podrozdziałach koncepcje w~celu uzyskania dużej redukcji złożoności obliczeniowej.

  Podrozdział~\ref{s_supplementary_algorithms} przybliża algorytmy niezwiązane bezpośrednio z~dziedziną pokrycia wierzchołkowego, które stanowią jednak wartościowe narzędzia wykorzystywane przez techniki opisane w~podrozdziałach poprzedzających.

  Algorytmy opisane w podrozdziałach~\ref{s_kernelization} oraz~\ref{s_ckx} nazywane będą dalej \emph{algorytmami głównymi}.
}
\subsection{Specyfikacja wewnętrzna}
\par{
  Rozdział~\ref{s_internals} obejmuje opis wykorzystanych technologii, narzędzi i bibliotek zewnętrznych, architektury załączonego kodu źródłowego, wybranych pakietów oraz implementacji niektórych algorytmów przedstawionych w~opisie zagadanienia.
}
\subsection{Badania eksperymentalne i analiza wyników}
\par{
  Rozdział~\ref{results} poświęcony jest w~całości prezentacji i analizie wyników badań eksperymentalnych.
}
\subsection{Podsumowanie i kierunki dalszych prac}
\par{
  Rozdział~\ref{summary} zawiera podsumowanie pracy oraz opis napotkanych problemów wraz z~propozycjami usprawnień.
}

  \chapter{Wstęp}\label{Chapter_Wstep}

  \section{Cel}\label{Section_Aim}

  \section{Układ pracy}\label{Section_Layout}

\chapter{Zagadnienie }\label{Chapter_Domain}
  \section{Opis dziedziny problemu}\label{Section_Problematyka}

  \section{Podstawowe definicje}\label{Section_Definicje}

  \section{Techniki przetwarzania wstępnego}\label{Section_preprocessing}

  Przed przystąpieniem do właściwego procesu zawężania dziedziny grafu
  rozpatrywanej w poszukiwaniu pokrywy wierzchołkowej, wykonać można (i~należy)
  zestaw prostych procedur o~niewielkiej złożoności czasowej.
  Procedury te pozwalają na pozbycie się z~grafu wierzchołków trywialnych,
  o~których przynależności do pokrywy decydują proste cechy związane
  z~najbliższym sąsiedztwem danwego wierzchołka.


  Elementy składowe przetwarzania wstępnego:
  \begin{enumerate}
    \item Usunięcie wszystkich wierzchołków izolowanych
      % Are cross-references allowed? E.g. refer here to the definition of an
      % isolated vertex.
    \item Usunięcie wsyzstkich wierzchołków stopnia 1., a~następnie usunięcie
      wszystkich wierzchołków izolowanych.
    \item Usunięcie wszystkich wierzchołków stopnia 2., których sąsiedztwo jest
      połączone, wraz z sąsiedztwem. Następnie, usunięcie wszystkich
      wierzchołków izolowanych.
    \item Dopóki istnieją wierzchołki stopnia 2., dla każdego wierzchołka $u$ 
      stopnia 2. o niepołączonym sąsiedztwie $(w,v)$:
      \begin{itemize}
        \item[-] Każdą krawędź $(v,v_i)$ zastąp krawędzią $(u, v_i)$.
        \item[-] Każdą krawędź $(w,w_i)$ zastąp krawędzią $(u, w_i)$.
        \item[-] Usuń wierzchołki $v$ oraz $w$.
      \end{itemize}
  \end{enumerate}


\chapter{Wybrane algorytmy }\label{Chapter_Algorytmy}


  \chapter{Specyfikacja wewnętrzna}\label{s_internals}
\section{Wykorzystane technologie}\label{ss_internals-technologies}
W przeciągu ostatniej dekady proces tworzenia oprogramowania uległ znaczącym
transformacjom.
Ogromna popularyzacja sektora technologii informacyjnych, którą zawdzięczamy głównie rozwojowi internetu, przyczyniła się do skupienia dużo większej uwagi na kwestiach zarówno sposobów wytwarzania oraz charakteru oprogramowania jak i na narzędziach temu służących.

W środowisku informatyki biznesowej odchodzi się od klasycznych, liniowych procesów tworzenia oprogramowania na rzecz szerokiego grona rozwiązań zaliczanych do grupy tzw. metodyk zwinnych. 
Głównym celem każdej metodyk zwinnych jest zwiększenie ilości dostarczanych produktów (w tym wypadku rozwiązań informatycznych) w danym czasie, zachowując jednocześnie jak najwyższy poziom ich jakości.
Prawidłowo wdrożone metodyki zwinne pozwalają na zacieśnienie pętli komunikacyjnej pomiędzy interesariuszami biznesowymi oraz technicznymi projektów, owocując częstszą wymianą bardziej szczegółowych informacji.
Bezpośrednio przekłada się to na zwiększoną przezroczystość procesu realizacji projektu.
Konsekwencja ta otwiera nowe pole możliwości związanych z planowaniem, podziałem prac, raportowaniem i prognozowaniem postępów prac oraz weryfikacją prognoz przez monitorowanie stanu realizacji za pośrednictwem metryk.

\par{
Zwiększona częstotliwość dostarczania wartości biznesowej wiąże się z potrzebą wynajdowania bardziej nowoczesnych narzędzi oraz praktyk wspierających taki tryb operowania.
Wobec zmieniających się wymagań dziedziny technologii informacyjnych, z biegiem lat na znaczeniu zyskały dawne koncepcje takie jak:

\begin{itemize}
  \item enkapsulacja,
  \item wyższe poziomy abstrakcji kodu,
  \item paradygmat programowania obiektowego,
  \item paradygmat programowania funkcyjnego,
  \item programowanie systemowe,
  \item programowanie współbieżne i asynchroniczne.
\end{itemize}
}

\par{
Wraz z adaptacją tychże pomysłów, opisywanych już w pracach sprzed zgoła czterdziestu laty, równolegle narodziły się i na ich podstawie rozwijane są nowe idee, do których zaliczyć można:
\begin{itemize}
  \item Paradygmat programowania hybrydowego, łączący w sobie elementy funkcyjne zarówno jak i obiektowe. Dzięki zaletom elementów paradygmatu obiektowego, wysoce i zrozumiale dla człowieka zorganizowane dane mogą być przetwarzane za pomocą potężnych szkicy logicznych, złożonych z funkcji za pomocą implementacji pojęć należących do paradygmatu funkcyjnego takich jak monady i kombinatory.
  \item Biblioteki wysokopoziomowe, enkapsulujące funkcjonalności narzędziowe i działania związane mocno z architekturą komputera. Biblioteki wysokopoziomowe zwiększają efektywność inżyniera oprogramowania przez wprowadzenie użytecznych i zrozumiałych idiomów programistycznych, często wywodzących się bezpośrednio z matematyki lub realizujących zestawy niskopoziomowych operacji sięgających języka pośredniego lub kodu assemblerowego jako całość logiczną oraz nazwaną w sposób zrozumiały dla człowieka. Ważną funkcją bibliotek wysokopoziomowych jest również dążenie do maskowania rozbieżności związanych z docelową platformą uruchomieniową kodu, co stanowi znaczące wsparcie dla programisty, pozwalając na pisanie w większości przypadków homogenicznego kodu, który nie jest uzależniony od platformy, na której ma zostać użyty.
  \item Automatyczne zarządzanie pamięcią za pomocą tzw. mechanizmów odśmiecania pamięci (\emph{Garbage Collection}, popularnie skracane do \emph{GC}). Praktyka pokazuje, iż w dużym odsetku systemów przeznaczenia ogólnego, przez które rozumie się większość zastosowań biznesowych i przemysłowych, nie wymaga tak ywsokiej precyzji, jaką oferuje ręczne zarządzanie pamięcią. Biorąc pod uwagę duży nakład pracy programisty potrzebny do prawidłowego ręcznego zarządzania pamięcia, bardziej opłacalne okazuje się poświęcenie ułamka wydajności systemu przez wprowadzenie cyklicznego odśmiecania pamięci w oparciu o pewne reguły rozpoznawania bloków do usunięcia, zwalniając jednocześnie z tego obowiązku inżyniera. Dla systemów zawierających komponenty bardziej wyspecjalizowane, lub wymagające przetwarzania w czasie rzeczywistym, nowoczesne języki oferują podejście mieszane, pozwalając na ręczne zarządzanie pamięcią w krytycznych miejscach aplikacji.
  \item Wprowadzanie mechanizmów synchronizacji wątków oraz prymitywów do zarządzania asynchronicznym lub współbieżnym wykonywaniem kodu jako obywateli pierwszej klasy nowoczesnych języków programowania. Popularnymi obecnie koncepcjami w tej dziedzinie są:
  \begin{itemize}
    \item ̆\emph{funkcje zwrotne} (Callback functions), przekazywane jako punkty powrotu w momencie zakończenia przetwarzania asynchronicznego,
    \item \emph{kanały} (channels), stanowiące abstrakcje w postaci obiektów umożliwiających dwukierunkową transmisję danych dowolnego typu pomiędzy komponentami systemu lub systemami,
    \item \emph{zdarzenia} (events), będące abstrakcją w postaci obiektów niosących informacje o zajściu określonych warunkóœ w asynchronicznym przepływie sterowania aplikacji.
  \end{itemize}
\end{itemize}
}

\par{
  Pojęcie ``narzędzia'' w XXI. wieku znacznie zyskało na pojemności i~nie ogranicza się już wyłącznie aspektów samego języka lub nieodzownych elementów ściśle z~nim związanych jak kompilatory czy linkery.
  Narastający nacisk kładzie się również na rozwijanie tak zwanych \emph{ekosystemów} wokół języków programowania.
  Pojęcie ekosystemu stanowi dość liberalne określenie grupy funkcjonalności oraz aplikacji służących wsparciu programisty przez m.in.:
  \begin{itemize}
    \item automatyzację podstawowych zadań,
    \item analizę a nawet zmiany lub przepisywanie kodu na podstawie badania drzewa składni abstrakcyjnej,
    \item uruchamianie testów jednostkowych oraz sprawnościowych,
    \item tworzenie profili wydajnościowych aplikacji ze względu na zużycie czasu procesora lub pamięci RAM w oparciu o śledzenie przepływu kontroli w aplikacji,
    \item podpowiedzi oraz dopełnianie na bieżąco pisanego kodu.
  \end{itemize}
}

\par{
  Większe zaplecze narzędziowe umożliwia wykorzystywanie grup aplikacji należących do ekosystemu w celu spełnienia założeń leżących u podstaw metodyk zwinnych, skupiających się na jak najczęstym dostarczaniu wartościowych produktów lub komponentów produktu wysokiej jakości, w ramach przejrzystego procesu twórczego.
}

\subsection{Język programowania Go} % (fold)
\label{sss_go}
\par{
Go jest statycznie typowanym, imperatywnym, strukturalnym językiem programowania, którego historia rozpoczęła się w 2007. roku w firmie Google. Autorzy oryginalnej specyfikacji wraz z implementacją to: Rob Pike, Robert Griesemer oraz Ken Thompson.
Najnowsza stabilna wersja języka na dzień pisania niniejszej pracy to 1.3.3.
}
\par{
Składnia Go silnie nawiązuje do języka C --- dokonano jednak wielu modyfikacji skupiających się przede wsyzstkim na jej uproszczeniu, eliminacji możliwości popełniania błędów oraz zwiększeniu zwięzłości.
Dużo uwagi poświęca się również pielęgnacji ekosystemu wokół Go w celu uczynienia go narzędziem jak najłatwiejszym i najbardziej praktycznym w użyciu.
Dla osiągnięcia tych założeń zastosowano wzorce znane zarówno z języków statycznie jak i dynamicznie typowanych.
}
\par{Deklaracja i inicjalizacja zmiennych odbywa się przy pomocy mechanizmu domniemania typów, w większości przypadków zwalniającego programistę z obowiązku jawnego oznaczania typu zmiennych oraz metod. Zamiast zapisu \texttt{int~x~=~0;}, znanego z języka C, stosuje się tu krótszy zapis \texttt{x~:=~0}. Warto również zwrócić uwagę na brak wymagania stawiania średników jako zakończeń wyrażeń.
}
 \par{Mimo możliwości korzystania ze wskaźników, bezpośredni dostęp do nich jest niemożliwy, co zapobiega błędom związanym z niezgodnością typów. W połączeniu ze statycznym typowaniem oznacza to, iż programista nie jest w stanie wprowadzić rozbieżności na poziomie typów zmiennych prowadzących do awarii aplikacji niewykrytych przez kompilator. Konsekwencją zablokowania bezpośredniego dostępu do wskaźników jest również brak możliwości wykonywania na nich działań arytmetycznych.
 }
 \par{
Dzięki mechanizmowi Garbage Collection, język zapobiega wyciekom pamięci wynikającym z nieprawidłowego zarządzania wskaźnikami. W aktualnej wersji wykorzystywana jest współbieżna wersja algorytmu \textit{mark and sweep}.
}
\par{
 W związku z faktem, iż Go jest językiem strukturalnym, brak w nim pojęcia obiektu. Uproszczonym odpowiednikiem jest struktura, definiowana słowem kluczowym \texttt{struct}.
}
  \par{
  Jedną z najbardziej radykalnych decyzji podczas tworzenia specyfikacji języka stanowi rezygnacja ze standardowego mechanizmu dziedziczenia. W miejsce dziedziczenia wirtualnego zastosowano system interfejsów, gdzie dana struktura implementuje określony interfejs wtedy i tylko wtedy gdy wystawia pełen zestaw publicznych metod zgodnych z jego deklaracją. Odpowiednikiem dziedziczenia klasycznego w Go jest osadzanie typów.
  Przykład osadzenia typów \texttt{Reader} i \texttt{Writer} w nowo utworzonym interfejsie \texttt{ReaderWriter}.
  \begin{lstlisting}
    type ReaderWriter interface {
      Reader
      Writer
    }
  \end{lstlisting}
  Struktury implementujące interfejs \texttt{ReaderWriter} implementują również interfejsy osadzone.
  Osadzanie typów w strukturach wygląda podobnie, należy jednak oznaczyć je symbolem wskaźnika.
  \begin{lstlisting}
    type ReaderWriter struct {
      *Reader
      *Writer
    }
  \end{lstlisting}
  Kluczowa różnica pomiędzy dziedziczeniem a osadzaniem typów polega na właściwości, iż metody typu osadzanego zostają włączone do typu zewnętrznego~---~jednak podczas wywołania danej metody, jej odbiorcą jest instancja typu osadzonego. \cite{godoc:embedding}
}
\par {
W celu uproszczenia składni głównie względem C++, argumentowanym przez jednego z autorów w notatce \cite{Pike:LessIsMore}, wyłączono ze specyfikacji wiele funkcji oferowanych przez podobne języki, poza wymienionymi wcześniej.
  \begin{itemize}
    \item Przeciążanie metod i operatorów,
    \item cykliczne zależności pomiędzy pakietami,
    \item asercje,
    \item programowanie generyczne.
  \end{itemize}
}
\par{
  Dla wygody programisty wprowadzono do Go zestaw podstawowych typów, wyrażających elementy brakujące zdaniem autorów w czystym C.
  \begin{itemize}
    \item \emph{Plastry} (slices), zapisywane jako \texttt{[]typ}, wskazują na tablicę obiektów przechowywanych w pamięci, przechowując wskaźnik do początku danego plastra, jego długość oraz \emph{pojemność}, określającą liczebność elementów plastra, która wymagać alokacji dodatkowej pamięci w celu rozszerzenia odpowiadającej tablicy.
    \item Niezmienne ciągi znaków (typ \texttt{string}), zawierające przeważnie tekst w kodowaniu UTF-8. Mogą jednak przechowywać dowolne bajty.
    \item Tablica haszująca, zapisywana jako \texttt{map[typ\_klucza]typ\_wartości}.
  \end{itemize}
}
\par{
  Go jest językiem kompilowanym do kodu bajtowego. Istnieją dwa oficjalnie wspierane zestawy kompilatorów:
  \begin{itemize}
    \item \texttt{gc} wraz z narzędziami dla architektur \texttt{amd64} oraz \texttt{i386}, posiadający wydajny optymalizator, oraz dla architektury \texttt{ARM},
    \item \texttt{gccgo}, będący nakładką na \texttt{gcc}.
  \end{itemize}
  Każdy z zestawów wspiera platformy DragonFly BSD, FreeBSD, Linux, NetBSD, OpenBSD, OS X (Darwin), Plan 9, Solaris oraz Windows. Wyjątkiem jest kompilator \texttt{gc} na architekturę \texttt{ARM}, wspierający wyłącznie platformy Linux, FreeBSD oraz NetBSD. \cite{godoc:compilers}
  Łańcuch narzędzi związany z procesem kompilacji tworzy statycznie linkowane, natywne binarne pliki wykonywalne bez zewnętrznych zależności.
}

\par{
  Ostatnim elementem, o którym należy wspomnieć, są wbudowane mechanizmy służące do programowania współbieżnego.
}
\subsection{Pakiet GLPK} % (fold)
\label{ss_internals_glpk}
\par{
  Pakiet GLPK (GNU Linear Programming Kit) jest biblioteką wysokopoziomową przeznaczoną do rozwiązywania problemów dużej skali z~dziedziny programowania liniowego, częściowo całkowitoliczbowego i~pokrewnych.
  Stanowi on część projektu GNU i~udostępniany jest w~ramach licencji GNU GPLv3.
  Na całość pakietu składa się zestaw funkcji napisanych w~języku ANSI C, zorganizowanych w~wywoływalną bibliotekę.
  Korzystanie z~pakietu może być realizowane w~dwojaki sposób.
  \begin{enumerate}
    \item Problemy modelować można za pomocą języka GNU MathProg --- tak przygotowane modele rozwiązuje się za pomocą narzędzia \texttt{glpsol}.
    \item Definicja problemów może odbywać się bezpośrednio w~kodzie aplikacji przez opisanie jej konstrukcjami udostępnianymi przez GLPK jako biblioteki języka C. Metoda ta jest znacznie bardziej elastyczna, gdyż nie mamy tu do czynienia ze statycznym modelem, lecz można go budować i~modyfikować z~wykorzystaniem wszelkich narzędzi udostępnianych przez język programowania.
  \end{enumerate}
}
\par{
  Pakiet GLPK korzysta z~ulepszonej metody simpleks oraz metody punktów wewnętrznych do rozwiązywania problemów programowania liniowego oraz algorytmu drzewa poszukiwań z~ograniczeniami wraz z~algorytmem Gomory'ego dla problemów programowania częściowo całkowitoliczbowego.
}
% subsection pakiet_glpk_ (end)
\subsection{Narzędzia pomocnicze}\{ss_internals_misc}
\label{ss_technologies_misc}
\subsubsection{\textbf{Pakiet Graphviz}}
\label{sss_technologies_misc_graphviz}
\par{
  Pakiet Graphviz (skrót od \emph{Graph Visualization Software}) jest zbiorem narzędzi,których rozwój zapoczątkowała firma AT\&T Labs Research, służących do rysowania grafów oraz udostępniających funkcje pakietu w ramach bibliotek dla aplikacji.
  Pakiet opublikowany jest w ramach licencji EPL.
  Wizualizacja grafów odbywa się na podstawie modeli w języku skryptowym DOT.
  W zakres możliwości pakietu wchodzą również przydatne funkcje wyróżniania elementów prezentowanego grafu przez kolorowanie węzłów lub krawędzi, możliwość tabelarycznego rozmieszczania węzłów oraz definiowanie własnych stylów krawędzi czy też wstawianie hiperłącz oraz dowolnych kształtów.
  Narzędzia wchodzące w skład pakietu:
  \begin{itemize}
    \item \texttt{dot} --- służy do tworzenia hierarchicznych obrazów grafów skierowanych,
    \item \texttt{neato} --- służy do tworzenia obrazów grafów nieskierowanych o niewielkiej liczbie węzłów (do ok. 100) za pomocą algorytmu minimalizującego globalną funkcję energii grafu,
    \item \texttt{fdp} --- podobnie jak \texttt{neato}, jednak funkcja rozmieszczająca działa w oparciu o redukcję sił połączeń w grafie,
    \item \texttt{sfdp} --- zmodyfikowana wersja \texttt{fdp}, przystosowana do obrazowania grafów o wysokiej liczbie węzłów,
    \item \texttt{twopi} oraz \texttt{circo} --- służą do obrazowania grafów w układzie kołowym,
    \item \texttt{dotty} --- stanowi graficzny interfejs użytkownika do wizualizacji oraz edycji grafów,
    \item \texttt{lefty} --- programowalny graficzny element kontrolujący wyswietlający grafy odczytane z pliku DOT i umożliwiający użytkownikowi wykonywanie na nich operacji za pomocą myszy.
  \end{itemize}
}

% subsection go (end)

\section{Szczegóły techniczne implementacji}\label{s_internals_implementation}
\subsection{Architektura aplikacji}\label{ss_internals_architecture}
\par{
  Zgodnie z kanonem podyktowanym przez specyfikację języka Go, kod aplikacji podzielony został na \emph{pakiety}.
  Należy zaznaczyć, iż pakiety w rozumienia języka Go są czymś innym niż pakiety opisywane w podrozdziale~\ref{ss_internals_misc}.
  W kontekście obecnego rozdziału przez pakiety rozumie się logiczne jednostki organizacji kodu w języku Go, pakiety omawiane w podrozdziale~\ref{ss_internals_misc} stanowią zewnętrzne biblioteki, z których korzystano w ramach prac implementacyjnych.
  Projektując strukturę kodu, nacisk położony został na logiczne rozdzielenie części programu związanych z poszczególnymi aspektami domeny problemu.
  W związku z tym zdefiniowano następujące pakiety:
  \begin{itemize}
    \item \texttt{graph} --- pakiet zawierający strukturę reprezentującą graf wraz ze zbiorem metod i funkcji do odczytywania związanych z nim danych, tworzenia oraz modyfikacji jego konstrukcji;
    \item \texttt{preprocessing} --- pakiet zawierający zestaw funkcji realizujących zadanie przetwarzania wstępnego grafu, realizując koncepcje opisane w podrozdziale~\ref{Section_preprocessing};
    \item \texttt{kernelization} --- pakiet zawierający implementacje koncepcji związanych z redukcją dziedziny do jądra problemu pokrycia wierzchołkowego, opisanych w podrozdziale~\ref{s_kernelization};
    \item \texttt{vc} --- pakiet zawierający implementacje algorytmów rozwiązujących problem pokrycia wierzchołkowego: podstawowego algorytmu rekurencyjnego opisanego pseudokodem~\ref{alg_VC1} w podrozdziale~\ref{s_vertex_cover_domain} oraz implementacji własnej metody opartej na drzewie poszukiwań z ograniczeniami, opisanej w podrozdziale~\ref{ss_internals_bnb};
    \item \texttt{matching} --- pakiet zawierający funkcje związane z poszukiwaniem skojarzeń w grafie, opisywane w podrozdziale~\ref{s_supplementary_algorithms}
    \item \texttt{containers} --- pakiet zawierający implementacje wykorzystywanych w pracy struktur danych nie udostępnionych przez biblioteki standardowe języka Go takich jak kolejka priorytetowa i stos;
    \item \texttt{graphviz} --- pakiet stanowiącą ograniczoną do minimum w kontekście oferowanych funkcji bibliotekę opakowującą pakiet Graphviz, opisywany w podrozdziale~\ref{sss_technologies_misc_graphviz};
    \item \texttt{utility} --- pakiet zawierający funkcje pomocnicze, niezwiązane konkretnie z żadnym aspektem domeny pracy, przydatne jednak podczas implementacji opisywanych koncepcji, jak na przykład wypisywanie dowolnych wartości do standardowego wyjścia w celu ułatwienia analizy wykonywania programu;
    \item \texttt{main} --- pakiet stanowiący punkt wejściowy programu. Tutaj znajdują się testy logiczne oraz wydajnościowe stanowiące źródło danych do badań eksperymentalnych przeprowadzonych w rozdziale~\ref{s_experimental_study}.
  \end{itemize}
}
\par{
  Organizacja kodu zorientowana na pakiety wykazuje kilka kluczowych zalet, z których główną jest luźne powiązanie komponentów odpowiadających za poszczególne aspekty domeny problemu oraz funkcje potrzebne do ich obsługi.
  Dzięki temu o wiele łatwiej jest pisać testy jednostkowe, izolując poszczególne części logiki w celu weryfikacji poprawności ich działania. 
  Bezpośrednio przekłada się to na rozszerzalność kodu dla osób wykorzystujących lub rozwijających go w przyszłości celem podjęcia nowych wyzwań z zakresu opisywanej w niniejszej pracy problematyki.
  Kod otoczony siecią testów jednostkowych jest wytrzymały --- można poddawać go dowolnym modyfikacjom bez obaw o utracenie poprawności działania poszczególnych jego elementów.
  Dodatkową zaletą płynącą z tego faktu jest zwiększona przejrzystość wynikająca z podziału całości logiki na jasno określone części o dużej czytelności oraz wartości.
  Testy jednostkowe pełnią również rolę dokumentacji i przykładów zastosowania funkcji, a także oczekiwanych względem testowych danych wejściowych wyników.
}
\subsection{Opis wybranych pakietów}\label{ss_internals_important_packages}
\subsubsection{Pakiet \texttt{graph}}
\par{
  W obliczu charakterystyki pracy to znaczy podjęcia się analizy i implementacji szerokiej gamy technik związanych z parametryzacją złożoności obliczeniowej pokrycia wierzchołkowego, kwestią wymagającą uwagi jest sposób reprezentacji grafu.
  W związku ze zróżnicowaniem podejść do problemu, reprezentacja struktur grafowych za pomocą wyłącznie macierzy incydencji lub też listy incydencji okazuje się nie tyle niewystarczająca, co nieoptymalna.
  Podczas gdy niektóre z technik operują na grafie przede wszystkim przy pomocy wierzchołków (głównie metody oparte na sformułowaniu problemu jako instancji przepływu w sieci), inne zdecydowanie traktują graf jako zbiór krawędzi pomiędzy pewnymi punktami.
  Jako iż jednym z pomniejszych celów pracy jest stworzenie ogólnej platformy użytecznej do testowania rozwiązań różnych problemów związanych z grafami, zdecydowano się zaimplementować strukturę grafu jako reprezentację zawierającą zarówno macierz incydencji jak i listę krawędzi.
}
\par{
  \begin{lstlisting}[language=go, caption=Typy reprezentujące wierzchołki grafu.]
  type Vertex int
  type Vertices []Vertex
  \end{lstlisting}
  W celu uniknięcia potrzeby uważania tablicy haszujących, gdzie klucze stanowią wierzchołki, zdecydowano się na wprowadzenie metody \texttt{Vertex.ToInt() int}, konwertującej dowolny wierzchołek do odpowiadającej mu wartości całkowitoliczbowej.
  Zabieg ten pozwala na zastosowanie zwykłych tablic w miejsce tablic haszujących, co pozwala na obniżenie złożoności obliczeniowej związanych z takimi strukturami operacji.

  Wartości wierzchołków często występują obok liczb całkowitych, pełniących zazwyczaj w tym kontekście rolę indeksów tablic.
  Aby łatwiej było odróżnić instancje wierzchołków od zwykłych liczb całkowitych, przyjęto konwencję numeracji wartości wierzchołków grafu od 1.
  \begin{lstlisting}[language=go, caption=Typy reprezentujące krawędzie grafu.]
  type Edge struct {
    From      Vertex
    To        Vertex
    isDeleted bool
  }

  type Edges []*Edge
  \end{lstlisting}

  Krawędzie grafu nie są reprezentowane idiomatycznie, to znaczy jako pary wierzchołków stanowiących ich zakończenia.
  Wynika to z praktycznej potrzeby związanej ze zwijaniem krawędzi w ramach operacji przetwarzania wstępnego (podrozdział~\ref{Section_preprocessing}), działania algorytmu Edmondsa (podrozdział~\ref{ss_edmonds_blossom}) oraz algorytmu Chen, Kanj, Xia (podrozdział~\ref{s_chen_kanj_xia}).
  Praktyka pokazuje, iż znacznie łatwiejsze i bardziej wydajne jest oznaczanie krawędzi grafu jako usuniętych zamiast właściwego ich usuwania --- głównie ze względu na fakt, iż w większości przypadków zostają one w późniejszym momencie przywrócone do grafu na przykład w celu określenia pełnej postaci odnalezionej ścieżki powiększającej.

  \begin{lstlisting}[language=go, caption=Struktura reprezentująca graf.]
  type Graph struct {
    Vertices                Vertices
    Edges                   Edges
    CurrentVertexIndex      int
    IsVertexDeleted         []bool
    degrees                 []int
    neighbors               [][]*Edge
    numberOfVertices        int
    numberOfEdges           int
    isRegular               bool
    needToComputeRegularity bool
  }
  \end{lstlisting}

  Podczas implementacji rozwiązań opisywanych w niniejszej pracy problemów napotkano na trzy podstawowe sposoby poruszania się po grafie.
  Każdy z tych sposobów znajduje odzwierciedlenie w implementacji struktury --- istnieje metoda realizująca dany sposób przy zachowaniu najmniejszej możliwej złożoności dla grafu $G=(V, E)$: $O(1)$, $O(|V|)$, lub $O(|E|)$.
  \\\\\underline{Iteracja po wszystkich wierzchołkach}\\
  \par{
      Metoda \texttt{ForAllVertices(action func(v Vertex, chan<- done bool))} iteruje po kolekcji \texttt{Vertices}, zapewniając  podstawową złożoność operacji na poziomie $O(|V|)$, z wyłączeniem złożoności operacji realizowanych przez funkcję \texttt{action}. Kanał \texttt{done} służy do informowania o potrzebie przerwania iteracji.\\
      Metody \texttt{ForAllVerticesOfDegree(degree int, action func(v Vertex, chan<- done bool))} oraz \texttt{ForAllVerticesOfDegreeGeq(degree int, action func(v Vertex, chan<- done bool))} korzystają z metody \texttt{ForAllVertices}, nakładając dodatkowe ograniczenia związane ze stopniem wierzchołków, które mają zostać uwzględnione w iteracjach.
  }
  \\\\\underline{Iteracja po wszystkich krawędziach}\\\
  \par{
      Analogicznie jak w przypadku wierzchołków, metoda \texttt{ForAllEdges(action func(edge *Edge, done chan<- bool))} iteruje po kolekcji \texttt{Edges}, zachowując złożoność operacji na poziomie $O(E)$ z wyłączeniem złożoności funkcji \textt{action}.
  }
  \\\\\underline{Uzyskiwanie dostępu do krawędzi pomiędzy danymi wierzchołkami}\\
  \par{
      Metoda \texttt{GetEdgeByCoordinates(from, to int) *Edge} korzysta z macierzy incydencji \texttt{neighbors}, pobierając z niej krawędź łączącą wierzchołki, których wartości skonwertowane do liczb całkowitych odpowiadają podanym współrzędnym.

      Przy okazji omawiania tejże metody należy zwrócić uwagę na następujące dwie kwestie.
      \begin{enumerate}
        \item Macierz incydencji \texttt{neighbors} zamiast wartości logicznych typu \texttt{bool} zawiera wskaźniki na krawędzie istniejące w kolekcji \texttt{Edges}.
        Rozwiązanie to podyktowane jest znów praktyką --- w zdecydowanej większości przypadków fakt istnienia danej krawędzi wiązał się z potrzebą wykonania działań z nią związanych, gdzie przechowywanie wskaźników na krawędzie pozwoliło na uniknięcie dodatkowych jawnych odwołań do kolekcji \texttt{Edges}.
        \item W związku z faktem, iż w ramach niniejszej pracy podjęto się analizy rozwiązań wersji problemów dotyczących grafu o nieskierowanych krawędziach, macierz \texttt{neighbors} zawiera wskaźnik do tej samej krawędzi zarówno dla współrzędnych wprost $(x, y)$, jak i zestawu transponowanego $(y, x)$.
        Rozwiązanie takie zostało przyjęte, mimo pewnego narzutu zajętości pamięci, za pożądane głównie ze względu na spójność właściwej struktury grafu z jego logiczną reprezentacją.
        Dodatkowo, zachowanie takiego rozłożenia wskaźników pomaga w bardziej zrozumiały sposób posługiwać się strukturą sieci przepływowej.
      \end{enumerate}
  }
  \\\\\underline{Operacje wykonywane na zbiorze wierzchołków sąsiednich}\\
  \par{
    Bardzo szeroki zakres działań implementowanych algorytmów w grafie $G=(V,E)$ opiera się na operacjach wykonywanych na sąsiedztwie wierzchołka $N(v); v \in V$.
    Zdecydowanie najczęściej realizowana jest iteracja po wierzchołkach sąsiednich, na drugim miejscu znajdują się operacje wymagające uzyskania sąsiedztwa jako kolekcji lub zbioru w rozumieniu struktury danych.
    W tym celu bardzo pomocna okazuje się macierz incydencji.
    Dla pierwszego przypadku została zaimplementowana metoda \texttt{ForAllNeigbors(v Vertex, action func(edge *Edge, done chan<- bool))}, oparta na rozumowaniu analogicznym do tego, na którym oparto metodę \texttt{ForAllEdges}, jednak iterująca po danym rzędzie macierzy \texttt{neighbors}, indeksowanym przekształconą do liczby całkowitej wartością wierzchołka $v$ z ograniczeniem uwzględniania w iteracjach wyłącznie istniejących wskaźników (to znaczy \textt{edge != nil}).

    Drugi przypadek obsługiwany jest przez metody \texttt{GetNeighbors(v Vertex) Neighbors} oraz \texttt{GetNeighborsWithSet(v Vertex) (Neighbors, mapset.Set)}.

    Metoda \texttt{GetNeighbors} zwraca odpowiedni rząd macierzy incydencji opakowany w typ pomocniczy, natomiast metoda \texttt{GetNeighborsWithSet} dodatkowo w ramach wewnętrznej iteracji konstruuje zbiór wierzchołków (w rozumieniu struktury danych) stanowiących sąsiedztwo.
    Wszystkie opisane metody związane z operacjami na sąsiedztwie wierzchołka $v$ zachowują złożoność $O(|N(v)|)$ z wyłączeniem złożoności funkcji \texttt{action}.
  }
}
  \chapter{Badania eksperymentalne}
\label{results}
\section{Dane testowe}
\par{
  Przygotowanie zestawu danych testowych jasno odzwierciedlającego ograniczenie złożoności obliczeniowej rozwiązania problemu pokrycia wierzchołkowego do wielomianowej przez zastosowanie technik redukcji dziedziny do jądra problemu nie jest łatwe.
  Opisywane techniki bardzo silnie zależą od charakterystycznych cech struktury grafu, które trudno jest opisać w prosty sposób parametrami generacji losowej.
  Dlatego też aby przygotować zestaw losowych grafów testowych przyjęto metodę generacji opartą na rozkładzie losowym stopni wierzchołków grafu w~połączeniu z~funkcją określającą współczynnik selektywności.
  Stopień każdego wierzchołka w~generowanym grafie wybierany jest losowo z~przedziału $\left<1, k\right>$ zgodnie z~rozkładem $P(k) \propto \frac{1}{k}$.
  Uzyskanie zbioru grafów losowych o~różnych parametrach owocowało bardzo niespójnymi wynikami czasów ich przetwarzania z~pomocą opisywanych w~niniejszej pracy technik, co spowodowane jest tym, że istnienie struktur redukowalnych w~grafie nie jest uzależnione wprost od liczebności wierzchołków czy też krawędzi.
  Po ,,wypróbowaniu'' kilku postaci funkcji określającej współczynnik selektywności udało się utworzyć zbiór zapewniający zbliżone do oczekiwanych pod względem przyrostu czasu przetwarzania wyniki --- przyjęta została postać $P(i, k) \propto \frac{1}{1+|i-k|}$, gdzie $i$ stanowi stopień rozpatrywanego wierzchołka.
  Następnie, w celu uściślenia monotoniczności wyników, metodą prób i~błędów dobrano wartość parametru $k=5$.
  Dla tak wybranych parametrów wygenerowano przedstawiony w~Tabeli~\ref{tab_testdata} zbiór grafów $G=(V, E)$, mających pokrycia wierzchołkowe $C$.\\
  \begin{table}
    \begin{center}
    \caption{Wartości opisujące grafy stanowiące zbiór danych testowych.}
    \begin{tabular}{| c | c | c | c |}
      \hline
      l.p. & $|V|$ & $|E|$ & $|C|$ \\ \hline
      1 & 100 & 115 & 47 \\
      2 & 200 & 246 & 96 \\
      3 & 300 & 367 & 152 \\
      4 & 400 & 508 & 199 \\
      5 & 500 & 589 & 241 \\
      6 & 600 & 744 & 290 \\
      7 & 700 & 860 & 348 \\
      8 & 800 & 964 & 396 \\
      9 & 900 & 1113 &  445 \\
      10 & 1000 &  1218 &  501 \\ \hline
    \end{tabular} 
    \begin{tabular}{| c | c | c | c |}
      \hline
      l.p. & $|V|$ & $|E|$ & $|C|$ \\ \hline
      11 & 1100 & 1360 &  543 \\
      12 & 1200 & 1446 &  590 \\
      13 & 1300 & 1571 &  651 \\
      14 & 1400 & 1717 &  693 \\
      15 & 1500 & 1758 &  739 \\
      16 & 1600 & 2010 &  804 \\
      17 & 1700 & 2079 & 1018 \\
      18 & 1800 & 2219 & 1095 \\
      19 & 1900 & 2326 & 1119 \\
      20 & 2000 & 2463 & 1163 \\ \hline
    \end{tabular}
    \end{center}
    \label{tab_testdata}
  \end{table}
}
\section{Porównanie i~analiza szybkości działania opisanych algorytmów}
\par{
  W celu zbadania efektywności zaimplementowanych algorytmów w~redukcji czasu rozwiązania problemu pokrycia wierzchołkowego wykonano serię testów, podczas których mierzono czas wykonywania następujących operacji:

  \begin{itemize}
    \item Wyznaczenie pokrycia wierzchołkowego metodą siłową, opisaną Pseudokodem~\ref{alg_VC1} dla zestawu grafów o~pomniejszonych rozmiarach.
    \item Wyznaczenie pokrycia wierzchołkowego algorytmem własnym, opisanym Pseudokodem~\ref{alg_VC2}.
    \item Wyznaczenie pokrycia wierzchołkowego algorytmem własnym po redukcji koron.
    \item Wyznaczenie pokrycia wierzchołkowego algorytmem własnym po redukcji dziedziny opartej na algorytmie przepływu w~sieci.
    \item Wyznaczenie pokrycia wierzchołkowego algorytmem własnym po wykonaniu operacji przetwarzania wstępnego i~redukcji koron.
    \item Wyznaczenie pokrycia wierzchołkowego algorytmem własnym po wykonaniu operacji przetwarzania wstępnego i~redukcji dziedziny opartej na algorytmie przepływu w~sieci.
    \item Redukcja dziedziny do jądra problemu przez usunięcie koron.
    \item Redukcja dziedziny do jądra problemu oparta na algorytmie przepływu w~sieci.
  \end{itemize}
}
\subsection{Wyznaczanie pokrycia wierzchołkowego}
\par{
  Zgodnie z~oczekiwaniami, wyznaczenie pokrycia wierzchołkowego metodą siłową, opisaną Pseudokodem~\ref{alg_VC1}, wymagało wykładniczego czasu.
  Powoduje to, że analiza grafów o~rozmiarach przedstawionych w~Tabeli~\ref{tab_testdata} trwa bardzo długo, a wyniki pomiarów związanych z~metodą siłową zaciemniają obraz wyników pozostałych testów.
  W związku z~tym pomiar czasu działania metody siłowej został zrealizowany dla osobnego zestawu siedmiu grafów wygenerowanych opisaną metodą o rozmiarach od dziesięciu do trzydziestu wierzchołków z~krokiem pięciu wierzchołków.
  Niestety, struktura jednego z~wygenerowanych grafów okazała się niekorzystna dla rozgałęzień algorytmu siłowego --- wyniki pomiarów czasu działania siłowej metody wyznaczania pokrycia wierzchołkowego zawiera Tabela~\ref{tab_vc_naive}.
  \begin{table}
    \begin{center}
      \caption{Czas wyznaczania pokrycia wierzchołkowego metodą siłową.}
      \begin{tabular}{| c | c | c | c |}
        \hline
        l.p. & $|V|$ & $|E|$ & czas \\ \hline
        1 & 10 & 8 & 323.884$ \mu$s \\
        2 & 15 & 20 & 9.52204 ms \\
        3 & 20 & 20 & 30.605758 ms \\
        4 & 25 & 33 & 614.454963 ms \\
        5 & 30 & 36 & 28.922161504 s \\
        6 & 35 & 35 & 1 m 27.328533967 s \\
        7 & 40 & 69 & \textit{wyłączono po 2 godzinach} \\ \hline
      \end{tabular} 
    \end{center}
    \label{tab_vc_naive}
  \end{table}
}
\subsubsection{\textbf{Czas wyznaczania pokrycia wierzchołkowego bez przetwarzania wstępnego}}\label{time_vc}
\par{
  Porównanie czasów wyznaczania pokrycia wierzchołkowego bez uprzedniego zastosowania technik przetwarzania wstępnego przedstawia Rysunek~\ref{fig_results_vc}.
  \begin{figure}
    \centering
      \includegraphics[width=\textwidth]{results-vc}
    \caption{Czas wyznaczania pokrycia wierzchołkowego bez przetwarzania wstępnego.}
    \label{fig_results_vc}
  \end{figure}
  Wyraźnie widoczny jest pozytywny wpływ zastosowania redukcji dziedziny do jądra problemu przez rozwiązanie przeformułowania problemu do egzemplarza problemu przepływu w~sieci.
  Za prawdopodobne uzasadnienie tego, że redukcja koron nie oferuje tak dużego przyspieszenia jak technika oparta na przepływie w~sieci uznano dwa czynniki:
  \begin{itemize}
    \item Grafy stanowiące dane testowe są ubogie w~korony.
    Wynika to z~postaci funkcji określającej współczynnik selektywności wierzchołków. Struktury koron wymagają bardzo specyficznych relacji zachodzących między zbiorami wierzchołków, mianowicie istnienia zbiorów niezależnych o~połączonym sąsiedztwie.
    W ramach procesu przygotowywania danych testowych udało się uzyskać grafy, dla których redukcja dziedziny uzyskana przez usunięcie koron była znacznie większa, jednak grafy te nie pozwalały na uzyskanie wyników dających przedstawić się w postaci funkcji monotonicznej.
    \item Zaimplementowana technika wyznaczania koron w~grafie oparta jest na zbyt restrykcyjnym uściśleniu NT--redukcji do serii operacji działających bezpośrednio na konkretnych egzemplarzach skojarzeń.
    Najprawdopodobniej stanowi to również powód opisanych w~Podsumowaniu problemów napotkanych przy implementacji algorytmu Chena, Kanji oraz Xia.
    Pozostałe zaimplementowane egzemplarze NT--redukcji wyznaczają podzbiory różniące się od tych uzyskiwanych przez redukcję proponowaną w pracy~\cite{KernelizationAlgorithms04}.
    Dalsze pole do badań może stanowić weryfikacja lub bardziej ogólne postacie NT--redukcji --- wykonywane za pomocą algorytmów programowania liniowego --- zapewnią lepsze rezultaty dla grafów o~strukturze zbliżonej do tych, które wykorzystano jako dane testowe.
  \end{itemize}

  Ze względu na brak optymalizacji opisywanych w~niniejszej pracy technik nie udało się wykonać pomiarów czasu wyznaczania pokrycia wierzchołkowego bez przetwarzania wstępnego dla grafów zawierających więcej niż 1600 wierzchołków.
  Pomimo nieco gorszych od oczekiwanych rezultatów należy zauważyć, że każda z~przedstawionych charakterystyk jest o~wiele rzędów wielkości lepsza niż charakterystyka uzyskana z~pomiarów czasu działania metody siłowej.
}
\subsubsection{\textbf{Czas wyznaczania pokrycia wierzchołkowego po przetwarzaniu wstępnym}}

  Porównanie czasów wyznaczania pokrycia wierzchołkowego po uprzednim zastosowaniu technik przetwarzania wstępnego przedstawia Rysunek~\ref{fig_results_vc_p}.
  \begin{figure}
    \centering
      \includegraphics[width=\textwidth]{results-vc-p}
      \caption{Czas wyznaczania pokrycia wierzchołkowego po przetwarzaniu wstępnym.}
    \label{fig_results_vc_p}
  \end{figure}
  Zastosowanie prostych technik przetwarzania wstępnego opisanych w~podrozdziale~\ref{Section_preprocessing} pozwala na dodatkowe skrócenie czasu wyznaczania pokrycia wierzchołkowego w~grafach testowych o~około dwa rzędy wielkości.

  Ponieważ różnica w~czasie wyznaczania pokrycia wierzchołkowego z~zastosowaniem techniki opartej na~przepływie w~sieci między wariantem bez przetwarzania wstępnego a~wariantem z~zastosowaniem przetwarzania wstępnego jest słabo widoczna, Rysunek~\ref{fig_nf_comp} zestawia charakterystyki czasowe tych przypadków.
  \begin{figure}
    \centering
      \includegraphics[width=\textwidth]{nf-comp}
    \caption{Wpływ przetwarzania wstępnego na czas wyznaczania pokrycia wierzchołkowego w~oparciu o~przepływ w~sieci.}
    \label{fig_nf_comp}
  \end{figure}

\subsection{Algorytmy przetwarzania wstępnego i~redukcji dziedziny do jądra problemu}
  Redukcja dziedziny do jądra problemu w~praktyce może mieć sens jedynie wtedy, gdy czas potrzebny wykonanie operacji zawężających przestrzeń poszukiwań pokrycia wierzchołkowego jest wielomianowy.
  W celu weryfikacji szybkości działania implementacji opisywanych technik redukcji dziedziny oraz przetwarzania wstępnego zmierzono czas potrzebny na wykonanie tych operacji dla zbioru grafów testowych. 

\subsubsection{\textbf{Algorytmy redukcji dziedziny do jądra problemu}}
  Porównanie czasów wyznaczania pokrycia wierzchołkowego bez uprzedniego zastosowania technik przetwarzania wstępnego przedstawia Rysunek~\ref{fig_results_k}.

  \begin{figure}
    \centering
      \includegraphics[width=\textwidth]{results-k}
      \caption{Czas redukcji dziedziny bez przetwarzania wstępnego.}
    \label{fig_results_k}
  \end{figure}

  Powodem niekorzystnej złożoności czasowej działania algorytmu redukcji dziedziny przez rozwiązanie przeformułowania problemu do egzemplarza problemu przepływu w~sieci jest brak optymalizacji implementacji koncepcji opisanej w~pracy~\cite{KernelizationAlgorithms04}.
  Szczegóły oraz proponowane usprawnienia opisane są w Podsumowaniu.
  Ponieważ charakterystyka wyników pomiarów czasu działania tego algorytmu jest zdeformowana i~nie ukazuje prawdziwej jego szybkości, a także ze względu na bardzo długi czas oczekiwania, ograniczono dziedzinę przetwarzanych przez niego grafów do zawierających najmniejszą liczbę wierzchołków.
  W związku z~zaciemnieniem na wykresie wartości wyników pomiarów czasu działania algorytmu redukcji koron, Rysunek~\ref{fig_results_k_crown} przedstawia wyłącznie tę charakterystykę.

  \begin{figure}
    \centering
      \includegraphics[width=\textwidth]{results-k-crown}
    \caption{Czas redukcji dziedziny przez wyznaczenie i~usunięcie koron bez przetwarzania wstępnego.}
    \label{fig_results_k_crown}
  \end{figure}

  Rysunek~\ref{fig_results_k_crown_p} przedstawia wpływ uprzedniego zastosowania algorytmów przetwarzania wstępnego na szybkość działania algorytmu redukcji koron.
  \begin{figure}
    \centering
      \includegraphics[width=\textwidth]{results-k-crown-p}
    \caption{Wpływ przetwarzania wstępnego na szybkość działania algorytmu redukcji koron.}
    \label{fig_results_k_crown_p}
  \end{figure}

  Zgodnie z~oczekiwaniami, zastosowanie procedur przetwarzania wstępnego przed przystąpieniem do wykonania algorytmu redukcji dziedziny do jądra problemu wpływa pozytywnie na szybkość jego działania.
  Jest to spowodowane zmniejszeniem liczby wierzchołków i~krawędzi grafu przed rozpocząciem algorytmu kernelizacji.

  Rysunek~\ref{fig_results_p} przedstawia charakterystykę czasu przetwarzania wstępnego w~zależności od liczby wierzchołków grafu.
  Pokazuje to, że operacja przetwarzania wstępnego jest wykonywana szybko nawet dla grafów o~dużej liczbie wierzchołków. Mając na uwadze przedstawione na poprzedzających wykresach korzyści płynące z~jej zastosowania nasuwa się wniosek, że warto tę operację stosować zawsze przed rozpoczęciem rozwiązywania problemu pokrycia wierzchołkowego, bez względu na zastosowane w~dalszych etapach metody redukcji dziedziny do jądra problemu oraz algorytmy wyznaczające pokrycie wierzchołkowe.
  \begin{figure}
    \label{fig_results_p}
    \centering
      \includegraphics[width=\textwidth]{results-p}
    \caption{Czas przetwarzania wstępnego.}
  \end{figure}
\section{Środowisko testowe i wykonywanie testów}
\par{
  W celu wykonania testów należy skopiować katalog zawarty na nośniku dołączonym do pracy, a następnie zdefiniować zmienną środowiskową \texttt{\$GOPATH} tak, by wskazywała na ścieżkę do podkatalogu \texttt{src}.
  W następnej kolejności należy zmienić bieżącą ścieżkę na podkatalog \texttt{src/main}.
  Do wykonania pomiarów służy polecenie \texttt{go run *.go -measure N}, gdzie \texttt{N} stanowi wyrażenie regularne zawierające część nazwy przypadku testowego.
  Wyczerpująca lista zdefiniowanych przypadków testowych zawarta jest w~pliku \texttt{test-cases.go}.

  Przedstawione wyniki testów szybkości działania algorytmów uzyskano na komputerze o nastepujących paramterach:
  \begin{itemize}
    \item Procesor Intel® Core™ i7-4800MQ (4 $\times$ 2.7 GHz)
    \item 32 GB DDR3L 1600 MHz
  \end{itemize}


  W celu realizacji testów jednostkowych należy zmienić bieżącą ścieżkę na katalog wybranego pakietu projektu i~wywołać komendę \texttt{go test}.
}
\subsection*{Generacja danych testowych}
\par{
  Istnieje możliwość wykonania testów na innych niż przedstawione w niniejszym rozdziale danych.
  Generację grafów losowych realizuje skrypt \texttt{random-graphs.py}, umieszczony w~głównym katalogu aplikacji.
  Jest on wykonywalny z~linii poleceń.
  Wywołanie skryptu może być parametryzowane następującymi argumentami:
  \begin{itemize}
    \item \texttt{-v}, określający liczebności zbiorów wierzchołków wygenerowanych grafów.
      Wywołanie skryptu z~wartością parametru \texttt{-v 5} wygeneruje graf o~pięciu wierzchołkach, wartość \texttt{-v 5,10} spowoduje generację dwóch grafów odpowiednio o~pięciu i~dziesięciu wierzchołkach, natomiast wartość \texttt{-v 5-20,5} zaowocuje generacją czterech grafów odpowiednio o~pięciu, dziesięciu, piętnastu oraz dwudziestu wierzchołkach.
    \item \texttt{-d}, określający współczynnik selektywności wierzchołków określonego stopnia w~generowanych grafach.
      Argument ten może przyjmować identyczne postaci jak argument \texttt{-v}.
    \item \texttt{-l}, określający algorytm wizualizacji grafów. Dopuszczalne wartości opisane są w podrozdziale~\ref{sss_internals_misc_graphviz}.
    \item \texttt{-o}, określający format obrazu wynikowego. Dopuszczalne wartości są określone formatami obsługiwanymi przez pakiet Graphviz.
  \end{itemize}

  Domyślnie skrypt generuje wyniki do katalogu \texttt{results}.
  Pliki wynikowe mają postać \texttt{V\_D.dot} oraz \texttt{V\_D.jpeg}, gdzie \texttt{V} jest wartością parametru \texttt{-v}, a  \texttt{D} jest wartością parametru \texttt{-d} dla wygenerowanego grafu.
}

  \chapter{Podsumowanie i kireunki dalszych prac}
\label{summary}
\section{Proponowane usprawnienia}
\section{Podsumowanie}

  \addcontentsline{toc}{chapter}{\bibname}
  \bibliography{Main}

  % \renewcommand{\appendixname}{Dodatek}

  \end{document}
