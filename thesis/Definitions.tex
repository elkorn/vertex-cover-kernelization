\section{Podstawowe definicje}\label{Section_Definicje}

\begin{definition}
  Krawędź $e=(u,v)$ uznaje się za \emph{przystającą} do wierzchołka $v$,
  jeżeli $u=v \lor w=v$.
\end{definition}

\begin{definition}
  Krawędź $e=(a,b)$ uznaje się za \emph{pokrytą} przez zbiór wierzchołków \\
  $V=\{a_0, a_1, \ldots, a_p\}$, jeżeli przystaje ona do co najmniej jednego
  wierzchołka $v \in V$.
\end{definition}

\begin{definition}
  W grafie $G=(V,E)$, mianem \emph{stopnia} wierzchołka $v \in V$ określa się 
  liczbę krawędzi $e \in E$ przystających do wierzchołka $v$.
  Stopień wierzchołka $v$ wyraża się w postaci $d(v)$.
\end{definition}

\begin{definition}\thlabel{def_isolated_vertex}
  \emph{Wierzchołkiem izolowanym} nazywa się wierzchołek $v; d(v)=0$.
\end{definition}

\begin{definition}
  \emph{Sąsiedztwem} wierzchołka $v; d(v)=p$ w grafie $G=(V,E)$ nazywa się zbiór 
  wierzchołków $N={v_0, v_1, \ldots, v_p}$ taki, że 
  $\forall_{v_n \in N}{(v,v_n) \in E \lor (v_n,v) \in E}$.
  Sąsiedztwo wierzchołka $v$ wyraża się w postaci $N(v)$.
\end{definition}

\begin{definition}
  \emph{Sąsiedztwem połączonym} wierzchołka $v; d(v)=p$ w grafie $G=(V,E)$ 
  nazywa się zbiór wierzchołków $N={v_0, v_1, \ldots, v_p}$ taki, że 
  $\not \exists_{v_n, v_m \in N}{(v_n,v_n) \in E \lor (v_n,v_m) \in E}$.
\end{definition}

\begin{definition}
  \emph{Pokrywą wierzchołkową} grafu $G=(V,E)$ nazywa się taki zbiór wierzchołków
  $VC \subseteq V$, że każda krawędź $e \in E$ jest pokryta przez $VC$.
\end{definition}

\begin{definition}
  Przez \emph{rozmiar} pokrywy wierzchołkowej $VC$ rozumie się liczebność ($\|VC\|$)
  zbioru wierzchołków reprezentującą tę pokrywę.
\end{definition}

\begin{definition}
  \emph{Pokrywą wierzchołkową o optymalnym rozmiarze}, \emph{pokrywą 
  wierzchołkową optymalnego rozmiaru} lub \emph{optymalną pokrywą wierzchołkową} 
  grafu $G=(V,E)$ nazywa się pokrywę wierzchołkową $VC_{opt}$ taką, że dla 
  zbioru $VCS \in G$ wszystkich pokryw wierzchołkowych,
  $\forall_{VC \in VCS \setminus VC_{opt}}{\|VC_{opt}\| < \|VC\|}$.
\end{definition}

\begin{definition}
  \emph{Zbiorem niezależnym} nazywa się taki zbiór wierzchołków\\
  $V_i=\{v_0, v_1, \ldots, v_p \}, V_i \in V$ grafu $G=(V,E)$, że
  $\forall_{v_i, v_j \in V_i}{(v_i, v_j) \notin E \land (v_j, v_i) \notin E}$.
\end{definition}

