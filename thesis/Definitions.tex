\section{Podstawowe definicje}\label{Section_Definicje}

Jeżeli nie zaznaczono inaczej, każda z nastepujących definicji odnosi się do
grafu $G=(V,E)$.

\begin{definition}
  Krawędź $e=(u,v)$ uznaje się za \emph{przystającą} do wierzchołka $v$,
  jeżeli $u=v \lor w=v$.
\end{definition}

\begin{definition}
  Krawędź $e=(a,b)$ uznaje się za \emph{pokrytą} przez zbiór wierzchołków \\
  $V=\{a_0, a_1, \ldots, a_p\}$, jeżeli przystaje ona do co najmniej jednego
  wierzchołka $v \in V$.
\end{definition}

\begin{definition}
  Mianem \emph{stopnia} wierzchołka $v \in V$ określa się 
  liczbę krawędzi $e \in E$ przystających do wierzchołka $v$.
  Stopień wierzchołka $v$ wyraża się w postaci $d(v)$.
\end{definition}

\begin{definition}\thlabel{def_isolated_vertex}
  \emph{Wierzchołkiem izolowanym} nazywa się wierzchołek $v; d(v)=0$.
\end{definition}

\begin{definition}
  \emph{Sąsiedztwem} wierzchołka $v; d(v)=p$ nazywa się zbiór 
  wierzchołków $N={v_0, v_1, \ldots, v_p}$ taki, że 
  $\forall_{v_n \in N}{(v,v_n) \in E \lor (v_n,v) \in E}$.
  Sąsiedztwo wierzchołka $v$ wyraża się w postaci $N(v)$.
\end{definition}

\begin{definition}
  \emph{Sąsiedztwem połączonym} wierzchołka $v; d(v)=p$ nazywa się zbiór wierzchołków 
  $N={v_0, v_1, \ldots, v_p}$ spełniający warunek:
  $\not \exists_{v_n, v_m \in N}{(v_n,v_n) \in E \lor (v_n,v_m) \in E}$.
\end{definition}

\begin{definition}
  \emph{Pokrywą wierzchołkową} grafu nazywa się taki zbiór wierzchołków
  $VC \subseteq V$, że każda krawędź $e \in E$ jest pokryta przez $VC$.
\end{definition}

\begin{definition}
  Przez \emph{rozmiar} pokrywy wierzchołkowej $VC$ rozumie się liczebność ($\|VC\|$)
  zbioru wierzchołków reprezentującą tę pokrywę.
\end{definition}

\begin{definition}
  \emph{Pokrywą wierzchołkową o optymalnym rozmiarze}, \emph{pokrywą 
  wierzchołkową optymalnego rozmiaru} lub \emph{optymalną pokrywą wierzchołkową} 
  nazywa się pokrywę wierzchołkową $VC_{opt}$ taką, że dla zbioru $VCS \in G$ 
  wszystkich pokryw wierzchołkowych, 
  $\forall_{VC \in VCS \setminus VC_{opt}}{\|VC_{opt}\| < \|VC\|}$.
\end{definition}

\begin{definition}
  \emph{Zbiorem niezależnym} nazywa się zbiór wierzchołków\\
  $V_i=\{v_0, v_1, \ldots, v_p \}, V_i \in V$ spełniający warunek:
  $\forall_{v_i, v_j \in V_i}{(v_i, v_j) \notin E \land (v_j, v_i) \notin E}$.
\end{definition}

\begin{definition}
  \emph{Grafem dwudzielnym} lub \emph{bigrafem} nazywa się graf o zbiorze 
  wierzchołków, który można rozdzielić na dwa zbiory niezależne $U, V$ takie, że
  $\forall_{u \in U}: \exists_{v \in V}: (u,v) \in E$.
  Równoważym jest stwierdzenie, iż graf dwudzielny stanowi graf nie posiadający
  żadnego cyklu o nieparzystej długości.


  Graf dwudzielny oznaczać można jako $G=(U,V,E)$, nawiązując do poszczególnych
  zbiorów wierzchołków oraz krawędzi między nimi.
\end{definition}

\begin{definition}\thlabel{def_matching}
  \emph{Skojarzeniem} nazywa się zbiór krawędzi $M=\{e_0, e_1, \ldots, e_p\}$
  nie posiadających wspólnych wierzchołków: ${\not\exists{\{e_i=(u_i,v_i),
  e_j=(u_j, v_j)\} \in M}: u_i=u_j \lor u_i=v_j \lor u_j,v_i \lor u_=v_j}$.
  Za skojarzenie uznać można również cały graf zawierający jedynie krawędzie
  spełniające powyższy warunek.
\end{definition}

\begin{definition}
  Wierzchołek $v$ uznaje się za \emph{skojarzony}, jeżeli pokrywa dowolną
  krawędź należącą do skojarzenia.
  W przeciwnym wypadku, wierzchołek jest \emph{nieskojarzony}.
\end{definition}

\begin{definition}\thlabel{def_maximal_matching}
  \emph{Największe skojarzenie} $M_{\max}$ jest skojarzeniem, do którego nie można
  dołączyć żadnej nowej krawędzi bez złamania warunku wymaganego do istnienia
  skojarzenia. 
  Innymi słowy, skojarzenie jest największe, jeżeli wierzchołki zawartych w nim
  krawędzi pokrywają każdą krawędź $e \in E$.
\end{definition}

\begin{definition}\thlabel{def_maximum_matching}
  \emph{Skojarzenie maksymalne} lub \emph{skojarzenie o największej liczebności}
  $M_{\textrm{MAX}}$ to skojarzenie zawierające największą możliwą liczbę krawędzi.
  W grafie może istnieć wiele skojarzeń maksymalnych.\\
  Należy zauważyć, iż $M=M_{\textrm{MAX}} \Rightarrow M=M_{\max}$.
\end{definition}

\begin{definition}
  \emph{Ścieżką} nazywa się skończoną lub nieskończoną sekwencję krawędzi 
  $P \subseteq E$ łączących ze sobą sekwencję różnych wierzchołków
  $V_p \subseteq V$.
  Ścieżka pomiędzy wierzchołkami $v,w$ opisywana może być jako $P_{v,w} = (v
  \rightarrow \ldots \rightarrow w)$.
  \begin{example}{Przykład}\\
    $G=(V,E), V=\{v_0, v_1, \ldots\}, E=\{(v_a, v_b)| \{v_a, v_b\} \in V\}\\
    P_{v_0, v_n}=\{(v_0, v_1), (v_1, v_2), \ldots, (v_{n-1}, v_n)\}$\\

    W grafie nieskierowanym, ta sama ścieżka może przybrać różne postaci, gdyż
    skierowanie krawędzi łączących określone wierzchołki nie ma znaczenia.
  \end{example}
\end{definition}

\begin{definition}
  Mając dane skojarzenie ${M=\{e_{M0}, e_{M1}, \ldots, e_{Mp}\}, e \in M \Rightarrow e \in E}$,
  za \emph{ścieżkę M-przemienną} uznaje się każdą ścieżkę $P=\{e_0, e_1, \ldots,
  e_{n-1}\}$, spełniającą warunek: 
  ${\forall{\{e_{i-1}, e_i\} \in P}: (e_{i-1} \in M \land e_{i} \notin M) \oplus
  (e_{i-1} \notin M \land e_{i} \in M)}$.
\end{definition}