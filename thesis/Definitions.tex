\section{Podstawowe Definicje}\label{s_definitions}

Jeżeli nie zaznaczono inaczej, każda z nastepujących Definicji odnosi się do
grafu $G=(V,E)$.

\begin{definition}
  Krawędź $e=(u,v)$ uznaje się za \emph{przystającą} (sąsiednią) do wierzchołka $v$,
  jeżeli $(u=v) \lor (w=v)$.
\end{definition}

\begin{definition}
  Krawędź $e=(a,b)$ uznaje się za \emph{pokrytą} przez zbiór wierzchołków \\
  $V=\{a_0, a_1, \ldots, a_p\}$, jeżeli przystaje ona do co najmniej jednego
  wierzchołka $v \in V$.
\end{definition}

\begin{definition}
  \emph{Stopniem} wierzchołka $v \in V$ określa się 
  liczbę krawędzi $e \in E$ przystających do wierzchołka $v$.
  Stopień wierzchołka $v$ oznacza się przez $d(v)$.
\end{definition}

\begin{definition}\thlabel{def_regular}
  Graf uznaje się za \emph{regularny} jeżeli wszystkie należące do niego wierzchołki są tego samego stopnia: $\forall_{v \in V}: \forall_{u \in V}:{d(v)=d(u)}$.
\end{definition}

\begin{definition}\thlabel{def_n_regular}
  Graf uznaje się za \emph{$N$-regularny} jeżeli graf ten jest regularny i dowolny wierzchołek grafu jest stopnia $N$.
\end{definition}

\begin{definition}\thlabel{def_isolated_vertex}
  \emph{Wierzchołkiem izolowanym} nazywa się wierzchołek $v$, dla którego $d(v)=0$.
\end{definition}

\begin{definition}
  \emph{Sąsiedztwem} wierzchołka $v$, gdzie $d(v)=p$ nazywa się zbiór 
  wierzchołków $N=\{v_0, v_1, \ldots, v_p\}$ taki, że 
  $\forall_{v_n \in N}{(v,v_n) \in E \lor (v_n,v) \in E}$.
  Sąsiedztwo wierzchołka $v$ wyraża się w postaci $N(v)$.
\end{definition}

\begin{definition}
  \emph{Sąsiedztwem połączonym} wierzchołka $v$, dla którego $d(v)=p$ nazywa się zbiór wierzchołków 
  $N={v_0, v_1, \ldots, v_p}$ spełniający warunek:\\
  $\not \exists_{v_n, v_m \in N}:{(v_n,v_n) \in E \lor (v_n,v_m) \in E}$.
\end{definition}

\begin{definition}\thlabel{def_vc}
  \emph{Pokryciem wierzchołkowym} grafu nazywa się taki zbiór wierzchołków
  $C \subseteq V$, że każda krawędź $e \in E$ jest pokryta przez $C$.
\end{definition}

\begin{definition}
  Przez \emph{rozmiar} lub \emph{liczebność} pokrycia wierzchołkowego $C$ rozumie się liczebność ($|C|$)
  zbioru wierzchołków reprezentującą tę pokrywę.
\end{definition}

\begin{definition}
  \emph{Pokryciem wierzchołkowym o optymalnym rozmiarze}, \emph{pokrywą 
  wierzchołkową optymalnego rozmiaru} lub \emph{optymalną pokryciem wierzchołkowym} 
  nazywa się pokrycie wierzchołkowe $C_{\textnormal{OPT}}$ takie, że dla zbioru wszystkich pokryć wierzchołkowych $S \in V$ spełniona jest własność $\forall_{C \in S \setminus C_{\textnormal{OPT}}}{|C_{\textnormal{OPT}}| < |C|}$.
\end{definition}

\begin{definition}\thlabel{def_independent_set}
  \emph{Zbiorem niezależnym} nazywa się zbiór wierzchołków\\
  $V_i=\{v_0, v_1, \ldots, v_p \}, V_i \subseteq V$ spełniający warunek:
  $\forall_{v_i, v_j \in V_i}{(v_i, v_j) \notin E \land (v_j, v_i) \notin E}$.
\end{definition}

\begin{definition}
  \emph{Grafem dwudzielnym} lub \emph{bigrafem} nazywa się graf o zbiorze 
  wierzchołków, który można rozdzielić na dwa zbiory niezależne $U, V$ takie, że spełniona jest własność
  $\forall_{u \in U}: \exists_{v \in V}: (u,v) \in E$.
  Równoważym jest sTwierdzenie, iż graf dwudzielny stanowi graf nie mający
  żadnego cyklu o nieparzystej długości.

  Graf dwudzielny oznaczać można jako $G=(U,V,E)$, nawiązując do poszczególnych
  podzbiorów wierzchołków oraz krawędzi między nimi.
\end{definition}

\begin{definition}\thlabel{def_matching}
  \emph{Skojarzeniem} nazywa się zbiór krawędzi $M=\{e_0, e_1, \ldots, e_p\}$
  nie mających wspólnych wierzchołków.
  \[\not\exists_{\{e_i=(u_i,v_i), e_j=(u_j, v_j)\} \in M}: (u_i=u_j) \lor (u_i=v_j) \lor (u_j=v_i) \lor (u_j=v_j)\]
  Za skojarzenie uznać można również cały graf zawierający jedynie krawędzie
  spełniające powyższy warunek.
\end{definition}

\begin{definition}
  Wierzchołek $v$ uznaje się za \emph{skojarzony}, jeżeli pokrywa dowolną
  krawędź należącą do skojarzenia.
  W przeciwnym razie, wierzchołek jest \emph{nieskojarzony}.
\end{definition}

\begin{definition}\thlabel{def_maximal_matching}
  \emph{Największe skojarzenie} $M_{\max}$ jest skojarzeniem, do którego nie można
  dołączyć żadnej nowej krawędzi bez naruszenia warunku wymaganego do istnienia
  skojarzenia. 
  Innymi słowy, skojarzenie jest największe, jeżeli wierzchołki zawartych w nim
  krawędzi pokrywają każdą krawędź $e \in E$.
\end{definition}

\begin{definition}\thlabel{def_maximum_matching}
  \emph{Skojarzenie maksymalne} lub \emph{skojarzenie o największej liczebności}
  $M_{\textnormal{MAX}}$ to skojarzenie zawierające największą możliwą liczbę krawędzi.
  W grafie może istnieć wiele skojarzeń maksymalnych.\\
  Należy zauważyć, iż~$M=M_{\textnormal{MAX}} \Rightarrow M=M_{\max}$.
\end{definition}

\begin{definition}
  \emph{Ścieżką} nazywa się skończoną lub nieskończoną sekwencję krawędzi 
  $P \subseteq E$ łączących ze sobą sekwencję różnych wierzchołków
  $V_p \subseteq V$.
  Ścieżka pomiędzy wierzchołkami $v$ oraz $w$ opisywana może być jako $P_{v,w} = (v \rightarrow \ldots \rightarrow w)$.
  Przykład:
  \[G=(V,E), V=\{v_0, v_1, \ldots\}, E=\{(v_a, v_b)| \{v_a, v_b\} \in V\}\]
  \[P_{v_0, v_n}=\{(v_0, v_1), (v_1, v_2), \ldots, (v_{n-1}, v_n)\}\]
  W grafie nieskierowanym, ta sama ścieżka może przybrać różne postaci, gdyż
  skierowanie krawędzi łączących określone wierzchołki nie ma znaczenia.
\end{definition}

\begin{definition}\thlabel{def_alternating_path}
  Mając dane skojarzenie ${M=\{e_{M0}, e_{M1}, \ldots, e_{Mp}\}, e \in M \Rightarrow e \in E}$,
  za \emph{ścieżkę M-przemienną} uznaje się każdą ścieżkę $P=\{e_0, e_1, \ldots,
  e_{n-1}\}$, spełniającą warunek: 
  ${\forall{\{e_{i-1}, e_i\} \in P}: (e_{i-1} \in M \land e_{i} \notin M) \oplus
  (e_{i-1} \notin M \land e_{i} \in M)}$.
\end{definition}

\begin{definition}\thlabel{def_induced_graph}
  Mianem grafu \emph{zaindukowanego} pewnym zbiorem $W \subseteq V$ określa się graf $G^\prime=(V^\prime, E^\prime)$, którego zbiór wierzchołków ograniczony jest do zbioru $W$, a zbiór krawędzi ograniczony jest wyłącznie do krawędzi pokrytych przez wierzchołki należące do zbioru $W$.
  Formalnie zbiór wierzchołków indukowanego grafu określony jest jako $V^\prime = W$, a zbiór jego krawędzi jako $E^\prime=\{e=(u,v)|e\in E \land (u \in W \lor v \in W)\}$.
\end{definition}
