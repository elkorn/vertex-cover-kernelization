\subsection{Rozwiązywanie problemu maksymalnego przepływu w~sieci}\label{ss_max_flow}
\par{
  Problem przepływu w sieci jest kwestią o dużym znaczeniu dla opisywanego w podrozdziale \ref{Kernelization_network_flow} sformułowania redukcji dziedziny do jądra problemu pokrycia wierzchołkowego jako egzemplarza problemu przepływu w sieci dla odpowiadającego grafu dwudzielnego. 
  Wymagana jest zatem solidna podstawa umożliwiająca rozwiązanie tegoż problemu w czasie wielomianowym.
}
\subsubsection{\textbf{Metoda Forda-Fulkersona}}
\par{
  Metoda Forda-Fulkersona stanowi podstawę kilku algorytmów rozwiązujących problem maksymalnego przepływu o~różnych czasach działania.
  Kluczowymi pojęciami, na których operuje metoda Forda-Fulkersona są \emph{sieci residualne}, \emph{ścieżki powiększające} oraz \emph{przekroje}.
  Pojęcia te mają istotne znaczenie w~praktyce rozwiązywania problemów dotyczących przepływów w~sieciach i~stanowią zasadnicze koncepcje w~twierdzeniu o~maksymalnym przepływie i~minimalnym przekroju.
  \begin{theorem}[Twierdzenie o~maksymalnym przepływie i~minimalnym przekroju Forda-Fulkersona]\thlabel{th_max_flow}
    Przyjmując za $f$ przepływ w~sieci $N=(V,E)$ ze źródłem $s$ i~ujściem $t$, następujące warunki są równoważne:
    \begin{enumerate}[(1)]
      \item Przepływ $f$ jest maksymalny w~sieci $N$.
      \item Sieć residualna $N_f$ nie zawiera ścieżek powiększających.
      \item Dla pewnego przekroju $(S, T)$ w~sieci $N$ zachodzi $|f|=c(S,T)$.
    \end{enumerate}
  \end{theorem}
  Dowód powyższego twierdzenia przedstawiony zostanie pod koniec niniejszego podrozdziału, po objaśnieniach pojęć składowych.
}
\par{
  Metoda Forda-Fulkersona jest metodą iteracyjną z~zerową wartością początkową przepływu w~sieci: $\forall_{\{u,v\}\in V}: f(u,v)=0$.
  W każdej iteracji wartość przepływu zostaje zwiększona przez odnalezienie ścieżki powiększającej, stanowiącej ścieżkę ze źródła $s$ do ujścia $t$, której pojemność pozwala na przesłanie dodatkowego przepływu.
  Proces ten powtarzany jest do momentu gdy sieć nie zawiera żadnej ścieżki powiększającej.
  Zgodnie z~Twierdzeniem \ref{th_max_flow}, przepływ otrzymany w~tym momencie jest maksymalny.
  \begin{algorithm}
    \caption{Pseudokod postępowania metody Forda-Fulkersona}\label{alg_fordFulkerson}
    \begin{algorithmic}[1]
      \Function{Ford-Fulkerson-Method}{N, s, t}

        \algorithmicrequire{sieć $N=(V, E)$, źródło $s$, ujście $t$}

        \algorithmicensure{przepływ $f$}

        \State $f \gets 0$
        \While{istnieje ścieżka powiększająca $P_a$}

          \State{powiększ przepływ $f$ wzdłuż $P_a$}
        \EndWhile
        \State\textbf{return} f
      \EndFunction
  \end{algorithmic}
  \end{algorithm}
}
\subsubsection{\textbf{Sieci residualne}}
\par{
  W dowolnej sieci przepływowej $N=(V, E_f)$ dodatkowy przepływ z~wierzchołka $u$ do $v$ nieprzekraczający przepustowości $c(u, v)$ krawędzi łączącej te wierzchołki stanowi \emph{przepustowość residualną}, definiowaną jako $c_f(u ,v) = c(u, v) - f(u, v)$.
  \begin{definition}\thlabel{def_residual_net}
    Dla dowolnej sieci przepływowej $N$ i~pewnego przepływu $f$, \emph{sieć residualna} $N_f$ jest siecią składającą się z~krawędzi dopuszczających dodatkowy dodatni przepływ.
    Formalnie sieć residualną definiuje się jako zbiór $E_f=\{(u, v) \in V \x V: c_f(u,v) > 0\}$.
    Krawędziami $E_f$ w~sieci residualnej $N_f$, zwanymi \emph{krawędziami residualnymi}, są albo krawędzie należące do zbioru $E$ sieci $N$ albo krawędzie do nich przeciwne.
  \end{definition}
  \begin{theorem}
    Krawędź $(u, v)$ może pojawić się w sieci residualnej $N_f$ wtedy i~tylko wtedy, gdy co najmniej jedna z~krawędzi $\{(u, v), (v, u)\}$ występuje w~sieci pierwotnej, co daje liczbę krawędzi residualnych $|E_f| \leq 2|E|$.
  \end{theorem}
  \begin{bproof}
    Jeżeli dla pewnej krawędzi $(u, v) \in E$ zachodzi $f(u, v) < c(u, v)$, to wtedy $c_f(u, v) = c(u, v) - f(u, v) > 0$, co z~definicji sieci residualnej świadczy, iż krawędź $(u, v) \in E_f$ jest w~niej zawarta.

    Jeżeli dla pewnej krawędzi $(u, v) \in E$ zachodzi $f(u, v) > c(u, v)$, to wtedy $c_f(u, v) = c(u, v) - f(u, v) < 0$, co z~definicji sieci residualnej świadczy, iż krawędź przeciwna $(v, u) \in E_f$  jest w~niej zawarta.

    Jeżeli żadna z~krawędzi $\{(u, v), (v, u)\}$ nie występuje w~pierwotnej sieci, to zachodzi wtedy $c(u, v) = c(v, u) = f(u, v) = f(v, u) = c_f(u, v) = c_f(v, u)=0$.
    Żadna z~tych krawędzi nie dopuszcza dodatniego przepływu, co zgodnie z~Definicją~\ref{def_residual_net} wyklucza ją ze zbioru krawędzi residualnych.
  \end{bproof}
}

\subsubsection{\textbf{Ścieżki powiększające}}
\par{
  \begin{definition}
    Dla danej sieci $N=(V,E)$ i~przepływu $f$ \emph{ścieżka powiększająca} to każda ścieżka ze źródła $s$ do ujścia $t$ w~sieci residualnej $N_f$.
  \end{definition}
  Zgodnie z~Definicją~\ref{def_residual_net}, każda krawędź $(u, v)$ na ścieżce powiększającej dopuszcza dodatkowy dodatni przepływ bez naruszenia warunku przepustowości.
  Największą możliwą wartość, o~którą zwiększyć można przepływ na każdej krawędzi ścieżki powiększającej $P_a$ stanowi \emph{przepustowość residualną ścieżki powiększającej}, definiowaną w następujący sposób.
  \begin{definition}
    \emph{Przepustowość residualna ścieżki powiększającej} $P_a$ odpowiada najmniejszej przepustowości residualnej spośród przynależących do niej krawędzi, $c_f(P_a) = min\{c_f(u, v):(u, v)\in P_a\}$.
  \end{definition}
  \begin{definition}\thlabel{def_flow_values}
    Przepływ $f_{P_a}$ na ścieżce powiększającej $P_a$ o~wartości $|F_{P_a}| = c_f(P_a) > 0$ definiowany jest w~oparciu o~następującą funkcję $f_p: V \times V \rightarrow \mathbb{R}$: 
    \begin{equation*}
    f_{P_a}(u, v) = \left\{
    \begin{array}{rl}
    c_f(P_a) & \textnormal{jeżeli } (u, v) \in P_a,\\
    -c_f(P_a) & \textnormal{jeżeli } (v, u) \in P_a,\\
    0 & \textnormal{w przeciwnym wypadku }.
    \end{array} \right.
    \end{equation*}
  \end{definition}
}
\subsubsection{\textbf{Przekroje w~sieciach}}
\par{
  Ostatnim z~pojęć potrzebnych do pełnego zrozumienia metody Forda-Fulkersona są przekroje w~sieciach.
  Stanowią one jednocześnie ważny element całości, gdyż maksymalny przepływ w~sieci bezpośrednio koreluje z~przepustowością najmniejszego jej przekroju.
  \begin{definition}
    Dla sieci $N=(V, E)$ ze źródłem $s$ i~ujściem $t$ każdy podział zbioru $V$ na podzbiory $S$ oraz $T=V\setminus S$ stanowi \emph{przekrój} $(S,T)$ w~sieci $N$.

    Jeżeli w~sieci zdefiniowany jest przepływ $f$, to \emph{przepływem netto} przez przekrój $(S, T)$ definiowany jest jako $f(S, T)$.

    \emph{Przepustowości} przekroju $(S, T)$ oznacza się jako $c(S, T)$.

    \emph{Przekrój minimalny} w~sieci $N$ stanowi przekrój, którego przepustowość jest najmniejsza spośród przekrojów istniejących w~$N$.
  \end{definition}
}
\par{
  Dodać należy, iż przepływ netto przez przekrój $(S, T)$ składa się z~dodatnich przepływów w~obu kierunkach; to znaczy dodatniego przepływu z~$S$ do $T$, branego ze znakiem plus oraz dodatniego przepływu z~$T$ do $S$, branego ze znakiem minus.
  Przepustowość przekroju $c(S, T)$ jest jednak obliczna wyłącznie na podstawie przepustowości krawędzi prowadzących z~$S$ do $T$.
}
\par{
  Dowolny przekrój $(S, T)$ w~sieci $N=(V, E)$ ze źródłem $s$ i~ujściem $t$ ma dwie kluczowe cechy czyniące tę koncepcję kluczową dla rozwiązania problemu największego przepływu.
  \begin{enumerate}
    \item Wartość dowolnego przepływu $f$ w~sieci $N$ jest równa wartości przepływu netto $f(S,T)$.
    \item Wartość dowolnego przepływu $f$ w~sieci $N$ jest nie większa niż przepustowość przekroju $(S, T)$.
  \end{enumerate}
}
\par{
  Po opisaniu niezbędnych elementów składowych można przedstawić dowód twierdzenia o~maksymalnym przepływie i~minimalnym przekroju.
\\
\\
  \begin{bproof}{(Dla twierdzenia o~maksymalnym przepływie i~minimalnym przekroju)\\}
    \underline{(1) $\implies$ (2)}: Jeżeli dla maksymalnego przepływu $f$ w~sieci $N$ sieć residualna $N_f$ zawierałaby ścieżkę powiększającą $P_a$, wartość przepływu $f$ musiałaby zostać powiększona o wartość przepływu $f_{P_a}$.
    Implikuje to, iż wartość przepływu $f$ musiałaby być większa niż $|f|$, co przeczy założeniu, że $f$ jest maksymalny.

    \underline{(2) $\implies$ (3)}: Założyć należy, że sieć residualna $N_f=(V_f, E_f)$ dla sieci $N$ ze źródłem $s$ i ujściem $t$ nie zawiera żadnej ścieżki powiększającej.
    Przyjąć należy istnienie przekroju $(S, T)$, $S=\{v \in V:\textnormal{ istnieje ścieżka z }s\textnormal{ do }v\textnormal{ w~}N_f\}$ oraz zbioru $T=V\setminus S$.
    W oparciu o cechę 1 dowolnego przekroju w~sieci stwierdzić można, że $s \in S$ oraz $t \notin S$, gdyż nie istnieje ścieżka z $s$ do $t$ w~sieci residualnej $N_f$.
    Dla każdej pary wierzchołków $u \in S, v \in T$ musi zachodzić równość $f(u, v)=c(u, v)$, gdyż w~przeciwnym wypadku spełniony byłby warunek klasyfikujący krawędź $(u,v)$ jako krawędź residualną --- w~konsekwencji następowałoby $v \in S$, co przeczy założeniu, że $v \in T$.
    Widać zatem, że $|f|=f(S, T)=c(S, T)$.

    \underline{(3) $\implies$ (1)}: Mając na uwadze warunek (3): $|f|=c(S,T)$, z cechy 2 dowolnego przekroju sieci wynika, że $f$ stanowi przepływ maksymalny w~sieci $N$.
  \end{bproof}
}
\subsubsection{\textbf{Algorytm Edmondsa-Karpa}}\label{sss_edmonds_karp}
W konkretnej implementacji metody Forda-Fulkersona powiększanie przepływu o przepływ residualny odbywa się przez aktualizację przepływu $f[u,v]$\footnote{Nawiasami kwadratowymi oznaczany jest dostęp do zmiennej tablicowej.} między każdą parą~wierzchołków $u,v \in V$.
Proces ten powtarzany jest do dopóty, dopóki sieć zawiera ścieżkę powiększającą.
\par{
  \begin{algorithm}
    \caption{Podstawowy algorytm Forda-Fulkersona}\label{alg_fordFulkersonConcrete}
    \begin{algorithmic}[1]
      \Function{Ford-Fulkerson}{N, s, t}

        \algorithmicrequire{sieć $N=(V, E)$, źródło $s$, ujście $t$}

        \algorithmicensure{przepływ $f$}

        \For{$\forall_{(u,v) \in E}$}

        \State{$f[u, v] \gets 0$}
        \State{$f[v, u] \gets 0$}
        \EndFor
        \While{istnieje ścieżka powiększająca $P_a$}\label{shortestPath}

          \State{$c_f \gets min\{c_f(u, v): (u, v)\in P_a\}$}
          \For{$\forall_{(u,v) \in P_a}$}

            \State{$f[u, v] \gets f[u, v] + c_f(p)$}
            \State{$f[v, u] \gets -f[u, v]$}
          \EndFor
        \EndWhile
        \State\textbf{return} f
      \EndFunction
    \end{algorithmic}
  \end{algorithm}
  Efektywność działania algorytmu w oparciu o metodę Forda-Fulkersona jest w~większości uzależniona od sposobu wyznaczania ścieżki powiększającej przez wiersz \algref{alg_fordFulkersonConcrete}{shortestPath}.
  }
\par{
  Jednym z praktycznie stosowanych algorytmów, gwarantującym wielomianową złożoność czasową jest algorytm Edmondsa-Karpa.
  Stanowi on uściślenie pseudokodu \ref{alg_fordFulkersonConcrete} przez zastosowanie przeszukiwania wszerz sieci przepływowej, co zapewnia, że ścieżka powiększająca $P_a$ wyznaczona przez wiersz \algref{alg_fordFulkersonConcrete}{shortestPath} będzie za każdym razem najkrótsza z możliwych.
  \begin{definition}
    \emph{Długość} ścieżki w sieci residualnej $N_f=(V_f, E_f)$ pomiędzy wierzchołkami $u, v \in V_f$ stanowi wartosć $\delta_f(u, v)$, określająca liczbę krawędzi składowych najkrótszej ścieżki z wierzchołka $u$ do wierzchołka $v$ przy założeniu jednostkowej długości każdej z krawędzi składowych.
  \end{definition}
  \begin{definition}
    Krawędź $(u, v)$ w sieci residualnej $N_f=(V_f, E_f)$ o źródle $s$ i ujściu $t$ jest \emph{krytyczna} na ścieżce powiększającej $P_a$ jeżeli przepustowość residualna ścieżki $c_f(P_a)$ jest równa przepustowości residualnej tej krawędzi, to znaczy $c_f(P_a)=c_f(u, v)$.
  \end{definition}
  \begin{theorem}\thlabel{th_augentation_complexity}
    Łączna liczba powiększeń przepływu w algorytmie Edmondsa-Karpa dla dowolnej sieci przepływowej $N=(V, E)$ o źródle $s$ i ujściu $t$ wynosi $O(|V||E|)$.
  \end{theorem}
  \begin{bproof}
    Każda krawędź krytyczna na dowolnej ścieżce powiększającej zostaje usunięta z sieci residualnej $N_f$ w oparciu o Definicję~\ref{def_residual_net}.
    Prowadzi to do wniosku, że co najmniej jedna krawędź na dowolnej ścieżce ścieżce powiększającej musi być krytyczna.
    Założyć należy istnienie wierzchołków $u, v \in V$ takich, że $(u, v) \in E$.
    W związku z założeniem algorytmu Edmondsa-Karpa --- mówiącemu, że ścieżki powiększające są najkrótsze --- w momencie, gdy krawędź $(u, v)$ jest krytyczna po raz pierwszy, zachodzi $\delta_f(s, v)=\delta_f(s, u)+1$.
    Powiększenie przepływu wzdłuż krawędzi $(u, v)$ powoduje jej usunięcie z sieci residualnej $N_f$, zgodnie z Definicją \ref{def_residual_net}.
    Dodatkowo, krawędź ta nie może zostać na powrót włączona do sieci residualnej dopóty, dopóki przepływ pomiędzy wierzchołkami $u$ i $v$ nie ulegnie zmniejszeniu.
    Zgodnie z Definicją \ref{def_flow_values}, jedyną sytuacją pozwalającą na takie zdarzenie jest znalezienie się krawędzi $(v, u)$ na pewnej ścieżce powiększającej.
    Przepływ w sieci $N$ w momencie odnalezienia takiej ścieżki oznaczony zostanie jako $f^\prime$.
    Zgodnie z dotychczasowym rozumowaniem łatwo zauważyć, że spełniona jest równość $\delta_{f^\prime}(s, u)=\delta_{f^\prime}(s, v) + 1$.
    Ponieważ $\delta_{f^\prime}(s, v) \leq \delta_{f^\prime}(s, u)$, oraz w związku z tym, że dla żadnego wierzchołka $v \in V\setminus \{s, t\}$ odległość $\delta_f(s, v)$ nie może maleć, następująca własność musi być spełniona. \begin{align*}
        \delta_{f^\prime}(s,u)=\delta_{f^\prime}(s,v)+1\\
                           \geq\delta_f(s, v)+1\\
                              =\delta_f(s, u)+2
    \end{align*}
    Oznacza to, że pomiędzy kolejnymi momentami, w których krawędź $(u, v)$ staje się krytyczna, odległość wierzchołka $u$ od źródła $s$ rośnie co najmniej o 2.
    Jeżeli krawędź $(u, v)$ należy do ścieżki krytycznej, to musi zachodzić $u \neq t$.
    Prowadzi to do obserwacji, że żaden z wierzchołków $\{s, u, t\}$ nie może stanowić wierzchołka pośredniego na dowolnej najkrótszej ścieżce ze źródła $s$ do wierzchołka $u$.
    Na tej podstawie stwierdzić można, że jeżeli wierzchołek $u$ przestanie być osiągalny ze źródła $s$ wzdłuż ścieżki powiększającej, to jego odległość od $s$ może wzrosnąć co najwyżej do $|V|-2$.
    W efekcie krawędź $(u, v)$ może stać się krytyczna co najwyżej $\frac{|V|-2}{2}$ razy.
    Ponieważ w sieci residualnej może istnieć co najwyżej $O(|E|)$ par wierzchołków, pomiędzy którymi istnieje krawędź stwierdzić można, że łączna liczba krawędzi krytycznych istniejących w czasie działania algorytmu wynosi co najwyżej $O(\frac{|V|-2}{2}|E|)=O(|V||E|)$.
  \end{bproof}

  Poza wyznaczaniem ścieżek powiększających, pozostałe operacje w algorytmie można zaimplementować jako działania ograniczone złożonością $O(|E|)$, co w efekcie daje całkowitą złożoność czasową $O(|V||E|^2)$.
  Biorąc pod uwagę grafy pełne, gdzie $|E| = |V|^2$, złożoność wynosi $O(|V|^5)$.
}