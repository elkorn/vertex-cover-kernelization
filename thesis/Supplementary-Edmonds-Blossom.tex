\subsection{Odnajdywanie maksymalnego skojarzenia w grafie}\label{ss_edmonds}
\par{
  Każdy z opisywanych w pracy algorytmów działający w oparciu o twierdzenie\\Nemhausera-Trottera \ref{nt_lp}, to znaczy sformułowanie redukcji dziedziny do jądra problemu pokrycia wierzchołkowego jako przepływu w sieci (podrozdział \ref{Kernelization_network_flow}) oraz algorytm redukcji koron (podrozdział \ref{ss_kernelization_crown_reduction}), wymaga w pewnym momencie odnalezienia maksymalnego skojarzenia grafu rozumianego według definicji \ref{def_maximum_matching}.
  Podrozdział skupia się na opisie problemu odnajdywania maksymalnego skojarzenia w dowolnych grafach oraz przedstawia wybrane do jego rozwiązania narzędzie --- algorytm skurczania kwiatów Edmondsa
}

\subsubsection{\textbf{Problem odnajdywania maksymalnego skojarzenia w grafach}}
\par{
  Zgodnie z definicją \ref{def_matching}, w dowolnym grafie $G=(V,E)$~skojarzenie $M$ stanowi zbiór krawędzi $e \in E$, które nie współdzielą ze sobą żadnych wierzchołków: \[\not\exists_{\{(u_1,v_1), (u_2, v_2)\} \in M}:{u_1 = u_2 \lor u_1 = v_2 \lor v_1 = u_2 \lor v_1 = v_2}.\]
  Skojarzenie maksymalne opisywane jest twierdzeniem Berge'a, na którym opiera się definicja \ref{def_maximum_matching}.
  \begin{theorem}[Twierdzenie Berge'a]\thlabel{th_Berge}
    Skojarzenie $M$ w grafie $G$ jest maksymalne wtedu i tylko wtedy gdy nie istnieje ścieżka powiększająca $P_a$ taka, że dla dowolnej pary kolejno następujących po sobie krawędzi $\{e_1, e_2\} \in P_a$ prawdziwe jest $e_1 \in M \oplus e_2 \in M$.
  \end{theorem}
   W kontekście poszukiwania skojarzeń poprzez ścieżki powiększające, graf można traktować jako odpowiednik sieci przepływowej o jednostkowych pojemnościach każdej z krawędzi.
}
\par{
  W przypadku grafów dwudzielnych z krawędziami o jednostkowej, problem odnalezienia maksymalnego dopasowania sprowadza się do rozwiązania problemu maksymalnego przepływu.
  Problem przeformułować należy poprzez dodanie do sieci przepływowej wierzchołków stanowiących źródło oraz ujście zgodnie z zasadami opisanymi w podrozdziale \ref{ss_max_flow}, a następnie postępować zgodnie z algorytmem Edmondsa-Karpa, opisanym w podrozdziale \ref{sss_edmonds_karp} pamiętając, iż rozwiązaniem problemu maksymalnego dopasowania jest przepływ prowadzący w jedną stronę pomiędzy podziałami grafu.
  Poprawność takiego sformułowania problemu odnalezienia maksymalnego skojarzenia jako problemu maksymalnego przepływu w sieci jest ugruntowana w fakcie, że każda krytyczna krawędź zostaje usunięta z sieci residualnej --- a branie pod uwagę przepływu tylko zapewnia brak wspólnych wierzchołków wśród krawędzi uczestniczących w rozwiązaniu.
}
\par{
  Problem odnajdywania maksymalnego skojarzenia w dowolnym grafie nie jest zadaniem trywialnym --- wymaga zatem odpowiedniego, działającego w czasie wielomianowym narzędzia jego rozwiązywania w celu możliwości efektywnego rozwiązania problemów głównych.
  \[
    TODO: Write about the generic approach to finding maximum matchings in general graphs.
  \]
  Narzędzie takie stanowi algorytm skurczania kwiatów Edmondsa, zwany w dalszej części podrozdziału algorytmem Edmondsa lub po prostu algorytmem.
}

\subsubsection{\textbf{Algorytm skurczania kwiatów Edmondsa}}
\par{
  Algorytm Edmondsa poszukuje ścieżek powiększających w grafie stanowiącym odpowiednik sieci przepływowej o jednostkowych pojemnościach każdej z krawędzi.
  Realizacja odbywa się w oparciu o drzewo poszukiwań ze specjalnym traktowaniem występujących w grafie cykli o nieparzystej długości.
  Algorytm Edmondsa przedstawiany jest w ten sposób, ponieważ najczęściej zestawia się go jako działającą w dowolnych grafach alternatywę dla innych algorytmów poszukujących maksymalnego skojarzenia w grafach dwudzielnych.
  Cykle o nieparzystej długości są charakterystyczne jedynie dla grafów niedwudzielnych i określane są mianem \emph{kwiatów}.
}
% \par{
%   Algorytm działa  według następujących kroków.
%   \begin{enumerate}
%     \item Poszukuj skierowanej ścieżki w grafie $H$ zaczynającej się od pewnego wolnego wierzchołka $v_f$ do pewnego dobrego wierzchołka $v_g$, posiadającego co najmniej jeden wolny wierzchołek sąsiedni.
%     \item Jeżeli nie odnaleziono tak zdefiniowanej ścieżki,  jest powiększająca
%   \end{enumerate}
  
%   Jeżeli ścieżka taka zostanie odnaleziona i zawiera ona skurczone kwiaty, algorytm rozwija każdy z nich w celu otrzymania właściwej formy znalezionej ścieżki powiększającej.
%   Tak odnaleziona ścieżka zostaje wykorzystania do rozszerzenia częściowego maksymalnego skojarzenia grafu o jedną krawędź.
%   W sytuacji gdy algorytm znajduje się w stanie gdzie nie istnieje ścieżka powiększająca w zredukowanej instancji grafu, pewnym jest również jej brak w grafie wejściowym.
% }
\par{
  \begin{definition}
    W dowolnym grafie $G=(V, E)$, mianem \emph{kwiatu} $B$ względem skojarzenia $M \in E$ określa się cykl o nieparzystej długości i największej liczbie krawędzi składowych skojarzonych w $M$. 
  \end{definition}
  Z powyższej definicji wynika, że przykładowo jeżeli pewien cykl składa się z $2k + 1$ wierzchołków oznacza to, że $k$ krawędzi musi zawierać się w pewnym skojarzeniu $M$ aby cykl ten mógł być kwiatem.
  \begin{definition}
    Poprzez \emph{skurczenie} kwiatu $B$ w grafie $G=(V, E)$ rozumie się operację składającą się z następujących czynności.
    \begin{enumerate}
       \item Zastąpieniu zbioru wszystkich wierzchołków $V_B=\{v: v \in B\}$ pojedynczym wierzchołkiem $v_B$. Konsekwencja zastąpienia każdego z wierzchołków kwiatu jest usunięcie przystających do niego krawędzi.
       \item Połączeniem wierzchołka $v_B$ krawędziami ze wszystkimi wierzchołkami spoza kwiatu, które sąsiadowały z nim przed wykonaniem kroku 1: \[V_N=\{v: v \in V \setminus V_B, \exists_{u \in V}:(u,v)\in E \lor (v, u)\in E\}.\]
     \end{enumerate}
  \end{definition}
  Konsekwencją skurczenia kwiatu $B=(V_B, E_B)$ jest zmniejszenie skojarzenia $M$ do $M^\prime=M\E_B$.
  W celu zwiększenia efektywności odnajdywania skierowanych ścieżek w grafie $H$, algorytm przechowuje dane o odwiedzonych krawędziach w strukturze lasu.\\
  Algorytm Edmondsa działa według następującego schematu.
  \begin{algorithm}
    \caption{Algorytm Edmondsa}\label{alg_Edmonds}
    \begin{algorithmic}[1]
      \Function{Edmonds}{G}

        \algorithmicrequire{graf $G$}

        \algorithmicensure{maksymalne skojarzenie $M$}

        \State{$M$ \gets \emptyset}

        \State{$H$ \gets $G$}

        \State{$N$ \gets $M$}

        \State{$P$ \gets $\{v: v\text{ jest wolne w } M\}$}\label{init_edmonds}

        \State{$NP$ \gets \emptyset}

        \State{$E_F$ \gets \emptyset}

        \State{$V_F$ \gets $P \bigcup NP$}

        \State{$F$ \gets \Call{Las}{V_F, E_F}}

        \State{Wybierz krawędź $(u, v); u \in P, v \notin NP$}\label{pick_edge}
        \If{taka krawędź nie istnieje}
          \State{\textbf{return} $M$}
        \EndIf
        \If{$v \notin V_F$}
          \State{$NP$ \gets $NP \bigcup \{v\}$}

          \State{$P$ \gets $P \bigcup \{M(v)\}$}

          \State{$E_F$ \gets $E_F \bigcup \{(v, M(v))\}$}
        \ElsIf{$v \in P$ oraz $(u, v) \in E_F$}
          \Comment{Odnaleziono kwiat B.}
          \State{$b$ \gets \Call{Skurcz}{B}}

          \State{$H$ \gets $H\B$}

          \State{$N$ \gets $N\B$}

          \State{$P$ \gets $P \bigcup \{b\}$}

          \State{Idź do kroku~\ref{pick_edge}}
        \Elsif{$v \in P$ oraz $(u, v) \notin E_F$}
          \State{Znajdź w $G$ ścieżkę powiększającą $\rho$ względem $M$}

          \State{Rozszerz $M$}
          \State{Idź do kroku~\ref{init_edmonds}}
        \EndIf
      \EndFunction
    \end{algorithmic}
  \end{algorithm}
}