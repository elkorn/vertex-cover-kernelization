\subsection{Odnajdywanie maksymalnego skojarzenia w~grafie}\label{ss_edmonds}
\par{
  Każdy z~opisywanych w~pracy algorytmów działający w~oparciu o~twierdzenie\\Nemhausera--Trottera \ref{nt_lp}, to znaczy sformułowanie redukcji dziedziny do jądra problemu pokrycia wierzchołkowego jako przepływu w~sieci (podrozdział \ref{Kernelization_network_flow}), algorytm redukcji koron (podrozdział \ref{ss_kernelization_crown_reduction}) oraz algorytm Chena, Kanji oraz Xia (podrozdział~\ref{s_ckx}), wymaga w~pewnym momencie odnalezienia maksymalnego skojarzenia grafu rozumianego według definicji \ref{def_maximum_matching}.
  Podrozdział skupia się na opisie problemu odnajdywania maksymalnego skojarzenia w~dowolnych grafach oraz przedstawia wybrane do jego rozwiązania narzędzie --- algorytm skurczania kwiatów Edmondsa
}

\subsubsection{\textbf{Problem odnajdywania maksymalnego skojarzenia w~grafach}}
\par{
  Zgodnie z~Definicją \ref{def_matching}, w~dowolnym grafie $G=(V,E)$~skojarzenie $M$ stanowi zbiór krawędzi $e \in E$, które nie współdzielą ze sobą żadnych wierzchołków: \[\not\exists_{\{(u_1,v_1), (u_2, v_2)\} \in M}:{u_1 = u_2 \lor u_1 = v_2 \lor v_1 = u_2 \lor v_1 = v_2}.\]
  Skojarzenie maksymalne opisywane jest twierdzeniem Berge'a, na którym opiera się Definicja \ref{def_maximum_matching}.
  \begin{theorem}[Twierdzenie Berge'a]\thlabel{th_Berge}
    Skojarzenie $M$ w~grafie $G$ jest maksymalne wtedy i~tylko wtedy, gdy nie istnieje ścieżka powiększająca $P_a$ taka, że dla dowolnej pary kolejno następujących po sobie krawędzi $\{e_1, e_2\} \in P_a$ prawdziwe jest $e_1 \in M \oplus e_2 \in M$.
  \end{theorem}
   Należy zwrócić uwagę, iż ścieżka powiększająca $P_a$ opisywana w~Twierdzeniu~\ref{th_Berge} jest $M$-przemienna w~rozumieniu definicji~\ref{def_alternating_path}.
   W kontekście poszukiwania skojarzeń przez ścieżki powiększające, graf można traktować jako odpowiednik sieci przepływowej o~jednostkowych pojemnościach każdej z~krawędzi.
}
\par{
  W przypadku grafów dwudzielnych z~krawędziami o~jednostkowej wadze, problem odnalezienia maksymalnego skojarzenia sprowadza się do rozwiązania problemu maksymalnego przepływu.
  Problem przeformułować należy przez dodanie do sieci przepływowej wierzchołków stanowiących źródło oraz ujście zgodnie z~zasadami opisanymi w~podrozdziale \ref{ss_max_flow}, a~następnie postępować zgodnie z~algorytmem Edmondsa-Karpa, opisanym w~podrozdziale \ref{sss_edmonds_karp} pamiętając, iż rozwiązaniem problemu maksymalnego skojarzenia jest przepływ prowadzący w~jedną stronę pomiędzy podziałami grafu.
  Poprawność sformułowania problemu odnalezienia maksymalnego skojarzenia jako problemu maksymalnego przepływu w~sieci jest ugruntowana tym, że każda krytyczna krawędź zostaje usunięta z~sieci residualnej --- a~branie pod uwagę przepływu tylko w~jedną stronę zapewnia brak wspólnych wierzchołków wśród krawędzi uczestniczących w~rozwiązaniu.
}
\par{
  Problem odnajdywania maksymalnego skojarzenia w~dowolnym grafie nie jest zadaniem trywialnym --- wymaga zatem odpowiedniego, działającego w~czasie wielomianowym narzędzia jego rozwiązywania w~celu możliwości efektywnego rozwiązania problemów głównych.
  Narzędzie takie stanowi algorytm skurczania kwiatów Edmondsa, zwany w~dalszej części podrozdziału algorytmem Edmondsa lub po prostu algorytmem.
}
\subsubsection{\textbf{Poszukiwanie ścieżek powiększających zaczynających się w~dowolnym wierzchołku grafu}}
\par{
  Aby móc dowieść poprawności algorytmu Edmondsa, należy wyjaśnić kilka dodatkowych koncepcji, na których jest on oparty.
  U podstaw teoretycznych algorytmu leży skonstruowany z~grafu wejściowego $G=(V, E)$ graf skierowany $H=(V_H, E_H)$ taki, że $V_H=V$ oraz $E_H=\{(u, v)|\exists_{w\in V\}}:(u, w) \notin M \land (w, v) \in M\}$.
  W objaśnianiu toku rozumowania pomocne są również następujące charakterystyczne zbiory wierzchołków.
  \begin{itemize}
    \item Zbiór \emph{wolnych} wierzchołków $V_F=\{v | v \notin M\}$.
    \item Zbiór \emph{dobrych} wierzchołków $V_G=\{v | N(v) \cap V_G \setminus \{v\} \neq \emptyset\}$. Wierzchołek $v\in V$ nazywany jest dobrym w~grafie gdy ma co najmniej jeden nieskojarzony przez zbiór $M$ wierzchołek sąsiedni.
  \end{itemize}
  
  Odnajdywanie ścieżek powiększających w~skierowanym grafie $H$ jest krokiem pozwalającym na odkrycie \emph{prostych} ścieżek powiększających w~grafie $G$.
  \begin{definition}
    \emph{Prostą} ścieżkę w~grafie $G=(V, E)$ stanowi ścieżka, która nie prowadzi dwa razy przez ten sam wierzchołek $v \in V$.
  \end{definition}
  \begin{definition}
    Dla pewnego grafu $G=(V, E)$ oraz grafu skierowanego $H=(V_H,E_H)$, skonstruowanego z~grafu $G$, mianem ścieżki \emph{pseudorozszerzającej} określa się prostą ścieżkę w~grafie $H$ prowadzącą od pewnego wierzchołka $v \in v_H$ do pewnego dobrego wierzchołka $w \in V_H$.
  \end{definition}

  Udowodnienie następujących twierdzeń kładzie fundamenty teoretyczne potrzebne do późniejszego uzasadnienia poprawności przyjętego podejścia do algorytmu Edmondsa.
  \begin{theorem}
    W grafie $G=(V, E)$ dla grafu skierowanego $H=(V_H,E_H)$ skonstruowanego z~grafu $G$ oraz skojarzenia $M$, $M$-przemienna ścieżka powiększająca o~początku w~wierzchołku $v \in V$ może istnieć wtedy i~tylko wtedy jeżeli istnieje skierowana ścieżka $\rho$ prowadząca od wierzchołka $v$ do dobrego wierzchołka $w$, która odpowiada prostej ścieżce w~grafie $G$.
  \end{theorem}
  \begin{bproof}
    Założyć należy istnienie w~grafie $G=(V, E)$ $M$-przemiennej ścieżki $\tau$ zaczynającej się od wierzchołka $v \in V$ i~kończącej w~dobrym wierzchołku $w \in V$. Przyjąć należy wierzchołek $u \in V$ jako należący do sąsiedztwa wierzchołka $w$, $u \in N(w)$, znajdującego się na ścieżce $\tau$.
    Zgodnie z~założeniem twierdzenia, że wierzchołek $w$ jest dobry, wierzchołek $u$ również musi być dobry.
    Łatwo zauważyć zatem, iż ścieżce składowej $\tau^\prime \subset \tau$ pomiędzy wierzchołkami $v$ oraz $u$ odpowiada prosta ścieżka $\tau_H \in E_H$, prowadząca od wierzchołka $v$ do wierzchołka $u$ --- jedynymi krawędziami mogącymi stanowić różnicę pomiędzy $\tau$ oraz $\tau_H$ są krawędzie należące do nawrotów w~ścieżce $\tau$.
  \end{bproof}
  \begin{theorem}
    Jeżeli $M$-przemienna ścieżka $\tau$ w~grafie $G$ odpowiadająca pseudopowiększającej ścieżce $\tau_H$ w~grafie $H$ nie jest powiększająca oznacza to, iż zawiera ona cykl o~nieparzystej długości.
  \end{theorem}
  \begin{bproof}
    Fakt, iż ścieżka $\tau$ nie jest powiększająca implikuje, że nie jest ona również prosta, to znaczy zawiera krawędzie nawrotowe.
    Jest to równoznaczne ze stwierdzeniem, iż ścieżka $\tau$ zawiera cykl.
    Jeżeli cykl ten miałby być parzystej długości, ścież $\tau_H$ musiałaby co najmniej dwukrotnie przechodzić przez pewien wierzchołek, co przeczy założeniu, iż stanowi jest ona prostą ścieżką skierowaną.
  \end{bproof}
  Mając na uwadze te zależności, można przejść do analizy algorytmu Edmondsa.
}
\subsubsection{\textbf{Algorytm skurczania kwiatów Edmondsa}}\label{ss_edmonds_blossom}
\par{
  Algorytm Edmondsa poszukuje ścieżek powiększających w~grafie stanowiącym odpowiednik sieci przepływowej o~jednostkowych pojemnościach każdej z~krawędzi.
  Realizacja odbywa się w~oparciu o~drzewo poszukiwań ze specjalnym traktowaniem występujących w~grafie cykli o~nieparzystej długości.
  Algorytm Edmondsa przedstawiany jest w~ten sposób, ponieważ najczęściej zestawia się go jako działającą w~dowolnych grafach alternatywę dla innych algorytmów poszukujących maksymalnego skojarzenia w~grafach dwudzielnych.
  Cykle o~nieparzystej długości są charakterystyczne jedynie dla grafów niedwudzielnych i~określane są mianem \emph{kwiatów}.
}
\par{
  \begin{definition}
    W dowolnym grafie $G=(V, E)$, \emph{kwiat} $B$ względem skojarzenia $M \subseteq E$ stanowi cykl o~nieparzystej długości i~największej liczbie krawędzi składowych skojarzonych w~$M$. 
  \end{definition}
  Z powyższej definicji wynika, że przykładowo jeżeli pewien cykl składa się z~$2k + 1$ wierzchołków oznacza to, że $k$ krawędzi musi zawierać się w~pewnym skojarzeniu $M$ aby cykl ten mógł być kwiatem.
  \begin{definition}
    przez \emph{skurczenie} kwiatu $B$ w~grafie $G=(V, E)$ rozumie się operację składającą się z~następujących czynności.
    \begin{enumerate}
       \item Zastąpieniu zbioru wszystkich wierzchołków $V_B=\{v: v \in B\}$ pojedynczym wierzchołkiem $v_B$. Konsekwencja zastąpienia każdego z~wierzchołków kwiatu jest usunięcie przystających do niego krawędzi.
       \item Połączeniem wierzchołka $v_B$ krawędziami ze wszystkimi wierzchołkami spoza kwiatu, które sąsiadowały z~nim przed wykonaniem kroku 1: \[V_N=\{v: v \in V \setminus V_B, \exists_{u \in V}:(u,v)\in E \lor (v, u)\in E\}.\]
     \end{enumerate}
  \end{definition}
  Konsekwencją skurczenia kwiatu $B=(V_B, E_B)$ jest zmniejszenie skojarzenia $M$ do $M^\prime = M \setminus E_B$.
  W celu zwiększenia efektywności odnajdywania skierowanych ścieżek w~grafie $H$, algorytm przechowuje dane o~odwiedzonych krawędziach w~strukturze lasu.\\
  Algorytm Edmondsa działa według schematu opisanego pseudokodem~\ref{alg_Edmonds}.
  \begin{algorithm}
    \caption{Algorytm Edmondsa}\label{alg_Edmonds}
    \begin{algorithmic}[1]
      \Function{Edmonds}{G}

        \algorithmicrequire{graf $G=(V, E)$}

        \algorithmicensure{maksymalne skojarzenie $M$}

        \State $M \gets \emptyset, H=(V_H, E_H) \gets G, W \gets M$
        \State $P \gets \{v: v \notin M\}, W \gets \emptyset, E_F \gets \emptyset$\label{init_edmonds}
        \State $F \gets \Call{Las}{P \cup W, E_F}$
        \State{Wybierz krawędź $(u, v); u~\in P, v \notin W$}\label{pick_edge}
        \If{taka krawędź nie istnieje}
          \State{\textbf{return} $M$}
        \EndIf
        \If{$v \notin P \cup W$}\label{put_matched_in_odd}
          \State $N \gets N \cup \{v\}$
          \State $P \gets P \cup \{M(v)\}$
          \State $E_F \gets E_F \cup \{(v, M(v))\}$
        \ElsIf{$v \in P$ oraz $(u, v) \in E_F$}\label{even_connected}
          \Comment{Odnaleziono kwiat B.}
          \State $b \gets \Call{Skurcz}{B}$
          \State $H \gets H\setminus B$
          \State $W \gets W\setminus B$
          \State $P \gets P \cup \{b\}$
          \State{Idź do kroku~\ref{pick_edge}}
        \ElsIf{$v \in P$ oraz $(u, v) \notin E_F$}\label{even_disjoint}
          \State\Comment{(Istnieje $W$-przemienna ścieżka powiększająca $\eta \in H$)}
          \State{Znajdź $M$-przemienną ścieżkę powiększającą $\rho \in G$}
          \State{Idź do kroku~\ref{init_edmonds}}
        \EndIf
      \EndFunction
    \end{algorithmic}
  \end{algorithm}
  \begin{theorem}\thlabel{th_edmonds_max_matching}
    W momencie zakończenia działania algorytmu, zbiór $N$ stanowi maksymalne skojarzenie w~grafie $H$.
  \end{theorem}
  Aby dowieść poprawności tego twierdzenia, przygotować należy metrykę pozwalającą na sprawdzenie czy dane skojarzenie rzeczywiście jest maksymalne.
  W oparciu o~notatki~\cite{Tutte-Berge:notes} oraz artykuł~\cite{cohen:hal-00358468} (stanowiące syntezę koncepcji pracy~\cite{Tutte-Berge:original}),  można posłużyć się następującą własnością, u~podstaw której leży wzór Tutte-Berge'a.

  Dla dowolnego zbioru wierzchołków $U$, przyjąć należy oznaczenie $\circ (G \setminus U)$ jako określenie liczby spójnych składowych grafu $G^\prime=(V^\prime, E^\prime)$ indukowanego z~grafu $G=(V \setminus U, E)$ o~nieparzystej liczebności.
  Dla każdego zbioru $U \subseteq V$ dowolne skojarzenie $M \subseteq E^\prime$ musi kojarzyć mniej niż $\circ (G \setminus U)$ wierzchołków --- przynajmniej jeden wierzchołek dla każdej ze składowych w~$G^\prime$ musi pozostać nieskojarzony.
  Co więcej, wierzchołki te skojarzone mogą być jedynie ze zbiorem wierzchołków $U$.
  \begin{definition}[ (Oparta na wzorze Tutte-Berge'a)]\thlabel{def_edmonds_max_matching}
    Skojarzenie $M$ jest maksymalne, jeżeli liczba wierzchołków nieskojarzonych przez $M$ wynosi $|V| - 2|M| \geq \circ (G \setminus U) - |U|$, a~zatem $|M|\leq \frac{|V|+|U| - \circ (G \setminus U)}{2}$.
  \end{definition}
  \begin{definition}
  \emph{Zbiór zaświadczający} to zbiór wierzchołków spełniający wzór Tutte-Berge'a.
  \end{definition}
  W przypadku spełnienia założeń Definicji~\ref{def_edmonds_max_matching}, zbiór $U$ zaświadcza o~tym, że skojarzenie $M$ jest maksymalne.
  W pseudokodzie~\ref{alg_Edmonds} zbiorem zaświadczającym jest zbiór $N$.
  Zaznaczyć należy, iż nazewnictwo zmiennych $P$ oraz $N$ jest nieprzypadkowe --- wierzchołki należące do zbioru $P$ znajdują się w~parzystej odległości od korzenia pewnego drzewa $T \in F$, natomiast wierzchołki należące do zbioru $N$, analogicznie, w~odległości nieparzystej od korzenia drzewa $T$.
  Przez odległość rozumie się długość ścieżki prowadzącej do danego wierzchołka z~korzenia drzewa $T$.\\\\
  \begin{bproof} (Dla twierdzenia~\ref{th_edmonds_max_matching})\\
    W momencie zakończenia działania algorytmu na grafie $G=(V, E)$, zbiór wierzchołków $V$ jest podzielony na trzy podzbiory: $P$, $N$ oraz podzbiór $R = V \setminus P \setminus N$.
    Zbiór $N$ jest niezależny w~grafie $H=(V_H, E_H)$ (w rozumieniu definicji~\ref{def_independent_set}) --- w~przeciwnym przypadku warunek~\algref{alg_Edmonds}{even_connected} lub~\algref{alg_Edmonds}{even_disjoint} algorytmu Edmondsa byłby spełniony co przeczy założeniu, iż algorytm zakończył działanie.
    Kroki w~momencie spełnienia warunku~\algref{alg_Edmonds}{put_matched_in_odd} algorytmu gwarantują, że dla każdego wierzchołka $v_N \in N$ dowolny skojarzony z~nim przez zbiór $W$ wierzchołek zostanie umieszczony w~zbiorze $P$.
    Tym samym zagwarantowana jest właściwość, że każdy wierzchołek $v_P \in P$ jest skojarzony przez zbiór $W$ wyłącznie z~innym wierzchołkiem $v_N \in N$.
    Krok~\algref{alg_Edmonds}{pick_edge} zapewnia, że w~momencie zakończenia działania algorytmu nie może istnieć krawędź $(v_P, v_R); v_P \in P, v_R \in R$ pomiędzy zbiorami $P$ oraz $R$.
    Łatwo zauważyć, że po usunięciu z~dziedziny zbioru $P$ każdy wierzchołek $v_P \in P$ sam w~sobie stanowi spójną składową grafu $H$.
    Ponieważ wszystkie wierzchołki $v_R \in R$ są skojarzone przez zbiór $W$ w~ramach zbioru $R$ można stwierdzić, że po usunięciu zbioru $N$ z~dziedziny składowe należące do zbioru $R$ są parzystej liczebności.
    Prowadzi to do wniosku, że spełnione są założenia wzoru Tutte-Berge'a, przybliżone w~artykule~\cite{cohen:hal-00358468}.
    \begin{align*}
        \circ(H \setminus N) = |P| &= |N| + |R|\\
        |N| + |V_H| - \circ(H \setminus N) = |V| - |R| &= 2|W|\\
        \frac{|N| + |V_H| - \circ (H \setminus N)}{2} &= |W|
    \end{align*}
    Zgodnie z~dotychczasowym tokiem rozumowania, spełnienie powyższej równości dowodzi, iż zbiór $N$ zaświadcza o~tym, że skojarzenie $W$ jest maksymalne w~grafie $H$.
  \end{bproof}
  W celu uzyskania pełnego oglądu poprawności algorytmu Edmondsa, należy uzasadnić poprawność operacji zwijania struktur kwiatów w~celu poszukiwania maksymalnych skojarzeń.
  \begin{theorem}
    Przyjąwszy istnienie pewnego skojarzenia $W$ w~grafie $H=(V_H, E_H)$ oraz kwiatu $B$, $W \setminus B$ jest maksymalne w~$H \setminus B$ wtedy i~tylko wtedy gdy skojarzenie $W$ jest maksymalne w~$H$.
  \end{theorem}
  \begin{bproof}
    Jeżeli skojarzenie $W \setminus B$ nie byłoby maksymalne, w~grafie $H$ musiałaby istnieć $W$-przemienna ścieżka powiększająca $\eta$.
    Kroki algorytmu w~razie spełnienia warunku~\algref{alg_Edmonds}{even_disjoint} w~pseudokodzie~\ref{alg_Edmonds} zajmują się tym przypadkiem.

    Ścieżka $\eta$ zaczyna się od korzenia pewnego drzewa $T_u$ w~lesie $F$, zawierającego wierzchołek $u$ (w rozumieniu warunku~\algref{alg_Edmonds}{pick_edge} w~pseudokodzie~\ref{alg_Edmonds}) prowadzi do wierzchołka $u$, a~następnie przez wierzchołek $v$ dociera do korzenia drzewa $T_v \in F$, zawierającego wierzchołek $v$.
    Uzyskanie $M$-przemiennej ścieżki powiększającej w~grafie $G$ jest możliwe przez rozwinięcie znajdujących się na ścieżce skurczonych kwiatów i~
    skojarzeń.

    Założyć należy, że graf $H=(V_H, E_H)$ otrzymano z~grafu $G=(V, E)$ przez skurczenie kwiatu $B$ do wierzchołka $b$.
    W takiej sytuacji jeżeli ścieżka $\eta$ nie przechodzi przez wierzchołek $b$ oznacza to, że musi być ona również $M$-przemienną ścieżką powiększającą w~grafie $G$. W przeciwnym przypadku --- gdy ścieżka $\eta$ zawiera w~sobie wierzchołek $b$ --- mogą wystąpić dwie sytuacje:
    \begin{enumerate}
      \item Ścieżka $\eta$ kończy się w~wierzchołku $b$.
      Może to nastąpić wyłącznie w~przypadku, gdy w~kwiecie $B$ występuje korzeń zawierającego go drzewa $T$, w~przeciwnym razie wierzchołek $b$ nie byłby wolny w~grafie $H$.
      W tejże sytuacji zakończeniem ścieżki $\eta$ w~grafie $G$ jest pewien wierzchołek $v_B \in B$, z~którego można dotrzeć do korzenia podążajac we właściwym kierunku --- to znaczy tym, który zawiera skojarzoną przez zbiór $M$ krawędź łączącą kwiat $B$ z~najbliższym wierzchołkiem z~zewnątrz.
      \item Ścieżka $\eta$ przechodzi przez wierzchołek $b$.
      Przyjąć należy, że ścieżka $\eta$ w~grafie $G$ dociera do pewnego wierzchołka $x$ należącego do kwiatu $B$.
      W takiej sytuacji jedyna skojarzona przez zbiór $M$ krawędź wychodząca z~kwiatu $B$ prowadzi z~pewnego wierzchołka $y$ do korzenia drzewa $T$.
      Zaczynając w~wierzchołku $x$ można zatem ponownie wybrać odpowiedni kierunek wzdłuż krawędzi kwiatu $B$ w~celu zachowania $M$-przemienności, osiągnąć wierzchołek $y$ i~postępować dalej zgodnie z~kształtem ścieżki $\eta$.
    \end{enumerate}

    Powyższy szkic mogących nastąpić sytuacji dowodzi, że w~momencie zakończenia działania algorytmu skojarzenie $W$ musi być maksymalne w~grafie $H$.
  \end{bproof}
  \begin{theorem}
    Algorytm Edmondsa znajduje maksymalne dopasowanie grafu $G=(V, E)$ w~czasie $O(|V|^4)$.
  \end{theorem}
  \begin{bproof}
    Podpierając się rozumowaniem przedstawionym w~dowodzie twierdzenia~\ref{th_augentation_complexity} oraz dowodami zawartymi w~niniejszym podrozdziale łatwo zauważyć, że algorytm Edmondsa w~trakcie działania dokona co najwyżej $\frac{|V|}{2}$ powiększeń.
    Pomiędzy kolejnymi powiększeniami istnieje możliwość dodania co najwyżej $|E|$ krawędzi do lasu $F$.
    Ponieważ kwiat stanowi cykl nieparzystej długości, zawiera on przynajmniej 3 krawędzie --- a~zatem pomiędzy kolejnymi powiększeniami istnieje możliwość skurczenia co najwyżej $\frac{|V|}{3}$ kwiatów.
    W związku z~potrzebą utrzymywania lasu $F$, dodawanie do niego kolejnej krawędzi poprzedzone jest sprawdzeniem co najwyżej $O(|E|)$ zawartych krawędzi w~celu zachowania ich unikalności w~ramach drzewa.
    Łatwo zauważyć, że złożoność obliczeniowa wynosić będzie $O(|E||V|^2)$.
    W możliwie najgęstszym grafie spełniona jest równość $|E|=|V|^2$ --- pesymistyczna złożoność obliczeniowa wynosić będzie zatem $O(|V|^2*|V|^2)=O(|V|^4)$.
  \end{bproof}
}