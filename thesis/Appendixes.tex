\chapter{Środowisko uruchomieniowe}
\section{Zależności}
\par{
  W celu uruchomienia załączonej aplikacji wymagane jest zainstalowanie następujących zależności:
  \begin{itemize}
    \item Środowisko uruchomieniowe języka Go w wersji co najmniej 1.3.3.
    \item Niestandardowe biblioteki:
    \begin{itemize}
      \item \texttt{github.com/deckarep/golang-set}
      \item \texttt{github.com/lukpank/go-glpk/glpk}
    \end{itemize}
    \item Pakiet GLPK.
  \end{itemize} 
  Aby móc generować grafy losowe, należy zainstalować środowisko uruchomieniowe dla języka Python w wersji co najmniej 4.3.2 oraz pakiet \texttt{graph-tool}.
}
\section{Instalacja i uruchamianie testów}
\par{
  W celu uruchomienia testów należy skopiować katalog zawarty na nośniku dołączonym do pracy, a następnie zdefiniować zmienną środowiskową \texttt{\$GOPATH} tak, by wskazywała na ścieżkę do podkatalogu \texttt{src}.
  W następnej kolejności należy zmienić bieżącą ścieżkę na podkatalog \texttt{src/main}.
  Do uruchomienia pomiarów służy polecenie \texttt{go run *.go -run N}, gdzie \texttt{N} stanowi wyrażenie regularne zawierające część nazwy przypadku testowego.
  Wyczerpująca lista zdefiniowanych przypadków testowych zawarta jest w pliku \texttt{test-cases.go}.

  Wyniki testów szybkości działania algorytmów zawarte w badaniach eksperymentalnych uzyskano na komputerze o następujących parametrach:
  \begin{itemize}
    \item Procesor Intel® Core™ i7-4800MQ (4 $\times$ 2.7 GHz)
    \item 32 GB DDR3L 1600MHz
  \end{itemize}

  W celu uruchomienia testów jednostkowych należy zmienić bieżącą ścieżkę na katalog wybranego pakietu projektu i wywołać komendę \texttt{go test}.
}
\section{Generacja grafów losowych}

\chapter{Grafy testowe}
\par{
  W niniejszym dodatku zebrano wizualizacje przykładowych grafów testowych wykorzystanych przy badaniach eksperymentalnych. 
}