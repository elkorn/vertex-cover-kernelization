\chapter{Podsumowanie i~kierunki dalszych prac}
\label{summary}
\section*{Proponowane usprawnienia i~kierunki rozwoju}\label{s_improvements}
\addtocounter{section}{1}
  \subsubsection{\textbf{Pełna implementacja algorytmu Chena, Kanji oraz Xia.}}\label{sss_problems_ckx}

  Ze względu na wysoką złożoność analityczną algorytmu Chena, Kanji oraz Xia nie udało się zaimplementować wszystkich elementów składowych wymaganych przez główny algorytm w~stopniu pozwalającym na przeprowadzenie badań eksperymentalnych z~jego wykorzystaniem.
  Głównym problemem był dobór NT--dekompozycji, którą posługuje się funkcja \textsc{Zwijanie}, działająca zgodnie z~pseudokodem~\ref{alg_ckx_gf}.
  Problem ten polegał na zastosowaniu dekompozycji wyznaczającej korony w~grafie, zaimplementowanej zgodnie z~wytycznymi opisywanymi w~pracy~\cite{KernelizationAlgorithms04} i działająca zgodnie z~pseudokodem~\ref{alg_findCrown} w kontekście odszukiwania pseudokoron.
  Proponowane jest zbadanie wpływu innego rodzaju NT--dekompozycji na poprawność działania algorytmu \textsc{Zwijanie}.
  Kod źródłowy zawiera kilka przykładowych implementacji NT--dekompozycji uzyskiwanych w~oparciu na rozwiązaniu sformułowania problemu jako zadania programowania liniowego.

  Usprawnienie logiki związanej z~funkcją \textsc{Zwijanie} spowoduje, że implementacja algorytmu Chena, Kanji oraz Xia będzie bardziej wydajna.
  Dalsze usprawnienia będą dotyczyć części algorytmu zajmującej się utrzymywaniem kolejki priorytetowej struktur, która usuwa krotki o~wartości $q = 0$ oraz wykonuje algorytm siłowy w~sytuacji, gdy algorytm główny dociera do punktu, gdzie zachodzi $k \leq 7$.

  \subsubsection{\textbf{Badania eksperymentalne dla różnych implementacji procedur składowych algorytmu Chena, Kanji oraz Xia pominiętych w~pracy źródłowej.}}

  Omawiany algorytm jest skomplikowany i żmudny w~analizie.
  Duża złożoność esencjonalna problemu prowadzi do zastosowania pewnych skrótów myślowych w~treści pracy~\cite{ImprovedBounds10}.
  Poczynione przez autorów założenia mówiące, że algorytm niejawnie wykonuje pewne operacji znacząco utrudniają implementację, a w~szczególności implementację zoptymalizowaną.
  Operacje te w~praktyce są niebagatelnymi kwestiami i~stanowią pole do dalszych badań i optymalizacji pozwalających na poprawę średniej złożoności obliczeniowej stworzonego rozwiązania.

  \subsubsection{\textbf{Poprawa wydajności czasowej implementacji techniki redukcji dziedziny w~oparciu o~algoretm przepływu w~sieci.}}

  Pozytywny wpływ technik redukcji dziedziny do jądra problemu pokrycia wierzchołkowego jest widoczny i~zaświadcza o~poprawności implementacji.
  Uzyskane wyniki testów wydajnościowych pokazują jednak, że implementacje opisywanych technik zrealizowane w~ramach niniejszej pracy wymagają optymalizacji w~celu uzyskania wydajności bliższej teoretycznej.
  W szczególności implementacja algorytmu redukcji opartej o~przepływ w~sieci boryka się z problemami wydajnościowymi w~związku z~zastosowanym sposobem wyznaczania zbioru $R$, zawierającego wierzchołki osiągalne ze zbioru $S$ przez $M$-przemienne ścieżki.
  Proponowane jest usprawnienie tej części kodu przez zastosowanie odpowiedniego, ogólnie przyjętego za efektywny i~szybki algorytmu pozwalającego określić osiągalność wierzchołka w~grafie począwszy od danego innego wierzchołka.

  \subsubsection{\textbf{Poprawa złożoności pamięciowej.}}

  Cel pracy stanowiła analiza i~implementacja omówionych algorytmów.
  Zaproponowane rozwiązania mogą być zoptymalizowane również w kontekście zużycia pamięci.
  Przedkładając czytelność kodu nad zużycie pamięci, uwagę skupiono na odzwierciedleniu wyników pośrednich w~postaci zmiennych oraz jak najłatwiejszym korzystaniu ze struktur danych, na których opierają się implementowane techniki.
  W załączonej implementacji wszelkie struktury inicjowane są z maksymalną pesymistyczną pojemnością względem analizowanego grafu, nie biorąc pod uwagę na przykład wierzchołków czy krawędzi usuniętych z~grafu.
  Proponowaną optymalizacją jest stworzenie wersji kodu oszczędniej korzystającej ze zmiennych, a także bardziej konserwatywnie rezerwującej potrzebną pamięć.
  Na uwadze warto mieć również wykorzystanie przez język Go mechanizmu odśmiecania pamięci --- implementacje algorytmów muszą być skonstruowane w~sposób sprzyjający jak najszybszym 

  \subsubsection{\textbf{Zbadanie wpływu różnych wariantów NT--dekompozycji na szybkość działania i~efektywność zaimplementowanych algorytmów.}}

  Zastosowane w~niniejszej pracy uściślenie operacji NT--redukcji nie jest oparte bezpośrednio na rozwiązaniu zadania programowania liniowego, lecz na iteracyjnym poruszaniu się po skojarzeniu grafu w~celu wyznaczenia korony.
  Badania eksperymentalne uwidoczniły pewne problemy spowodowane tym podejściem.
  Istnieje potencjał poprawy efektywności algorytmów redukujących dziedzinę w bezpośrednim oparciu na koncepcji NT--dekompozycji.
  W ramach niniejszej pracy zaimplementowano kilka przykładowych NT--dekompozycji (NT--dekompozycja wg. Bar-Yehudy oraz NT--dekompozycja wg. J.F. Bussa).
  Jako kierunek dalszych prac proponowane jest przeprowadzenie badań eksperymentalnych wpływu zastosowania poszczególnych NT--dekompozycji w~algorytmie redukcji koron oraz algorytmie Chena, Kanji oraz Xia na efektywność zawężania dziedziny grafu.

\section*{Podsumowanie i~wnioski}
\addtocounter{section}{1}
  W~ramach niniejszej pracy omówiono oraz zaimplementowano pięć algorytmów służących do redukcji złożoności czasowej rozwiązania problemu pokrycia wierzchołkowego grafu. Przeprowadzono badania szybkości działania dla czterech z~nich.
  Przypadki testowe obejmowały wyznaczanie pokrycia wierzchołkowego i~działanie algorytmów redukcji dziedziny do jądra problemu na niezmodyfikowanym grafie wejściowym oraz na grafie poddanym uprzedniemu wykonaniu algorytmu przetwarzania wstępnego.

  W oparciu o~wyniki badań eksperymentalnych określenie najszybszego spośród zaimplementowanych algorytmów nie jest możliwe.
  Głównym powodem jest skupienie się na poprawności czytelności stworzonych rozwiązań, pozostawiając szybkość i zużycie pamięci na drugim planie, co opisano w~podrozdziale~\ref{s_improvements}.
  Algorytm rozwiązujący przeformułowanie problemu jako zadania przepływu w~sieci wykazuje największą efektywność w redukcji dzedziny --- czas wyznaczania pokrycia wierzchołkowego w~grafie zredukowanym omawianą techniką był najniższy spośród wszystkich metod dla większości przypadków testowych.
  Pomimo koncepcyjnej poprawności i~wysokiej efektywności w~redukcji dziedziny, czas jej działania w~porównaniu do pozostałych metod jest znacznie dłuższy, co oznacza, że stworzone w~ramach niniejszej pracy rozwiązanie nie jest zalecane do użytku praktycznego.
  W przypadku algorytmu redukcji koron, ciężko mówić o uniwersalności tego podejścia.
  Wiąże się to z~uzależnieniem szybkości działania tej metody od bardzo charakterystycznych struktur występujących w~grafie.
  W związku ze sposobem generacji losowych grafów testowych, były one ubogie w~te struktury, co spowodowało spadek efektywności zawężania dziedziny przez algorytm redukcji koron, widoczny na charakterystykach porównujących czas wyznaczania pokrycia wierzchołkowego dla grafu przetworzonego poszczególnymi metodami.

  Mimo napotkanych problemów, drogą analizy i~badań eksperymentalnych udało się wykazać pozytywny wpływ omawianych technik na złożoność czasową rozwiązania problemu pokrycia wierzchołkowego dla grafów ogólnych.

  Oprócz zaproponowanych implementacji, badań oraz~analizy zostały również zaproponowane usprawnienia i~potencjalne rozszerzenia, które zwiększą jakość oraz poprawią wydajność omawianych metod.
  Interesujące z~perspektywy dalszych badań jest zastosowanie innych wariantów koncepcji wykorzystywanych przez zaimplementowane algorytmy --- takich jak NT--dekompozycja --- w celu zbadania wpływu ich zmiany na złożoność czasową algorytmów oraz ich efektywność w redukcji dziedziny grafu.