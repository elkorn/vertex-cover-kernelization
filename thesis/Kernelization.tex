\section{Techniki zawężania dziedziny}\label{Section_kernelization}

Proces zawężania dziedziny realizowany jest przed właściwym poszukiwaniem
pokrywy wierzchołkowej. Polega on na usunięciu z dziedziny problemu
wierzchołków, które z~pewnością nie należą lub z~pewnością należą do optymalnej
pokrywy wierzchołkowej grafu wejściowego o~rozmiarze $k\prime \leq k$.
Dzięki parametryzacji maksymalną wartością oczekiwanego rozmiaru wyniku,
istnieje możliwość zakończenia przetwarzania z~odpowiedzią negatywną bez
angażowania właściwej logiki wyszukiwania pokrywy wierzchołkowej, co pozytywnie
wpływa na średnią złożoność obliczeniową procesu.

Opisane algorytmy zawężające dziedzinę problemu są od siebie niezależne.
Wynikiem działania każdego z~nich jest zbiór wierzchołków nietrywialnych do
rozpatrzenia z~perspektywy podjęcia decyzji o~przynależności do optymalnej
pokrywy wierzchołkowej.

\subsection{Usuwanie węzłów wysokiego stopnia}\label{section_kernelization_high-degree}

\begin{theorem}
  Każdy wierzchołek $v; d(v) > k $ musi należeć do optymalnej pokrywy wierzchołkowej 
  $VC; \|VC\| \leq k$.
\end{theorem}
\begin{proof}
  W~celu uzyskania pokrywy wierzchołkowej podgrafu $G\prime=(V\prime,E\prime)$
  grafu $G=(V,E)$, gdzie $V\prime=\{v_0, v_1, \ldots, v_k, v_{k+1}\}, V\prime
  \subseteq V$ oraz \\
  $E\prime=\{(v_0,v_1), (v_0,v_2), \ldots, (v_0, v_k), (v_0,v_{k+1})\}$,
  należy pokryć każdą krawędź $e \in E\prime$.
  Łatwo zauważyć, iż \\ jeżeli $ve_1=\{v_0,v_1\}, ve_2=\{v_0,v_2\}, \ldots,
  ve_k=\{v_0,v_k\},ve_{k+1}=\{v_0,v_{k+1}\}$,
  to ${VC_1=ve_1 \cup ve_2 \cup \ldots \cup ve_k \cup ve_{k+1}}$ spełnia warunki 
  wymagane do uzyskania statusu pokrywy wierzchołkowej, jednak $\|VC_1\| = k +1$.\\
  Jeżeli $VC_2=ve_1 \cap ve_2 \cap \ldots \cap ve_k \cap ve_{k+1}; VC_2 \neq \emptyset$,
  stwierdzić można, iż $VC_2$ nadal stanowi pokrywę wierzchołkową $G\prime$ oraz
  $\|VC_2\|=1$.
  Gdyby usunąć wierzchołek należący do $VC_2$, $VC_3=ve_1 \oplus ve_2 \oplus \ldots \oplus ve_k \oplus ve_{k+1}$ nadal
  również stanowi pokrywę wierzchołkową, jednak $\|VC_3\|=k$, a~zatem $VC_3$ nie
  może należeć do pokrywy wierzchołkowej o~rozmiarze $k\prime \leq k$.
\end{proof}

\begin{theorem}
  Procedura usuwania węzłów wysokiego stopnia zwraca właściwy wynik w czasie
  $O(n^2)$.
\end{theorem}
\begin{proof}
  ...
\end{proof}
