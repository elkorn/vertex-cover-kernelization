\section{Techniki redukcji dziedziny do jądra problemu}\label{s_kernelization}

Proces redukcji dziedziny do jądra problemu realizowany jest przed wyznaczaniem pokrycia wierzchołkowego. 
Polega on na usunięciu z dziedziny problemu wierzchołków, które z~pewnością nie należą lub z~pewnością należą do optymalnego pokrycia wierzchołkowego $C$ grafu wejściowego o~rozmiarze $|C| = k^\prime \leq k$.
Dzięki parametryzacji maksymalną wartością oczekiwanego rozmiaru wyniku istnieje możliwość zakończenia przetwarzania odpowiedzią negatywną bez
angażowania algorytmu wyznaczającego pokrycie wierzchołkowe, co pozytywnie
wpływa na średnią złożoność obliczeniową procesu.

Opisane algorytmy zawężające dziedzinę problemu są od siebie niezależne.
Wynikiem działania każdego z~nich jest zbiór wierzchołków nietrywialnych do
rozpatrzenia z~perspektywy podjęcia decyzji o~przynależności do optymalnego
pokrycia wierzchołkowego.

\subsection{Usuwanie węzłów wysokiego stopnia}\label{section_kernelization_high-degree}

\begin{theorem}
  Każdy wierzchołek $v; d(v) > k$ musi należeć do optymalnej pokrywy wierzchołkowej 
  $VC; |VC| \leq k$.
\end{theorem}
\begin{bproof}
  W~celu uzyskania pokrywy wierzchołkowej podgrafu $G\prime=(V\prime,E\prime)$
  grafu $G=(V,E)$, gdzie $V\prime=\{v_0, v_1, \ldots, v_k, v_{k+1}\}, V\prime
  \subseteq V, d(v_0)=k$ oraz \\
  $E\prime=\{(v_0,v_1), (v_0,v_2), \ldots, (v_0, v_k), (v_0,v_{k+1})\}$,
  należy pokryć każdą krawędź $e \in E\prime$.
  Łatwo zauważyć, iż \\ jeżeli $ve_1=\{v_0,v_1\}, ve_2=\{v_0,v_2\}, \ldots,
  ve_k=\{v_0,v_k\},ve_{k+1}=\{v_0,v_{k+1}\}$,
  to ${VC_1=ve_1 \bigcup ve_2 \bigcup \ldots \bigcup ve_k \bigcup ve_{k+1}}$ spełnia warunki 
  wymagane do uzyskania statusu pokrywy wierzchołkowej, jednak $|VC_1| = k +1$.\\
  Jeżeli $VC_2=ve_1 \bigcap ve_2 \bigcap \ldots \bigcap ve_k \bigcap ve_{k+1}; VC_2 \neq \emptyset$,
  stwierdzić można, iż $VC_2$ nadal stanowi pokrywę wierzchołkową $G\prime$ oraz
  $|VC_2|=1$.
  Gdyby usunąć wierzchołek należący do $VC_2$, $VC_3=ve_1 \oplus ve_2 \oplus \ldots \oplus ve_k \oplus ve_{k+1}$ nadal
  również stanowi pokrywę wierzchołkową, jednak $|VC_3|=k$.
  Dowolny zbiór $VC_4=VC_3 \setminus \{v\}$ nie spełnia jednak warunków pokrywy
  wierzchołkowej $G\prime$.
  Na tej podstawie stwierdzić można, iż dowolna pokrywa wierzchołkowa 
  $VC, |VC| \leq k$ grafu $G$ niezawierająca $v_0$, musi zawierać całe jego
  sąsiedztwo. Pokrywa ta nie może być jednak optymalna dla $k > 1$.
  Obserwacja ta prowadzi do wniosku, iż optymalna pokrywa wierzchołkowa
  $|VC_{opt}|$ grafu $G$ musi zawierać każdy wierzchołek $v \in V, d(v) \geq k$.
\end{bproof}

\begin{theorem}
  Procedura usuwania wierzchołków wysokiego stopnia realizowana jest w czasie
  $O(n^2)$.
\end{theorem}
\begin{bproof}
  Aby określić stopień dowolnego wierzchołka $v$ w grafie $G=(V,E)$, należy 
  dokonać $O(n), n=|V|$ porównań w celu wyznaczenia jego sąsiedztwa.
  Operacja musi zostać zrealizowana $n$ razy w celu określenia stopnia
  wszystkich wierzchołków $G$, co w rezultacie daje złożoność $O (n^2)$.
\end{bproof}

Zastosowanie algorytmu usuwania wierzchołków wysokiego stopnia w połączeniu 
z~technikami przetwarzania wstępnego ogranicza rozmiar dziedziny problemu przez
fakt, iż każdy wierzchołek $v, v \in V\prime$ jest stopnia $d(v)$ takiego, iż
$3 \leq d(v) \leq k\prime$.

\begin{theorem}
  Jeżeli mianem $G\prime$ określa się graf o pokrywie wierzchołkowej rozmiaru
  $k\prime$, który nie zawiera wierzchołka $v, d(v) > k\prime \lor d(v) > 3$, to
  wtedy $n\prime \leq \frac{k\prime^2}{3} + k\prime$.
\end{theorem}
\begin{bproof}
  Przyjąć należy ${VC,|VC|=k\prime}$ jako pokrywę wierzchołkową grafu
  $G\prime$.
  Dopełnienie $\overline{VC}$ zbioru $VC$ stanowi niezależny zbiór
  $n\prime-k\prime$ wierzchołków.
  Przyjąć należy zbiór $F=\{f_0,f_1, \ldots, f_p\}, f \in F \Rightarrow f \in E\prime$
  krawędzi pokrytych przez $\overline{VC}$.
  W związku z faktem, iż $\forall_{v \in \overline{VC}}{d(v) \geq 3}$, każdy
  wierzchołek $v \in \overline{VC}$ musi posiadać przynajmniej 3 wierzchołki sąsiednie
  $w \in C$.
  Prowadzi to do wniosku, iż $|F| \geq 3(n\prime - k\prime)$.
  Liczba krawędzi pokrytych przez $C$ nie może być mniejsza niż $|F|$ --- nie
  może być również większa niż $k\prime|C|$, ponieważ po wykonaniu
  przetwarzania wstępnego oraz procedury usuwania węzłów wysokiego stopnia,
  $\forall_{v \in V}{|N(v)|\leq~k\prime}$.
  Pamiętając, że $|C|=k\prime$, można stwierdzić, iż
  ${3(n\prime-k\prime)\leq|F|\leq~k\prime^2}$, co sprowadza się do
  ${n\prime\leq\frac{k\prime^2}{3}+k\prime}$.
\end{bproof}



\subsection{Sformułowanie problemu pokrycia wierzchołkowego jako problemu programowania liniowego}\label{section_kernelization_lp_formulation}

Proces rozwiązywania problemu pokrycia wierzchołkowego może być rozwiązany
przy zastosowaniu heurystyki zaproponowanej w pracy~\cite{hochbaum82}, opartej 
na~programowaniu całkowitoliczbowym.
Opisywany algorytm~\cite[rozdz.~4.2.2]{abukhzam03} korzysta z~ heurystyki proponowanej w~pracy~\cite{hochbaum82} 
w~następujący sposób.

\subsubsection{\textbf{Rozwiązanie problemu oryginalnego}}\label{ss_lp_original}
Każdemu wierzchołkowi $u \in V$ grafu $G=(V,E)$ należy przypisać wartość $X_u
\in \{0, 1\}$, z zachowaniem następujących własności:
\begin{enumerate}
  \item $\sum_{u \in V}X_u = \min$,
  \item $\{u,v\} \in E \implies X_u + X_v \geq 1$.
\end{enumerate}

Funkcja celu programu liniowego zwraca dolną granicę rozmiaru pokrycia wierzchołkowego $|C|$.
Zbiór rozwiązań prawdopodobnych składa się z~funkcji $V \to \{0, 1\}$,
spełniających warunek 2.
W związku z faktem, iż programowanie całkowitoliczbowe samo w sobie stanowi
problem $\mathcal{NP}$-zupełny, dokonać należy relaksacji do postaci programu liniowego, co
zapewni szerszy zakres prawdopodobnych rozwiązań.

W~pracy~\cite{khuller02} zaproponowana została relaksacja przez zamianę wartości 
$X_u \in \{0,1\}$ na $X_u \geq 0$.
Należy zauważyć, iż wartość $O_{LP}$ (ang. \emph{Linear Programming, LP}) zwracana przez rozwiązanie postaci 
liniowej jest zawsze ograniczona z dołu przez wartość $O_{\textnormal{IP}}$ zwracaną przez 
rozwiązanie postaci całkowitoliczbowej.
Co więcej, w~pracy~\cite{khuller02} udowodniono, że $O_{\textnormal{IP}} \leq 2*O_{LP}$.
Zależność ta wynika z twierdzenia Nemhausera--Trottera korzystającego
z~własności, że w~dowolnym ekstremum rozwiązania relaksacji programu
całkowitoliczbowego do postaci liniowej zmienne przyjmują wartość 
$X_u \in \{0, \frac{1}{2}, 1\}$.

Definiując $V_0 = \{u : X_u=0\}, V_{\frac{1}{2}}=\{u: X_u=\frac{1}{2}\},
V_1=\{u: X_u=1\}$, twierdzenie zapisać można jak następuje.

\begin{theorem}[Pierwsze twierdzenie Nemhausera--Trottera]\thlabel{nt_lp}
  Istnieje optymalne rozwiązanie $O$ sformułowania problemu pokrycia wierzchołkowego jako problemu programowania liniowego o następujących właściwościach.
  \begin{enumerate}[(a)]
    \item $O \subset V_1 \cup V_{\frac{1}{2}}$.
    \item $V_1 \subset O$.
  \end{enumerate}
\end{theorem}
W celu sprowadzenia powyższej relaksacji do przypadku rozwiązania
sparametryzowanego problemu pokrycia wierzchołkowego, zdefiniować należy zbiór 
$\{ X_u : u \in V \}$ zawierający wartości przypisywane wierzchołkom grafu 
$G=(V,E)$ przez funkcję celu oraz zbiory:\\
\begin{align*}
P&=\left\{u \in V | X_u>\frac{1}{2}\right\},\\
Q&=\left\{u \in V | X_u=\frac{1}{2}\right\},\\
R&=\left\{u \in V | X_u<\frac{1}{2}\right\}.
\end{align*}
Istotą redukcji dziedziny problemu do jądra jest dołączenie wszystkich
wierzchołków $u_P \in P$ do częściowego pokrycia wierzchołkowego $C$ oraz 
usunięcie z~niego wszystkich wierzchołków $u_R \in R$.
Graf wynikowy $G^\prime=(V^\prime, E^\prime)$ indukowany jest elementami $Q$: 
zbiorem wierzchołków $V^\prime=Q$ oraz zbiorem krawędzi\\$E^\prime=\{e=(v, w)| e \in E, \{v, w\} \in Q\}$.
\begin{theorem}
  Istnieje optymalne pokrycie wierzchołkowe $C \in G$, dla którego spełnione są własności $P \subset C$ oraz $C \cap R = \emptyset$.
\end{theorem}
\begin{bproof}
  Należy założyć pewne rozwiązanie całkowitoliczbowego sformułowania problemu 
  pokrycia wierzchołkowego $O_{\textnormal{IP}}$ oraz zbiory 
  ${A = P \setminus O_{\textnormal{IP}}, B = R \cap O_{\textnormal{IP}}}$.
  Zachodzi $N(B) \cap Q = \emptyset$, co zapewnia właściwość 2 sformułowania, którego rozwiązaniem jest $O_{\textnormal{IP}}$.

  Jeżeli $|A|<|B|$, to zastąpienie zbioru $B$ przez $A$ w rozwiązaniu $O_{\textnormal{IP}}$ (ang. \emph{Integer Programming, IP}) spowodowałoby odkrycie przynajmniej jednej krawędzi grafu --- wykluczając tym samym tak otrzymane pokrycie wierzchołkowe jako rozwiązanie.
  W prypadku gdy $|A|>|B|$, musiałaby istnieć możliwość otrzymania rozwiązania sformułowania liniowego lepszego niż $O_{\textnormal{IP}}$ przez ustanowienie $\epsilon = \min\{X_v-\frac{1}{2}: v \in A\}$, a~następnie
  zastąpienie $\forall{u \in B}:X_u \leftarrow X_u + \epsilon$ oraz $\forall{v \in A}: X_v \leftarrow X_v -\epsilon$.
  Jest to niemożliwe, ponieważ wynik $O_{\textnormal{IP}}$ stanowi optymalne rozwiązanie sformułowania liniowego w~oparciu o~pierwsze twierdzenie Nemhausera--Trottera~(\ref{nt_lp}).
  Nasuwa się konkluzja, że jedyny przypadek z~jakim można mieć w tym miejscu do czynienia to $|A|=|B|$.
  Przypadek ten jest trywialny --- w~celu orzymania optymalnego pokrycia wierzchołkowego wystarczy zastąpić zbiór $A$ zbiorem $B$.
\end{bproof}
Prezentowany algorytm redukuje dziedzinę do jądra problemu o~rozmiarze $n^\prime=|V|-|P|-|R|$.
Wartość wynikowa parametru określającego maksymalny rozmiar optymalnego pokrycia wierzchołkowego zmniejszona zostaje do $k^\prime=k-|P|$.
\begin{theorem}
  Nie istnieje optymalne pokrycie wierzchołkowe $C^\prime_{\textnormal{OPT}}\in G^\prime$ o rozmiarze $|C^\prime_{\textnormal{OPT}}|>\Sigma_{u\in Q}X_u=\frac{|Q|}{2}$.
\end{theorem}
\begin{bproof}
  Zauważmy, że rozmiar funkcji celu sformułowania liniowego ogranicza od dołu rozmiar funkcji celu sformułowania całkowitoliczbowego.
  W przeciwnym wypadku, procedura rozwiązująca początkowe sformułowanie liniowe problemu, którego wynik stanowi zbiór $Q$, nie byłaby w stanie zapewnić optymalnego rozwiązania, co byłoby sprzeczne z założeniami sformułowania.
\end{bproof}
\par{
  W świetle powyższego dowodu widać, że można zakończyć działanie procesu poszukiwania pokrycia wierzchołkowego $C_{\textnormal{OPT}}$ o liczebności $|C_{\textnormal{OPT}}|\leq k$, udzielając odpowiedzi negatywnej gdy $|Q|>2k^\prime$.
  Warto dodać, że powyższe sformułowanie algorytmu jest niepraktyczne dla grafów o~dużym zagęszczeniu ze względu na liczbę warunków ograniczających sformułowania równą $|E|$.
  Właściwym podejściem do takich przypadków jest przekształcenie problemu z~minimalizacyjnego do dualnego problemu maksymalizacyjnego, w~którym liczba warunków ograniczających równa będzie $|V|$.
}
\subsubsection{\textbf{Rozwiązanie problemu dualnego}}
\par{
  Ponieważ koszt dowolnego prawdopodobnego rozwiązania problemu
  dualnego do oryginalnego sformułowania liniowego problemu pokrycia
  wierzchołkowego~(\ref{ss_lp_original}) stanowi dolną granicę dla optimum
  (\ref{ss_lp_original}) poprzez słabą dualność. 
  Konstrukcja sformułowania liniowego dualnego problemu maksymalizacyjnego wygląda
  następująco.\\
  Każdej krawędzi $e=(u,v) \in E$ grafu $G=(V,E)$ należy przypisać wartość
  $Y_{(u,v)} \geq 0$, z zachowaniem następujących własności:
  \begin{enumerate}
    \item $\sum_{(u,v) \in E}Y_{(u,v)} = \max$,
    \item $\forall_{v \in V}:\sum_{u:(u,v)\in E}Y_{(u,v)} \leq 1$,
    \item $\forall_{(u,v) \in E}: Y_{(u,v)} \geq 0$.
  \end{enumerate}
}
\subsection{Formulacja problemu jako instancji przepływu w sieci}\label{Kernelization_network_flow}
Algorytm ten, zaproponowany w~\cite{KernelizationAlgorithms04}, opiera się na
algorytmie użytym w~\cite{Niedermeier02} do udowodnienia twierdzenia
Neumhausera-Trottera.

\begin{nt}\thlabel{theorem_nt}
  Dla grafu $G=(V,E), |V|=n, |E|=m$, dwa rozłączne zbiory $C_0 \subseteq V,
  V_0 \subseteq V$ mogą zostać oblicone w czasie $O(\sqrt{n}m)$ przy zachowaniu
  następujących własności.
  \begin{enumerate}
    \item Dla podgrafu $G[V_0]$ założyć należy istnienie pokrywy $D \subseteq
      V_0$. W następstwie, $C := D \bigcup C_0$ stanowi pokrywę wierzchołkową~$G$.
    \item Istnieje optymalna pokrywa wierzchołkowa $S$ grafu $G$ z $C_0
      \subseteq S$.
    \item Podgraf $G[V_0]$ posiada optymalną pokrywę wierzchołkową rozmiaru co
      najmniej $\frac{[V_0]}{2}$. 
  \end{enumerate}
\end{nt}

Algorytm definiuje graf dwudzielny $B$ na podstawie grafu $G$, odnajduje pokrywę
wierzchołkową $VC_B$ grafu $B$ poprzez odnalezienie maksymalnego skojarzenia $B$,
a następnie przydziela wartości wierzchołkom $G$ w oparciu o przynależność do
$VC_B$.
Implementacja algorytmu z~\cite{Niedermeier02} zrealizowana jest przez
przekształcenie $B$ w instancję problemu przezpływu w sieci i rozwiązania tejże
przy pomocy algorytmu Forda-Fulkersona.\footnote{Algorytm Forda-Fulkersona
  zastosowany został w niniejszej pracy, oryginalna implementacja oparta została
o algorytm Dinica.}
Złożoność czasowa algorytmu wynosi $O(\sqrt{n}m)$, zgodnie z~twierdzeniem
Neumhausera-Trottera. 
W związku z faktem, iż w grafie może istnieć maksymalnie~$n^2$ krawędzi, można
wyrazić tę złożoność~w~formie $O(n^\frac{5}{2})$.
Rozmiar zredukowanej dziedziny problemu ograniczony jest do $2k$.

\begin{enumerate}
  \item Przekształć graf $G=(V,E)$ w graf dwudzielny $H=(U,F)$ zgodnie z
    następującymi zasadami:\\
    $A=\{A_v|v \in V\}\\
    B=\{B_v|v \in V\}\\
    U=A_v \bigcup B_v
    F=\{(A_v, B_v)|(v,u) \in E \lor (u,v) \in E\}$
  \item Przekształć graf dwudzielny $H$ w graf przepływu w sieci $H\prime$:
    \begin{itemize}
      \item dodaj węzeł źródłowy $v_s$, połączony z każdym wierzchołkiem $v_a
        k\in A$ krawędziami skierowanymi $(v_s, v_a)$,
      \item dodaj węzeł docelowy $v_z$, połączony z każdym wierzchołkiem $v_b
        \in B$ krawędziamiy skierowanymi $(v_b, v_z)$,
      \item wszystkie krawędzie $f \in F$ skieruj $(v_a, v_b)$,
      \item każdej krawędzi $h \in H\prime$ nadaj pojemność $c(f)=1$.
    \end{itemize}
  \item Znajdź maksymalny przepływ $MF$ w $H\prime$.
  \item Zbiór $M=MF \bigcap F$ stanowi maksymalne skojarzenie $H$.
  \item Znajdź pokrywę wierzchołkową $H$, bazując na $M$.
    \begin{itemize}
      \item Jeżeli $\forall_{v in U}{v \in M}$, pokrywę
        wierzchołkową~stanowi całość zbioru $A$ lub $B$.
      \item Przy licznościach zbiorów $A$, $B$ oraz wagach krawędzi w $H\prime$
        wiadomo, iż ${\forall_{v_A in A}{v_A \in M} \iff \forall_{v_B in B}{v_B \in M}}$.
        Na tej podstawie można, stwierdzić, że $\exists_{v_A \in A}{v_A \notin
        M}$.
        Skonstruuj zatem 3 zbiory $S$, $R$ oraz $T$ wierzchołków. Zbiór
        $S = \{v_{Au}|v_{Au} \in A \land v_{Au} \notin M\}$ zawiera wszystkie
        nieskojarzone wierzchołki ze zbioru $A$.
        $R$ stanowi zbiór wszystkich wierzchołków $v_A \in A$ osiąglnych z $S$
        poprzez M-przemienne ścieżki. \\
        $T=\{v_T|v_T \in N(R), v_R \in R, ((v_R,v_M) \in M \lor (v_M,v_R)) \in M\}$ 
        jest zbiorem zawierającym wierzchołki sąsiednie względem $R$ wzdłuż 
        ścieżek zawartych w skojarzeniu $M$.
        Pokrywę wierzchołkową grafu dwudzielnego $H$ stanowi zbiór 
        ${VC=(A \setminus S \setminus R) \bigcup T}, |VC|=|M|$.
    \end{itemize}
  \item Przypisz wagi wszystkim wierzchołkom $v \in V$ w odniesieniu do $VC$:
    \begin{itemize}
      \item $\{A_v, B_v\} \in VC \Rightarrow W_v=1$,
      \item $A_v \in VC \land B_v \notin VC \lor A_v \notin VC \land B_v \in
        VC \Rightarrow W_v=0.5$,
      \item $\{A_v, B_v\} \notin VC \Rightarrow W_v=0$
    \end{itemize}
    W przypadku 1., wszystkim wierzchołkom nadać należy wagę $W_v=0.5$.
  \item Zdefiniuj graf wynikowy jako 
    $G\prime=(V\prime, E\prime), V\prime=\{v \in V|W_v=0.5\}$.
    Wynikowy rozmiar dziedziny problemu zdefiniuj jako 
    ${k\prime=k-x, x=|\{v\in~V|W_v=1\}}|$.
\end{enumerate}

Utworzenie grafu dwudzielnego jest wartościowe z punktu widzenia problemu
pokrycia wierzchołkowego poprzez korelację maksymalnego dopasowania w grafie
dwudzielnym z optymalną pokrywą wierzchołkową. 
Zależność ta sformułowana została jako twierdzenie K\"oniga:

\begin{konig*}
  W dowolnym grafie dwudzielnym, ilość krawędzi zawarta w maksymalnym
  dopasowaniu jest równa rozmiarowi optymalnej pokrywy wierzchołkowej tego
  grafu.
\end{konig*}

\begin{theorem}\label{theorem_nf1}
  Wynikiem kroku 5.\ algorytmu jest poprawna pokrywa wierzchołkowa $VC$
  grafu $H$.
\end{theorem}
\begin{bproof}
  (Dla przypadku 1.) \\
  $A = VC \oplus B = VC$.
  Na tej podstawie stwierdzić można, iż ${|A|=|B|=|M|}$.
  Przyjmując $VC = A$, $\forall{f=(u,v), f\in F}: u \in VC \oplus v \in VC$, co
  w konsekwencji oznacza, iż każda krawędź $f$ jest pokryta przez $VC$, czyniąc
  $VC$ prawidłową pokrywą wierzchołkową.
\end{bproof}
\begin{bproof}
  (Dla przypadku 2.) \\
  Istnieją zbiory $S, R \subset T$ oraz $T \subset B$ oraz istnieje pokrywa 
  wierzchołkowa \\ $VC=(A \setminus S \setminus R) \bigcup T$.
  ${\forall{e=(x,y), e \in E}: x \in S \oplus x \in R \oplus x \in (A \setminus S
  \setminus R)}$.


  Każdy z przypadków rozpatrywany jest osobno:
  \begin{itemize}
    \item \underline{$x \in S$}: $x$ jest nieskojarzony---jeżeli $M$ ma być skojarzeniem 
      maksymalnym, $y$ musi być skojarzony.
      Prowadzi to do wniosku, iż $\exists{e_M=(w,y)}: e_M \in M$.
      Na podstawie wytycznych algorytmu wiadomo, iż w takiej sytuacji $w \in R$
      oraz, co ważniejsze, $y \in T$---oznacza to, że $e$ jest pokryta przez $M$.
    \item \underline{$x \in R$}: $\exists{e_M=(x,w), e_M\in M}: w \in T$. \\
      $w=y \implies y \in T$.
      W przeciwnym wypadku, gdy $w \neq y$, pewnym jest, że $\exists{e_M=(z,w),
      z \in R \oplus z \in S}: e_M \in M$.
      Dodatkowo wiadomo,\\że ${\exists_{e_{M2}=(v,y)}: e_{M2} \in M}$.
      Jeżeli ta zależność miałaby nie być spełniona, oznaczałoby to, iż zbiór $M$
      nie stanowi maksymalnego skojarzenia---zamiast krawędzi $(x,w)$ musiałby
      zawierać krawędzie $\{(x,y),(z,w)\}$.
      W efekcie $v \in R, y \in T$, tak więc krawędź $e$ jest pokryta przez $VC$.
    \item \underline{$x \in A \setminus S \setminus R$}: Przypadek trywialny,
      pokrycie krawędzi $e$ wynika z definicji samej pokrywy $VC$.
  \end{itemize}
\end{bproof}
\begin{theorem}
  Pokrywa stanowiąca wynik kroku 5.\ algorytmu jest rozmiaru $|VC| = |M|$. 
\end{theorem}
\begin{bproof}
  Z definicji, $|S| = |V| - |M|; |A \setminus S|=|(A \setminus S
  \setminus R) \bigcup R|=|M|$.\\
  Na podstawie faktu, iż każdy wierzchołek $v_R in R$ jest skojarzony oraz
  każdy z wierzchołków $v_T \in T$ jest osiągalny z $R$ przez ścieżki złożone z
  krawędzi $e_M \in M$ stwierdzić można, że $|T|=|R|$.
  To prowadzi do wniosku: $|(A\setminus S\setminus R)\bigcup T|=|((A \setminus
  S \setminus R) \bigcup R)|=|M|$.
\end{bproof}
\begin{theorem}\label{theorem_nf2}
  Pokrywa stanowiąca wynik kroku 5.\ algorytmu jest optymalna.
\end{theorem}
\begin{bproof}
  Graf $H$ jest grafem dwudzielnym, a rozmiar jego maksymalnego skojarzenia
  wynosi $|M|$.
  W oparciu o twierdzenie K\"oniga, rozmiar optymalnej pokrywy wierzchołkowej
  grafu $H$ równy jest liczebności jego maksymalnego skojarzenia.
\end{bproof}
\begin{theorem}
  Wynik kroku 6.~algorytmu stanowi realne rozwiązanie formulacji problemu jako
  zagadnienia programowania liniowego.
\end{theorem}
\begin{bproof}
  Jednym z warunków formulacji problemu jako zadania programowania liniowego jest
  $\forall_{e=(u,v) \in E}: W_u + W_v \geq 1$.
  Krok 6.\ przypisuje wagi wierzchołkom grafu $G$: ${forall_{v \in V}: W_v \in
  \{0, 0.5, 1\}}$.
  W związku z charakterystyką przekształcenia grafu $G$ w graf $H$, 
  $(x,y) \in E \implies \{(A_x, B_y), (A_y, B_x)\} \in H$.
  W oparciu o~dowody twierdzeń~\ref{theorem_nf1} i~\ref{theorem_nf2} można
  stwierdzić, iż jeżeli przynajmniej jeden z wierzchołków każdej krawędzi $f \in
  F$ zawarty jest w pokrywie $VC$, to $(\{A_x, B_x\} \in VC) \oplus (\{A_y,
  B_y\} \in VC) \oplus (\{A_x, B_y\} \in VC) \oplus (\{A_y, B_x\} \in VC)$.
  Widać zatem, że każda krawędź $e$ ma przypisaną prawidłową wagę.
\end{bproof}

\subsection{Redukcja koron}
\label{ss_kernelization_crown_reduction}
\par{
  Pojęcie \emph{korony grafu} spopularyzowane zostało przede wszystkim dzięki dorobkowi naukowemu M. Fellowsa oraz F. Abu-Khzam.
  Niniejsza praca czerpie z~literatury tychże autorów, głównie z~pracy~\cite{KernelizationAlgorithms04}.
  Znaczenie koron w~grafach jest kluczowe szczególnie dla parametryzowanego wariantu problemu pokrycia wierzchołkowego --- specyficzna budowa tej struktury otwiera drogę do efektywnej redukcji dziedziny poszukiwań do jądra problemu pokrycia wierzchołkowego w~grafie.
  Wiele algorytmów wiodących prym pod względem złożoności obliczeniowej opiera się w dużym stopniu właśnie na identyfikacji oraz przetwarzaniu koron lub struktur podobnych koronom.
  Prócz algorytmu analizowanego w~niniejszym podrozdziale, podjęto się również analizy i~implementacji innego algorytmu wykorzystującego struktury koron, zaproponowanego w~pracy~\cite{ImprovedBounds10}, któremu poświęcono podrozdział~\ref{s_ckx}.
}
\subsubsection{\textbf{Kontekst struktur koron w~grafach}}
\label{sss_kernelization_crown_context}
\par{
  Bardzo ważna dla opisywanej koncepcji jest praca~\cite{chlebik:crown}, gdzie określono miejsce koron wśród szerszej klasy tak zwanych \emph{struktur zmniejszających zaangażowanie} w grafach.
  \begin{definition}
    Dla grafu $G=(V, E)$ i pewnego podzbioru $U \subseteq V$ ze zbiorem wierzchołków sąsiednich $N(U)$ \emph{strukturę zmniejszającą zaangażowanie} stanowi uporządkowana para zbiorów $(I \subseteq V, N(I) \subseteq V)$ spełniająca następujące własności.
    \begin{enumerate}
      \item $I \neq \emptyset$ stanowi niezależny zbiór wierzchołków w~grafie $G$.
      \item $N(I)$ stanowi optymalne pokrycie wierzchołkowe grafu indukowanego z~grafu $G$ zbiorem wierzchołków $I \cup N(I)$.
    \end{enumerate}
  \end{definition}
  Możliwość identyfikacji struktury $(I, N(I))$ zmniejszającej zaangażowanie w~grafie $G=(V, E)$ jest ważna ze względu na to, że każdy podzbiór $C=N(I)\cup C^\prime$, gdzie $C^\prime$ stanowi optymalne pokrycie wierzchołkowe dla grafu $G[V \setminus (I\cup N(I))]$ stanowi optymalne pokrycie wierzchołkowe grafu $G$.
  Zmniejszenie zaangażowania za pomocą struktury $(I, N(I))$ polega na zaangażowaniu algorytmu korzystającego z~tejże struktury w~odnajdywanie wyłącznie rozwiązań $C$ spełniających własność $C \cap (I \cup N(I)) = N(I)$, usuwając zbiór $I \cup N(I)$ z dziedziny poszukiwań w grafie $G$, redukując tym samym egzepmlarz problemu pokrycia wierzchołkowego do pomniejszonego grafu $G[V \setminus (I \cup N(I))]$.
}
\par{
  Charakterystycznym rodzajem struktur zmiejszających zaangażowanie w~grafie są tak zwane \emph{NT--dekompozycje}, stanowiące wynik \emph{NT--redukcji} --- operacji opartej na sformułowaniu problemu pokrycia wierzchołkowego jako egzemplarza problemu programowania liniowego i~rozwiązania go zgodnie z~Twierdzeniem~\ref{nt_lp}\footnote{Sformułowanie problemu jako relaksacji liniowej jest jednym ze sposobów podejścia do identyfikacji struktur zmniejszających zaangażowanie, istnieją również inne sposoby --- jednakże wspólnym mianownikiem wszystkich tych metod są korzenie sięgające twierdzenia~\ref{nt_lp}.}
  \begin{definition}\thlabel{def_nt_decomposition}.
    \emph{NT--dekompozycja} (dekompozycja Nemhausera--Trottera) stanowi specjalny przypadek struktury $(I, N(I))$ zmniejszającej zaangażowanie w~grafie $G=(V, E)$ zidentyfikowanej przez zastosowanie procesu \emph{NT--redukcji} (redukcji Nemhausera--Trottera).
    Proces ten posługuje się funkcją $x: V \rightarrow \left\{0, \frac{1}{2}, 1\right\}$ różną od $x \equiv \frac{1}{2}$, która określa wagę wierzchołków grafu w~kontekście przynależności do optymalnego pokrycia wierzchołkowego.
  \end{definition}
  Przyjmując $V_i^x=\{u \in V| x(u)=i\}$ dla każdej wartości $i\in \{0, \frac{1}{2}, 1\}$, zbiór $V_0^x$ jest niezależny i~niepusty w~grafie $G$, zachowana również zostaje własność $V_1^x = N(V_0^x)$.
  Z twierdzenia~\ref{nt_lp} wynika, że zbiór $V_1^x$ stanowi optymalne pokrycie wierzchołkowe grafu $G[V_0^x \cup N(V_0^x)]$.
  Prowadzi to do wniosku, że para $(V_0^x, N(V_0^x))$ stanowi prawidłową strukturę zmniejszającą zaangażowanie w~grafie $G$.
  Zaangażowanie zostaje zmniejszone do zbioru rozwiązań mających część wspólną ze zbiorem $V_0^x \cup V_1^x$ w zbiorze $V_1^x$, a~przestrzeń poszukiwań zredukowana zostaje do grafu $G[V_\frac{1}{2}^x]$.
}
\par{
  Autorzy pracy~\cite{chlebik:crown} obalają stwierdzenie jakoby koncepcja koron miała być ortogonalna względem NT--redukcji --- postulują, że korony stanowią wyspecjalizowaną podklasę NT--dekompozycji o~właściwościach opisanych w~następującym podrozdziale.
}
\subsubsection{\textbf{Właściwości koron}}
\label{sss_kernelization_crown_main}
\begin{definition}\thlabel{def_crown}
  \emph{Koronę} grafu $G=(V, E)$ stanowi uporządkowana para podzbiorów wierzchołków $(I \subseteq V, H \subseteq V)$ o następujących właściwościach.
  \begin{enumerate}
    \item $I \neq \emptyset$ stanowi zbiór niezależny w~grafie $G$.
    \item $H=N(I)$.
    \item Istnieje skojarzenie $M=\{e_0, e_1, \ldots, e_p\}$, dla którego zachodzi $\forall_{e_M=(u,v) \in M}: (u\in I \land v\in H) \lor (u \in H \land v \in I)$ oraz $\forall_{v_h \in H}\exists_{e_M=(u,v)\in M}: u = v_h \oplus v = v_h$.
    \item Spełniona jest nierówność $|H| \leq |I|$. (własność przechodnia z~własności 3)
  \end{enumerate}
\end{definition}
\begin{definition}
  Zbiór $H$ stanowi \emph{głowę korony}.
\end{definition}
\begin{definition}\thlabel{def_crown_head}
  \emph{Szerokość korony} stanowi liczebność zbioru $|H|$.
\end{definition}
\begin{definition}\thlabel{def_strict_crown}
  Korona \emph{ścisła} jest szczególnym rodzajem korony $(I, H)$, dla której spełniona jest nierówność $|H| < |I|$.
\end{definition}
\begin{definition}\thlabel{def_equal_crown}
  Korona \emph{równa} jest szczególnym rodzajem korony $(I, H)$, dla której spełniona jest równość $|H| = |I|$.
\end{definition}
\begin{definition}\thlabel{def_trivial_crown}
  Egzemplarz korony $(I, H)$, gdzie zbiory składowe $I$ oraz $H$ są puste jest \emph{trywialny}.
  Wynika to z~braku korzyści płynących z~wykorzystania takiego egzemplarza do redukcji przestrzeni poszukiwań pokrycia wierzchołkowego.
\end{definition}
\begin{definition}\thlabel{th_crown_free_graph}
  Graf $G$, w którym wyznaczyć można jedynie trywialny egzemplarz korony stanowi graf~\emph{wolny od koron} lub po prostu graf nieposiadający koron.
\end{definition}
\begin{theorem}\thlabel{th_crown_vc}
  Jeżeli graf $G=(V,E)$ zawiera koronę $(I,H)$, to istnieje takie optymalne pokrycie wierzchołkowe $C_{\textnormal{OPT}} \subseteq V$, że $H \subseteq C_{\textnormal{OPT}}$ oraz $I \not\subseteq C_{\textnormal{OPT}}$.
\end{theorem}
\begin{bproof}
  Z własności 3\ Definicji~\ref{def_crown}\ wynika, że każde pokrycie wierzchołkowe $C$ musi zawierać przynajmniej jeden wierzchołek $v_H \in H$.
  Na tej podstawie stwierdzić można, że $|C|\geq|H|$.
  Taki rozmiar pokrycia osiągnąć można przez zastąpienie zbioru $C$ zbiorem $C\cup H$.
  Należy w~tym miejscu oznaczyć, że wierzchołki $v_H$ są użyteczne w~kontekście możliwości pokrywania krawędzi $e \notin M$, podczas gdy wierzchołki $v_I \in
  I$ nie mają tej cechy.
  Łatwo zatem zauważyć, że $|C \cup H| \leq |C \cup I|$.
  Wniosek płynący z~tej obserwacji jest jednoznaczny: istnieje optymalne pokrycie wierzchołkowe $C_{\textnormal{OPT}}$ zawierające zbiór $H$ i wykluczające zbiór $I$.
\end{bproof}
W celu wyznaczenia korony w~grafie zastosować można algorytm działający zgodnie z~pseudokodem~\ref{alg_findCrown}.
Rezultatem działania algorytmu jest korona $(I,H)$, na którą składają się zbiory $I=I_N$ oraz $H=H_N$.
\begin{algorithm}
  \caption{Algorytm wyznaczający koronę w~grafie $G$}\label{alg_findCrown}
  \begin{algorithmic}[1]
    \Function{WyznaczKorone}{$G$, $k$}

    \algorithmicrequire{graf $G$, maksymalna liczebność $k$ pokr. wierzch.}

    \algorithmicensure{korona $(I, H)$}

    \State{$M_1\gets$ największe skojarzenie $G$}
    \State $O \gets \emptyset$
    \ForAll{$v \in V$}
      \If{$\neg\exists_{(u, w) \in M_1}: u=v \lor w=v$}
        \State $O \gets O \cup \{v\}$
      \EndIf
    \EndFor
    \If{$|M_1| \geq k$} \Comment{nie istnieje pokr. wierzch. $C$ o liczebności $|C| \leq k$}
      \State\textbf{return} nil
    \EndIf
  \State{$M_2 \gets$ maksymalne skojarzenie na krawędziach $O\leftrightarrow N(O)$}
  \If{$|M_2| > k$} \Comment{nie istnieje pokr. wierzch. $C$ o liczebności $|C| \leq k$}
    \State{\textbf{return} nil}
  \EndIf
  \State{$I_0 \gets \{v_O|v_O\in O, \neg\exists_{(u,v)\in M_2}: u=v_O\lor v=v_O\}$}
  \State($n \gets 0$)
  \While{$I_{n-1} \neq I_n$}\label{findCrown_while}
    \State{$H_n \gets N(I_n)$}\label{findCrown_makeH}
    \State{$I_{n+1} \gets I_n\cup N_{M_2}(H_n)$}\label{findCrown_makeI}
    \State{$n \gets n+1$}
  \EndWhile\label{findCrown_endWhile}
  \State $N \gets n$
  \State{\textbf{return} $(I_N,H_N)$}
  \EndFunction
\end{algorithmic}
\end{algorithm}
\begin{theorem}
  Algorytm~\ref{alg_findCrown}\ wyznacza koronę pod warunkiem, że $I_0\neq\emptyset$.
\end{theorem}
\begin{bproof} Udowodnione zostaną trzy kolejne własności wymienione w~Definicji~\ref{def_crown}.
  \begin{enumerate}
    \item Ponieważ $M_1$ stanowi największe skojarzenie w~grafie $G$, zbiory $O$ oraz $I \subset O$ są niezależne.
    \item  Z definicji wynika, że $H=N(I_{N-1})$.
      Z warunku zakończenia pętli~\algref{alg_findCrown}{findCrown_while} wynika, że $I=I_N=I_{N-1}$.
      Na tej podstawie mamy $H=N(I)$.
    \item Założyć należy istnienie elementu $h \in H$ takiego, że dla dowolnego wierzchołka $u \in V$ nie istnieje krawędź $(u, h) \in E$ ani $(h, u) \in E$ skojarzona przez zbiór $M_2$ w grafie $G$.
    Rezultatem budowy $H$ byłaby zatem ścieżka rozszerzająca $P$ o nieparzystej długości. 
    Warunkiem przynależności $h \in H$ jest istnienie nieskojarzonego wierzchołka $v_O \in O$ stanowiącego początek tejże ścieżki.
    W takim przypadku wynikiem wiersza~\ref{findCrown_makeH}\ algorytmu byłaby zawsze krawędź nieskojarzona, podczas gdy wynikiem wiersza~\ref{findCrown_makeI}\ byłaby krawędź stanowiąca część~skojarzenia.
    Proces ten powtarzałby się do momentu osiągnięcia wierzchołka $h$.
    Utworzona ścieżka prowadzi między dwoma nieskojarzonymi wierzchołkami, będąc zarazem $M_2$-przemienną.
    Istnienie takiej ścieżki oznaczałoby możliwość zwiększenia skojarzenia $M_2$ przez wykonanie operacji $M_2=M_2\oplus P$, co stoi w~opozycji do założenia, że $M_2$ stanowi skojarzenie maksymalne.
    Obserwacja ta prowadzi do stwierdzenia, że każdy wierzchołek $h \in H$ musi być skojarzony w $M_2$.
    Skojarzenie użyte w~strukturze korony to skojarzenie $M_2$ z dziedziną ograniczoną do krawędzi pomiędzy wierzchołkami należącymi do zbiorów $H$ oraz $I$.
  \end{enumerate}
\end{bproof}
\par{
  Rezultatem jednej iteracji algorytmu redukcji korony (wiersze~\algref{alg_findCrown}{findCrown_while} -- \algref{alg_findCrown}{findCrown_endWhile}) jest graf
  $G^\prime=(V^\prime, E^\prime)$ składający się ze zbiorów $V^\prime=V\setminus H \setminus I$ oraz $E^\prime = E \setminus \{H\leftrightarrow I\}$, gdzie zbiór $\{H\leftrightarrow I\}$ zawiera krawędzie pomiędzy wierzchołkami należącymi do zbiorów $H$ oraz $I$.
  Rozmiar dziedziny problemu ulega zmniejszeniu do wartości $n^\prime=n-|I|-|H|$, natomiast wartość parametru spada do $k^\prime=k-|H|$, ponieważ każdy z~wierzchołków $h \in H$ musi należeć do optymalnego pokrycia wierzchołkowego, co udowodniono dla twierdzenia~\ref{th_crown_vc}.
  Zauważmy, że jeżeli w~grafie istnieje maksymalne skojarzenie $M_{\textnormal{MAX}}$ o rozmiarze $|M_{\textnormal{MAX}}| > k$, to wykluczone jest istnienie optymalnego pokrycia wierzchołkowego $C_{\textnormal{OPT}}$ o liczebności $|C_{\textnormal{OPT}}|\leq k$.
  Jeżeli więc rozmiar dowolnego z~odnalezionych skojarzeń $M_1, M_2$ jest większy niż $k$, to algorytm może zakończyć działanie, udzielając odpowiedzi negatywnej --- czyli zwrócić trywialny (zgodnie z~Definicją~\ref{def_trivial_crown}) egzemplarz korony.
  Zależność ta pozwala również zdefiniować górną granicę rozmiaru grafu wynikowego $|G^\prime|$.
}
\begin{theorem}\thlabel{th_crown_domain_reduction}
  Jeżeli utworzone w~ramach algorytmu~\ref{alg_findCrown} skojarzenia $M_1$ oraz $M_2$ zawierają co najwyżej $k$ krawędzi, to zbiór wierzchołków $D=V \setminus I \setminus H$, stanowiący jądro egzemplarza problemu pokrycia wierzchołkowego jest rozmiaru $|D| \leq 3k$.
\end{theorem}
\begin{bproof}
  Ponieważ skojarzenie $M_1$ o rozmiarze $|M_1| \leq k$ stanowi zbiór krawędzi, to zbiór wierzchołków tychże krawędzi $V_{M_1}=\{v, u|v, u \in V, (u,v)\in M_1 \lor (v,u) \in M_1\}$ musi być rozmiaru $ |V_{M_1}| \leq 2k$ --- z tego wynika, że $|O| \geq n-2k$.
  Ponieważ mamy $|M_2| \leq k$, istnieje co najwyżej $k$ wierzchołków $v_O \in O$ skojarzonych przez $M_2$.
  Łatwo zauważyć, że w~takim przypadku istnieje co najmniej $n-3k$ wierzchołków $v_O \in O$ nieskojarzonych przez $M_2$ --- są one zawarte w~zbiorze $I_0$, a~zatem także w~zbiorze~$I$.
  Ten tok rozumowania prowadzi do wniosku, że rozmiar zbioru wierzchołków stanowiącego jądro egzemplarza problemu pokrycia wierzchołkowego wynosi $|V \setminus I \setminus H| \leq 3k$.
\end{bproof}
\par{
  Kształt odnalezionej przez algorytm korony jest podyktowany strukturą wybranego największego skojarzenia $M_1$.
  Rozsądnym krokiem jest również wykonanie jednej iteracji algorytmów przetwarzania wstępnego przed każdą kolejną iteracją redukcji koron --- usunięcie korony z~dużym prawdopodobieństwem prowadzić będzie do powstania wierzchołków niskiego stopnia, redukowalnych za pomocą przetwarzania wstępnego.
  Oznacza to, że pożądane jest wykonywanie algorytmu redukcji korony wielokrotnie, wykorzystując różne największe skojarzenia tak, by zidentyfikować i~zredukować jak największą liczbę koron, co pozwoli na maksymalne zawężenie dziedziny problemu.
  Najbardziej złożoną obliczeniowo częścią algorytmu jest wyznaczenie maksymalnego skojarzenia $M_2$, zrealizowane w~niniejszej pracy za pomocą algorytmu kwiatów Edmondsa\footnote{
    Oryginalna implementacja, przedstawiona w~pracy~\cite{KernelizationAlgorithms04}, oparta jest o~przeformułowanie problemu pokrycia wierzchołkowego do egzemplarza problemu przepływu w~sieci, rozwiązanego za pomocą algorytmu Dinica, o~wynikowej złożoności czasowej $O(n^{5/2})$.
  }.
}
\begin{theorem}
  Implementacja algorytmu redukcji korony grafu $G=(V,E)$ o rozmiarach $|V|=n$ oraz $|E|=m$ w~oparciu o~algorytm kwiatów Edmondsa wyznacza koronę w~czasie $O(n^{4})$.
\end{theorem}
\begin{bproof}
  Z~teoretycznego punktu widzenia, dwie najbardziej obciążające operacje to wyznaczenie w~grafie maksymalnego skojarzenia $M_2$ oraz odnalezienie największego skojarzenia $M_1$.
  Aby wyznaczyć największe skojarzenie $M_1$, należy sprawdzić wszystkie krawędzie $e\in E$ w celu poszukiwania wspólnych wierzchołków.
  Wykorzystując koncepcję oznaczania wierzchołków krawędzi dołączanych do skojarzenia jako odwiedzonych, złożoność operacji sprowadza się do $O(m)$.
  W grafie może znajdować się maksymalnie $O(n^{2})$ krawędzi.
  W celu odnalezienia maksymalnego skojarzenia zastosowano algorytm Edmondsa, którego złożoność wynosi $O(n^{4})$, co opisano w~podrozdziale~\ref{ss_edmonds}.
  Powyższe obserwacje prowadzą do wniosku, iż algorytm redukcji korony wyznacza koronę w~grafie $G$ w~czasie $O(n^{4} + n^{2})=O(n^{4})$.
\end{bproof}


