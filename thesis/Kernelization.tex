\section{Techniki zawężania dziedziny}\label{Section_kernelization}

Proces zawężania dziedziny realizowany jest przed właściwym poszukiwaniem
pokrywy wierzchołkowej. Polega on na usunięciu z dziedziny problemu
wierzchołków, które z~pewnością nie należą lub z~pewnością należą do optymalnej
pokrywy wierzchołkowej grafu wejściowego o~rozmiarze $k\prime \leq k$.
Dzięki parametryzacji maksymalną wartością oczekiwanego rozmiaru wyniku,
istnieje możliwość zakończenia przetwarzania z~odpowiedzią negatywną bez
angażowania właściwej logiki wyszukiwania pokrywy wierzchołkowej, co pozytywnie
wpływa na średnią złożoność obliczeniową procesu.

Opisane algorytmy zawężające dziedzinę problemu są od siebie niezależne.
Wynikiem działania każdego z~nich jest zbiór wierzchołków nietrywialnych do
rozpatrzenia z~perspektywy podjęcia decyzji o~przynależności do optymalnej
pokrywy wierzchołkowej.

\subsection{Usuwanie węzłów wysokiego stopnia}\label{section_kernelization_high-degree}

\begin{theorem}
  Każdy wierzchołek $v; d(v) > k$ musi należeć do optymalnej pokrywy wierzchołkowej 
  $VC; \|VC\| \leq k$.
\end{theorem}
\begin{proof}
  W~celu uzyskania pokrywy wierzchołkowej podgrafu $G\prime=(V\prime,E\prime)$
  grafu $G=(V,E)$, gdzie $V\prime=\{v_0, v_1, \ldots, v_k, v_{k+1}\}, V\prime
  \subseteq V, d(v_0)=k$ oraz \\
  $E\prime=\{(v_0,v_1), (v_0,v_2), \ldots, (v_0, v_k), (v_0,v_{k+1})\}$,
  należy pokryć każdą krawędź $e \in E\prime$.
  Łatwo zauważyć, iż \\ jeżeli $ve_1=\{v_0,v_1\}, ve_2=\{v_0,v_2\}, \ldots,
  ve_k=\{v_0,v_k\},ve_{k+1}=\{v_0,v_{k+1}\}$,
  to ${VC_1=ve_1 \bigcup ve_2 \bigcup \ldots \bigcup ve_k \bigcup ve_{k+1}}$ spełnia warunki 
  wymagane do uzyskania statusu pokrywy wierzchołkowej, jednak $\|VC_1\| = k +1$.\\
  Jeżeli $VC_2=ve_1 \bigcap ve_2 \bigcap \ldots \bigcap ve_k \bigcap ve_{k+1}; VC_2 \neq \emptyset$,
  stwierdzić można, iż $VC_2$ nadal stanowi pokrywę wierzchołkową $G\prime$ oraz
  $\|VC_2\|=1$.
  Gdyby usunąć wierzchołek należący do $VC_2$, $VC_3=ve_1 \oplus ve_2 \oplus \ldots \oplus ve_k \oplus ve_{k+1}$ nadal
  również stanowi pokrywę wierzchołkową, jednak $\|VC_3\|=k$.
  Dowolny zbiór $VC_4=VC_3 \setminus \{v\}$ nie spełnia jednak warunków pokrywy
  wierzchołkowej $G\prime$.
  Na tej podstawie stwierdzić można, iż dowolna pokrywa wierzchołkowa 
  $VC, \|VC\| \leq k$ grafu $G$ niezawierająca $v_0$, musi zawierać całe jego
  sąsiedztwo. Pokrywa ta nie może być jednak optymalna dla $k > 1$.
  Obserwacja ta prowadzi do wniosku, iż optymalna pokrywa wierzchołkowa
  $\|VC_{opt}\|$ grafu $G$ musi zawierać każdy wierzchołek $v \in V, d(v) \geq k$.
\end{proof}

\begin{theorem}
  Procedura usuwania wierzchołków wysokiego stopnia realizowana jest w czasie
  $O(n^2)$.
\end{theorem}
\begin{proof}
  Aby określić stopień dowolnego wierzchołka $v$ w grafie $G=(V,E)$, należy 
  dokonać $O(n), n=\|V\|$ porównań w celu wyznaczenia jego sąsiedztwa.
  Operacja musi zostać zrealizowana $n$ razy w celu określenia stopnia
  wszystkich wierzchołków $G$, co w rezultacie daje złożoność $O (n^2)$.
\end{proof}

Zastosowanie algorytmu usuwania wierzchołków wysokiego stopnia w połączeniu z
technikami przetwarzania wstępnego ogranicza rozmiar dziedziny problemu przez
fakt, iż każdy wierzchołek $v, v \in V\prime$ jest stopnia $d(v)$ takiego, iż
$3 \leq d(v) \leq k\prime$.

\begin{theorem}
  Jeżeli mianem $G\prime$ określa się graf o pokrywie wierzchołkowej rozmiaru
  $k\prime$, który nie zawiera wierzchołka $v, d(v) > k\prime \lor d(v) > 3$, to
  wtedy $n\prime \leq \frac{k\prime^2}{3} + k\prime$.
\end{theorem}
\begin{proof}
  Przyjąć należy ${VC,\|VC\|=k\prime}$ jako pokrywę wierzchołkową grafu
  $G\prime$.
  Dopełnienie $\overline{VC}$ zbioru $VC$ stanowi niezależny zbiór
  $n\prime-k\prime$ wierzchołków.
  Przyjąć należy zbiór $F=\{f_0,f_1, \ldots, f_p\}, f \in F \Rightarrow f \in E\prime$
  krawędzi pokrytych przez $\overline{VC}$.
  W związku z faktem, iż $\forall_{v \in \overline{VC}}{d(v) \geq 3}$, każdy
  wierzchołek $v \in \overline{VC}$ musi posiadać przynajmniej 3 wierzchołki sąsiednie
  $w \in C$.
  Prowadzi to do wniosku, iż $\|F\| \geq 3(n\prime - k\prime)$.
  Liczba krawędzi pokrytych przez $C$ nie może być mniejsza niż $\|F\|$ --- nie
  może być również większa niż $k\prime\|C\|$, ponieważ po wykonaniu
  przetwarzania wstępnego oraz procedury usuwania węzłów wysokiego stopnia,
  $\forall_{v \in V}{\|N(v)\|\leq~k\prime}$.
  Pamiętając, że $\|C\|=k\prime$, można stwierdzić, iż
  ${3(n\prime-k\prime)\leq\|F\|\leq~k\prime^2}$, co sprowadza się do
  ${n\prime\leq\frac{k\prime^2}{3}+k\prime}$.
\end{proof}

\subsection{Formulacja problemu jako instancji przepływu w sieci}\label{Kernelization_network_flow}
Algorytm ten, zaproponowany w~\cite{KernelizationAlgorithms04}, opiera się na
algorytmie użytym w~\cite{Niedermeier02} do udowodnienia teorii
Neumhausera-Trottera.

\begin{nt*}\thlabel{theorem_nt}
  Dla grafu $G=(V,E), \|V\|=n, \|E\|=m$, dwa rozłączne zbiory $C_0 \subseteq V,
  V_0 \subseteq V$ mogą zostać oblicone w czasie $O(\sqrt{n}m)$ przy zachowaniu
  następujących własności.
  \begin{enumerate}
    \item Dla podgrafu $G[V_0]$ założyć należy istnienie pokrywy $D \subseteq
      V_0$. W następstwie, $C := D \bigcup C_0$ stanowi pokrywę wierzchołkową~$G$.
    \item Istnieje optymalna pokrywa wierzchołkowa $S$ grafu $G$ z $C_0
      \subseteq S$.
    \item Podgraf $G[V_0]$ posiada optymalną pokrywę wierzchołkową rozmiaru co
      najmniej $\frac{[V_0]}{2}$. 
  \end{enumerate}
\end{nt*}

Algorytm definiuje graf dwudzielny $B$ na podstawie grafu $G$, odnajduje pokrywę
wierzchołkową $VC_B$ grafu $B$ poprzez odnalezienie największego skojarzenia $B$,
a następnie przydziela wartości wierzchołkom $G$ w oparciu o przynależność do
$VC_B$.
Implementacja algorytmu z~\cite{Niedermeier02} zrealizowana jest przez
przekształcenie $B$ w instancję problemu przezpływu w sieci i rozwiązania tejże
przy pomocy algorytmu Forda-Fulkersona.\footnote{Algorytm Forda-Fulkersona
  zastosowany został w niniejszej pracy, oryginalna implementacja oparta została
o algorytm Dinica.}
Złożoność czasowa algorytmu wynosi $O(\sqrt{n}m)$, zgodnie z~teoria
Neumhausera-Trottera.
Rozmiar zredukowanej dziedziny problemu ograniczony jest do $2k$.

\begin{enumerate}
  \item Przekształć graf $G=(V,E)$ w graf dwudzielny $H=(U,F)$ zgodnie z
    następującymi zasadami:\\
    $A=\{A_v|v \in V\}\\
    B=\{B_v|v \in V\}\\
    U=A_v \bigcup B_v
    F=\{(A_v, B_v)|(v,u) \in E \lor (u,v) \in E\}$
  \item Przekształć graf dwudzielny $H$ w graf przepływu w sieci $H\prime$:
    \begin{itemize}
      \item[-] dodaj węzeł źródłowy $v_s$, połączony z każdym wierzchołkiem $v_a
        k\in A$ krawędziami skierowanymi $(v_s, v_a)$,
      \item[-] dodaj węzeł docelowy $v_z$, połączony z każdym wierzchołkiem $v_b
        \in B$ krawędziamiy skierowanymi $(v_b, v_z)$,
      \item[-] wszystkie krawędzie $f \in F$ skieruj $(v_a, v_b)$,
      \item[-] każdej krawędzi $h \in H\prime$ nadaj pojemność $c(f)=1$.
    \end{itemize}
  \item Znajdź maksymalny przepływ $MF$ w $H\prime$.
  \item Zbiór $M=MF \bigcap F$ stanowi największe skojarzenie $H$.
  \item Znajdź pokrywę wierzchołkową $H$, bazując na $M$.
    \begin{itemize}
      \item[-] Jeżeli $\forall_{v in U}{v \in M}$, pokrywę
        wierzchołkową~stanowi całość zbioru $A$ lub $B$.
      \item[-] Przy licznościach zbiorów $A$, $B$ oraz wagach krawędzi w $H\prime$
        wiadomo, iż ${\forall_{v_A in A}{v_A \in M} \iff \forall_{v_B in B}{v_B \in M}}$.
        Na tej podstawie można, stwierdzić, że $\exists_{v_A \in A}{v_A \notin
        M}$.
        Skonstruuj zatem 3 zbiory $S$, $R$ oraz $T$ wierzchołków. Zbiór
        $S = \{v_{Au}|v_{Au} \in A \land v_{Au} \notin M\}$ zawiera wszystkie
        nieskojarzone wierzchołki ze zbioru $A$.
        $R$ stanowi zbiór wszystkich wierzchołków $v_A \in A$ osiąglnych z $S$
        poprzez M-przemienne ścieżki. \\
        $T=\{v_T|v_T \in N(R) \land v_R \in R \land ((v_R,v_M) \in M \lor (v_M,v_R)) \in M\}$ 
        jest zbiorem zawierającym wierzchołki sąsiednie względem $R$ wzdłuż 
        ścieżek zawartych w skojarzeniu $M$.
        Pokrywę wierzchołkową grafu dwudzielnego $H$ stanowi zbiór 
        ${VC=(A \setminus S \setminus R) \bigcup T}, \|VC\|=\|M\|$.
    \end{itemize}
  \item Przypisz wagi wszystkim wierzchołkom $v \in V$ w odniesieniu do $VC$:
    \begin{itemize}
      \item[-] $\{A_v, B_v\} \in VC \Rightarrow W_v=1$,
      \item[-] $A_v \in VC \land B_v \notin VC \lor A_v \notin VC \land B_v \in
        VC \Rightarrow W_v=0.5$,
      \item[-] $\{A_v, B_v\} \notin VC \Rightarrow W_v=0$
    \end{itemize}
    W przypadku 1., wszystkim wierzchołkom nadać należy wagę $W_v=0.5$.
  \item Zdefiniuj graf wynikowy jako 
    $G\prime=(V\prime, E\prime), V\prime=\{v \in V|W_v=0.5\}$.
    Wynikowy rozmiar dziedziny problemu zdefiniuj jako 
    ${k\prime=k-x, x=\|\{v\in~V|W_v=1\}}\|$.
\end{enumerate}

