\section{Techniki redukcji dziedziny do jądra problemu}\label{s_kernelization}

Proces redukcji dziedziny do jądra problemu realizowany jest przed wyznaczaniem pokrycia wierzchołkowego. 
Polega on na usunięciu z~dziedziny problemu wierzchołków, które z~pewnością nie należą lub z~pewnością należą do optymalnego pokrycia wierzchołkowego $C$ grafu wejściowego o~rozmiarze $|C| = k^\prime \leq k$.
Dzięki parametryzacji problemu maksymalną wartością oczekiwanego rozmiaru pokrycia wierzchołkowego istnieje możliwość zakończenia przetwarzania odpowiedzią negatywną bez
angażowania algorytmu wyznaczającego pokrycie wierzchołkowe, co pozytywnie wpływa na średnią złożoność obliczeniową procesu.

Opisane algorytmy zawężające dziedzinę problemu są od siebie niezależne.
Wynikiem działania każdego z~nich jest zbiór wierzchołków nietrywialnych do
rozpatrzenia z~perspektywy podjęcia decyzji o~przynależności do optymalnego
pokrycia wierzchołkowego.

\subsection{Usuwanie wierzchołków wysokiego stopnia}\label{section_kernelization_high-degree}

\begin{theorem}
  Każdy wierzchołek $v$ stopnia $d(v) > k$ musi należeć do optymalnego pokrycia wierzchołkowego 
  $C$ rozmiaru $|C| \leq k$.
\end{theorem}
\begin{bproof}
  W~celu uzyskania pokrycia wierzchołkowego podgrafu $G^\prime=(V^\prime,E^\prime)$
  grafu $G=(V,E)$, gdzie $V^\prime=\{v_0, v_1, \ldots, v_k, v_{k+1}\} \subseteq V$, $d(v_0)=k$ oraz \\
  $E^\prime=\{(v_0,v_1), (v_0,v_2), \ldots, (v_0,v_{k+1})\}$ należy pokryć każdą krawędź $e \in E^\prime$.
  Zakładając istnienie zbiorów \[V_1=\{v_0,v_1\}, V_2=\{v_0,v_2\}, \ldots,
  V_k=\{v_0,v_k\},V_{k+1}=\{v_0,v_{k+1}\}\]
  zbiór $C_1=V_1 \cup V_2 \cup \ldots \cup V_k \cup V_{k+1}$ stanowi pokrycie wierzchołkowe, jednak $|C_1| = k +1$.\\
  Jeżeli $C_2=V_1 \cap V_2 \cap \ldots \cap V_k \cap V_{k+1} \neq \emptyset$, to zbiór $C_2$ nadal stanowi pokrycie wierzchołkowe grafu $G^\prime$ o liczebności
  $|C_2|=1$.
  Gdyby usunąć wierzchołek należący do zbioru $C_2$, $C_3=V_1 \oplus V_2 \oplus \ldots \oplus V_k \oplus V_{k+1}$ nadal
  również stanowi pokrycie wierzchołkowe, jednak $|C_3|=k$.
  Dowolny zbiór $C_4=C_3 \setminus \{v\}$ nie spełnia jednak warunków pokrycia wierzchołkowego $G^\prime$.
  Na tej podstawie stwierdzić można, iż dowolne pokrycie wierzchołkowe grafu $G$ 
  $C$ o rozmiarze $|C| \leq k$ wykluczające wierzchołek $v_0$ musi zawierać całe jego
  sąsiedztwo. Pokrycie to nie może być jednak optymalne dla $k > 1$.
  Obserwacja ta prowadzi do wniosku, że optymalne pokrycie wierzchołkowe
  $C_{\textnormal{opt}}$ grafu $G$ musi zawierać każdy wierzchołek $v \in V$ stopnia $d(v) \geq k$.
\end{bproof}

\begin{theorem}
  Procedura usuwania wierzchołków wysokiego stopnia realizowana jest w~czasie $O(n^2)$.
\end{theorem}
\begin{bproof}
  Aby określić stopień dowolnego wierzchołka $v$ w grafie $G=(V,E)$ należy 
  dokonać $O(|V|)$ porównań w celu wyznaczenia jego sąsiedztwa.
  Operacja musi zostać zrealizowana $|V|$ razy w~celu określenia stopnia
  wszystkich wierzchołków $v \in V$, co w~rezultacie daje złożoność $O (|V|^2)$.
\end{bproof}

Zastosowanie algorytmu usuwania wierzchołków wysokiego stopnia w~połączeniu 
z~technikami przetwarzania wstępnego ogranicza rozmiar dziedziny problemu przez
to, że każdy wierzchołek $v \in V^\prime$ jest stopnia $3 \leq d(v) \leq k^\prime$.

\begin{theorem}
  Jeżeli $G^\prime$ stanowi graf o~pokryciu wierzchołkowym rozmiaru $k^\prime$, który nie zawiera wierzchołka $v$ stopnia $d(v) > k^\prime \lor d(v) > 3$, to
  wtedy zawiera on co najwyżej $\frac{k^{\prime2}}{3} + k^\prime$ wierzchołków.
\end{theorem}
\begin{bproof}
  Przyjąć należy $C$ jako pokrycie wierzchołkowe grafu $G^\prime$ rozmiaru $|C|=k^\prime$.
  Dopełnienie $\overline{C}$ zbioru $C$ stanowi niezależny zbiór
  $|V^\prime|-k^\prime$ wierzchołków.
  Przyjąć należy zbiór $F=\{f_0,f_1, \ldots, f_p\} \subseteq E^\prime$ krawędzi pokrytych przez $\overline{C}$.
  Ponieważ zachodzi $\forall_{v \in \overline{C}}{d(v) \geq 3}$, każdy
  wierzchołek $v \in \overline{C}$ musi mieć przynajmniej 3 wierzchołki sąsiednie
  $w \in C$.
  Prowadzi to do wniosku, iż $|F| \geq 3(|V^\prime| - k^\prime)$.
  Liczba krawędzi pokrytych przez zbiór $C$ nie może być mniejsza niż $|F|$ --- nie
  może być również większa niż $k^\prime|C|$, ponieważ po wykonaniu
  przetwarzania wstępnego oraz procedury usuwania wierzchołków wysokiego stopnia spełniona musi być własność $\forall_{v \in V}{|N(v)|\leq~k^\prime}$.
  Ponieważ $|C|=k^\prime$, można stwierdzić, iż zbiór $F$ jest rozmiaru ${3(|V^\prime|-k^\prime)\leq|F|\leq~k^{\prime2}}$, co daje ${|V^\prime|\leq\frac{k^{\prime2}}{3}+k^\prime}$.
\end{bproof}
\subsection{Sformułowanie problemu pokrycia wierzchołkowego jako problemu programowania liniowego}\label{section_kernelization_lp_formulation}

Proces rozwiązywania problemu pokrycia wierzchołkowego może być rozwiązany
przy zastosowaniu heurystyki zaproponowanej w pracy~\cite{hochbaum82}, opartej 
na~programowaniu całkowitoliczbowym.
Opisywany algorytm~\cite[rozdz.~4.2.2]{abukhzam03} korzysta z~ heurystyki proponowanej w~pracy~\cite{hochbaum82} 
w~następujący sposób.

\subsubsection{\textbf{Rozwiązanie problemu oryginalnego}}\label{ss_lp_original}
Każdemu wierzchołkowi $u \in V$ grafu $G=(V,E)$ należy przypisać wartość $X_u
\in \{0, 1\}$, z zachowaniem następujących własności:
\begin{enumerate}
  \item $\sum_{u \in V}X_u = \min$,
  \item $\{u,v\} \in E \implies X_u + X_v \geq 1$.
\end{enumerate}

Funkcja celu programu liniowego zwraca dolną granicę rozmiaru pokrycia wierzchołkowego $|C|$.
Zbiór rozwiązań prawdopodobnych składa się z~funkcji $V \to \{0, 1\}$,
spełniających warunek 2.
W związku z faktem, iż programowanie całkowitoliczbowe samo w sobie stanowi
problem $\mathcal{NP}$-zupełny, dokonać należy relaksacji do postaci programu liniowego, co
zapewni szerszy zakres prawdopodobnych rozwiązań.

W~pracy~\cite{khuller02} zaproponowana została relaksacja przez zamianę wartości 
$X_u \in \{0,1\}$ na $X_u \geq 0$.
Należy zauważyć, iż wartość $O_{LP}$ (ang. \emph{Linear Programming, LP}) zwracana przez rozwiązanie postaci 
liniowej jest zawsze ograniczona z dołu przez wartość $O_{\textnormal{IP}}$ zwracaną przez 
rozwiązanie postaci całkowitoliczbowej.
Co więcej, w~pracy~\cite{khuller02} udowodniono, że $O_{\textnormal{IP}} \leq 2*O_{LP}$.
Zależność ta wynika z twierdzenia Nemhausera--Trottera korzystającego
z~własności, że w~dowolnym ekstremum rozwiązania relaksacji programu
całkowitoliczbowego do postaci liniowej zmienne przyjmują wartość 
$X_u \in \{0, \frac{1}{2}, 1\}$.

Definiując $V_0 = \{u : X_u=0\}, V_{\frac{1}{2}}=\{u: X_u=\frac{1}{2}\},
V_1=\{u: X_u=1\}$, twierdzenie zapisać można jak następuje.

\begin{theorem}[Pierwsze twierdzenie Nemhausera--Trottera]\thlabel{nt_lp}
  Istnieje optymalne rozwiązanie $O$ sformułowania problemu pokrycia wierzchołkowego jako problemu programowania liniowego o następujących właściwościach.
  \begin{enumerate}[(a)]
    \item $O \subset V_1 \cup V_{\frac{1}{2}}$.
    \item $V_1 \subset O$.
  \end{enumerate}
\end{theorem}
W celu sprowadzenia powyższej relaksacji do przypadku rozwiązania
sparametryzowanego problemu pokrycia wierzchołkowego, zdefiniować należy zbiór 
$\{ X_u : u \in V \}$ zawierający wartości przypisywane wierzchołkom grafu 
$G=(V,E)$ przez funkcję celu oraz zbiory:\\
\begin{align*}
P&=\left\{u \in V | X_u>\frac{1}{2}\right\},\\
Q&=\left\{u \in V | X_u=\frac{1}{2}\right\},\\
R&=\left\{u \in V | X_u<\frac{1}{2}\right\}.
\end{align*}
Istotą redukcji dziedziny problemu do jądra jest dołączenie wszystkich
wierzchołków $u_P \in P$ do częściowego pokrycia wierzchołkowego $C$ oraz 
usunięcie z~niego wszystkich wierzchołków $u_R \in R$.
Graf wynikowy $G^\prime=(V^\prime, E^\prime)$ indukowany jest elementami $Q$: 
zbiorem wierzchołków $V^\prime=Q$ oraz zbiorem krawędzi\\$E^\prime=\{e=(v, w)| e \in E, \{v, w\} \in Q\}$.
\begin{theorem}
  Istnieje optymalne pokrycie wierzchołkowe $C \in G$, dla którego spełnione są własności $P \subset C$ oraz $C \cap R = \emptyset$.
\end{theorem}
\begin{bproof}
  Należy założyć pewne rozwiązanie całkowitoliczbowego sformułowania problemu 
  pokrycia wierzchołkowego $O_{\textnormal{IP}}$ oraz zbiory 
  ${A = P \setminus O_{\textnormal{IP}}, B = R \cap O_{\textnormal{IP}}}$.
  Zachodzi $N(B) \cap Q = \emptyset$, co zapewnia właściwość 2 sformułowania, którego rozwiązaniem jest $O_{\textnormal{IP}}$.

  Jeżeli $|A|<|B|$, to zastąpienie zbioru $B$ przez $A$ w rozwiązaniu $O_{\textnormal{IP}}$ (ang. \emph{Integer Programming, IP}) spowodowałoby odkrycie przynajmniej jednej krawędzi grafu --- wykluczając tym samym tak otrzymane pokrycie wierzchołkowe jako rozwiązanie.
  W prypadku gdy $|A|>|B|$, musiałaby istnieć możliwość otrzymania rozwiązania sformułowania liniowego lepszego niż $O_{\textnormal{IP}}$ przez ustanowienie $\epsilon = \min\{X_v-\frac{1}{2}: v \in A\}$, a~następnie
  zastąpienie $\forall{u \in B}:X_u \leftarrow X_u + \epsilon$ oraz $\forall{v \in A}: X_v \leftarrow X_v -\epsilon$.
  Jest to niemożliwe, ponieważ wynik $O_{\textnormal{IP}}$ stanowi optymalne rozwiązanie sformułowania liniowego w~oparciu o~pierwsze twierdzenie Nemhausera--Trottera~(\ref{nt_lp}).
  Nasuwa się konkluzja, że jedyny przypadek z~jakim można mieć w tym miejscu do czynienia to $|A|=|B|$.
  Przypadek ten jest trywialny --- w~celu orzymania optymalnego pokrycia wierzchołkowego wystarczy zastąpić zbiór $A$ zbiorem $B$.
\end{bproof}
Prezentowany algorytm redukuje dziedzinę do jądra problemu o~rozmiarze $n^\prime=|V|-|P|-|R|$.
Wartość wynikowa parametru określającego maksymalny rozmiar optymalnego pokrycia wierzchołkowego zmniejszona zostaje do $k^\prime=k-|P|$.
\begin{theorem}
  Nie istnieje optymalne pokrycie wierzchołkowe $C^\prime_{\textnormal{OPT}}\in G^\prime$ o rozmiarze $|C^\prime_{\textnormal{OPT}}|>\Sigma_{u\in Q}X_u=\frac{|Q|}{2}$.
\end{theorem}
\begin{bproof}
  Zauważmy, że rozmiar funkcji celu sformułowania liniowego ogranicza od dołu rozmiar funkcji celu sformułowania całkowitoliczbowego.
  W przeciwnym wypadku, procedura rozwiązująca początkowe sformułowanie liniowe problemu, którego wynik stanowi zbiór $Q$, nie byłaby w stanie zapewnić optymalnego rozwiązania, co byłoby sprzeczne z założeniami sformułowania.
\end{bproof}
\par{
  W świetle powyższego dowodu widać, że można zakończyć działanie procesu poszukiwania pokrycia wierzchołkowego $C_{\textnormal{OPT}}$ o liczebności $|C_{\textnormal{OPT}}|\leq k$, udzielając odpowiedzi negatywnej gdy $|Q|>2k^\prime$.
  Warto dodać, że powyższe sformułowanie algorytmu jest niepraktyczne dla grafów o~dużym zagęszczeniu ze względu na liczbę warunków ograniczających sformułowania równą $|E|$.
  Właściwym podejściem do takich przypadków jest przekształcenie problemu z~minimalizacyjnego do dualnego problemu maksymalizacyjnego, w~którym liczba warunków ograniczających równa będzie $|V|$.
}
\subsubsection{\textbf{Rozwiązanie problemu dualnego}}
\par{
  Ponieważ koszt dowolnego prawdopodobnego rozwiązania problemu
  dualnego do oryginalnego sformułowania liniowego problemu pokrycia
  wierzchołkowego~(\ref{ss_lp_original}) stanowi dolną granicę dla optimum
  (\ref{ss_lp_original}) poprzez słabą dualność. 
  Konstrukcja sformułowania liniowego dualnego problemu maksymalizacyjnego wygląda
  następująco.\\
  Każdej krawędzi $e=(u,v) \in E$ grafu $G=(V,E)$ należy przypisać wartość
  $Y_{(u,v)} \geq 0$, z zachowaniem następujących własności:
  \begin{enumerate}
    \item $\sum_{(u,v) \in E}Y_{(u,v)} = \max$,
    \item $\forall_{v \in V}:\sum_{u:(u,v)\in E}Y_{(u,v)} \leq 1$,
    \item $\forall_{(u,v) \in E}: Y_{(u,v)} \geq 0$.
  \end{enumerate}
}
\subsection{Sformułowanie problemu pokrycia wierzchołkowego jako egzemplarza problemu przepływu w~sieci}\label{Kernelization_network_flow}
Algorytm ten, zaproponowany w~pracy~\cite{KernelizationAlgorithms04}, opiera się na
algorytmie użytym w~pracy~\cite{Niedermeier02} do udowodnienia następującego twierdzenia
Nemhausera--Trottera.

\begin{theorem}[Drugie twierdzenie Nemhausera--Trottera]\thlabel{theorem_nt}
  Dla grafu $G=(V,E)$, gdzie $|V|=n$ i $|E|=m$, dwa rozłączne zbiory $C_0 \subseteq V$ oraz $V_0 \subseteq V$ mogą zostać wyznaczone w~czasie $O(\sqrt{n}m)$ przy zachowaniu następujących własności.
  \begin{enumerate}
    \item Dla podgrafu $G[V_0]$ należy założyć istnienie pokrycia wierzchołkowego $D \subseteq V_0$.
    Zbiór $C = D \cup C_0$ stanowi pokrycie wierzchołkowe~$G$.
    \item Istnieje optymalne pokrycie wierzchołkowe $S$ grafu $G$, dla którego zachodzi $S \supseteq C_0$.
    \item Podgraf $G[V_0]$ ma optymalne pokrycie wierzchołkowe rozmiaru co najmniej $|V_0|/2$. 
  \end{enumerate}
\end{theorem}

Algorytm buduje graf dwudzielny $B$ na podstawie grafu $G$, wyznacza pokrycie wierzchołkowe $C_B$ grafu $B$ przez odnalezienie maksymalnego skojarzenia bigrafu $B$,
a następnie przydziela wagi wierzchołkom $G$ w oparciu o~przynależność do $C_B$.
Implementacja algorytmu z~pracy~\cite{Niedermeier02} zrealizowana jest przez
przekształcenie zbioru $B$ w egzemplarz problemu przepływu w~sieci i~rozwiązania tegoż za pomocą algorytmu Forda-Fulkersona\footnote{Algorytm Forda-Fulkersona
  zastosowany został w niniejszej pracy, oryginalna implementacja wykorzystuje w~tym celu algorytm Dinica.}.
Złożoność czasowa algorytmu wynosi $O(\sqrt{n}m)$, zgodnie z~pierwszym twierdzeniem Nemhausera--Trottera. 
Ponieważ w grafie może istnieć maksymalnie~$n^2$ krawędzi, można wyrazić tę złożoność~w~formie $O(n^{5/2})$.
Rozmiar zredukowanej dziedziny problemu ograniczony jest do $2k$.

Algorytm redukujący dziedzinę poszukiwań do jądra problemu pokrycia wierzchołkowego poprzez sformułowanie problemu jako egzemplarza problemu przepływu w~sieci działa zgodnie z~następującym schematem.
\begin{enumerate}
  \item Przekształć graf $G=(V,E)$ w graf dwudzielny $H=(U,F)$ zgodnie z
    następującymi zasadami:\\
    $A=\{A_v|v \in V\}\\
    B=\{B_v|v \in V\}\\
    U=A_v \cup B_v
    F=\{(A_v, B_v)|(v,u) \in E \lor (u,v) \in E\}$
  \item Przekształć graf dwudzielny $H$ w graf przepływu w~sieci $H^\prime$:
    \begin{itemize}
      \item dodaj węzeł źródłowy $v_s$ połączony z~każdym wierzchołkiem $v_a
        k\in A$ krawędziami skierowanymi $(v_s, v_a)$,
      \item dodaj węzeł docelowy $v_z$, połączony z~każdym wierzchołkiem $v_b
        \in B$ krawędziamiy skierowanymi $(v_b, v_z)$,
      \item każdą krawędź $f=(v_a, v_b) \in F$ skieruj tak, by spełnić własność $v_a \in a~\land v_b \in B$,
      \item każdej krawędzi $h \in H^\prime$ nadaj pojemność $c(f)=1$.
    \end{itemize}
  \item Znajdź maksymalny przepływ $MF$ w sieci $H^\prime$.
  \item Zbiór $M=MF \cap F$ stanowi maksymalne skojarzenie $H$.
  \item Znajdź pokrycie wierzchołkowe $H$ na podstawie skojarzenia $M$.
    \begin{itemize}
      \item Jeżeli zachodzi $\forall_{v \in U}{v \in M}$, to pokrycie wierzchołkowe stanowi całość zbioru $A$ lub $B$.
      \item W~przeciwnym razie, przy licznościach zbiorów $A$ i $B$ oraz wagach krawędzi w~egzemplarzu problemu przepływu w~sieci $H^\prime$ dowolny wierzchołek $v_A \in A$ jest skojarzony przez zbiór $M$ wtedy i~tylko wtedy, gdy odpowiadający mu wierzchołek $v_B \in B$ jest również skojarzony przez zbiór $M$ --- musi więc istnieć pewien wierzchołek $v_A \in A$, dla którego zachodzi $v_A \notin M$.
        Skonstruuj zatem trzy zbiory wierzchołków $S$, $R$ oraz $T$.
        Zbiór $S = \{v_{Au}|v_{Au} \in a~\land v_{Au} \notin M\}$ zawiera wszystkie nieskojarzone wierzchołki ze zbioru $A$.
        Zbiór $R$ zawiera wszystkie wierzchołki $v_A \in A$ osiąglne ze zbioru $S$ poprzez $M$-przemienne ścieżki. \\
        Zbiór $T=\{v_T|v_T \in N(R), v_R \in R, ((v_R,v_M) \in M \lor (v_M,v_R)) \in M\}$ zawiera wierzchołki sąsiadujące z~wierzchołkami należącymi do zbioru $R$ wzdłuż ścieżek zawartych w~skojarzeniu $M$.
        Pokrycie wierzchołkowe grafu dwudzielnego $H$ stanowi zbiór $C=(A \setminus S \setminus R) \cup T$ o liczebności $|C|=|M|$.
    \end{itemize}
  \item Przypisz następujące wagi wszystkim wierzchołkom $v \in V$ ze względu na przynależność do pokrycia wierzchołkowego $C$:
    \begin{equation*}
    W_v = \left\{
    \begin{array}{rl}
    1 & \textnormal{jeżeli } \{A_v, B_v\} \in C,\\
    0.5 & \textnormal{jeżeli } (A_v \in C \land B_v \notin C) \lor (A_v \notin C \land B_v \in C),\\
    0 & \textnormal{jeżeli } \{A_v, B_v\} \notin C.
    \end{array} \right.
    \end{equation*}

    W przypadku gdy zachodzi $\forall_{v \in U}{v \in M}$, wszystkim wierzchołkom nadać należy wagę $W_v=0.5$.
  \item Zdefiniuj graf wynikowy jako 
    $G^\prime=(V^\prime, E^\prime), V^\prime=\{v \in V|W_v=0.5\}$.
    Wynikowy rozmiar dziedziny problemu zdefiniuj jako 
    $k^\prime=k-x$, gdzie $x=|\{v\in~V|W_v=1\}|$.
\end{enumerate}
Utworzenie grafu dwudzielnego jest wartościowe z~punktu widzenia problemu
pokrycia wierzchołkowego przez korelację maksymalnego dopasowania w~grafie
dwudzielnym z~optymalnym pokryciem wierzchołkowym. 
Zależność ta sformułowana została jako twierdzenie K\"oniga.
\begin{theorem}[Twierdzenie K\"oniga]
  W dowolnym grafie dwudzielnym, liczba krawędzi zawarta w~maksymalnym
  dopasowaniu jest równa rozmiarowi optymalnego pokrycia wierzchołkowego tego
  grafu.
\end{theorem}
\begin{theorem}\label{theorem_nf1}
  Wynikiem kroku 5 algorytmu jest poprawna pokrycie wierzchołkowe $C$ grafu $H$.
\end{theorem}
\begin{bproof}
  W przypadku 1 zachodzi $A = C$ albo $B = C$.
  Na tej podstawie stwierdzić można, że ${|A|=|B|=|M|}$.
  Jeżeli $C = A$, to spełniona jest własność $\forall_{f=(u,v), f\in F}: u \in C \oplus v \in C$, co
  w konsekwencji oznacza, że każda krawędź $f$ jest pokryta przez zbiór $C$.
  Zbiór $C$ stanowi zatem pokrycie wierzchołkowe.
  W przypadku 2 istnieją zbiory $S, R \subset T$ i $T \subset B$ oraz pokrycie wierzchołkowe $C=(A \setminus S \setminus R) \cup T$.
  Każda krawędź $e=(x, y) \in E$ spełnia własność $x \in S$, $x \in R$ albo $x \in (A \setminus S \setminus R)$.
  Każdy z~przypadków rozpatrywany jest osobno:
  \begin{itemize}
    \item \underline{$x \in S$}: Wierzchołek $x$ jest nieskojarzony --- jeżeli $M$ ma być skojarzeniem maksymalnym, to $y$ musi być skojarzony.
      Prowadzi to do wniosku, iż $\exists{e_M=(w,y)}: e_M \in M$.
      Na podstawie przebiegu algorytmu wiadomo, że w~takiej sytuacji zachodzi $w \in R$
      oraz, co ważniejsze, $y \in T$ --- krawędź $e$ jest więc pokryta przez $M$.
    \item \underline{$x \in R$}: Zachodzi $\exists{e_M=(x,w), e_M\in M}: w \in T$. \\
      Jeżeli zachodzi $w=y$, to spełniona musi być włásność $y \in T$.
      W przeciwnym razie, gdy $w \neq y$, istnieje krawędź $(z,w)$ skojarzona przez zbiór $M$, dla której zachodzi albo $z \in R$ albo $z \in S$.
      Dodatkowo wiadomo, że istnieje jeszcze jedna krawędź $(v, y)$ skojarzona przez zbiór $M$.
      Jeżeli ta zależność miałaby nie być spełniona, to zbiór $M$ nie stanowiłby maksymalnego skojarzenia --- zamiast krawędzi $(x,w)$ musiałby zawierać krawędzie $\{(x,y),(z,w)\}$.
      W efekcie spełnione są własności $v \in R$ oraz $y \in T$, tak więc krawędź $e$ jest pokryta przez zbiór $C$.
    \item \underline{$x \in a~\setminus S \setminus R$}: Przypadek trywialny,
      pokrycie krawędzi $e$ wynika z~definicji pokrycia wierzchołkowego $C$.
  \end{itemize}
\end{bproof}
\begin{theorem}
  Pokrycie wierzchołkowe stanowiące wynik kroku 5\ algorytmu jest rozmiaru $|C| = |M|$. 
\end{theorem}
\begin{bproof}
  Z definicji wynika, że $|S| = |V| - |M|$ oraz $|A \setminus S|=|(A \setminus S
  \setminus R) \cup R|=|M|$.\\
  Ponieważ każdy wierzchołek $v_R in R$ jest skojarzony przez zbiór $M$ oraz każdy z~wierzchołków $v_T \in T$ jest osiągalny z $R$ przez ścieżki złożone z~krawędzi $e_M \in M$, spełniona jest własność $|T|=|R|$.
  To prowadzi do wniosku, że prawdziwa jest równość $|(A\setminus S\setminus R)\cup T|=|((A \setminus S \setminus R) \cup R)|=|M|$.
\end{bproof}
\begin{theorem}\label{theorem_nf2}
  Pokrycie wierzchołkowe stanowiące wynik kroku 5\ algorytmu jest optymalne.
\end{theorem}
\begin{bproof}
  Graf $H$ jest grafem dwudzielnym, a~rozmiar jego maksymalnego skojarzenia wynosi $|M|$.
  Z~twierdzenia K\"oniga wynika, że rozmiar optymalnego pokrycia wierzchołkowego grafu dwudzielnego $H$ równy jest liczebności jego maksymalnego skojarzenia.
\end{bproof}
\begin{theorem}
  Wynik kroku 6~algorytmu stanowi rozwiązanie sformułowania problemu pokrycia wierzchołkowego jako egzemplarza problemu programowania liniowego.
\end{theorem}
\begin{bproof}
  Jednym z~warunków sformułowania problemu pokrycia wierzchołkowego jako egzemplarza problemu programowania liniowego jest $\forall_{e=(u,v) \in E}: W_u + W_v \geq 1$.
  Krok 6\ przypisuje następujące wagi wierzchołkom grafu $G$: $\forall_{v \in V}: W_v \in \{0, 0.5, 1\}$.
  W związku z~charakterystyką przekształcenia grafu $G$ w graf $H$, istnienie pewnej krawędzi $(x,y) \in E$ implikuje istnienie pary krawędzi $\{(A_x, B_y), (A_y, B_x)\} \in H$.
  Z dowodów twierdzeń~\ref{theorem_nf1} i~\ref{theorem_nf2} wynika, że jeżeli przynajmniej jeden z~wierzchołków każdej krawędzi $f \in F$ zawarty jest w~pokryciu wierzchołkowym $C$, to spełniona jest jedna z~własności: $(\{A_x, B_x\} \in C)$, $(\{A_y, B_y\} \in C)$, $(\{A_x, B_y\} \in C)$ albo $(\{A_y, B_x\} \in C)$.
  Widać zatem, że każda krawędź $e$ ma przypisaną prawidłową wagę.
\end{bproof}

\subsection{Redukcja koron}\label{ss_kernelization_crown_reduction}
\begin{definition}\thlabel{def_crown}
  Mianem \emph{korony} grafu $G=(V, E)$ nazywa się uporządkowaną parę
  podzbiorów wierzchołków $(I, H), I \subseteq V, H \subset V$, zachowujących
  następujące własności.
  \begin{enumerate}
    \item $I \neq \emptyset$ stanowi zbiór niezależny w $G$.
    \item $H=N(I)$.
    \item Istnieje skojarzenie $M=\{e_0, e_1, \ldots, e_p\}, \forall_{e_M=(u,v) \in
      M}: (u\in I \land v\in H) \lor (u \in H \land v \in I)$ takie, że
      $\forall_{v_h \in H}\exists_{e_M=(u,v)\in M}: u = v_h \oplus v = v_h$.
  \end{enumerate}
\end{definition}
\begin{definition}
  Zbiór $H$ określa się mianem \emph{głowy korony}.
\end{definition}
\begin{definition}
  Mianem \emph{szerokości korony} określa się $\|H\|$.
\end{definition}
\begin{theorem}\thlabel{th_crown_vc}
  Jeżeli graf $G=(V,E)$ zawiera koronę $(I,H)$, istnieje optymalna pokrywa 
  wierzchołkowa $VC_{OPT} \in V, H \in VC_{OPT}, I \notin VC_{OPT}$.
\end{theorem}
\begin{bproof}
  Z własności 3.\ definicji~\ref{def_crown}.\ wynika, że każda pokrywa 
  wierzchołkowa $VC$ musi zawierać przynajmniej jeden wierzchołek $v_H \in H$.
  Na tej podstawie stwierdzić można, że $\|VC\|\geq\|H\|$.
  Taki rozmiar pokrywy osiągnąć można przez umieszczenie $VC=VC\bigcup H$.
  Należy w~tym miejscu zazanczyć, że wierzchołki $v_H$ są użyteczne w~kontekście
  możliwości pokrywania krawędzi $e \notin M$, podczas gdy wierzchołki $v_I \in
  I$ nie posiadają tej cechy.
  Mając to na uwadze łatwo zauważyć, że $\|VC \bigcup H\| \leq \|VC \bigcup
  I\|$.
  Wniosek płynący z tej obserwacji jest jednoznaczny: istnieje optymalna pokrywa
  wierzchołkowa $VC_{OPT}; H \in VC_{OPT}, I \notin VC_{OPT}$.
\end{bproof}
W celu odnalezienia korony w grafie, zastosować można następujący algorytm.
\begin{algorithm}
  \caption{Algorytm odnajdujący koronę w grafie $G$}\label{alg_findCrown}
  \begin{algorithmic}[1]
    \Function{findCrown}{G, k}
    \State{$M_1\leftarrow$ największe dopasowanie $G$}
    \State{$O\leftarrow v \in V, \neg\exists_{e_{M_1}=(u, w) \in M_1}: u=v \lor w=v$}
    \If{$\|M_1\| \geq k$}
    \State\textbf{return} nil\Comment{$\neg\exists{VC_{OPT} \in V}: \|VC_{OPT}\| \leq k$}
  \EndIf
  \State{$M_2 \leftarrow$ maksymalne dopasowanie na krawędziach $O\leftrightarrow N(O)$}
  \If{$\|M_2\| > k$}
  \State{\textbf{return} nil\Comment{$\neg\exists{VC_{OPT} \in V}: \|VC_{OPT}\|\leq k$}}
\EndIf
\State{$I_0 \leftarrow v_O\in O, \neg\exists_{e_{M_2}=(u,v)\in M_2}: u=v_O\lor v=v_O$}
\State($n \leftarrow 0$)
\While{$I_{n-1} \neq I_n$}\label{findCrown_while}
\State{$H_n \leftarrow N(I_n)$}\label{findCrown_makeH}
\State{$I_{n+1} \leftarrow I_n\bigcup N_{M_2}(H_n)$}\label{findCrown_makeI}
\State{$n \leftarrow n+1$}
\EndWhile
\State{\textbf{return} $(I_n,H_n)$}\Comment{$n=N$}
  \EndFunction
\end{algorithmic}
\end{algorithm}
Rezultatem działania algorytmu jest korona $(I,H); I=I_N, H=H_N$.

\begin{theorem}
  Algorytm~\ref{alg_findCrown}.\ jest w stanie odnaleźć koronę pod warunkiem, że
  $I_0\neq\emptyset$.
\end{theorem}
\begin{bproof} (Spełnienie własności 1.\ definicji~\ref{def_crown}.)
  \par{
    Bazując na fakcie, iż $M_1$ stanowi największe dopasowanie $G$, stwierdzić
    można, że zarówno $O$ jak i $I \subset O$ stanowią zbiory niezależne.
  }
\end{bproof}
\begin{bproof} (Spełnienie własności 2.\ definicji~\ref{def_crown}.)
  \par{
    Z definciji wynika $H=N(I_{N-1})$.
    Z warunku zakończenia pętli\algref{findCrown}{findCrown_while} wynika 
    $I=I_N=I_{N-1}$.
    Na tej podsawie widocznym jest, że $H=N(I)$.
  }
\end{bproof}
\begin{bproof} (Spełnienie własności 3.\ definicji~\ref{def_crown}., dowód przez
  sprzeczność)\par{
    Założyć należy istnienie elementu $h \in H, \neg\exists_{e_{M_2}=(u,v) \in
  M_2}: u=h \lor v = h$.
  Rezultatem budowy $H$ byłaby zatem ścieżka rozszerzająca $P$ o nieparzystej
  długości. 
  Warunkiem przynależności $h \in H$ jest istnienie nieskojarzonego wierzchołka
  $v_O \in O$, stanowiącego początek tejże ścieżki.
  W takim wypadku, wynikiem linii~\ref{findCrown_makeH}.\ algorytmu byłaby
  zawsze krawędź nieskojarzona, podczas gdy wynikiem
  linii~\ref{findCrown_makeI}.\ byłaby  krawędź stanowiąca część skojarzenia.
  Proces ten powtarzałby się do momentu osiągnięcia wierzchołka $h$.
  Utworzona ścieżka rozpościera się zatem pomiędzy dwoma nieskojarzonymi
  wierzchołkami, będąc zarazem $M_2$-przemienną.
  Istnienie takiej ścieżki oznaczałoby możliwość zwiększenia dopasowania $M_2$
  poprzez wykonanie operacji $M_2=M_2\oplus P$, co stoi w opozycji do
  założenia, iż $M_2$ stanowi skojarzenie maksymalne.
  Obserwacja ta prowadzi do stwierdzenia, iż każdy wierzchołek $h \in H$ musi
  być skojarzony w $M_2$.
  Właściwe dopasowanie użyte w strukturze korony to dopasowanie $M_2$, z
  dziedziną ograniczoną do krawędzi $H \leftrightarrow I$.
}
\end{bproof}

Rezultatem jednej iteracji algorytmu redukcji korony jest graf
$G\prime=(V\prime, E\prime);\\V\prime=V\setminus H \setminus I, E\prime = E
\setminus \{H\leftrightarrow I\}$.

Rozmiar dziedziny problemu ulega zmniejszeniu do wartości
$n\prime=n-\|I\|-\|H\|$, natomiast wartość parametru spada do $k\prime=k-\|H\|$,
z uwagi na fakt, że każdy z wierzchołków $h \in H$ musi należeć do optymalnej
pokrywy wierzchołkowej, co udowodniono dla twierdzenia~\ref{th_crown_vc}.
Należy zaobserwować, iż jeżeli w grafie istnieje maksymalne skojarzenie
$M_{MAX}, \|M_{MAX}\| > k$, wyklucza to istnienie optymalnej pokrywy
wierzchołkowej $VC_{OPT}, \|VC_{OPT}\|\leq k$. Zatem, jeżeli rozmiar dowolnego 
z odnalezionych skojarzeń $M_1, M_2$  jest większy niż $k$, algorytm może
zakończyć działanie, udzielając ekwiwalentu odpowiedzi negatywnej.
Zależność ta pozwala również zdefiniować górną granicę rozmiaru grafu wynikowego
$G\prime$.

\begin{theorem}
  $\|M_1\| \leq k, \|M_2\| \leq k \implies \|V\prime \setminus I \setminus H\|
  \leq 3k$.
\end{theorem}
\begin{bproof}
  Ponieważ skojarzenie $M_1, \|M_1\| \leq k$ stanowi zbiór krawędzi, wnioskować
  można, iż $V_{M_1}=\{v, u|v, u \in V, (u,v)\in M_1 \lor (v,u) \in M_1\}, \|V_{M_1}\| \leq
  2k$.
  Z tego wynika, iż $\|O\| \geq n-2k$.
  W związku z faktem, iż $\|M_2\| \leq k$ istnieje co najwyżej $k$ wierzchołków
  $v_O \in O$ skojarzonych przez $M_2$.
  Łatwo zauważyć, iż w takim wypadku istnieje co najmniej $n-3k$ wierzchołków
  $v_O \in O$ nieskojarzonych przez $M_2$---są one zawarte w $I_0$, a zatem
  także w $I$.
  Ten tok rozumowania prowadzi do wniosku, iż $\|V \setminus I \setminus H\|
  \leq 3k$.
\end{bproof}

Należy zauważyć, że kształt odnalezionej przez algorytm korony podyktowany jest
strukturą wybranego największego skojarzenia $M_1$.
Prowadzi to do wniosku, iż pożądane jest wykonywanie algorytmu redukcji korony
wielokrotnie, wykorzystując różne największe skojarzenia tak, by zlokalizować i
zredukować jak największą ilość koron, co pozwoli na maksymalną redukcję
dziedziny problemu.
Rozsądnym krokiem jest również wykonanie jednej iteracji algorytmów przetwarzania
wstępnego przez każdą kolejną iteracją redukcji korony---usunięcie korony z
dużym prawdopodobieństwem prowadzić będzie do powstania wierzchołków niskiego
stopnia.
Najbardziej złożoną obliczeniowo częścią algorytmu jest odnalezienie skojarzenia
maksymalnego $M_2$, zrealizowane w niniejszej pracy przy pomocy algorytmu
kwiatów Edmondsa.
\footnote{
  Oryginalna implementacja, opisywana
  w~\cite{KernelizationAlgorithms04} oparta jest o reformulację problemu do
  instancji zadania przepływu w sieci, rozwiązanej za pomocą algorytmu Dinica, o
  wynikowej złożoności czasowej $O(n^{\frac{5}{2}})$.
}

\begin{theorem}
  Implementacja algorytmu redukcji korony grafu $G=(V,E); \|V\|=n,\|E\|=m$ 
  w~oparciu o algorytm kwiatów Edmondsa znajduje koronę w~czasie $O(n^{4})$.
\end{theorem}
\begin{bproof}
  Z~teoretycznego punktu widzenia, dwie najbardziej obciążające operacje to
  odnalezienie skojarzenia maksymalnego $M_2$ oraz odnalezienie skojarzenia
  największego $M_1$.
  W celu odnalezienia skojarzenia największego $M_1$ grafu należy
  sprawdzić wszystkie krawędzie $e\in E$ w celu poszukiwania wspólnych
  wierzchołków.
  Wykorzystując koncepcję zaznaczania wierzchołków krawędzi dołączanych do 
  skojarzenia jako odwiedzonych, złożoność operacji sprowadza się do $O(m)$.
  W grafie może znajdować się maksymalnie $O(n^{2})$ krawędzi.
  W celu odnalezienia dopasowanie maksymalnego zastosowano algorytm Edmondsa,
  którego złożoność wynosi $O(n^{4})$, co
  opisano w~podrozdziale~\ref{ss_edmonds}.
  Powyższe obserwacje prowadzą do wniosku, iż algorytm redukcji korony jest w
  stanie znaleźć koronę w grafie w czasie $O(n^{4} + n^{2})=O(n^{4})$.
\end{bproof}


