\section{Ogólne metody wykorzystane w algorytmach}\label{s_methods}

\subsection{Programowanie liniowe}\label{ss_lp}
\par{
  Programowanie liniowe, zwane również optymalizacją liniową, stanowi sposób
  osiągania najlepszego możliwego wyniku (zazwyczaj maksimum lub minimum) w modelu 
  matematycznym o wymaganiach określonych nierównościami liniowymi.
}
\par{
  Formalnie, mianem programowania liniowego określa technikę się optymalizacji 
  liniowej \emph{funkcji celu}, poddawanej \emph{ograniczeniom} w postaci równań 
  lub nierówności liniowych.
  \emph{Region dopuszczalnych rozwiązań} stanowi wypukły wielokąt stanowiący 
  zbiór powstały w wyniku przecięcia skończonej ilości półpłaszczyn wyznaczanych 
  przez nierówności liniowe.
  Funkcja celu to funkcja liniowa $z(x) \in \mathbb{R}$ zdefiniowana na regionie
  dopuszczalnych rozwiązań.
  Algorytm programowania liniowego odnajduje punkt na wielościanie,
  gdzie $z$ osiąga wartość optymalną w kontekście formulacji zadania jeżeli taki
  punkt istnieje. 
  Przykładowy zapis kanonicznej formy wyrażania programów liniowych:\\\par
  Zmaksymalizować funkcję celu:
  \begin{align*}
    z(x)={c^T}x
  \end{align*}\par
  Przy ograniczeniach: \begin{align*}
    Ax \leq b\\
    x\geq 0
  \end{align*}

  Gdzie:
  \begin{itemize}
    \item[-] $x$ stanowi wektor zmiennych do wyznaczenia,
    \item[-] $b$ oraz $c$ to wektory znanych współczynników,
    \item[-] $A$ to macierz znanych współczynników,
    \item[-] ${(\cdot)}^\mathrm{T}$ oznacza macierz transponowaną,
    \item[-] $Ax \leq b, x\geq 0$ to ograniczenia określające wypukły wielokąt,
      na którym optymalizowana jest funkcja celu $c^{T}x$.
  \end{itemize}
}
\par{
  W celu otrzymania optymalnego rozwiązania problemu programowania liniowego
  stosuje się jeden z  \emph{algorytmów programowania liniowego}.
  Do grupy algorytmów programowania liniowego zalicza się m.in.\ algorytm
  simplex lub algorytm ``na krzyż'' (criss-cross).
}
\par{
  Programowanie liniowe znajduje zastosowanie w wielu dziedzinach nauki i przemysłu. 
  Przykładem może być biznes i ekonomia zarówno jak i szeroko pojęta inżynieria.
  Techniki programowania liniowego są użyteczne przy problemach związanych z
  planowaniem, trasowaniem, harmonogramowaniem, przydziałem zadań oraz
  projektowaniem.
  Wynika to z faktu, iż wiele rzeczywistych problemów w dziedzinie badań
  operacyjnych, mających na celu optymalizację procesów decyzyjnych w praktyce,
  może zostać wyrażone w postaci zadań programowania liniowego.
  Wiele algorytmów rozwiązujących większe problemy optymalizacyjne wykorzystuje
  programowanie liniowe do rozwiązywania podproblemów częściowych jako zadania 
  programowania liniowego.
}
\subsubsection{Dualność problemów programowania liniowego}\label{sss_lp_duality}
\par{
  Cechą każdego problemu wyrażonego jako zadanie programowania liniowego,
  nazywanego problemem \emph{pierwotnym}, jest dwoistość.
  Oznacza to, że problem pierwotny może zostać przekształcony w odpowiadający mu
  problem \emph{wtórny}, zapewniający górną granicę optimum problemu
  pierwotnego. 
  Przykładowo, pierwotny problem:\par
  Zmaksymalizować $c^{T}x$ przy ograniczeniach $Ax\leq b, x \geq 0$;\\
  zastąpić można odpowiadającym mu \emph{symetrycznym} problemem wtórnym,\par
  Zminimalizować $b^{T}y$ przy ograniczeniach $A^{T}y \geq c, y \geq 0$.\\
}
\par{
  Twierdzenia dualności oparte są na dwóch fundamentalnych koncepcjach:
  \begin{enumerate}
    \item Problem wtórny symetrycznego problemu dualnego dwoistego programu
      liniowego stanowi pierwotny program liniowy.
    \item Każde prawdopodobne rozwiązanie programu liniowego stanowi
      ograniczenie optimum funkcji celu jego problemu dualnego.
  \end{enumerate}

  \begin{weakduality*}
    Wartość funkcji celu problemu dualnego dla dowolnego prawdopodobnego
    rozwiązania jest większa bądź równa wartości funkcji celu pierwotnego dla
    dowolnego prawdopodobnego rozwiązania.
  \end{weakduality*}
  \begin{strongduality*}
    Jeżeli problem pierwotny posiada rozwiązanie optymalne $x*$, to problem wtórny
    również posiada rozwiązanie optymalne $y*; c^{T}x*=b^{T}y*$.
  \end{strongduality*}
}

\par{
  W kontekście niniejszej pracy, dualność problemów problemów programowania
  liniowego jest szczególnie wartościowa ze względu na fakt, iż problemem wtórnym
  względem problemu pokrycia wierzchołkowego dowolnego grafu jest problem
  maksymalnego skojarzenia grafu.
}

\subsubsection{Programowanie całkowitoliczbowe i relaksacje}\label{sss_ilp_relaxations}
\par{
  Programem liniowym całkowitoliczbowym nazywa się każdy program liniowy z
  dodatkowym ograniczeniem aby $\forall_{x_n \in x}: x_n \in
  \mathbb{Z}~\refstepcounter{equation}(\theequation)\label{ilp_bound}$.
  Ograniczenie to zwane jest \emph{warukiem całkowitoliczbowości}.
  Większość problemów NP-trudnych jest wyrażalne w postaci programu liniowego
  całkowitoliczbowego.
  Prowadzi to do wniosku, iż sam problem programowania liniowego
  całkowitoliczbowego jest NP-trudny, co zaproponowano w~\cite{Kar72}.
}
\par{  
  Charakterystyczną poddziedziną programowów całkowitoliczbowych są
  \emph{binarne programy całkowitoliczbowe}, charakteryzującę się zamianą
  ograniczenia z programu całkowitoliczbowego~\eqref{ilp_bound}, na następujące:
  $\forall_{x_n \in x}: x_n \in \{0, 1\}~\refstepcounter{equation}(\theequation)\label{bilp_bound}$.
  Jedyne problemy programowania całkowitoliczbowego rozpatrywane w niniejszej
  pracy dotyczyć będą wyłącznie binarnych programów całkowitoliczbowych.
}
\par{
  W celu przekształcenia problemu NP-trudnego, wyrażonego w postaci binarnego 
  programu całkowitoliczbowego, do problemu rozwiązywalnego w czasie 
  wielomianowym, wyrażonego w postaci programu liniowego, należy dokonać 
  \emph{relaksacji}.
  Istotą relaksacji jest zastąpienie warunku całkowitoliczbowości binarnego
  programu całkowitoliczbowego~\eqref{bilp_bound} na mniej restrykcyjne
  ograniczenie $\forall_{x_n\in x}: 0\leq x_n\leq 1$.
}
\par {
  Istotną cechą relaksacji jest jej \emph{dokładność}, która może zostać
  stwierdzona gdy współrzędne każdego z wierzchołków regionu dopuszczalnych 
  rozwiązań są liczbami całkowitymi.
  Mając do czynienia z dokładną relaksacją, można na jej podstawie wprost 
  rozwiązać odpowiadający program całkowitoliczbowy w czasie wielomianowym.
  W tym celu wykonać należy następujące kroki:
  \begin{enumerate}
    \item otrzymać optimum $x*$ programu liniowego,
    \item odnaleźć wierzchołek $x\prime, z(x\prime)=z(x*)$,
      \item zwrócić $x\prime$ jako rozwiązanie programu całkowitoliczbowego.
    \end{enumerate}
}

\subsection{Redukcja do jądra problemu}\label{subsection_kernelization}
\subsection{Drzewa poszukiwań z ograniczeniem}\label{subsection_bound_search_trees}
