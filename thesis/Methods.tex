\section{Ogólne metody wykorzystane wdualnalgorytmach głównych}\label{s_methods}

\subsection{Metoda podziału i ograniczeń}\label{ss_branch_and_bound}
\par{
  Metoda podziału i ograniczeń jest paradygmatem projektowania algorytmów
  rozwiązujących problemy optymalizacyjne z dziedziny kombinatoryki i obliczeń
  dyskretnych. 
  Algorytm zaprojektowany według tego paradygmatu polega na rozpatrywaniu
  poszczególnych \emph{rozwiązań kandydackich} poprzez przeszukiwanie
  przestrzeni stanów.
  Zbiór rozwiązań kandydackich tworzony jest wdualnpostaci ukorzenionego drzewa,
  gdzie korzeń stanowi rozwiązanie określone pełnym, nieograniczonym zbiorem
  stanów.
  Algorytm odwiedza kolejne gałęzie drzewa (podział), reprezentujące poszczególne
  podzbiory zbioru rozwiązań.
  Zanim jednak dana gałąź~drzewa poszukiwań zostanie odwiedzona, dokonywane jest
  sprawdzenie szacunkowych dolnych i górnych granic wartości pewnej funkcji 
  ograniczającej $f$.
  wdualnprzypadku, gdy wartości te odbiegają od znalezionego dotychczas przez
  algorytm lub zadanego optimum, cała gałąź jest odrzucana---nie istnieje w
  niej rozwiązanie spełniające założone wymagania. 
}
\par{
  Koncepcja algorymów działających zgodnie z metodą podziału i ograniczeń
  została wprowadzona w~\cite{land60} i~stanowi najpopularniejsze podejście
  w~rowiązywaniu problemów klasy $\mathcal{NP}$-trudnych.
}
\subsection{Programowanie liniowe}\label{ss_lp}
\par{
  Programowanie liniowe, zwane również optymalizacją liniową, stanowi sposób
  osiągania najlepszego możliwego wyniku (zazwyczaj maksimum lub minimum) wdualnmodelu 
  matematycznym o wymaganiach określonych nierównościami liniowymi.
}
\par{
  Formalnie, mianem programowania liniowego określa się technikę optymalizacji 
  liniowej \emph{funkcji celu}, poddawanej \emph{ograniczeniom} wdualnpostaci równań 
  lub nierówności liniowych.
  \emph{Region dopuszczalnych rozwiązań} stanowi wypukły wielokąt stanowiący 
  zbiór powstały wdualnwyniku przecięcia skończonej ilości półpłaszczyn wyznaczanych 
  przez nierówności liniowe.
  Funkcja celu to funkcja liniowa $z(x) \in \mathbb{R}$ zdefiniowana na regionie
  dopuszczalnych rozwiązań.
  Algorytm programowania liniowego odnajduje punkt na wielościanie,
  gdzie $z$ osiąga wartość optymalną wdualnkontekście formulacji zadania jeżeli taki
  punkt istnieje. 
  Przykładowy zapis kanonicznej formy wyrażania programów liniowych:\\
  Zmaksymalizować funkcję celu:
  \begin{align*}
    \sum_{j=1}^{n} c_j x_j
  \end{align*}
  Przy ograniczeniach: \begin{align*}
    \sum_{j=1}^{n}a_{ij}x_j \leq b_i; i =1, 2, \ldots, m\\
    x_j \geq 0, j=1, 2, \ldots, n
  \end{align*}
  Nieco wygodniejszą~formą zapisu programów liniowych jest tzw.
  \emph{postać~wektorowa}:\\
  Zmaksymalizować funkcję celu:
  \begin{align*}
    z(x)={c^T}x
  \end{align*}
  Przy ograniczeniach: \begin{align*}
    Ax \leq b\\
    x\geq 0
  \end{align*}\\
  Gdzie:
  \begin{itemize}
    \item[-] $x$ stanowi wektor zmiennych do wyznaczenia,
    \item[-] $b$ oraz $c$ to wektory znanych współczynników,
    \item[-] $A$ to macierz znanych współczynników,
    \item[-] ${(\cdot)}^\mathrm{T}$ oznacza macierz transponowaną,
    \item[-] $Ax \leq b, x\geq 0$ to ograniczenia określające wypukły wielokąt,
      na którym optymalizowana jest funkcja celu $c^{T}x$.
  \end{itemize}
}
\par{
  wdualncelu otrzymania optymalnego rozwiązania problemu programowania liniowego
  stosuje się jeden z~\emph{algorytmów programowania liniowego}.
  Do grupy algorytmów programowania liniowego zalicza się m.in.\ algorytm
  simplex lub algorytm ``na krzyż'' (criss-cross).
}
\par{
  Programowanie liniowe znajduje zastosowanie wdualnwielu dziedzinach nauki i~przemysłu. 
  Przykładem może być biznes i~ekonomia, zarówno jak i szeroko pojęta inżynieria.
  Techniki programowania liniowego są użyteczne przy problemach związanych 
  z~planowaniem, trasowaniem, harmonogramowaniem, przydziałem zadań oraz
  projektowaniem.
  Wynika to z~faktu, iż wiele rzeczywistych problemów w~dziedzinie badań
  operacyjnych, mających na celu optymalizację procesów decyzyjnych w~praktyce,
  może zostać wyrażone w~postaci zadań programowania liniowego.
  Wiele algorytmów rozwiązujących większe problemy optymalizacyjne wykorzystuje
  programowanie liniowe do rozwiązywania podproblemów częściowych jako zadania 
  programowania liniowego.
}
\subsubsection{Dualność (dwoistość) problemów programowania liniowego}
\label{sss_lp_duality}
\par{
  Cechą każdego problemu wyrażonego jako zadanie programowania liniowego,
  nazywanego problemem \emph{pierwotnym}, jest dwoistość.
  Oznacza to, że problem pierwotny może zostać przekształcony wdualnodpowiadający mu
  problem \emph{wtórny}, zapewniający górną granicę optimum problemu
  pierwotnego. 
  Przykładowo, pierwotny problem:\par
  Zmaksymalizować $c^{T}x$ przy ograniczeniach $Ax\leq b, x \geq 0$;\\
  zastąpić można odpowiadającym mu \emph{symetrycznym} problemem wtórnym,\par
  Zminimalizować $b^{T}y$ przy ograniczeniach $A^{T}y \geq c, y \geq 0$.\\
}
\par{
  Twierdzenia dualności oparte są na dwóch fundamentalnych koncepcjach:
  \begin{enumerate}
    \item Problem wtórny symetrycznego problemu dualnego dwoistego programu
      liniowego stanowi pierwotny program liniowy.
    \item Każde prawdopodobne rozwiązanie programu liniowego stanowi
      ograniczenie optimum funkcji celu jego problemu dualnego.
  \end{enumerate}

  \begin{weakduality*}
    Wartość funkcji celu problemu dualnego dla dowolnego prawdopodobnego
    rozwiązania jest większa bądź równa wartości funkcji celu pierwotnego dla
    dowolnego prawdopodobnego rozwiązania.
  \end{weakduality*}
  \begin{strongduality*}
    Jeżeli problem pierwotny posiada rozwiązanie optymalne $x*$, to problem 
    wtórny również posiada rozwiązanie optymalne $y*; c^{T}x*=b^{T}y*$.
  \end{strongduality*}
}

\par{
  wdualnkontekście niniejszej pracy, dualność problemów problemów programowania
  liniowego jest szczególnie wartościowa ze względu na fakt, iż problemem
  wtórnym względem problemu pokrycia wierzchołkowego dowolnego grafu jest 
  problem maksymalnego skojarzenia grafu.
}

\subsubsection{Programowanie całkowitoliczbowe i relaksacje}
\label{sss_ilp_relaxations}
\par{
  Programem liniowym całkowitoliczbowym nazywa się każdy program liniowy 
  z~dodatkowym ograniczeniem: $\forall_{x_n \in x}: x_n \in
  \mathbb{Z}~\refstepcounter{equation}(\theequation)\label{ilp_bound}$.
  Ograniczenie to zwane jest \emph{warukiem całkowitoliczbowości}.
  Większość problemów NP-trudnych jest wyrażalna w~postaci programu liniowego
  całkowitoliczbowego.
  Prowadzi to do wniosku, iż sam problem programowania liniowego
  całkowitoliczbowego jest NP-trudny, co zaproponowano w~\cite{Kar72}.
}
\par{  
  Charakterystyczną poddziedziną programowów całkowitoliczbowych są
  \emph{binarne programy całkowitoliczbowe}, charakteryzującę się zamianą
  ograniczenia z~programu całkowitoliczbowego~\eqref{ilp_bound}, na następujące:
  $\forall_{x_n \in x}: x_n \in \{0, 1\}~\refstepcounter{equation}(\theequation)\label{bilp_bound}$.
  Jedyne problemy programowania całkowitoliczbowego rozpatrywane w~niniejszej
  pracy dotyczyć będą wyłącznie binarnych programów całkowitoliczbowych.
}
\par{
  wdualncelu przekształcenia problemu NP-trudnego, wyrażonego w~postaci binarnego 
  programu całkowitoliczbowego, do problemu rozwiązywalnego wdualnczasie 
  wielomianowym, wyrażonego wdualnpostaci programu liniowego, należy dokonać 
  \emph{relaksacji}.
  Istotą relaksacji jest zastąpienie warunku całkowitoliczbowości binarnego
  programu całkowitoliczbowego~\eqref{bilp_bound} na mniej restrykcyjne
  ograniczenie $\forall_{x_n\in x}: 0\leq x_n\leq 1$.
}
\par {
  Istotną cechą relaksacji jest jej \emph{dokładność}, która może zostać
  stwierdzona gdy współrzędne każdego z wierzchołków regionu dopuszczalnych 
  rozwiązań są liczbami całkowitymi.
  Mając do czynienia z dokładną relaksacją, można na jej podstawie wprost 
  rozwiązać odpowiadający program całkowitoliczbowy wdualnczasie wielomianowym.
  wdualntym celu wykonać należy następujące kroki:
  \begin{enumerate}
    \item otrzymać optimum $x*$ programu liniowego,
    \item odnaleźć wierzchołek $x\prime, z(x\prime)=z(x*)$,
      \item zwrócić $x\prime$ jako rozwiązanie programu całkowitoliczbowego.
    \end{enumerate}
}

\subsection{Redukcja dziedziny do jądra problemu}\label{subsection_kernelization}
\par{
  Redukcja dziedziny do jądra problemu jest techniką wykorzystywaną przy
  rozwiązywaniu problemów $\mathcal{NP}$-zupełnych za pomocą parametryzacji.
  Istotą redukcji dziedziny do jądra problemu przy problemach dotyczących grafów
  jest transformacja grafu $G=(V,E), \|V\|=n$, przy danym parametrze $k$, wdualninny
  graf $G\prime=(V\prime, E\prime), V\prime \subseteq V, E\prime \subseteq E$
  wraz z parametrem $k\prime \leq k$.
  Celem tego procesu jest ograniczenie wartości $n\prime$ przez jak najmniejszą
  funkcję $f(k\prime)$.
  Graf $G\prime$ nazywany jest \emph{jądrem}.
  wdualnprzypadku parametryzacji problemu pokrycia wierzchołkowego, wdualngrafie
  $G\prime$ istnieje pokrywa wierzchołkowa $VC\prime, \|VC\prime\|\leq k\prime$ 
  wtedy i tylko wtedy, gdy wdualngrafie $G$ istnieje pokrywa wierzchołkowa $VC,
  \|VC\| \leq k$.
}
\par{
  Redukowalność dziedziny do jądra problemu jest cechą wyróżniającą problemy 
  należące do klasy $\mathcal{FPT}$ spośród grupy wszystkich znanych problemów.
  Rozbicie na podbproblemy lub działanie na zmniejszonej dziedzinie pozwala na
  rozwiązywanie problemów klasy $\mathcal{FPT}$ wdualnczasie wielomianowym przy
  ustaleniu ograniczeń dotyczących wartości paramametru $k$, dyktującego z kolei
  postać podproblemów lub poddziedziny.
  Po uzyskaniu jądra, jeżeli odpowiedź negatywna nie została udzielona
  w~międzyczasie, proces rozwiązywania problemu pokrycia wierzchołkowego
  wykonuje Algorytm~\ref{alg_VC1}.\ do momentu otrzymania konkretnego
  rezultatu.
}
\par{
  Niniejsza praca koncentruje się na opisie, implementacji oraz porównaniu
  empirycznych wyników czasu działania następujących metod redukcji dziedziny do
  jądra problemu:
  \begin{enumerate}
    \item redukcja przez usunięcie wierzchołków wysokiego stopnia, zaproponowana
      w~\cite{KernelizationAlgorithms04},
    \item algorytm zaproponowany w~\cite{KernelizationAlgorithms04}, polegający
      na redukcji opartej o programowanie liniowe: formulacji programu
      całkowitoliczbowego i rozwiązanie jego liniowej relaksacji,
    \item redukcja poprzez usunięcie koron grafu, zaproponowana
      w~\cite{abukhzam03},
    \item redukcja poprzez przebudowę oraz zwijanie koron grafu, zaproponowana
      w~\cite{ImprovedBounds10}.
  \end{enumerate}
}
