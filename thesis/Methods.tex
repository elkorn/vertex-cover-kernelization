\section{Ogólne metody wykorzystane w algorytmach}\label{section_methods}

\subsection{Programowanie liniowe}\label{subsection_lp}

Programowanie liniowe, zwane również optymalizacją liniową, stanowi sposób
osiągania najlepszego możliwego wyniku (zazwyczaj maksimum lub minimum) w modelu 
matematycznym o wymaganiach określonych nierównościami liniowymi.

Formalnie, mianem programowania liniowego określa technikę się optymalizacji 
liniowej \emph{funkcji celu}, poddawanej \emph{ograniczeniom} w postaci równań 
lub nierówności liniowych.
\emph{Region wykonalności} stanowi wypukły wielościan stanowiący zbiór powstały
w wyniku przecięcia skończonej ilości półprzestrzeni wyznaczanych przez
nierówności liniowe.
Funkcja celu to funkcja liniowa $f(x) \in \mathbb{R}$ zdefiniowana na regionie
wykonalności. Algorytm programowania liniowego odnajduje punkt na wielościanie,
gdzie $f$ osiąga wartość optymalną w kontekście formulacji zadania jeżeli taki
punkt istnieje.

Kanoniczna forma wyrażania programów liniowych wygląda następująco:

\subsection{Redukcja do jądra problemu}\label{subsection_kernelization}
\subsection{Drzewa poszukiwań z ograniczeniem}\label{subsection_bound_search_trees}
