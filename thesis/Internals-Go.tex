\subsection{Język programowania Go} % (fold)
\label{ss_internals_go}
\par{
Go jest statycznie typowanym, imperatywnym, strukturalnym językiem programowania, którego historia rozpoczęła się w 2007. roku w firmie Google. Autorzy oryginalnej specyfikacji wraz z implementacją to: Rob Pike, Robert Griesemer oraz Ken Thompson.
Najnowsza stabilna wersja języka na dzień pisania niniejszej pracy to 1.3.3.
}
\par{
Składnia Go silnie nawiązuje do języka C --- dokonano jednak wielu modyfikacji skupiających się przede wsyzstkim na jej uproszczeniu, eliminacji możliwości popełniania błędów oraz zwiększeniu zwięzłości.
Dużo uwagi poświęca się również pielęgnacji ekosystemu wokół Go w celu uczynienia go narzędziem jak najłatwiejszym i najbardziej praktycznym w użyciu.
Dla osiągnięcia tych założeń zastosowano wzorce znane zarówno z języków statycznie jak i dynamicznie typowanych.
}
\par{Deklaracja i inicjalizacja zmiennych odbywa się za pomocą mechanizmu domniemania typów, w większości przypadków zwalniającego programistę z obowiązku jawnego oznaczania typu zmiennych oraz metod. Zamiast zapisu \textit{\lstinline{int x = 0;}}, znanego z języka C, stosuje się tu krótszy zapis \textit{\lstinline{x := 0}}. Warto również zwrócić uwagę na brak wymagania stawiania średników jako zakończeń wyrażeń.
}
 \par{Mimo możliwości korzystania ze wskaźników, bezpośredni dostęp do nich jest niemożliwy, co zapobiega błędom związanym z niezgodnością typów. W połączeniu ze statycznym typowaniem oznacza to, iż programista nie jest w stanie wprowadzić rozbieżności na poziomie typów zmiennych prowadzących do awarii aplikacji niewykrytych przez kompilator. Konsekwencją zablokowania bezpośredniego dostępu do wskaźników jest również brak możliwości wykonywania na nich działań arytmetycznych.
 }
 \par{
Dzięki mechanizmowi Garbage Collection, język zapobiega wyciekom pamięci wynikającym z nieprawidłowego zarządzania wskaźnikami. W aktualnej wersji wykorzystywana jest współbieżna wersja algorytmu \textit{mark and sweep}.
}
\par{
 W związku z faktem, iż Go jest językiem strukturalnym, brak w nim pojęcia obiektu. Uproszczonym odpowiednikiem jest struktura, definiowana słowem kluczowym \textit{\lstinline{struct}}.
}
  \par{
  Jedną z najbardziej radykalnych decyzji podczas tworzenia specyfikacji języka stanowi rezygnacja ze standardowego mechanizmu dziedziczenia. W miejsce dziedziczenia wirtualnego zastosowano system interfejsów, gdzie dana struktura implementuje określony interfejs wtedy i tylko wtedy gdy wystawia pełen zestaw publicznych metod zgodnych z jego deklaracją. Odpowiednikiem dziedziczenia klasycznego w Go jest osadzanie typów.
  Przykład osadzenia typów \textit{\lstinline{Reader}} i \textit{\lstinline{Writer}} w nowo utworzonym interfejsie \textit{\lstinline{ReaderWriter}}.
  \begin{lstlisting}[language=go]
    type ReaderWriter interface {
      Reader
      Writer
    }
  \end{lstlisting}
  Struktury implementujące interfejs \textit{\lstinline{ReaderWriter}} implementują również interfejsy osadzone.
  Osadzanie typów w strukturach wygląda podobnie, należy jednak oznaczyć je symbolem wskaźnika.
  \begin{lstlisting}[language=go]
    type ReaderWriter struct {
      *Reader
      *Writer
    }
  \end{lstlisting}
  Kluczowa różnica pomiędzy dziedziczeniem a osadzaniem typów polega na właściwości, iż metody typu osadzanego zostają włączone do typu zewnętrznego --- jednak podczas wywołania danej metody, jej odbiorcą jest egzemplarz typu osadzonego. \cite{godoc:embedding}
}
\par {
W celu uproszczenia składni głównie względem C++, argumentowanym przez jednego z autorów w notatce \cite{Pike:LessIsMore}, wyłączono ze specyfikacji wiele funkcji oferowanych przez podobne języki, poza wymienionymi wcześniej.
  \begin{itemize}
    \item Przeciążanie metod i operatorów,
    \item cykliczne zależności pomiędzy pakietami,
    \item asercje,
    \item wyjątki --- powrócono do zwracania błędów jako wyników funkcji,
    \item programowanie generyczne.
  \end{itemize}
}
\par{
  Dla wygody programisty wprowadzono do Go zestaw podstawowych typów, wyrażających elementy brakujące zdaniem autorów w czystym C.
  \begin{itemize}
    \item \emph{Plastry} (slices), zapisywane jako \textit{\lstinline{[]type}}, wskazują na tablicę obiektów przechowywanych w pamięci, przechowując wskaźnik do początku danego plastra, jego długość oraz \emph{pojemność}, określającą liczebność elementów plastra, która wymagać alokacji dodatkowej pamięci w celu rozszerzenia odpowiadającej tablicy.
    \item Niezmienne ciągi znaków (typ \textit{\lstinline{string}}), zawierające przeważnie tekst w kodowaniu UTF-8. Mogą jednak przechowywać dowolne bajty.
    \item Tablica haszująca, zapisywana jako \textit{\lstinline{map[key\_type]value\_type}}.
  \end{itemize}
}
\par{
  Go jest językiem kompilowanym do kodu bajtowego. Istnieją dwa oficjalnie wspierane zestawy kompilatorów:
  \begin{itemize}
    \item \texttt{gc} wraz z narzędziami dla architektur \texttt{amd64} oraz \texttt{i386}, mający wydajny optymalizator, oraz dla architektury \texttt{ARM},
    \item \texttt{gccgo}, stanowiący nakładkę na \texttt{gcc}.
  \end{itemize}
  Każdy z zestawów wspiera platformy DragonFly BSD, FreeBSD, Linux, NetBSD, OpenBSD, OS X (Darwin), Plan 9, Solaris oraz Windows. Wyjątkiem jest kompilator \texttt{gc} na architekturę \texttt{ARM}, wspierający wyłącznie platformy Linux, FreeBSD oraz NetBSD. \cite{godoc:compilers}
  Łańcuch narzędzi związany z procesem kompilacji tworzy statycznie linkowane, natywne binarne pliki wykonywalne bez zewnętrznych zależności.
}

\par{
  Ostatnim elementem, o którym należy wspomnieć są wbudowane mechanizmy służące do programowania współbieżnego.
  Za bardzo wartościową cechę języka Go uznaje się zawarcie w jego specyfikacji oraz implementacji prymitywów zapewniających logiczne i wygodne w użyciu konstrukcje tworzące poziom abstrakcji ponad technikaliami związanymi z zarządzaniem wątkami i procesami oraz ich synchronizacją, zarówno jak i aspektami programowania asynchronicznego.
  Cechy te stanowią silny argument przemawiający za wartościowością Go jako języka docelowego dla aplikacji serwerowych o wysokiej przepustowości.
  \begin{itemize}
    \item Wyrażenie \textit{\lstinline{go}}, pozwalające na uruchomienie dowolnej funkcji w tzw. \emph{lekkim procesie}, enkapsulującym pojedynczy wątek jądra systemu i współdzielącym przestrzeń adresową z innymi wątkami wykorzystywanymi przez podobne procesy.
    \item Typ kanałowy, \textit{\lstinline{chan type}}, pozwalający na synchronizowaną wymianę informacji pomiędzy lekkimi procesami. Instancje kanałów obsługiwane są przede wszystkim za pomocą operatorów wysyłania (\textit{\lstinline{chan\_instance <- value}}) oraz odbioru (\textit{\lstinline{variable <- chan\_instance}}). Obydwie te operacje blokują wykonywanie programu w ramach danego procesu dopóki dowolny z innych procesów nie umieści danych w kanale.
    \item Wyrażenie \textit{\lstinline{select}}, będące rozwiązaniem podobnym do konstrukcji \textit{\lstinline{switch}} z różnicą polegającą na fakcie, iż predykaty przypadków nie są oparte na wartościach określonej zmiennej lecz na stanie kanałów w sensie pojawiania się na nich danych.
  \end{itemize}
  Podstawowe elementy umożliwiają budowę bardziej skomplikowanych mechanizmów zarządzania przepływem kontroli w systemach współbieżnych dostosowanych do poszczególnych przypadków użycia, takich jak pule robotników lub łańcuchy przetwarzania.
}