\subsection{Formulacja problemu jako instancji przepływu w sieci}\label{Kernelization_network_flow}
Algorytm ten, zaproponowany w~\cite{KernelizationAlgorithms04}, opiera się na
algorytmie użytym w~\cite{Niedermeier02} do udowodnienia twierdzenia
Neumhausera-Trottera.

\begin{theorem}[Drugie twierdzenie Nemhausera-Trottera]\thlabel{theorem_nt}
  Dla grafu $G=(V,E), |V|=n, |E|=m$, dwa rozłączne zbiory $C_0 \subseteq V,
  V_0 \subseteq V$ mogą zostać oblicone w czasie $O(\sqrt{n}m)$ przy zachowaniu
  następujących własności.
  \begin{enumerate}
    \item Dla podgrafu $G[V_0]$ założyć należy istnienie pokrywy $D \subseteq
      V_0$. W następstwie, $C := D \bigcup C_0$ stanowi pokrywę wierzchołkową~$G$.
    \item Istnieje optymalna pokrywa wierzchołkowa $S$ grafu $G$ z $C_0
      \subseteq S$.
    \item Podgraf $G[V_0]$ posiada optymalną pokrywę wierzchołkową rozmiaru co
      najmniej $\frac{[V_0]}{2}$. 
  \end{enumerate}
\end{theorem}

Algorytm definiuje graf dwudzielny $B$ na podstawie grafu $G$, odnajduje pokrywę
wierzchołkową $VC_B$ grafu $B$ poprzez odnalezienie maksymalnego skojarzenia $B$,
a następnie przydziela wartości wierzchołkom $G$ w oparciu o przynależność do
$VC_B$.
Implementacja algorytmu z~\cite{Niedermeier02} zrealizowana jest przez
przekształcenie $B$ w instancję problemu przezpływu w sieci i rozwiązania tejże
przy pomocy algorytmu Forda-Fulkersona.\footnote{Algorytm Forda-Fulkersona
  zastosowany został w niniejszej pracy, oryginalna implementacja oparta została
o algorytm Dinica.}
Złożoność czasowa algorytmu wynosi $O(\sqrt{n}m)$, zgodnie z~twierdzeniem
Neumhausera-Trottera. 
W związku z faktem, iż w grafie może istnieć maksymalnie~$n^2$ krawędzi, można
wyrazić tę złożoność~w~formie $O(n^\frac{5}{2})$.
Rozmiar zredukowanej dziedziny problemu ograniczony jest do $2k$.

\begin{enumerate}
  \item Przekształć graf $G=(V,E)$ w graf dwudzielny $H=(U,F)$ zgodnie z
    następującymi zasadami:\\
    $A=\{A_v|v \in V\}\\
    B=\{B_v|v \in V\}\\
    U=A_v \bigcup B_v
    F=\{(A_v, B_v)|(v,u) \in E \lor (u,v) \in E\}$
  \item Przekształć graf dwudzielny $H$ w graf przepływu w sieci $H\prime$:
    \begin{itemize}
      \item dodaj węzeł źródłowy $v_s$, połączony z każdym wierzchołkiem $v_a
        k\in A$ krawędziami skierowanymi $(v_s, v_a)$,
      \item dodaj węzeł docelowy $v_z$, połączony z każdym wierzchołkiem $v_b
        \in B$ krawędziamiy skierowanymi $(v_b, v_z)$,
      \item wszystkie krawędzie $f \in F$ skieruj $(v_a, v_b)$,
      \item każdej krawędzi $h \in H\prime$ nadaj pojemność $c(f)=1$.
    \end{itemize}
  \item Znajdź maksymalny przepływ $MF$ w $H\prime$.
  \item Zbiór $M=MF \bigcap F$ stanowi maksymalne skojarzenie $H$.
  \item Znajdź pokrywę wierzchołkową $H$, bazując na $M$.
    \begin{itemize}
      \item Jeżeli $\forall_{v in U}{v \in M}$, pokrywę
        wierzchołkową~stanowi całość zbioru $A$ lub $B$.
      \item Przy licznościach zbiorów $A$, $B$ oraz wagach krawędzi w $H\prime$
        wiadomo, iż ${\forall_{v_A in A}{v_A \in M} \iff \forall_{v_B in B}{v_B \in M}}$.
        Na tej podstawie można, stwierdzić, że $\exists_{v_A \in A}{v_A \notin
        M}$.
        Skonstruuj zatem 3 zbiory $S$, $R$ oraz $T$ wierzchołków. Zbiór
        $S = \{v_{Au}|v_{Au} \in A \land v_{Au} \notin M\}$ zawiera wszystkie
        nieskojarzone wierzchołki ze zbioru $A$.
        $R$ stanowi zbiór wszystkich wierzchołków $v_A \in A$ osiąglnych z $S$
        poprzez M-przemienne ścieżki. \\
        $T=\{v_T|v_T \in N(R), v_R \in R, ((v_R,v_M) \in M \lor (v_M,v_R)) \in M\}$ 
        jest zbiorem zawierającym wierzchołki sąsiednie względem $R$ wzdłuż 
        ścieżek zawartych w skojarzeniu $M$.
        Pokrywę wierzchołkową grafu dwudzielnego $H$ stanowi zbiór 
        ${VC=(A \setminus S \setminus R) \bigcup T}, |VC|=|M|$.
    \end{itemize}
  \item Przypisz wagi wszystkim wierzchołkom $v \in V$ w odniesieniu do $VC$:
    \begin{itemize}
      \item $\{A_v, B_v\} \in VC \Rightarrow W_v=1$,
      \item $A_v \in VC \land B_v \notin VC \lor A_v \notin VC \land B_v \in
        VC \Rightarrow W_v=0.5$,
      \item $\{A_v, B_v\} \notin VC \Rightarrow W_v=0$
    \end{itemize}
    W przypadku 1., wszystkim wierzchołkom nadać należy wagę $W_v=0.5$.
  \item Zdefiniuj graf wynikowy jako 
    $G\prime=(V\prime, E\prime), V\prime=\{v \in V|W_v=0.5\}$.
    Wynikowy rozmiar dziedziny problemu zdefiniuj jako 
    ${k\prime=k-x, x=|\{v\in~V|W_v=1\}}|$.
\end{enumerate}

Utworzenie grafu dwudzielnego jest wartościowe z punktu widzenia problemu
pokrycia wierzchołkowego poprzez korelację maksymalnego dopasowania w grafie
dwudzielnym z optymalną pokrywą wierzchołkową. 
Zależność ta sformułowana została jako twierdzenie K\"oniga:

\begin{theorem}[Twierdzenie K\"oniga]
  W dowolnym grafie dwudzielnym, ilość krawędzi zawarta w maksymalnym
  dopasowaniu jest równa rozmiarowi optymalnej pokrywy wierzchołkowej tego
  grafu.
\end{theorem}

\begin{theorem}\label{theorem_nf1}
  Wynikiem kroku 5.\ algorytmu jest poprawna pokrywa wierzchołkowa $VC$
  grafu $H$.
\end{theorem}
\begin{bproof}
  (Dla przypadku 1.) \\
  $A = VC \oplus B = VC$.
  Na tej podstawie stwierdzić można, iż ${|A|=|B|=|M|}$.
  Przyjmując $VC = A$, $\forall{f=(u,v), f\in F}: u \in VC \oplus v \in VC$, co
  w konsekwencji oznacza, iż każda krawędź $f$ jest pokryta przez $VC$, czyniąc
  $VC$ prawidłową pokrywą wierzchołkową.
\end{bproof}
\begin{bproof}
  (Dla przypadku 2.) \\
  Istnieją zbiory $S, R \subset T$ oraz $T \subset B$ oraz istnieje pokrywa 
  wierzchołkowa \\ $VC=(A \setminus S \setminus R) \bigcup T$.
  ${\forall{e=(x,y), e \in E}: x \in S \oplus x \in R \oplus x \in (A \setminus S
  \setminus R)}$.


  Każdy z przypadków rozpatrywany jest osobno:
  \begin{itemize}
    \item \underline{$x \in S$}: $x$ jest nieskojarzony---jeżeli $M$ ma być skojarzeniem 
      maksymalnym, $y$ musi być skojarzony.
      Prowadzi to do wniosku, iż $\exists{e_M=(w,y)}: e_M \in M$.
      Na podstawie wytycznych algorytmu wiadomo, iż w takiej sytuacji $w \in R$
      oraz, co ważniejsze, $y \in T$---oznacza to, że $e$ jest pokryta przez $M$.
    \item \underline{$x \in R$}: $\exists{e_M=(x,w), e_M\in M}: w \in T$. \\
      $w=y \implies y \in T$.
      W przeciwnym wypadku, gdy $w \neq y$, pewnym jest, że $\exists{e_M=(z,w),
      z \in R \oplus z \in S}: e_M \in M$.
      Dodatkowo wiadomo,\\że ${\exists_{e_{M2}=(v,y)}: e_{M2} \in M}$.
      Jeżeli ta zależność miałaby nie być spełniona, oznaczałoby to, iż zbiór $M$
      nie stanowi maksymalnego skojarzenia---zamiast krawędzi $(x,w)$ musiałby
      zawierać krawędzie $\{(x,y),(z,w)\}$.
      W efekcie $v \in R, y \in T$, tak więc krawędź $e$ jest pokryta przez $VC$.
    \item \underline{$x \in A \setminus S \setminus R$}: Przypadek trywialny,
      pokrycie krawędzi $e$ wynika z definicji samej pokrywy $VC$.
  \end{itemize}
\end{bproof}
\begin{theorem}
  Pokrywa stanowiąca wynik kroku 5.\ algorytmu jest rozmiaru $|VC| = |M|$. 
\end{theorem}
\begin{bproof}
  Z definicji, $|S| = |V| - |M|; |A \setminus S|=|(A \setminus S
  \setminus R) \bigcup R|=|M|$.\\
  Na podstawie faktu, iż każdy wierzchołek $v_R in R$ jest skojarzony oraz
  każdy z wierzchołków $v_T \in T$ jest osiągalny z $R$ przez ścieżki złożone z
  krawędzi $e_M \in M$ stwierdzić można, że $|T|=|R|$.
  To prowadzi do wniosku: $|(A\setminus S\setminus R)\bigcup T|=|((A \setminus
  S \setminus R) \bigcup R)|=|M|$.
\end{bproof}
\begin{theorem}\label{theorem_nf2}
  Pokrywa stanowiąca wynik kroku 5.\ algorytmu jest optymalna.
\end{theorem}
\begin{bproof}
  Graf $H$ jest grafem dwudzielnym, a rozmiar jego maksymalnego skojarzenia
  wynosi $|M|$.
  W oparciu o twierdzenie K\"oniga, rozmiar optymalnej pokrywy wierzchołkowej
  grafu $H$ równy jest liczebności jego maksymalnego skojarzenia.
\end{bproof}
\begin{theorem}
  Wynik kroku 6.~algorytmu stanowi realne rozwiązanie formulacji problemu jako
  zagadnienia programowania liniowego.
\end{theorem}
\begin{bproof}
  Jednym z warunków formulacji problemu jako zadania programowania liniowego jest
  $\forall_{e=(u,v) \in E}: W_u + W_v \geq 1$.
  Krok 6.\ przypisuje wagi wierzchołkom grafu $G$: ${forall_{v \in V}: W_v \in
  \{0, 0.5, 1\}}$.
  W związku z charakterystyką przekształcenia grafu $G$ w graf $H$, 
  $(x,y) \in E \implies \{(A_x, B_y), (A_y, B_x)\} \in H$.
  W oparciu o~dowody twierdzeń~\ref{theorem_nf1} i~\ref{theorem_nf2} można
  stwierdzić, iż jeżeli przynajmniej jeden z wierzchołków każdej krawędzi $f \in
  F$ zawarty jest w pokrywie $VC$, to $(\{A_x, B_x\} \in VC) \oplus (\{A_y,
  B_y\} \in VC) \oplus (\{A_x, B_y\} \in VC) \oplus (\{A_y, B_x\} \in VC)$.
  Widać zatem, że każda krawędź $e$ ma przypisaną prawidłową wagę.
\end{bproof}
