\subsection{Pakiet GLPK} % (fold)
\label{ss_internals_glpk}
\par{
  Pakiet GLPK (GNU Linear Programming Kit) jest biblioteką wysokopoziomową przeznaczoną do rozwiązywania problemów dużej skali z~dziedziny programowania liniowego, częściowo całkowitoliczbowego i~pokrewnych.
  Stanowi on część projektu GNU i~udostępniany jest w~ramach licencji GNU GPLv3.
  Na całość pakietu składa się zestaw funkcji napisanych w~języku ANSI C, zorganizowanych w~wywoływalną bibliotekę.
  Korzystanie z~pakietu może być realizowane w~dwojaki sposób.
  \begin{enumerate}
    \item Problemy modelować można za pomocą języka GNU MathProg --- tak przygotowane modele rozwiązuje się za pomocą narzędzia \texttt{glpsol}.
    \item Definicja problemów może odbywać się bezpośrednio w~kodzie aplikacji przez opisanie jej konstrukcjami udostępnianymi przez GLPK jako biblioteki języka C. Metoda ta jest znacznie bardziej elastyczna, gdyż nie mamy tu do czynienia ze statycznym modelem, lecz można go budować i~modyfikować z~wykorzystaniem wszelkich narzędzi udostępnianych przez język programowania.
  \end{enumerate}
}
\par{
  Pakiet GLPK korzysta z~ulepszonej metody simpleks oraz metody punktów wewnętrznych do rozwiązywania problemów programowania liniowego oraz algorytmu drzewa poszukiwań z~ograniczeniami wraz z~algorytmem Gomory'ego dla problemów programowania częściowo całkowitoliczbowego.
}
% subsection pakiet_glpk_ (end)