\chapter{Uwagi}\label{section_notes}

 Rozdział ten nie należy do ``oficjalnej'' pracy i nie będzie zawarty w jej
 finalnym wydaniu.
 Służy jedynie jako lista przypomnień oraz medium komunikacji z Promotorem.

\begin{note}\thlabel{why_bipartite_are_cool}
  Muszę uzupełnić informacje dot.\ wartościowości przekształcenia w graf
  dwudzielny w kontekście problemu pokrycia wierzchołkowego.
\end{note}
\begin{note}\thlabel{lp_impl}
  Na chwilę obecną, rozwiązanie LP nie jest jeszcze zaimplementowane.
  Muszę zaimplementować jak najprostszy mechanizm rozwiązywania formulacji
  liniowych, najprawdopodobniej w oparciu o~\cite{sedgewick11}.


  Obecnie jednak moja uwaga poświęcona jest implementacji techniki zawężania dziedziny
  opartej na krotkach z~\cite{ImprovedBounds10}.
\end{note}
\begin{note}
  Należy wprowadzic pojęcie algorytmu rozwiązującego formulacje liniowe. 
  Da się odczuć jego brak w sekcji~\ref{section_kernelization_lp_formulation}.
\end{note}
\begin{note}\thlabel{fpt_translation}
  Nie udało mi się do tej pory dotrzeć do żadnego (!) tłumaczenia
  ``fixed-parameter tractability''.
  Czy miałby Pan jakąś propozycję w to miejsce?
  Może spotkał się już Pan z próbami tłumaczenia tego sformułowania---uważam, że
  wartościowe byłoby odnalezienie lub utworzenie stosownego polskiego
  odpowiednika, godnie oddającego istotę sprawy. 
\end{note}
