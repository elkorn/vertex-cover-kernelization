\chapter{Uwagi robocze}\label{section_notes}
\par{
  Rozdział ten nie należy do ``oficjalnej'' pracy i nie będzie zawarty w jej
 finalnym wydaniu.
 Służy jedynie jako lista przypomnień oraz medium komunikacji z Promotorem.
}
\begin{note}
  Wydruki ``robocze'' pracy dostarczam w formie jednostronne --- am nadzieję, że
  ułatwi to Panu umieszczanie notatek bliżej powiązanych treści.
\end{note}
\begin{note}
  W kolejnej wersji pracy zamieszczone zostaną jeszcze następujące rozdziały:
  \begin{itemize}
    \item Abstrakt
    \item Cel pracy
    \item Układ pracy
    \item Specyfikacja wewnętrzna (wykorzystane technologie, opis
      implementacji)
    \item Specyfikacja zewnętrzna
    \item Analiza wyników
    \item Treść opisów algorytmów uupełniających
  \end{itemize}
  Układ pracy nadal wymaga nieco dodatkowej uwagi z mojej strony.
  Zamieszczam listę mankamentów strukturalnych i typograficznych, których 
  jestem~świadom na dzień \today:
  \begin{itemize}
    \item Notacja $G^\prime$ musi zostać zmieniona na $G^{\prime}$.
    \item Niektóre formuły wykraczają poza marginesy.
      Mam w tej kwestii dylemat, jakie postępowanie jest lepsze -~czy należy
      ``złamać'' długie formuły (pogarsza nieco ich czytelność), czy lepiej 
      przenosić je w całości do nowej linii (nieprzyjemny efekt wizualny).
      W pracy widać zastosowanie obydwóch technik i, szczerze, nadal nie jestem
      w stanie podjąć decyzji.
    \item Miejscami znaleć można trochę ``literówek''- poprawiam je na
      bieżąco, jednak niektóre umykają mojej uwadze.
      Ostatecznie nie będzie z tym problemu -~pod koniec przepuszczę pracę przez
      program sprawdzający poprawność pisowni w celu wyłapania wszelkich tego
      typu pomyłek.
    \item Przecięcia stron zdarzają się w niefortunnych miejscac --- ytuacja
      zostanie opanowana w ramach finalnego szlifu strukturalnego pracy, kiedy
      treść~zostanie zaakceptowana.
  \end{itemize}
\end{note}
\begin{note}
  Brakujące definicje:
  \begin{itemize}
    \item ścieżka rozszerzająca
    \item notacja $A \leftrightarrow B$, oznaczająca zbiór krawędzi pomiędzy
      wierzchołkami zbiorów $A$ oraz $B$.
  \end{itemize}
\end{note}
\begin{note}\thlabel{why_bipartite_are_cool}
  Muszę uzupełnić informacje dot.\ wartościowości przekształcenia w graf
  dwudzielny w kontekście problemu pokrycia wierzchołkowego.
\end{note}
\begin{note}\thlabel{lp_impl}
  Na chwilę obecną, rozwiązanie PL nie jest jeszcze zaimplementowane.
  Muszę zaimplementować jak najprostszy mechanizm rozwiązywania formulacji
  liniowych, najprawdopodobniej w oparciu o algorytm sympleksu, na
  podstawie~\cite{sedgewick11} oraz~\cite{Cormen:2001:IA:580470}.


  Obecnie jednak moja uwaga poświęcona jest implementacji algorytmu redukcji 
  przez przebudowę oraz zwijanie koron grafu z wykorzystaniem rozgałęziania w
  oparciu o krotki, zaproponowanego w~\cite{ImprovedBounds10}. 
\end{note}
\begin{note}
  Należy wprowadzic pojęcie algorytmu rozwiązującego formulacje liniowe. 
  Da się odczuć jego brak w sekcji~\ref{section_kernelization_lp_formulation}.
\end{note}
\begin{note}\thlabel{fpt_translation}
  Nie udało mi się do tej pory dotrzeć do żadnego (!) tłumaczenia
  ``fixed-parameter tractability''.
  Czy miałby Pan jakąś propozycję w to miejsce?
  Może spotkał się już Pan z próbami tłumaczenia tego sformułowani --- ważam, że
  wartościowe byłoby odnalezienie lub utworzenie stosownego polskiego
  odpowiednika, godnie oddającego istotę sprawy. 
\end{note}
\begin{note}
  W miarę możliwości prosiłbym o~weryfikację jakości dowodó --- iększość z~nich
  jest oczywiście w~bardzo dużym stopniu oparta na cytowanych pracach, jednak
  starałem się sformułować je w~sposób nie pozostawiający wątpliwości co do
  elementów składowych użytych w~toku rozumowania.
  Oryginalne prace, ze względu na ``wyższy poziom zaawansowania'' pomijały
  część~koncepcji, które dla mnie nie były tak samo oczywiste jak dla Autorów.
  Analiza tychże dowodów wymagała z~mojej strony dodatkowych notatek, których
  syntezy starałem się w~sposób elegancki dołączyć do dowodów przedstawionych w
  niniejszej pracy tak, by ich analiza nie wymagała przeglądania dodatkowych
  źródeł.
  Wyjątkiem od tej reguły są dowody oparte na twierdzeniach udowodnionych wiele
  lat temu (uznawszy je za ``dobrze znane'') lub zawarte w~pracach, których 
  treść w całości sprowadza się do prezentacji danego dowodu.
\end{note}
\begin{note}
  Złożoność czasowa moich implementacji algorytmów będzie miejscami niestety
  gorsza od proponowanej w~części teoretyczne --- ynika to z faktu podjęcia
  próby stworzenia w miarę jednolitej platformy do testowania wybranych
  algorytmów grafowych.
  Udało się stworzyć coś na wzór takiej platformy, jednak kosztem złożoności na
  chwilę obecną.
  Optymalizacja jest możliw --- bawiam się jednak, że czas nie pozwoli na jej
  realizację, a priorytetem wydaje się być merytoryka.

  Dodać trzeba, iż pozytywny wpływ opisywanych w pracy technik na czas
  rozwiązywania problemu pokrycia wierzchołkowego nadal jest jasno widoczny.
\end{note}
