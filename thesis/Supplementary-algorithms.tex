\section{Wybrane algorytmy uzupełniające}\label{s_supplementary_algorithms}
\par{
  Podrozdział ten przybliża algorytmy uzupełiające wykorzystane przy 
  implementacji algorytmów głównych, opisywanych 
  w~podrozdziale~\ref{s_kernelization}.
}

\subsection{Rozwiązywanie problemu maksymalnego przepływu w sieci}\label{ss_max_flow}
\subsubsection{Metoda Forda-Fulkersona}
\par{
  Metoda Forda-Fulkersona stanowi podstawę kilku algorytmów rozwiązujących problem maksmalnego przepływu, o różnych czasach działania.
  Kluczowymi pojęciami, na których operuje metoda Forda-Fulkersona są \emph{sieci residualne}, \emph{ścieżki powiększające} oraz \emph{przekroje}.
  Pojęcia te mają istotne znaczenie w praktyce rozwiązywania problemów dotyczących przepływów w sieciach i stanowią zasadnicze koncepcje w twierdzeniu o maksymalnym przepływie i minimalnym przekroju.
  \begin{theorem}[Twierdzenie o maksymalnym przepływie i minimalnym przekroju Forda-Fulkersona]\thlabel{th_max_flow}
    Przyjmując za $f$ przepływ w sieci $N=(V,E)$ ze źródłem $s$ i ujściem $t$, to następujące warunki są równoważne:
    \begin{enumerate}
      \item Przepływ $f$ jest maksymalny w sieci $N$.
      \item Sieć residualna $N_f$ nie zawiera ścieżek powiększających.
      \item Dla pewnego przekroju $(S, T)$ w sieci $N$ zachodzi $|f|=c(S,T)$.
    \end{enumerate}
  \end{theorem}
}
\par{
  Metoda Forda-Fulkersona jest metodą iteracyjną ze zerową wartością początkową przepływu w sieci: $\forall_{\{u,v\}\in V}: f(u,v)=0$.
  W każdej iteracji wartość przepływu zostaje zwiększona poprzez odnalezienie ścieżki powiększającej, stanowiącej ścieżkę ze źródła $s$ do ujścia $t$, której pojemność pozwala na przesłanie dodatkowego przepływu.
  Przepływ maksymalny obliczany jest przez aktualizację przepływu f[u,v]\footnote{Nawiasami kwadratowymi oznaczany jest dostęp do zmiennej tablicowej.} między każdą parą wierzchołków $\{u,v\} \in N$.
  Proces ten powtarzany jest do momentu gdy sieć nie zawiera żadnej ścieżki powiększającej.
  Zgodnie z twierdzeniem \ref{th_max_flow}, przepływ otrzymany w tym momencie jest maksymalny.
  Przymuje się, że przepustowości krawędzi $c(u, v)$ są dane wraz z grafem i $c(u, v)=0$ w przypadku gdy $(u, v) \notin E$.
  W zapisie algorytmu oznaczenie $c_f(u, v)$ oznacza \emph{przepustowość residualną} krawędzi $(u, v)$, natomiast $c_f(P_a)$ przepustowość residualną ścieżki powiększającej.
  Pojęcie przepustowości residualnej opisane jest w następującej podsekcji.
  \begin{algorithm}
    \caption{Podstawowy algorytm Forda-Fulkersona}\label{alg_fordFulkerson}
    \begin{algorithmic}[1]
      \Function{Ford-Fulkerson}{N, s, t}

        \algorithmicrequire{sieć $N=(V, E)$, źródło $s$, ujście $t$}

        \algorithmicensure{przepływ $f$}

        \For{$\forall_{(u,v) \in E}$}
        \Do
        \State{$f[u, v] \gets 0$}
        \State{$f[v, u] \gets 0$}
        \EndFor
        \While{istnieje ścieżka powiększająca $P_a$}
          \Do
          \State{$c_f \gets min\{c_f(u, v): (u, v)\in P_a\}$}
          \For{$\forall_{(u,v) \in P_a}$}
            \Do
            \State{$f[u, v] \gets f[u, v] + c_f(p)$}
            \State{$f[v, u] \gets -f[u, v]$}
          \EndFor
        \EndWhile
        \State\textbf{return} f
      \EndFunction
  \end{algorithmic}
  \end{algorithm}
}
\subsection{Algorytm kwiatów Edmondsa}\label{ss_edmonds}




