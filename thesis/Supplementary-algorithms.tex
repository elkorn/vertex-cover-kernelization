\section{Wybrane algorytmy uzupełniające}\label{s_supplementary_algorithms}
\par{
  Podrozdział ten przybliża algorytmy uzupełiające wykorzystane przy 
  implementacji algorytmów głównych, opisywanych 
  w~podrozdziale~\ref{s_kernelization}.
}

\subsection{Rozwiązywanie problemu maksymalnego przepływu w~sieci}\label{ss_max_flow}
\subsubsection{\textbf{Metoda Forda-Fulkersona}}
\par{
  Metoda Forda-Fulkersona stanowi podstawę kilku algorytmów rozwiązujących problem maksmalnego przepływu, o~różnych czasach działania.
  Kluczowymi pojęciami, na których operuje metoda Forda-Fulkersona są \emph{sieci residualne}, \emph{ścieżki powiększające} oraz \emph{przekroje}.
  Pojęcia te mają istotne znaczenie w~praktyce rozwiązywania problemów dotyczących przepływów w~sieciach i~stanowią zasadnicze koncepcje w~twierdzeniu o~maksymalnym przepływie i~minimalnym przekroju.
  \begin{theorem}[Twierdzenie o~maksymalnym przepływie i~minimalnym przekroju Forda-Fulkersona]\thlabel{th_max_flow}
    Przyjmując za $f$ przepływ w~sieci $N=(V,E)$ ze źródłem $s$ i~ujściem $t$, to następujące warunki są równoważne:
    \begin{enumerate}
      \item Przepływ $f$ jest maksymalny w~sieci $N$.
      \item Sieć residualna $N_f$ nie zawiera ścieżek powiększających.
      \item Dla pewnego przekroju $(S, T)$ w~sieci $N$ zachodzi $|f|=c(S,T)$.
    \end{enumerate}
  \end{theorem}
  Dowód powyższego twierdzenia przedstawiony zostanie pod koniec sekcji, po objaśnieniach pojęć składowych.
}
\par{
  Metoda Forda-Fulkersona jest metodą iteracyjną z~zerową wartością początkową przepływu w~sieci: $\forall_{\{u,v\}\in V}: f(u,v)=0$.
  W każdej iteracji wartość przepływu zostaje zwiększona poprzez odnalezienie ścieżki powiększającej, stanowiącej ścieżkę ze źródła $s$ do ujścia $t$, której pojemność pozwala na przesłanie dodatkowego przepływu.
  Przepływ maksymalny obliczany jest przez aktualizację przepływu f[u,v]\footnote{Nawiasami kwadratowymi oznaczany jest dostęp do zmiennej tablicowej.} między każdą parą wierzchołków $\{u,v\} \in N$.
  Proces ten powtarzany jest do momentu gdy sieć nie zawiera żadnej ścieżki powiększającej.
  Zgodnie z~twierdzeniem \ref{th_max_flow}, przepływ otrzymany w~tym momencie jest maksymalny.
  Przymuje się, że przepustowości krawędzi $c(u, v)$ są dane wraz z~grafem i~$c(u, v)=0$ w~przypadku gdy $(u, v) \notin E$.
  W zapisie algorytmu oznaczenie $c_f(u, v)$ oznacza \emph{przepustowość residualną} krawędzi $(u, v)$, natomiast $c_f(P_a)$ przepustowość residualną ścieżki powiększającej.
  Pojęcie przepustowości residualnej opisane jest w~następującej podsekcji.
  \begin{algorithm}
    \caption{Podstawowy algorytm Forda-Fulkersona}\label{alg_fordFulkerson}
    \begin{algorithmic}[1]
      \Function{Ford-Fulkerson}{N, s, t}

        \algorithmicrequire{sieć $N=(V, E)$, źródło $s$, ujście $t$}

        \algorithmicensure{przepływ $f$}

        \For{$\forall_{(u,v) \in E}$}
        \Do
        \State{$f[u, v] \gets 0$}
        \State{$f[v, u] \gets 0$}
        \EndFor
        \While{istnieje ścieżka powiększająca $P_a$}
          \Do
          \State{$c_f \gets min\{c_f(u, v): (u, v)\in P_a\}$}
          \For{$\forall_{(u,v) \in P_a}$}
            \Do
            \State{$f[u, v] \gets f[u, v] + c_f(p)$}
            \State{$f[v, u] \gets -f[u, v]$}
          \EndFor
        \EndWhile
        \State\textbf{return} f
      \EndFunction
  \end{algorithmic}
  \end{algorithm}
}
\subsubsection{\textbf{Sieci residualne}}
\par{
  W dowolnej sieci przepływowej $N=(V, E_f)$ dodatkowy przepływ z~wierzchołka $u$ do $v$, nie przekraczający przepustowości $c(u, v)$ krawędzi łączącej te wierzchołki określa się mianem \emph{przepustowości residualnej}, definiowanej jako $c_f(u ,v) = c(u, v) - f(u, v)$.
  Dla dowolnej sieci przepływowej $N$ i~pewnego przepływu $f$ sieć residualna $N_f$ jest siecią składającą się z~krawędzi dopuszczających dodatkowy dodatni przepływ. Formalnie sieć residualną definiuje się jako zbiór $E_f=\{(u, v) \in V \x V: c_f(u,v) > 0\}$.
  Krawędziami $E_f$ w~sieci $N_f$, zwanymi \emph{krawędziami residualnymi}, są albo krawędzie należące do zbioru $E$ sieci $N$ albo krawędzie do nich przeciwne.
  \begin{theorem}
    Krawędź $(u, v)$ może pojawić się w sieci residualnej $N_f$ wtedy i~tylko wtedy, gdy co najmniej jedna z~krawędzi $\{(u, v), (v, u)\}$ występuje w~sieci pierwotnej, co daje $|E_f| \leq 2|E|$.
  \end{theorem}
  \begin{bproof}
    Jeżeli dla pewnej krawędzi $(u, v) \in E$ zachodzi $f(u, v) < c(u, v)$, to wtedy $c_f(u, v) = c(u, v) - f(u, v) > 0$, co z~definicji sieci residualnej świadczy, iż krawędź $(u, v) \in E_f$ jest w~niej zawarta.

    Jeżeli dla pewnej krawędzi $(u, v) \in E$ zachodzi $f(u, v) > c(u, v)$, to wtedy $c_f(u, v) = c(u, v) - f(u, v) < 0$, co z~definicji sieci residualnej świadczy, iż krawędź przeciwna $(v, u) \in E_f$  jest w~niej zawarta.

    Jeżeli żadna z~krawędzi $\{(u, v), (v, u)\}$ nie występuje w~pierwotnej sieci, zachodzi wtedy $c(u, v) = c(v, u) = f(u, v) = f(v, u) = c_f(u, v) = c_f(v, u)=0$. Żadna z~tych krawędzi nie dopuszcza dodatniego przepływu, co zgodnie z~definicją sieci residualnej wyklucza ją ze zbioru krawędzi residualnych.
  \end{bproof}
}

\subsubsection{\textbf{Ścieżki powiększające}}
\par{
  \begin{definition}
    Dla danej sieci $N=(V,E)$ i~przepływu $f$ mianem \emph{ścieżki powiększającej} określa się każdą ścieżkę ze źródła $s$ do ujścia $t$ w~sieci residualnej $N_f$.
  \end{definition}
  Zgodnie z~definicją sieci residualnej każda krawędź $(u, v)$ na ścieżce powiększającej dopuszcza dodatkowy dodatni przepły bez naruszenia warunku przepustowości.
  Największą możliwą wartość, o~którą zwiększyć można przepływ na każdej krawędzi ścieżki powiększającej $P_a$ określa się mianem \emph{przepustowości residualnej ścieżki powiększającej}, definiowanej jak następuje.
  \begin{definition}
    Przepustowość residualna ścieżki powiększającej $P_a$ odpowiada najmniejszej przepustowości residualnej spośród przynależących do niej krawędzi, $c_f(P_a) = min\{c_f(u, v):(u, v)\in P_a\}$.
  \end{definition}
  \begin{definition}
    Przepływ $f_{P_a}$ na ścieżce powiększającej $P_a$ o~wartości $|F_{P_a}| = c_f(P_a) > 0$ definiowany jest w~oparciu o~następującą funkcję $f_p: V \times V \rightarrow \mathbb{R}$: 
    \begin{equation*}
    f_{P_a}(u, v) = \left\{
    \begin{array}{rl}
    c_f(P_a) & \text{jeżeli } (u, v) \in P_a,\\
    -c_f(P_a) & \text{jeżeli } (v, u) \in P_a,\\
    0 & \text{w przeciwnym wypadku }.
    \end{array} \right.
    \end{equation*}
  \end{definition}
}
\subsubsection{\textbf{Przekroje w~sieciach}}
\par{
  Ostatnim z~pojęć potrzebnych do pełnego zrozumienia metody Forda-Fulkersona są przekroje w~sieciach.
  Stanowią one jednocześnie ważny element całości, gdyż maksymalny przepływ w~sieci bezpośrednio koreluje z~przepustowością najmniejszego jej przekroju.
  \begin{definition}
    Dla sieci $N=(V, E)$ ze źródłem $s$ i~ujściem $t$ każdy podział zbioru $V$ na podzbiory $S$ oraz $T=V\setminus S$ określa się mianem \emph{przekroju} $(S,T)$ w~sieci $N$.

    Jeżeli w~sieci zdefiniowany jest przepływ $f$, to \emph{przepływem netto} przez przekrój $(S, T)$ definiowany jest jako $f(S, T)$.

    Mianem \emph{przepustowości} przekroju $(S, T)$ określa się $c(S, T)$.

    \emph{Przekrojem minimalnym} w~sieci $N$ nazywa się przekrój, którego przepustowość jest najmniejsza spośród przekrojów istniejących w~$N$.
  \end{definition}
}
\par{
  Nadmienić należy, iż przepływ netto przez przekrój $(S, T)$ składa się z~dodatnich przepływów w~obu kierunkach; to znaczy dodatniego przepływu z~$S$ do $T$, branego ze znakiem plus oraz dodatniego przepływu z~$T$ do $S$, branego ze znakiem minus.
  Przepustowość przekroju $c(S, T)$ jest jednak obliczna wyłącznie na podstawie przepustowości krawędzi prowadzących z~$S$ do $T$.
}
\par{
  Dowolny przekrój $(S, T)$ w~sieci $N=(V, E)$ ze źródłem $s$ i~ujściem $t$ posiada dwie kluczowe cechy czyniące tę koncepcję kluczową dla rozwiązania problemu największego przepływu.
  \begin{enumerate}
    \item Wartość dowolnego przepływu $f$ w~sieci $N$ jest równa wartości przepływu netto $f(S,T)$.
    \item Wartość dowolnego przepływu $f$ w~sieci $N$ jest nie większa niż przepustowość przekroju $(S, T)$.
  \end{enumerate}
}

\subsection{Algorytm kwiatów Edmondsa}\label{ss_edmonds}




