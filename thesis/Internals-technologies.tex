\section{Wykorzystane techniki i technologie}\label{ss_internals-technologies}
W przeciągu ostatniej dekady proces tworzenia oprogramowania uległ znaczącym
transformacjom.
Ogromna popularyzacja sektora technologii informacyjnych, którą zawdzięczamy głównie rozwojowi internetu, przyczyniła się do skupienia dużo większej uwagi na kwestiach zarówno sposobów wytwarzania oraz charakteru oprogramowania jak i na narzędziach temu służących.

W środowisku informatyki biznesowej odchodzi się od klasycznych, liniowych procesów tworzenia oprogramowania na rzecz szerokiego grona rozwiązań zaliczanych do grupy tzw. metodyk zwinnych. 
Głównym celem każdej metodyk zwinnych jest zwiększenie ilości dostarczanych produktów (w tym wypadku rozwiązań informatycznych) w danym czasie, zachowując jednocześnie jak najwyższy poziom ich jakości.
Prawidłowo wdrożone metodyki zwinne pozwalają na zacieśnienie pętli komunikacyjnej pomiędzy interesariuszami biznesowymi oraz technicznymi projektów, owocując częstszą wymianą bardziej szczegółowych informacji.
Bezpośrednio przekłada się to na zwiększoną przezroczystość procesu realizacji projektu.
Konsekwencja ta otwiera nowe pole możliwości związanych z planowaniem, podziałem prac, raportowaniem i prognozowaniem postępów prac oraz weryfikacją prognoz poprzez monitorowanie stanu realizacji za pośrednictwem metryk.

\par{
Zwiększona częstotliwość dostarczania wartości biznesowej wiąże się z potrzebą wynajdowania bardziej nowoczesnych narzędzi oraz praktyk wspierających taki tryb operowania.
Wobec zmieniających się wymagań dziedziny technologii informacyjnych, z biegiem lat na znaczeniu zyskały dawne koncepcje takie jak:

\begin{itemize}
  \item enkapsulacja,
  \item wyższe poziomy abstrakcji kodu,
  \item paradygmat programowania obiektowego,
  \item paradygmat programowania funkcyjnego,
  \item programowanie systemowe,
  \item programowanie współbieżne i asynchroniczne.
\end{itemize}
}

\par{
Wraz z adaptacją tychże pomysłów, opisywanych już w pracach sprzed zgoła czterdziestu laty, równolegle narodziły się i na ich podstawie rozwijane są nowe idee, do których zaliczyć można:
\begin{itemize}
  \item Paradygmat programowania hybrydowego, łączący w sobie elementy funkcyjne zarówno jak i obiektowe. Dzięki zaletom elementów paradygmatu obiektowego, wysoce i zrozumiale dla człowieka zorganizowane dane mogą być przetwarzane przy pomocy potężnych szkicy logicznych, złożonych z funkcji za pomocą implementacji pojęć należących do paradygmatu funkcyjnego takich jak monady i kombinatory.
  \item Biblioteki wysokopoziomowe, enkapsulujące funkcjonalności narzędziowe i działania związane mocno z architekturą komputera. Biblioteki wysokopoziomowe zwiększają efektywność inżyniera oprogramowania poprzez wprowadzenie użytecznych i zrozumiałych idiomów programistycznych, często wywodzących się bezpośrednio z matematyki lub realizujących zestawy niskopoziomowych operacji sięgających języka pośredniego lub kodu assemblerowego jako całość logiczną oraz nazwaną w sposób zrozumiały dla człowieka. Ważną funkcją bibliotek wysokopoziomowych jest również dążenie do maskowania rozbieżności związanych z docelową platformą uruchomieniową kodu, co stanowi znaczące wsparcie dla programisty, pozwalając na pisanie w większości przypadków homogenicznego kodu, który nie jest uzależniony od platformy, na której ma zostać użyty.
  \item Automatyczne zarządzanie pamięcią przy pomocy tzw. mechanizmów odśmiecania pamięci (\emph{Garbage Collection}, popularnie skracane do \emph{GC}). Praktyka pokazuje, iż w dużym odsetku systemów przeznaczenia ogólnego, przez które rozumie się większość zastosowań biznesowych i przemysłowych, nie wymaga tak ywsokiej precyzji, jaką oferuje ręczne zarządzanie pamięcią. Biorąc pod uwagę duży nakład pracy programisty potrzebny do prawidłowego ręcznego zarządzania pamięcia, bardziej opłacalne okazuje się poświęcenie ułamka wydajności systemu poprzez wprowadzenie cyklicznego odśmiecania pamięci w oparciu o pewne reguły rozpoznawania bloków do usunięcia, zwalniając jednocześnie z tego obowiązku inżyniera. Dla systemów zawierających komponenty bardziej wyspecjalizowane, lub wymagające przetwarzania w czasie rzeczywistym, nowoczesne języki oferują podejście mieszane, pozwalając na ręczne zarządzanie pamięcią w krytycznych miejscach aplikacji.
  \item Wprowadzanie mechanizmów synchronizacji wątków oraz prymitywów do zarządzania asynchronicznym lub współbieżnym wykonywaniem kodu jako obywateli pierwszej klasy nowoczesnych języków programowania. Popularnymi obecnie koncepcjami w tej dziedzinie są:
  \begin{itemize}
    \item ̆\emph{funkcje zwrotne} (Callback functions), przekazywane jako punkty powrotu w momencie zakończenia przetwarzania asynchronicznego,
    \item \emph{kanały} (channels), stanowiące abstrakcje w postaci obiektów umożliwiających dwukierunkową transmisję danych dowolnego typu pomiędzy komponentami systemu lub systemami,
    \item \emph{zdarzenia} (events), będące abstrakcją w postaci obiektów niosących informacje o zajściu określonych warunkóœ w asynchronicznym przepływie sterowania aplikacji.
  \end{itemize}
\end{itemize}
}

\par{
  Pojęcie ``narzędzia'' w XXI. wieku znacznie zyskało na pojemności i~nie ogranicza się już wyłącznie aspektów samego języka lub nieodzownych elementów ściśle z~nim związanych jak kompilatory czy linkery.
  Narastający nacisk kładzie się również na rozwijanie tak zwanych \emph{ekosystemów} wokół języków programowania.
  Pojęcie ekosystemu stanowi dość liberalne określenie grupy funkcjonalności oraz aplikacji służących wsparciu programisty przez m.in.:
  \begin{itemize}
    \item automatyzację podstawowych zadań,
    \item analizę a nawet zmiany lub przepisywanie kodu na podstawie badania drzewa składni abstrakcyjnej,
    \item uruchamianie testów jednostkowych oraz sprawnościowych,
    \item tworzenie profili wydajnościowych aplikacji ze względu na zużycie czasu procesora lub pamięci RAM w oparciu o śledzenie przepływu kontroli w aplikacji,
    \item podpowiedzi oraz dopełnianie na bieżąco pisanego kodu.
  \end{itemize}
}

\par{
  Większe zaplecze narzędziowe umożliwia wykorzystywanie grup aplikacji należących do ekosystemu w celu spełnienia założeń leżących u podstaw metodyk zwinnych, skupiających się na jak najczęstym dostarczaniu wartościowych produktów lub komponentów produktu wysokiej jakości, w ramach przejrzystego procesu twórczego.
}

\subsection{Język programowania Go} % (fold)
\label{sss_go}
\par{
Go jest statycznie typowanym, imperatywnym, strukturalnym językiem programowania, którego historia rozpoczęła się w 2007. roku w firmie Google. Autorzy oryginalnej specyfikacji wraz z implementacją to: Rob Pike, Robert Griesemer oraz Ken Thompson.
Najnowsza stabilna wersja języka na dzień pisania niniejszej pracy to 1.3.3.
}
\par{
Składnia Go silnie nawiązuje do języka C --- dokonano jednak wielu modyfikacji skupiających się przede wsyzstkim na jej uproszczeniu, eliminacji możliwości popełniania błędów oraz zwiększeniu zwięzłości.
Dużo uwagi poświęca się również pielęgnacji ekosystemu wokół Go w celu uczynienia go narzędziem jak najłatwiejszym i najbardziej praktycznym w użyciu.
Dla osiągnięcia tych założeń zastosowano wzorce znane zarówno z języków statycznie jak i dynamicznie typowanych.
}
\par{Deklaracja i inicjalizacja zmiennych odbywa się przy pomocy mechanizmu domniemania typów, w większości przypadków zwalniającego programistę z obowiązku jawnego oznaczania typu zmiennych oraz metod. Zamiast zapisu \texttt{int~x~=~0;}, znanego z języka C, stosuje się tu krótszy zapis \texttt{x~:=~0}. Warto również zwrócić uwagę na brak wymagania stawiania średników jako zakończeń wyrażeń.
}
 \par{Mimo możliwości korzystania ze wskaźników, bezpośredni dostęp do nich jest niemożliwy, co zapobiega błędom związanym z niezgodnością typów. W połączeniu ze statycznym typowaniem oznacza to, iż programista nie jest w stanie wprowadzić rozbieżności na poziomie typów zmiennych prowadzących do awarii aplikacji niewykrytych przez kompilator. Konsekwencją zablokowania bezpośredniego dostępu do wskaźników jest również brak możliwości wykonywania na nich działań arytmetycznych.
 }
 \par{
Dzięki mechanizmowi Garbage Collection, język zapobiega wyciekom pamięci wynikającym z nieprawidłowego zarządzania wskaźnikami. W aktualnej wersji wykorzystywana jest współbieżna wersja algorytmu \textit{mark and sweep}.
}
\par{
 W związku z faktem, iż Go jest językiem strukturalnym, brak w nim pojęcia obiektu. Uproszczonym odpowiednikiem jest struktura, definiowana słowem kluczowym \texttt{struct}.
}
  \par{
  Jedną z najbardziej radykalnych decyzji podczas tworzenia specyfikacji języka stanowi rezygnacja ze standardowego mechanizmu dziedziczenia. W miejsce dziedziczenia wirtualnego zastosowano system interfejsów, gdzie dana struktura implementuje określony interfejs wtedy i tylko wtedy gdy wystawia pełen zestaw publicznych metod zgodnych z jego deklaracją. Odpowiednikiem dziedziczenia klasycznego w Go jest osadzanie typów.
  Przykład osadzenia typów \texttt{Reader} i \texttt{Writer} w nowo utworzonym interfejsie \texttt{ReaderWriter}.
  \begin{lstlisting}
    type ReaderWriter interface {
      Reader
      Writer
    }
  \end{lstlisting}
  Struktury implementujące interfejs \texttt{ReaderWriter} implementują również interfejsy osadzone.
  Osadzanie typów w strukturach wygląda podobnie, należy jednak oznaczyć je symbolem wskaźnika.
  \begin{lstlisting}
    type ReaderWriter struct {
      *Reader
      *Writer
    }
  \end{lstlisting}
  Kluczowa różnica pomiędzy dziedziczeniem a osadzaniem typów polega na właściwości, iż metody typu osadzanego zostają włączone do typu zewnętrznego~---~jednak podczas wywołania danej metody, jej odbiorcą jest instancja typu osadzonego. \cite{godoc:embedding}
}
\par {
W celu uproszczenia składni głównie względem C++, argumentowanym przez jednego z autorów w notatce \cite{Pike:LessIsMore}, wyłączono ze specyfikacji wiele funkcji oferowanych przez podobne języki, poza wymienionymi wcześniej.
  \begin{itemize}
    \item Przeciążanie metod i operatorów,
    \item cykliczne zależności pomiędzy pakietami,
    \item asercje,
    \item programowanie generyczne.
  \end{itemize}
}
\par{
  Dla wygody programisty wprowadzono do Go zestaw podstawowych typów, wyrażających elementy brakujące zdaniem autorów w czystym C.
  \begin{itemize}
    \item \emph{Plastry} (slices), zapisywane jako \texttt{[]typ}, wskazują na tablicę obiektów przechowywanych w pamięci, przechowując wskaźnik do początku danego plastra, jego długość oraz \emph{pojemność}, określającą liczebność elementów plastra, która wymagać alokacji dodatkowej pamięci w celu rozszerzenia odpowiadającej tablicy.
    \item Niezmienne ciągi znaków (typ \texttt{string}), zawierające przeważnie tekst w kodowaniu UTF-8. Mogą jednak przechowywać dowolne bajty.
    \item Tablica haszująca, zapisywana jako \texttt{map[typ\_klucza]typ\_wartości}.
  \end{itemize}
}
\par{
  Go jest językiem kompilowanym do kodu bajtowego. Istnieją dwa oficjalnie wspierane zestawy kompilatorów:
  \begin{itemize}
    \item \texttt{gc} wraz z narzędziami dla architektur \texttt{amd64} oraz \texttt{i386}, posiadający wydajny optymalizator, oraz dla architektury \texttt{ARM},
    \item \texttt{gccgo}, będący nakładką na \texttt{gcc}.
  \end{itemize}
  Każdy z zestawów wspiera platformy DragonFly BSD, FreeBSD, Linux, NetBSD, OpenBSD, OS X (Darwin), Plan 9, Solaris oraz Windows. Wyjątkiem jest kompilator \texttt{gc} na architekturę \texttt{ARM}, wspierający wyłącznie platformy Linux, FreeBSD oraz NetBSD. \cite{godoc:compilers}
  Łańcuch narzędzi związany z procesem kompilacji tworzy statycznie linkowane, natywne binarne pliki wykonywalne bez zewnętrznych zależności.
}

\par{
  Ostatnim elementem, o którym należy wspomnieć, są wbudowane mechanizmy służące do programowania współbieżnego.
}
% subsection go (end)
