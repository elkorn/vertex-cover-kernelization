\chapter{Wstęp}\label{Chapter_Introduction}
\section{Cel}\label{Section_Aim}
\par{
  Jedną z~cech charakterystycznych problemów należących do klasy $\mathcal{NP}$ jest trudność rozwiązania ich (w przeciwieństwie do weryfikacji poprawności danej odpowiedzi) w~czasie umożliwiającym zastosowanie w~skali spotykanej w~praktyce w~informatyce oraz innych dziedzinach takich jak przemysł, biznes czy biologia.
  Richard Karp określa takiej trudności problem jako \emph{satysfakcjonująco} rozwiązany w~sytuacji gdy pewien algorytm jest w~stanie znaleźć jego rozwiązanie wykonując skończoną ilość kroków, ograniczoną pewnym wielomianem, którego zmienną stanowi rozmiar danych wejściowych --- mówi się wtedy o rozwiązaniu otrzymanym w~\emph{czasie wielomianowym}.
  Natura problemów tego stopnia trudności bardzo często związana jest z~domeną elementów policzalnych.
  Popularne i przydatne zarówno w~szerszym kontekście badań algorytmicznych jak i w~praktyce okazują się być zadania polegające na określaniu charakterystycznych właściwości grafów, macierzy całkowitoliczbowych, rodzin skończonych zbiorów, wzorów logicznych i podobnych im struktur.
}
\par{
  Niniejsza praca skupia się na analizie, implementacji oraz opisie wybranych metod rozwiązywania problemu pokrycia wierzchołkowego grafu z~wykorzystaniem technik zaliczanych do grupy metod parametryzacji.
  Problem pokrycia wierzchołkowego należy do klasy problemów $\mathcal{NP}$-zupełnych, stanowiących podzbiór problemów klasy $\mathcal{NP}$.
  Problem pokrycia wierzchołkowego został uwzględniony w~zestawie 21 problemów $\mathcal{NP}$-zupełnych w~pracy~\cite{DBLP:Karp10} Richarda Karpa z~roku 1972.
  Studium klasy problemów $\mathcal{NP}$-zupełnych od ponad 50 lat stanowi bardzo aktywną i obszerną dziedzinę algorytmiki oraz teorii obliczeń.
  Mimo niezwykle bogatego dorobku naukowego związanego z~analizą problemów $\mathcal{NP}$-zupełnych, pytanie czy klasy problemów $\mathcal{NP}$ i $\mathcal{P}$ są jednoznaczne nadal pozostaje otwarte --- nie udzielono na nie popartej konstruktywnymi dowodami odpowiedzi.
  Fakt ten warunkuje dalsze postępy w~tej dziedzinie i jednocześnie zachęca do zgłębiania opisywanej tematyki, oferując szerokie pole dla nowego wkładu w~jej rozwój.
  Głównym celem pracy jest przedstawienie fundamentalnej teoretycznej wiedzy dotyczącej podejścia do rozwiązywania problemów $\mathcal{NP}$-zupełnych opartego o techniki parametryzacji na przykładzie problemu pokrycia wierzchołkowego grafu oraz materializacji koncepcji teoretycznych w~postaci implementacji opisywanych algorytmów.
  Za cele dodatkowe pracy uznaje się przedstawienie skomplikowanej matematycznie problematyki w~sposób przystępny dla czytelnika o tle inżynierskim, stanowiące zrozumiałą podporę w~dalszych badaniach lub adaptacji przedstawionych rozwiązań do konkretnych problemów praktycznych oraz utworzenie możliwie jednolitej platformy ułatwiającej implementację i badania eksperymentalne nad algorytmami związanymi z~problemami poruszającymi tematykę grafów.
}
\section{Układ pracy}\label{Section_Layout}
\par{
  Praca podzielona została na cztery główne rozdziały, związane ściśle z~poszczególnymi etapami prowadzonych prac.
}
\subsection{Rozdział teoretyczny --- opis zagadnienia}
\par{
  Rozdział~\ref{Chapter_Domain} zawiera opis teoretyczny oraz analizę koncepcji związanych z~poruszaną tematyką.

  Podrozdział~\ref{Section_Domain} stanowi szczegółowe wprowadzenie do domeny problemu pokrycia wierzchołkowego oraz oględnie przedstawia uznane grupy technik ograniczania złożoności obliczeniowej algorytmów rozwiązujących problemy należące do zbioru problemów $\mathcal{NP}$-zupełnych.

  Podrozdziały~\ref{s_methods} oraz~\ref{s_kernelization} skupiają się na opisie i analizie konkretnych technik należących do grup opisanych w~podrozdziale~\ref{Section_Domain} --- w~szczególności do grupy technik parametryzacji --- wykorzystanych do realizacji założonych w~niniejszej pracy celów.

  Podrozdział~\ref{s_definitions} stanowi zbiór Definicji podstawowych pojęć wykorzystywanych w~dalszych częściach pracy, których znajomość jest wymagana do zrozumienia prezentowanego toku rozumowania.
  Pojęcia zdefiniowane w~ramach podrozdziału~\ref{s_definitions} są wykorzystywane w~analizie wszystkich następujących koncepcji --- dlatego też zostały zagregowane w~miejscu je poprzedzającym.
  Pojęcia związane bezpośrednio z~konkretnym algorytmem zawarte są w~odpowiadającym mu podrozdziale.

  Podrozdział~\ref{Section_preprocessing} przybliża proste techniki modyfikacji struktury grafu stanowiące uzupełnienie mające na celu zwiększenie efektywności algorytmów opisywanych w~podrozdziale~\ref{s_kernelization}.

  Podrozdział~\ref{s_kernelization} przedstawia analizę poszczególnych technik redukcji dziedziny do jądra problemu pokrycia wierzchołkowego zaproponowanych w~literaturze źródłowej.

  Podrozdział~\ref{s_ckx} poświęcony jest w~całości algorytmowi zaproponowanemu w~pracy~\cite{ImprovedBounds10} --- algorytm ten stanowi wyczerpującą całość, wykorzystującą i łączącą opisane w~poprzedzających podrozdziałach koncepcje w~celu uzyskania dużej redukcji złożoności obliczeniowej.

  Podrozdział~\ref{s_supplementary_algorithms} przybliża algorytmy niezwiązane bezpośrednio z~dziedziną pokrycia wierzchołkowego, które stanowią jednak wartościowe narzędzia wykorzystywane przez techniki opisane w~podrozdziałach poprzedzających.

  Algorytmy opisane w podrozdziałach~\ref{s_kernelization} oraz~\ref{s_ckx} nazywane będą dalej \emph{algorytmami głównymi}.
}
\subsection{Specyfikacja wewnętrzna}
\par{
  Rozdział~\ref{s_internals} obejmuje opis wykorzystanych technologii, narzędzi i bibliotek zewnętrznych, architektury załączonego kodu źródłowego oraz wybrane pakiety oraz implementacje niektórych algorytmów przedstawionych w~opisie zagadanienia.
}
\subsection{Badania eksperymentalne i analiza wyników}
\par{
  Rozdział~\ref{s_results} poświęcony jest w~całości prezentacji wyników badań eksperymentalnych.
}
\subsection{Podsumowanie i kierunki dalszych prac}
\par{
  Rozdział~\ref{s_summary} zawiera podsumowanie pracy oraz opis napotkanych problemów wraz z~propozycjami usprawnień.
}
