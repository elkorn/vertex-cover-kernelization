\section{Opis dziedziny problemu}\label{Section_Domain}
\subsection{Wybrane klasy złożoności problemów}\label{subsection_p_np}
\par{
  Przez \emph{klasę złożoności} rozumie się zbiór problemów o~podobnej złożoności mierzonej pewnym kryterium. 
  Proste klasy złożoności określić można za pomocą podstawowych czynników, takich jak rodzaj problemu (obliczeniowy, decyzyjny, optymalizacyjny etc.), model  obliczeniowy (deterministyczna lub niedeterministyczna maszyna Turinga etc.)
  oraz wymagane do obliczeń zasoby wraz ze związanymi z~nimi gwarancjami i~ograniczeniami (czas wielomianowy, liniowa przestrzeń etc.).
}

\subsubsection{Wybrane popularne klasy problemów:}
\label{sss_popular_cplx_classes}
\par{
\begin{itemize}
  \item Klasa $\mathcal{P}$ obejmuje problemy decyzyjne rozwiązywalne przez
    deterministyczną maszynę Turinga w~czasie wielomianowym.
  \item Klasa $\mathcal{NP}$ obejmuje problemy decyzyjne, dla których dowody na 
    odpowiedzi pozytywne są \emph{weryfikowalne} przez deterministyczną maszynę
    Turinga w~czasie wielomianowym.
  \item Problemy $\mathcal{NP}$-trudne stanowią problemy, do których w~czasie
    wielomianowym zredukować można wszystkie problemy klasy $\mathcal{NP}$.
  \item Problemy $\mathcal{NP}$-zupełne stanowią problemy, które są zarówno
    $\mathcal{NP}$ jak i $\mathcal{NP}$-trudne.
    Cechą charakterystyczną problemów $\mathcal{NP}$-zupełnych jest to, iż
    dowolne rozwiązanie problemu $\mathcal{NP}$-zupełnego jest weryfikowalne
    w~czasie wielomianowym, jednak nie jest znana efektywna metoda odnalezienia
    rozwiązania w~rozsądnym (tj.\ wielomianowym) czasie. 
    W~chwili obecnej wszystkie znane algorytmy rozwiązujące problemy 
    $\mathcal{NP}$-zupełne wymagają czasu ponadwielomianowego względem rozmiaru
    danych wejściowych.
    Istnieją jednak ogólne techniki rozwiązywania problemów obliczeniowych,
    które pozwalają na uzyskanie czasów wielomianowych dla niektórych problemów
    $\mathcal{NP}$-zupełnych przy zachowaniu pewnych ograniczeń.
    \begin{itemize}
      \item \underline{Aproksymacja}: poświęcenie dokładności rozwiązania
        w~celu uzyskania przyspieszenia procesu decyzyjnego, polegające na zbliżaniu się do
        optimum zamiast poszukiwania dokładnego rozwiązania.
      \item \underline{Parametryzacja}: często istnieje możliwość rozwiązania
        problemu w~czasie wielomianowym przez zastosowanie parametrów wpływających na warunki
        poszukiwań rozwiązania.
      \item \underline{Heurystyka}: zastosowanie algorytmów działających
        ,,wystarczająco'' dobrze w~większości przypadków, jednak co do których
        nie ma pewności, że zawsze zapewnią prawidłowy wynik w~rozsądnym czasie.
      \item \underline{Randomizacja}: przy dopuszczeniu niewielkiego
        prawdopodobieństwa porażki istnieje szansa na poprawienie
        średniego czasu działania przez zastosowanie elementu losowości w
        działaniu algorytmu.
    \end{itemize}
  \item $\mathcal{FPT}$, obejmująca problemy \emph{łatwe w rozwiązaniu względem stałych parametrów} (ang. fixed-parameter tractable problems).
\end{itemize}

Należy zaznaczyć, że mimo skupienia uwagi na podejściu opartym o~parametryzację złożoności obliczeniowej, rozpatrywane w~ramach niniejszej pracy metody wykorzystują również pozostałe techniki, w~szczególności z~grupy aproksymacji i~redukcji, dla podproblemów składowych lub redukcji problemu pokrycia wierzchołkowego.
}
\subsubsection{\textbf{Kwestia $\mathcal{P}=\mathcal{NP}$}}
\label{sss_problem_p_neq_np}
\par{
  Jednym z~uzasadnień badań nad problemem pokrycia wierzchołkowego, prócz
  praktycznych zastosowań, jest próba odpowiedzi na pytanie czy klasa problemów
  $\mathcal{NP}$ nie jest równoznaczna klasie problemów $\mathcal{P}$.
  Pytanie to należy do grupy tzw. ,,Problemów Milenijnych'', a za pierwszą
  prawidłową odpowiedź fundacja Clay Mathematics Institute oferuje nagrodę w
  wysokości miliona dolarów. 
  Do tej pory argumenty zarówno za $\mathcal{P}=\mathcal{NP}$ jak i~za
  $\mathcal{P}\neq\mathcal{NP}$ nie są oparte na ścisłym matematycznym
  rozumowaniu, a raczej na empirycznych obserwacjach otaczającego świata.
  Głównym argumentem za $\mathcal{P}\neq\mathcal{NP}$ jest brak znaczących
  postępów w~dziedzinie wyszukiwania wyczerpującego oraz stwierdzenia
  unaoczniające, iż gdyby $\mathcal{P}=\mathcal{NP}$ było spełnione, nie 
  istniałyby znaczące różnice w~trudności między rozwiązaniem 
  problemu $\mathcal{NP}$-zupełnego, a~zweryfikowaniem poprawności gotowego 
  jego rozwiązania --- co wydaje się być sprzeczne z~dotychczasowym doświadczeniem.
  Konsekwencje $\mathcal{P}=\mathcal{NP}$ byłyby również negatywne dla dziedziny
  kryptografii, która jawnie czerpie korzyści z~$\mathcal{P}\neq\mathcal{NP}$.
  Odpowiedź ta mogłaby stanowić zagrożenie dla bezpieczeństwa cyfrowego.
}

\subsection{Parametryzowana złożoność obliczeniowa}
\label{sss_parametric_complexity}
\par{
  Problemy klasy $\mathcal{FPT}$ są rozwiązywalne w~czasie $f(k)\cdot n^{O(1)}$ --- gdzie $n$ stanowi rozmiar danych wejściowych, a $k$ parametr ograniczający w~pewien sposób parametry wyniku poszukiwań --- dla pewnej obliczalnej funkcji $f$.
  Funkcja $f$ w~praktyce jest zazwyczaj funkcją wykładniczą, jak na przykład $2^{O(k)}$.
  Definicja dopuszcza jednak funkcje jescze bardziej strome.
  Najważeniejszą przesłanką sformułowania klasy $\mathcal{FPT}$ jest wykluczenie postaci funkcji $f(n,k)$, uniemożliwiającej rozwiązanie problemu $\mathcal{NP}$-zupełnego w~czasie lepszym niż wykładniczy.
}

\subsection{Problem pokrycia wierzchołkowego}\label{s_vertex_cover_domain}
\par{
  Problem pokrycia wierzchołkowego jest problemem decyzyjnym.
  Polega on na udzieleniu odpowiedzi na pytanie ,,Czy w~grafie $G=(V,E)$ dla zadanego $k$
  istnieje zbiór wierzchołków $C \in V$ o liczebności $|C| \leq k$ pokrywający każdą krawędź tego grafu?''.
  Pokrycie wierzchołkowe stanowi zbiór wierzchołków $C \subseteq$ spełniający zależność $V\forall_{e=(u,v) \in E}:u\in C\lor v\in C$.
  Problem pokrycia wierzchołkowego należy do klasy problemów $\mathcal{NP}$-zupełnych, co udowodniono w~pracy~\cite{Kar72}.
}
\par{
  Problem pokrycia wierzchołkowego jest popularny w~dziedzinie biologii obliczeniowej. 
  Do praktycznych zastosowań algorytmów rozwiązujących problem pokrycia wierzchołkowego można zaliczyć:
  \begin{itemize}
    \item odnajdywanie drzew filogenetycznych na podstawie informacji
      dotyczących domen białkowych,
    \item analiza genetycznych cech ilościowych,
    \item analiza danych na mikromacierzach DNA.\@
  \end{itemize}
  Jednym z~zastosowań poza polem biologii obliczeniowej są prace nad dynamicznym wykrywaniem wyścigów w~danych~\cite{O'Callahan:2003:HDD:781498.781528}.
}
\begin{theorem}
  Optymalne pokrycie wierzchołkowe grafu $G=(V,E)$ o rozmiarach $|V|=n, |E|=m$ może zostać odnalezione w~czasie $O(2^{n}m)$.
\end{theorem}
\begin{bproof}
  Aby zweryfikować czy dany podzbiór $V_s \subseteq V$ pokrywa każdą krawędź
  $e \in E$, należy wykonać $O(m)$ porównań.
  Aby wykonać operację na wszystkich podzbiorach $V_s$, należy wykonać tę
  czynność dla wszystkich zbiorów należących do zbioru potęgowego 
  $P(V)$ o liczebności $|P(V)| = 2^{n}$.
  Aby odnaleźć pokrycie wierzchołkowe wśród podzbiorów $V$, należy dla każdego z~nich zrealizować operację weryfikacji pokrycia krawędzi, co w~rezultacie daje 
  złożoność $O(2^{n}m)$.
\end{bproof}
\par{
  Jak widać, postać rozwiązania ,,wprost'' problemu pokrycia wierzchołkowego wymaga czasu wykładniczego względem rozmiaru grafu wejściowego.
  Jednak z~pomocą opisywanych w~niniejszej pracy technik istnieje możliwość takiej modyfikacji struktury danych wejściowych, by całkowity czas potrzebny na rozwiązanie obramować pesymistyczną~złożonością wielomianową względem ich rozmiaru oraz pewnej wartości parametru ograniczającego rozmiar wyznaczanego pokrycia wierzchołkowego.
}
\par{
  W~celu uproszczenia zapisów, wprowadzona zostanie notacja $O^{\star}$, pomijająca czynniki wielomianowe w~złożoności w~oparciu o~to, że dla górnej granicy złożoności podstawowego algorytmu rozwiązującego problem pokrycia wierzchołkowego teoretycznie nie mają one znaczenia.
  \begin{equation*}
    O^{\star}(f(x))=O(f(x) \cdot w(x))
  \end{equation*}
  gdzie $w(x)$ jest wielomianem.

  \begin{theorem}\thlabel{th_vc_naive}
    W dowolnym grafie $G=(V, E)$ problem pokrycia wierzchołkowego jest rozwiązywalny w~czasie $O^{\star}(2^k)$ i wielomianowej przestrzeni dla zadanego $k \geq 0$.
  \end{theorem}
  \begin{bproof}
    Zakładając istnienie zbioru $C \subseteq V$ kandydującego do miana pokrycia wierzchołkowego, dla każdej krawędzi $(u, v) \in E$ zachodzić musi własność  $u \in C \lor v \in C$.
    Jeżeli więc istnieje krawędź $(u, v) \in E$ łącząca dwa wierzchołki, z których żaden nie należy do zbioru $C$, to należy dodać jeden z~tych wierzchołków do zbioru $C$.
    Rekurencyjnie realizowane są obydwie możliwości.
    Z~definicji pokrycia wierzchołkowego wynika, że zaakceptowanie wierzchołka jako należącego do optymalnego pokrycia wierzchołkowego eliminuje potrzebę rozpatrywania tego wierzchołka w~kolejnych iteracjach algorytmu.
    W~związku z~tym przed rozpoczęciem każdej kolejnej iteracji wierzchołek podlegający rozpatrzeniu wraz z~przystającymi do niego krawędziami należy usunąć z~dziedziny problemu w~celu wyznaczenia pokrycia wierzchołkowego o~rozmiarze co najwyżej $k-1$ w~pozostałej części grafu.
    Kiedy rekurencja dociera do momentu, gdzie zachodzi $k=0$ i istnieje krawędź $e=(u,v) \in E$ niepokryta przez zbiór $C$ wiadomo, że rozwiązanie odnalezione w~tej gałęzi nie jest akceptowalne i~należy je odrzucić.
    Bez względu na ostateczną postać rozwiązania zadanego problemu, algorytm z~każdą iteracją konsekwentnie zmniejsza rozmiar parametru poszukiwanego częściowego pokrycia.
    Łatwo zauważyć, że istnieją dwa warunki zakończenia działania algorytmu:
    \begin{enumerate}
      \item Zbiór krawędzi dziedziny problemu $E$ jest pusty co oznacza, że wierzchołki włączone dotychczas do rozwiązania pokrywają wszystkie krawędzie grafu.
      Oznacza to, iż odnaleziono minimalne pokrycie wierzchołkowe o~rozmiarze co najwyżej $k$ --- algorytm kończy działanie odpowiedzią twierdzącą po $O^\star(2^{k^\prime \leq k})=O^\star(2^k)$ operacjach.
      \item Wartość parametru osiągnęła $k=0$ co oznacza, iż odnalezione minimalne częściowe pokrycie wierzchołkowe zawiera już $k$ wierzchołków, lecz nadal nie pokrywa każdej krawędzi grafu $e \in E$.
      Algorytm kończy działanie odpowiedzią przeczącą po $O^\star(2^k)$ iteracjach.
    \end{enumerate}
  \end{bproof}
  Przedstawiony w~dowodzie Twierdzenia~\ref{th_vc_naive} tok rozumowania odzwierciedlony jest w~pseudokodzie~\ref{alg_VC1}.
  \begin{algorithm}
    \caption{Algorytm siłowy rozwiązujący problem pokrycia wierzchołkowego}\label{alg_VC1}
    \begin{algorithmic}[1]
      \Function{C}{$G$, $k$}

        \algorithmicrequire{graf wejściowy $G=(V, E)$, największa dopuszczalna liczebność pokrycia wierzchołkowego $k$}

        \algorithmicensure{informacja czy istnieje pokrycie wierzchołkowe o~liczebności $\leq k$}

        \If {$E=\emptyset$}
          \Return{true}\Comment{Odnaleziono pokrycie wierzchołkowe o~liczebności $\leq k$}
        \EndIf
        \If {$k=0$}
          \Return{false}\Comment{Nie istnieje pokrycie wierzchołkowe o~liczebności $\leq k$}
        \EndIf
        \State $(u,v) \leftarrow e \in E$
        \State $G^\prime \gets G[V\setminus \{u\}]$\Comment{Graf indukowany zbiorem $V\setminus \{u\}$}
        \State \textbf{return} {C($G^\prime$, $k-1$) lub C($G^\prime$, $k-1$)}
      \EndFunction
    \end{algorithmic}
  \end{algorithm}
}
